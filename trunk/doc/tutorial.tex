
% 10pt is the smallest font for article
\documentclass{article}

\usepackage{graphicx}
\usepackage{epsf}
\usepackage{a4}
\usepackage{palatino}
\usepackage{euler}
\newcommand {\atilde} {$_{\char '176}$} % tilde(~) character

\title{Coot Tutorial}
%(Part of Protein Crystallography)}

%\author{Your Workshop Here}
%\author{M.Res. Functional Genomics}
%\author{CCP4 Workshop Bangalore}
%\author{CCP4 Workshop}
%\author{CSHL 2006 Workshop}
%\author{Osaka 2006 Workshop}
%\author{EMBO 2007 Workshop}
%\author{Soleil 2008 Workshop}
%\author{CSHL 2008 Workshop}
%\author{ChemBioBootCamp 2009 Workshop}
%\author{CCP4 School APS 2010}
%\author{BCA/CCP4 Summer School Oxford 2010}
%\author{BCA/CCP4 Summer School Oxford 2012}
%\author{Universit\"at Basel Workshop}
\author{X-ray Methods, CSHL 2013}

\begin{document}
\maketitle
\tableofcontents
%\listoffigures


%% what does changing contour level tell us?

%% effects of resolution - blur the map and see what we can see now

%% proteins are surrounded by solvent

%% given an electron density map and a sequence, where do we go next?
%% i.e. where does the chain go?

%% What are the differnt types of maps we typically look at? and why
%% (you can add some maths here)

%% lets zoom out and look at the crystall packing - how are the
%% crystals fromed from arrangement of the protein chains

%% where are the challenges now?  Where do we struggle in the process?

%% how do we know that we have assigned the sequence correctly? what
%% do we know about how proteins pack their side-chains (by
%% hydrophobicity) - and for membrane proteins it's the reverse...

%% add (somewhere) a cartoon for the formation of a peptide bond.
%% Which limits the rotation freedom.

%% And energy considerations with bonds between sp2 or sp3 hybridized atoms.
%% how does this lead to rotamers (library)?

%% Project: Verify3D for membrane proteins

%% what numerical methods are we using to build the protein here?



\newpage

\section{Mousing}

First, how do we move around and select things?

\vspace{0.5cm}
  \begin{tabular}{ll}
    Left-mouse Drag & Rotate view \\
    Ctrl Left-Mouse Drag &  Translates view \\
    Shift Left-Mouse &  Label Atom\\
    Right-Mouse Drag &  Zoom in and out\index{zoom}\\
    Middle-mouse & Centre on atom\\
    Scroll-wheel Forward &  Increase map contour level\\
    Scroll-wheel Backward &  Decrease map contour level
  \end{tabular}
\vspace{0.5cm}

\section{Getting Started}

In this tutorial, we will learn how to do the following:
\begin{enumerate}
\item Start Coot
\item Display coordinates
\item Display a map
\item Zoom in and out
\item Recentre on Different Atoms
\item Change the Clipping (Slab)
\item Recontour the Map
\item Change the Map Colour
\item Display rotamers and refine residue
% and maybe some other things.
\end{enumerate}

\subsection{Load tutorial data}

To use coot, in a terminal window, type:


\texttt{\$ coot}

When you first start \texttt{coot}, it should look
something like Figure \ref{fig:start-coot}.

\begin{figure}[htbp]
  \begin{center}
    \leavevmode
%    \includegraphics[width=70mm]{images/Start-Screenshot.png}
    \epsfxsize 70mm
    \epsffile{images/Start-Screenshot.ps}
    \caption{Coot at Startup}{Not much to see at present...
      Actually, after this, coot screenshots will be displayed with a
      white background, whereas you will see a black one}
    \label{fig:start-coot}
  \end{center}
\end{figure}

So let's read in the coordinates and data (and thus generate a map)

To load the files on which we will be working:

From Coot's main menu bar, select
\textsf{Extensions $\rightarrow$ Load tutorial model and data}


\begin{figure}[htbp]
  \begin{center}
    \leavevmode
%    \includegraphics[width=70mm]{images/Coordinates-Screenshot.png}
    \epsfxsize 70mm
    \epsffile{images/Coordinates-Screenshot.ps}
    \caption{Coot After Loading Coordinates}
    \label{fig:coordinates-coot}
  \end{center}
\end{figure}

\subsection{Adjust Virtual Trackball}
By default, Coot has a ``virtual trackball'' to relate the motion of
the molecule to the motion of the mouse.  Many people don't like this.

So you might like to try the following. In the Coot main menu-bar:

\textsf{Edit $\rightarrow$ Preferences\ldots}

\textsl{  [Coot displays a Preferences window]}

On the left toolbar select \textsf{General} and then on the right top 
notebook tab \textsf{HID}. Select \textsf{``Flat''} to change the mouse motion.

(Use the \textsf{``Spherical Surface''} option to turn it back to how
it is by default).


What is the difference? ``Flat'' mode is like the mode used by ``O''.
In both modes, dragging the mouse near the centre of the screen causes
the view to rotate about the X- or Y- axis. However in spherical mode
you can also rotate about the z-axis by dragging the mouse along an
edge of the window.

\subsection{Display maps}

We are at the stage where we are looking at the results of the
refinement.  The refinement programs stores its data (labelled lists
of structure factor amplitudes and phases) in an ``MTZ'' file.  Let's
take a look\ldots


\begin{itemize}
\item Select \textsf{``File''} from the Coot menu-bar
\item Select \textsf{``Auto Open MTZ''} menu item
\textsl{  [Coot displays a Dataset File Selection window]}
\item Select the filename \texttt{rnasa-1.8-all\_refmac1.mtz}
\end{itemize}

If you choose instead ``Open MTZ, cif or phs...'' you will see:
\begin{itemize}

\item
\textsl{  [Coot displays a Dataset File Selection window]}

\item Select the filename \texttt{rnasa-1.8-all\_refmac1.mtz}

\textsl{  [Coot displays a Dataset Column Label Selection window]}


\begin{figure}[htbp]
  \begin{center}
    \leavevmode
%    \includegraphics[width=30mm]{images/ColumnLabel-Screenshot.png}
    \epsfxsize 30mm
    \epsffile{images/ColumnLabel-Screenshot.ps}
    \caption{Coot MTZ Column Label Selection Window}
    \label{fig:mtz_columns}
  \end{center}
\end{figure}

Notice that you have a selection of different column labels for the
``Amplitudes'' and ``Phases'', however, let's use the defaults:
``FWT'' and ``PHWT''.

\item Press \textsf{``OK''} in the Column Label Window

Now open the MTZ file and select column labels ``DELFWT'' and ``PHDELWT''.

So now we have 2 maps (whether auto-opened or not).
\end{itemize}


\subsection{Zoom in and out}
To zoom in, click Right-mouse and drag it left-to-right\footnote{or
  up-to-down, if you prefer that}. To zoom out again, move the mouse
the opposite way.

\begin{figure}[htbp]
  \begin{center}
    \leavevmode
%    \includegraphics[width=70mm]{images/Map-Screenshot.png}
    \epsfxsize 70mm
    \epsffile{images/Map-Screenshot.ps}
    \caption{Coot after reading an MTZ file and zoomed in.}
    \label{fig:map_screenshot}
  \end{center}
\end{figure}

\subsection {Recentre on Different Atoms}
\begin{itemize}
\item Select \textsf{``Draw''} from the Coot menu-bar
\item Select \textsf{``Go To Atom\ldots''}

\textsl{  [Coot displays the Go To Atom window]}

\item Expand the tree for the ``A'' chain
\item Select \texttt{1 ASP} in the residue list
\item Click \textsf{``Apply''} in the Go To Atom window
  
\item At your leisure, use \textsf{``Next Residue''} and
  \textsf{``Previous Residue''} (or ``Space'' and ``Shift''
  ``Space''in the graphics window) to move along the chain.


\item Click Middle-mouse over an atom in the graphics window

\textsl{[Coot recentres on that atom]}

\item Ctrl Left-mouse \& Drag moves the view around.  If this is a too
  slow and jerky:
  \begin{itemize}
  \item Select \textsf{Edit $\rightarrow$ Preferences\ldots} from Coot's menu-bar
  \item Select the \textsf{``Maps''} toolbutton on the left
  \item Select the \textsf{``Dragged Map''} notebook tab.
  \item Select \textsf{``No''} in the ``Active Map on Dragging'' window
  \item Click \textsf{``OK''} in the ``Preferences'' window
  \end{itemize}
  Now the map is recontoured at the end of the drag, not at each
  step\footnote{which looks less good on faster computers.}.

  Another, additional way to make the movements faster is to change the 
  number of steps for the ``Smooth Recentering''. Again you find this in
  the ``Preferences'' (\textsf{Preferences $\rightarrow$ General $\rightarrow$ 
  Smooth Recentering}). Change the \textsf{``Number of Steps''} (the 
  default is 40) to something smaller, e.g. 20.


\begin{figure}[htbp]
  \begin{center}
    \leavevmode
%    \includegraphics[width=70mm]{images/GoToAtom-Screenshot.png}
    \epsfxsize 70mm
    \epsffile{images/GoToAtom-Screenshot.ps}
    \caption{Coot's Go To Atom Window (it doesn't look \emph{exactly} like this any more).}
    \label{fig:goto-atom}
  \end{center}
\end{figure}

\item You can display the contacts too, as you do this:

  \begin{itemize}
  \item Select \textsf{``Measures''} from the Coot menu-bar
  \item Select \textsf{``Environment Distances\ldots''}
  \item Click on the \textsf{``Show Residue Environment?''} check-button
    \begin{itemize}
    \item Also Click \textsf{``Label Atom?''} if you wish the
      C$\alpha$ atoms of the residues to be labelled.
    \end{itemize}
  \item Click \textsf{``OK''} in the Environment Distances window
  \item Click \textsf{``Apply''} in the Go To Atom window

    \emph{[You can't change the colour of the Environment distances]}
  \end{itemize}

  \begin{figure}[htbp]
    \begin{center}
      \leavevmode
%      \includegraphics[width=70mm]{images/Environment-Screenshot.png}
      \epsfxsize 70mm
      \epsffile{images/Environment-Screenshot.ps}
      \caption{Coot showing Atom Label and environment distances.}
      \label{fig:environment}
    \end{center}
  \end{figure}
\end{itemize}

You can turn off the Environment distances if you like.

\subsection{Change the Clipping (Slab)}

\begin{itemize}
\item Select \textsf{``Draw''} from the Coot menu-bar
\item Select \textsf{``Clipping\ldots''} from the sub-menu

\textsl{ [Coot displays a Clipping window]}

\item Adjust the slider to the clipping of your choice
\item Click \textsf{``OK''} in the Clipping window
\end{itemize}

Alternatively, you can use ``D'' and ``F''\footnote{think:
  \underline{D}epth of \underline{F}ield.} on the keyboard, or Control
Right-mouse up/down (Control Right-mouse left/right does
z-translation).

\subsection{Recontour the Map}
\begin{itemize}
\item Scroll your scroll-wheel forwards one click\footnote{don't click
    it \emph{down}.}

 \textsl{ [Coot recontours the map using a 0.05electron/\AA$^3$ higher
  contour level]}

\item Scroll your scroll-wheel forwards and backwards more notches and
  see the contour level changing.
  
\item If you don't have a wheel on your mouse you can use ``+'' and
  ``-'' on the keyboard.
  
\item Note that the \textsf{``Scroll''} button in the Display Manager
  allows you to select which map is affected by this\footnote{by
    default it is the last map, which is not necessarily the map that
    you want.}.

\end{itemize}

%% Alternatively, use the Display Control window to get
%% the Map ``Properties'' button and which gives you a dialog in which
%% you can change the contour level using based on the sigma level of the
%% map.

\subsection{Change the Map Colour}
\begin{itemize}
\item Select \textsf{``Edit''} from the Coot menu-bar
\item Select \textsf{``Map Colour''} in the sub-menu
\item Select \textsf{``1 xxx FWT PHWT''} in the sub-menu

\textsl{ [Coot displays a Map Colour Selection window]}

\item Choose a new colour by clicking on the colour widgets

\textsl{ [Coot changes the map colour to match the selection ]}

\item Click \textsf{``OK''} in the Map Colour Selection window

\end{itemize}

\subsection{Select a Map}
Select a map for model building. We prefer you to use the modelling toolbar
icon buttons (here ``Map'') displayed on the right side of the Coot window.
Note: If you are not sure which icon corresponds to which function look at
the displayed tips when over the button with the mouse. Or change
the display style of the buttons by clicking on the bottom arrow and
select another style to get only or additionally text displayed.

\begin{trivlist}

\item Right Toolbar: \textsf{``Map''} from the Modelling Toolbar

\item click OK (you want to select the map with ``\ldots FWT PHWT'')

\end{trivlist}

Alternatively you can use the Model/Fit/Refine window:

\begin{trivlist}

\item Menubar: \textsf{Calculate $\rightarrow$ Model/Fit/Refine...}

\textsl{ [Coot displays the Model/Fit/Refine window]}

\item Select \textsf{``Select Map\ldots''} from the Model/Fit/Refine
  window

\item click OK (you want to select the map with ``\ldots FWT PHWT'')

\end{trivlist}

Note: Most options found in the Model/Fit/Refine window are available in
the Modelling Toolbar too. You can change this with Preferences.
  

%% \subsection{Display rotamers and refine residue}
%% \begin{itemize}
%% \item Select ``Draw'' from the Coot menu-bar
%% \item Select ``Go To Atom...''
%% \item Select ``89 A PHE'' from the residue list
%% \item Click ``Apply''\footnote{or double click on the residue label.}
%% \item Select ``Calculate'' from the Coot menu-bar
%% \item Select ``Model/Fit/Refine\ldots'' from the sub-menu

\section{Model Building}

``So what's wrong with this structure?'' you might ask.

There are several ways to analyse structural problems and some of them
are available in Coot.

\begin{trivlist}
% \item Validate $\rightarrow$ Geometry Analysis $\rightarrow$ tutorial.pdb
\item \textsf{Validate $\rightarrow$ Density Fit Analysis $\rightarrow$ tutorial-modern.pdb}

\textsl{ [Coot displays a bar graph]}

\item Look at the graph.  The bigger and redder the bar the worse the
  geometry. There are 2 area of outstanding badness in the A chain,
  around 41A and 89A.
\item Let's look at 89A first - click on the block for 89A.
\end{trivlist}

\textsl{ [Coot moves the view so that 89A CA is at the centre of the
  screen ]}

\subsection{Rotamers}

\begin{itemize}
\item Examine the situation\ldots
  \emph{[ The sidechain is pointing the wrong way.  Let's Fix it...]}

\item Modelling Toolbar: Select the \textsf{``Rotamers''} button

\item Alternatively:

\subitem Menubar: \textsf{Calculate $\rightarrow$ Model/Fit/Refine...}

\textsl{ [Coot displays the Model/Fit/Refine window]}

\subitem Select \textsf{``Rotamers''} from the Model/Fit/Refine window.

\item In the graphics window, (left-mouse) click on an atom of residue
  89A (the C$\gamma$, say)

\textsl{ [Coot displays the ``Select Rotamer'' window]}

\item Choose the Rotamer that most closely puts the atoms into the
  side-chain density %(it's rotamer 1\footnote{not unusual, that.})

\item Click \textsf{``Accept''} in the ``Select Rotamer'' window

\textsl{ [Coot updates the coordinates to the selected rotamer]}

\item Click \textsf{``Real Space Refine Zone''} in the
  Model/Fit/Refine window\footnote{\ldots or use the modelling toolbar button}.
  
\item In the graphics window, click on an atom of residue 89A.  Click
  it again.
  
  \textsl{ [Coot displays the refined coordinates in white in the
    graphics and a new ``Accept Refinement'' window]}

\item Click \textsf{``Accept''} in the ``Accept Refinement'' window.
  
  \textsl{ [Coot updates the coordinates to the refined coordinates.
    89A now fits the density nicely.]}


\end{itemize}

OK. That's good.  

Now, how about if we just use Real Space Refinement only?

\begin{itemize}
\item Click \textsf{``Undo''} twice

  \textsl{ [Coot puts the sidechain back to the original position] } 
  
\item Click \textsf{``Real Space Refine Zone''} in the
  Modelling Toolbar \footnote{\ldots or in the Model/Fit/Refine window}.
  
\item In the graphics window, click on an atom of residue 89A.  Click
  it again.

  \textsl{ [Coot displays the refined coordinates in white in the
    graphics and a new ``Accept Refinement'' window]}
  
\item Now using left-mouse, click and drag on the intermediate (white)
  CZ atom of the PHE (if you mis-click the atom, the view will
  rotate).
  
\item Can you pull the atom around so that the side-chain fits the
  density?

  \emph{ [Yes, you can] } 

\end{itemize}
  
  \begin{figure}[htbp]
    \begin{center}
      \leavevmode
%      \includegraphics[width=70mm]{images/89A-Screenshot.png}
      \epsfxsize 70mm
      \epsffile{images/89A-Screenshot.ps}
      \caption{89A now fits the density nicely.}
      \label{fig:89a-coot}
    \end{center}
  \end{figure}

\subsection{More Real Space Refinement}

Now let's have a look at the other region of outstanding badness:
\begin{itemize}
\item Click on the graph block for 41 A

\textsl{ [Coot moves the view so that 41A CA is at the centre of the screen]}

\item Examine the situation\ldots
\item Residue 41 is in a mess and not fitting to the density.  Can you
  fix it?

  \emph{ [Yes, you can] } 
\item The trick is Real Space Refine that zone. So\ldots
\item You can either Real Space Refine a few residues (40, 41 and 42)
  or just 41.  Take your pick.
\item Click on \textsf{``Real Space Refine Zone''} in the
  Modelling Toolbar (Model/Fit/Refine window)
\item Select a range by clicking on atoms in the graphics window
  (either atoms in 40 then 42 or an atom in 41 twice)

\textsl{[Coot displays intermediate (white) atoms]}

\item Click and drag on some atoms until the atoms fit nicely in the
  density.  

  If you want to move a \emph{single atom} then \emph{Ctrl Left-mouse}
  to pick and move (just) that intermediate atom.
  
  \emph{[Note: Selecting and moving just the Carbonyl Oxygen is a
    good idea - use Ctrl Left-mouse to move just one atom.]}

\end{itemize}


\section{Blobology}

\subsection{Find Blobs}
To be found under Validate (called ``Unmodelled Blobs'').  

%Select the map and the coordinates (which are used to arrange the
%waters in the unit cell ``close to'' the protein).  

You can use the defaults in the subsequent pop-up. Press ``Find Blobs''
and wait a short while.

You will get a new window that tell you that it has found unexplaned
blobs.  Time to find out what they are.

\subsubsection{Blob 3}
Let's start from Blob 3 (the blobs are ordered biggest to smallest -
Blobs 3 and 4 (if you have it) are the smallest).

\begin{itemize}
\item Click on ``Blob 3''

\textsl{ [Coot centres the screen on a blob]}

\item Examine the situation\ldots

  \emph{[We need something tetrahedral there\ldots]}
  
\item \textsf{``Place Atom At Pointer''} on the Modelling 
 Toolbar \footnote{also available in Model/Fit/Refine window}

\textsl{ [Coot shows a Pointer Atom Type window]}

\item \textsf{``SO4''} in the new window\ldots
\item In the ``Pointer Atom Added to Molecule:'' frame, change ``New
  Molecule'' to `\textsf{`tutorial-modern.pdb''}
\item Click \textsf{OK}.
\item Examine the situation\ldots
\item the orientation is not quite right.
\item Let's Real Space Refine it (you should know what to do by now\ldots)
\item (\textsf{``Real Space Refine Zone''} then click an atom in the
  SO$_4$ twice.  Accept)

\textsl{[The SO$_4$ fits better now].}

\end{itemize}

Blob 4 is like Blob 3 (isn't it?)

\subsubsection{Blob 2}
Click on the button \textsf{``Blob 2''} and examine the density.
Something is missing from the model.  What?

This protein and has been co-crystallized with its ligand substrate.
That's what missing: 3' GMP.  So let's add it\ldots

\begin{itemize}
\item \textsf{File $\rightarrow$ Get Monomer\ldots}
\item Type \texttt{3GP} in the box (Use Upper Case).  Press return\ldots

\textsl{ [Coot pauses for a few seconds while LIBCHECK and REFMAC run]}

\textsl{ [3GP appears]}

It is generally easier to work without hydrogens, so let's delete
them

\item \textsf{``Delete\ldots''} from the Modelling Toolbar
  \footnote{Found in the Model/Fit/Refine window too}
\item \textsf{``Hydrogens in Residue''}
\item click on an atom in 3GP

\textsl{ [Hydrogens disappear]}

The 3GP is displaced from where we want it to be.

\item \textsf{``Rotate/Translate Zone''} from the Modelling Toolbar
\footnote{Exists in the Model/Fit/Refine window too}
\item click on an atom in 3GP twice

\textsl{ [Intermediate (white) atoms appear]}

\item Drag it or use the sliders to move the intermediate fragment to
  the approximately the right place.
\item Click \textsf{``OK''} in the slider window
\item Now optimize the fit using ``Real Space Refine Zone'' like we did
  before.

Now merge this ligand into the protein:
\item \textsf{Calculate $\rightarrow$ Merge Molecules\ldots}
\item Click (activate) ``monomer-3GP.pdb''
\item into Molecule:
  
  \textsf{tutorial-modern.pdb}

\item \textsf{Merge}

\end{itemize}

\subsubsection{Blob 1}

\begin{itemize}
\item Click on ``\textsf{Blob 1}''

\item Examine the situation\ldots  What is this density?

  \emph{[We need to add residues to the C-terminus of the A chain.]}

The missing residues are GLN, THR, CYS (QTC).

\item \textsf{Calculate $\rightarrow$ Fit Loop\ldots}

\textsl{ [Coot pops-ups a Fit Loop dialog]}

Add residue number \texttt{94} to \texttt{96} in the A chain.

The single letter codes of the extra residues are \texttt{QTC}

\item Make sure that you can see the unmodelled density well, then click \textsf{``Fit Loop''}

\item Watch\ldots (fun eh?)

\item Examine the density fit. 
  
  \textsl{ [It should be fine, except at the C-terminus.  We need to
    add an OXT atom to the residue]}

\item \textsf{Calculate $\rightarrow$ Other Modelling Tools\ldots}

\item \textsf{Add OXT to Residue\ldots}

\item \textsf{Add it}

\end{itemize}



\begin{center}
------------------------------------------------------

This is a far as I expect you to get.

------------------------------------------------------
\end{center}


\section{Extra Fun (if you have time)}

\subsection{Waters}
Fit waters to the structure
\begin{itemize}
\item \textsf{``Find Waters\ldots''} in the Other Modelling Tools
  dialog \footnote{In older versions of Coot, the Find Waters button
    can be found in the Model/Fit/Refine dialog or the water drop in
    the modelling toolbar}, then OK (using the defaults)

\textsl{ [Waters appear]}

\end{itemize}
To check the waters:
\begin{itemize}
\item Click on \textsf{``Measures''} in the main menubar.  
\item Click on \textsf{Environment Distances}
\item Click on the \textsf{``Show Residue Environment?''} check-button
\item Change the distances as you like
\item Use the Go To Atom widget to go to the first water (probably
  towards the end of the list of residues in Chain ``W'')
\item Use the Spacebar to navigate to the next residue (and Shift
  Spacebar to go backwards)
\item If you don't like the fast sliding\footnote{it's not very good
    for water checking} you can turn it off (if you haven't done so already):
  \begin{itemize}
  \item Click \textsf{Edit $\rightarrow$ Preferences\ldots} in the main menubar
  \item Click \textsf{General $\rightarrow$ ``Smooth Recentering...''}
  \item Click \textsf{``No''}
  \end{itemize}
\end{itemize}


\subsection{Working with NCS}

The tutorial data has two-fold NCS. With good data it is interesting
to investigate the differences between the NCS copies. This may be
done by comparing the NCS related electron density, comparing the NCS
related chains, or comparing the Ramachandran plots for related chains:

\begin{itemize}
\item Comparing the NCS related electron density.

Go to \textsf{Calculate/NCS maps} and select the first map and
model. Now, if you go to anywhere in the A chain, you will see two
sets of electron density - the original density for the A chain, and
the density for the B chain transformed to overlap the A chain
density. This gives you a visual comparison of any differences.

Look at the density in the B molecule. Why does the density not agree
here? (Hint: the NCS is not a 2-fold rotation in this case, it is
improper).

\item Comparing the NCS-related chains.

  Go to \textsf{Draw/NCS ghost control} and select \textsf{Display
    non-crystallographic ghosts}. You will now see a copy of the B
  chain superimposed on the A chain. The B chain is draw using thin
  bonds.

  Look for regions where the models differ. Using the \textsf{NCS
    maps} tool, check that the differences in the models are supported
  by the density.

\item Compare the NCS-related chains using ``NCS jumping''.

  This is an alternative way of examining NCS similarities and
  differences.  Use the Display Manager to undisplay the NCS maps you
  recently created above.  

  Use the ``Goto Atom'' dialog and press the ``Update from Current
  Position'' button.  

  \textsl{ [Coot shifts the centre of the screen to the closest
    displayed CA atom]}

  Identify the chain that you are looking at (for example, by
  double-clicking on an atom)

  Now press the ``o'' key on the keyboard.

  \textsl{ [Coot shifts the centre of the screen to a NCS-related
    chain]}

  Identify the chain that you are looking at now.

  Press ``o'' again.

  What is happening?

  (Note: ''o'' stands for ``\underline{O}ther NCS-related chain''.)


\item Comparing the Ramachandran plots for related chains.

  Go to \textsf{Validate/Kleywegt plot}. Check the \textsf{Use
    specific chain} options and select chains A and B. You will get a
  Ramachandran plot with equivalent residues from the A and B chains
  linked by arrows. Long arrows indicate significant differences.
  Click on a dot to view the residue concerned.

\end{itemize}

Another tool for identifying NCS differences is the
\textsf{Validate/NCS differences} graph. This gives a clickable
histogram of differences.


\subsection{Add Terminal Residue}
[To be found on the Modelling Toolbar\footnote{and in the 
  Model/Fit/Refine window}]

\begin{itemize}
\item Use the Go To Atom window to go to residue 72 A.

\item What do you see?
  
  \emph{[You see missing atoms (that is, you don't see them).  This residue
  is supposed to be a CYS!  Bad things have happened to it.  Let's get
  rid of this residue. ]}
  
  \emph{[There may also be waters in the side-chain positions too (they have
  to be removed too).]} 

\item \textsf{Delete\ldots}

\item (Make sure ``Residue/Monomer'' is active)

\item Click on an atom of residue 72 A.

\textsl{  [Residue 72 A disappears]}

\item Lets' put back an ALA.

\item \textsf{Add Terminal Residue\ldots}

\textsl{  [Coot adds intermediate white atoms and pops up a dialog]}

\item \textsf{Accept} the new atoms

OK, we now have a ALA.  We want a CYS:

\item \textsf{Mutate \& Auto Fit\ldots}

\item Click on an atom in the new ALA

\textsl{  [Coot pops up a residue type chooser]}

\item Choose residue type \textsf{CYS}

\textsl{  [Coot mutates and fits a CYS]}

\item Examine the situation.

\emph{[ There is an extra blob of density on that CYS.  What is it?]}
It is an alternative conformation.

Let's model it

\item \textsf{Add Alt Conf\ldots}

\textsl{  [Coot pops up a residue splitter dialog]}

\item Choose how you want to split your residue (the Right Way is an
  unresolved philosophical issue\ldots\footnote{Actually Jane Richardson may disagree with that statement.})

\item Click on an atom in 72 A
  
  \textsl{ [Coot adds intermediate white atoms and pops up a Rotamer
    Chooser]}
  
\item Choose the correct rotamer - it shouldn't be difficult to tell
  which is the closest.

\item \textsf{Accept} when you are happy with you choice.
  
\item Now optimize the fit of this new residue with Real Space
  Refinement.

The refinement of each alternate conformations is independent.

\item \textsf{Real Space Refine Zone}

\item click on (say) SG,B of 72A

\item Press ``A'' (on the keyboard)

  \textsl{ [Coot moves the atoms a bit]}

\item \textsf{Accept}

\item Now do the same with the other conformation.

   \emph{  [Ahhh!  Much better.]}

\end{itemize}


\subsection{Display Symmetry Atoms}
To be found on \textsf{``Draw'' $\rightarrow$ ``Cell \& Symmetry''}
menu item in the main menu.

Try:
\begin{itemize}
\item \textsf{Show Symmetry Atoms? $\rightarrow$ Yes}
\item \textsf{Symmetry as Calphas?} [on]
\item \textsf{Colour symmetry by molecule} [on]
\item \textsf{Symmetry Radius 30\AA}
\item \textsf{Colour Merge $\rightarrow$ 0.1}
\item \textsf{Show Unit Cell?} [on]
\item \textsf{OK}
\item Zoom out and have a look at how the molecules pack together.
\end{itemize}


\subsection{Skeletonization and Baton Building}

You can calculate the map skeleton in Coot directly:

\textsf{Calculate $\rightarrow$ Map Skeleton... $\rightarrow$ On.}

This can be used to ``baton build'' a map.  You can turn off the
coordinates and try it if you like (the Baton Building window can be
found by clicking \textsf{``Ca Baton Mode\ldots''} in the Other
Modelling Tools dialog\footnote{this is quite advanced}).
  
If you want to do this, I suggest you use Go To Atom and start residue
2 A (this allows you to build the complete A chain in the correct
direction (it takes about 15 minutes or so) and you can compare it to
the real structure afterwards.
  
  Remember, when you start, you are placing a CA at the baton
  \textbf{tip} and at the start you are placing atom CA 1.  This might
  seem that you are ``double-backing'' on yourself - which can be
  confusing the first time.

\subsection{Refine with Refmac}

Refmac is a program that does Maximum Likelihood refinement of the
model, and the interface to it can be found on the Model/Fit/Refine
dialog. (By default there is no icon on the toolbar for this action
- open the window from the calculate menu\footnote{or add your own 
``Run Refmac'' button to the toolbar or the main/top toolbar. To add 
the button to the modelling/side toolbar, open the ``Preference'' window.
Now go to \textsf{Preferences $\rightarrow$ General $\rightarrow$ 
Refinement Toolbar}. In the right hand window scroll down and tick the
box besides ``Run Refmac\ldots''. A corresponding button will appear on the 
modelling toolbar. Alternatively you can add your own ``Run Refmac\ldots'' 
button to the top toolbar of the Coot window (only in Python 
versions). Click Right-mouse on the toolbar (besides the ``Display Manager'' 
button) and you will see a pop-up menu. Select ``Manage Buttons'',  
tick the box besides ``Add Alt Conf'' and click ``Apply''. A new button 
named ``Add Alt Conf'' will appear on the toolbar.}). Now 
click on \textsf{``Run Refmac\ldots''} (at the bottom of the 
Model/Fit/Refine window).

The defaults should be fine...

\textsf{``Run Refmac''} in the new window\ldots

\ldots Wait\ldots

\emph{[The cycle continues\ldots]}

\begin{itemize}
\item Use \textsf{``Validate $\rightarrow$ Difference Map Peaks''} to
  find interesting features in the newly created map and model.
\item Python enabled version only: a dialog with geometric outliers and
  other interesting things (if they exist) detected by Refmac shows up
  \footnote{This is an analysis from the Refmac log file and can 
  alternatively be opened from \textsf{Extensions... $\rightarrow$ 
  Refine...  $\rightarrow$ Read REFMAC log}}.
\end{itemize}


\section{Use the EDS}

\begin{trivlist}
\item Let's validate a structure deposited in the PDB.

\item Use a web browser and go to \texttt{http://eds.bmc.uu.se/eds/}.  Use
the keyword search to find the structure of a particular
protein/ligand/author.  

\item \textsf{File $\rightarrow$ Fetch PDB \& Map Using EDS\ldots}
  
\item Get the accession code and paste it into the entry box that pops
  up.
  
\item Use the validation tools in Coot to examine how good a PDB file
  this is... 

\item \emph{[the Difference Map Peaks can be interesting]}

\end{trivlist}


\subsection{Make a (Pretty?) Picture}

(Depends on the appropriate supporting software being available):

\begin{itemize}
\item Arrange a nice view
\item Press F8
\item Wait a few seconds\ldots
\end{itemize}


\section*{Colophon}

This document is written using XEmacs 21.5 in \LaTeX{} using AUC\TeX{}
and is distributed with the Coot source code.

\end{document}
