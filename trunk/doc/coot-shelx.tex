
\documentclass{article}

\usepackage{epsf}
\usepackage{a4}
\usepackage{palatino}
\usepackage{euler}
\newcommand {\atilde} {$_{\char '176}$} % tilde(~) character

\title{Coot for SHELX Tutorial}

\author{Paul Emsley \& Judit Debreczeni}

\begin{document}
\maketitle
\tableofcontents
%\listoffigures


\section{Introduction}

This is a description of the presentation of Coot at the ACA meeting
in Salt Lake City, 2007.

It should also serve as a tutorial for Coot specific for use with
SHELXL.

Get the latest version.

\section{A SHELX Project}

A SHELX Project consists of these things...

.ins and .res as valid coordinate inputs

\subsection{Reading .lst}

needs chain unsplitting

atom split, disagreeable restraints.

\subsection{Reading .res}

reads atoms, fvars, does not interpret most other things.

\subsubsection{chain splitting}

on residue number gap.

\subsection{Reading .fcf}

converts to .fcf.cif



%% Now let's get on and do stuff.  Some test data

%% Read in sample 03srv164.res
\section{03srv164}

OK, elecron density.

Too much, cut down.

\subsection{hydrogens}
Navigate to difference map peak.  Yuck! Hydrogens in wrong place.

Delete them.

Add in HFIX command add it to the .res file.

Turn off difference map.

Rerun SHELXL - and bing! new structure has hydrogens in the right
place and new map has no diff peaks.

Good.

We could play similar games with other hydrogens, but we won't.

\subsection{symmetry}

Let's show symmetry.

Turn off density.

% OK we are done with 03srv164.

close.

% Bstern:

\section{Bstern}

\subsection{Adding Atoms}


\subsection{B-factor variance}

include a picture here.





\end{document}
