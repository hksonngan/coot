
\documentclass[twocolumn]{article}
\usepackage{a4}
\usepackage{palatino}
\usepackage{euler}
% \usepackage{fancyhdr}

\newcommand {\atilde} {$_{\char '176}$} % tilde(~) character
\setlength{\columnsep}{10mm}
%\textwidth 6.0in % doesn't change left margin
\oddsidemargin 0.2in 

\title{Coot Crib Sheet}
\begin{document}
\maketitle

\section{Keyboard}

\subsection{Dialog Shortcuts}
\begin{tabular}{ll}
  F5  & Post Model/Fit/Refine dialog \\
  F6  & Post Go To Atom window \\
  F7  & Post Display Control Window\\
  F8 & Raster3D ``Screenshot''
\end{tabular}


\subsection{Previous/Next Residue}

\begin{tabular}{ll}
  ``Space'' & Next Residue \\
  ``Shift'' ``Space'' & Previous Residue
\end{tabular}

\subsection{Closest Residue}
``P'' go to an atom of the closest residue (the ``CA'' atom if the
residue has one)

\subsection{Next NCS Chain}
``O'' - other NCS chain.

\subsection{Previous/Next Rotamer}

When in ``Rotamer'' mode, these keyboard short-cuts are
available\footnote{note: focus must be in the graphics window, not
  the Rotamer dialog}:

\begin{tabular}{ll}
  ``.'' & Next Rotamer \\
  ``,'' & Previous Rotamer
\end{tabular}

\subsection{Keyboard Chi Angles}
Instead of pressing the buttons in the Chi Angles button box, you can
use keyboard ``1'' for Chi1, ``2'' for Chi2 \emph{etc.}

\subsection{Keyboard Contouring}

Use \texttt{+} or \texttt{-} to change the contour level

\subsection{Keyboard Labelling}
``l'' to label closest atom

\subsection{Keyboard Go To Residue} 
   \texttt{Ctrl-G} then key in a residue number and (optionally) a chain-id 
      and press Return 

\subsection{Keyboard Translation}
\begin{tabular}{ll}
  Keypad 3 & Push View (+Z translation)\\
  Keypad . & Pull View (-Z translation)
\end{tabular}

\subsection{Keyboard Undo}
\begin{tabular}{ll}

  Ctrl-Z & Undo last modification   \\
  U & Undo last move/navigation     \\
\end{tabular}

\subsection{Keyboard Zoom and Clip}

\begin{tabular}{ll}

  N & Zoom out   \\
  M & Zoom in    \\
  D & Slim clip  \\
  F & Fatten clip\\
\end{tabular}

\subsection{Crosshairs}
C: cross-hairs

\subsection{Skeleton}
S: Generate skeleton around current point\footnote{if a skeleton is being
displayed}

\subsection{Continuous Rotate}
I: Toggle continuous spin

\subsection{Baton Mode}
B: toggle into baton rotate mode\footnote{rather than view rotate
  mode}

\newpage
\section{Mouse}
Mouse actions are occassionally augmented with keyboard modifiers:
  \vspace{5mm}

  \begin{tabular}{ll}
    Left-mouse Drag & Rotate view \\
    Ctrl Left-Mouse Drag &  Translates view \\
    Shift Left-Mouse Click &  Label Atom\\
    Right-Mouse Drag &  Zoom in and out\index{zoom}\\
    Shift Right-Mouse Drag & Change clipping and Translate in Screen Z \\
    & The movement is along orthogonal axes: \\
    & up+right/down+left shifts in z, \\ 
    &  up+left/down+right changes the slab \\
    Ctrl Shift Right-Mouse Drag &  Rotate View about Screen Z\\
    Middle-mouse Click & Centre on atom\\
    Scroll-wheel Forward &  Increase map contour level\\
    Scroll-wheel Backward &  Decrease map contour level
  \end{tabular}

  \vspace{5mm}
  Intermediate (white) atoms can be dragged around by clicking on
  them:

  \vspace{5mm}
\begin{tabular}{ll}
 Left-mouse Drag:     & Move all intermediate \\
                      & atoms by linear shear \\
 Left-mouse Drag  & as above with\\
  with ``A'' key:
                                 &  non-linear shear\\
 Left-mouse Drag & Move a single atom\\
 with ``Ctrl'': 
\end{tabular}

\section{Refinement Extras}
Use ''A'' to define a residue range\footnote{+/- \emph{n} residues
  from the current residue} with a single-click. Useful in Refinement
and Regularization.

\begin{itemize}
\item Click ``Real Space Refine Zone''
\item Click on an atom
\item Press the ``A'' key
\end{itemize}

\end{document}

% \begin{trivlist}
% \item Left-mouse Drag: Move all intermediate atoms by linear shear
% \item Left-mouse Drag with ``A'' key: as above with non-linear shear
% \item Left-mouse Drag with ``Ctrl'': Move a single atom
% \end{trivlist}

%   \vspace{5mm}

% \begin{tabular}{ll}
%  Left-mouse Drag: & Move all intermediate atoms by linear shear \\
%  Left-mouse Drag with ``A'' key: & as above with non-linear shear \\
%  Left-mouse Drawg with ``Ctrl'': & Move a single atom
% \end{tabular}
