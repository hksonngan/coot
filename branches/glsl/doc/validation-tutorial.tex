
% 10pt is the smallest font for article
\documentclass{article}

\usepackage{graphicx}
\usepackage{epsf}
\usepackage{a4}
\usepackage{palatino}
\usepackage{euler}
\newcommand {\atilde} {$_{\char '176}$} % tilde(~) character

\title{Tutorial: Validation with Coot}

% \author{CCP4 Workshop New Delhi 2010}
\author{BCA/CCP4 Summer School Oxford 2010}

\begin{document}
\maketitle
%\tableofcontents
%\listoffigures

\section{Preamble}

  We will validate the structure in which we fitted the
  3-aminobenzamide ligands.

\section{Difference Map Peaks}

  One of the best tools for finding problems is the difference maps
  peaks tool. So let's use that that.  

  You should be able to see the ligand sites and other interesting
  features. What the problem here? (For me it is peak 5, but if you
  have run refmac with the model from the ligand fitting, then it may
  be peak number 1.)  Fix the problem.

\section{Rotamer Analysis}
 
Another useful tool is the Rotamer Analysis.  Use that tool (in the
\textsf{Validate} menu) to find problematic residues (note: not all
big bars are necessarily wrong).

%   [ Note to demonstrator:]

%   LEU A 358 is back to front 
%   VAL A 567 is back to front

%   LYS A 263 model as an ALA
%   ILE A 289 round the wrong way

%   TRP A 427 round the wrong way

\section{Ramachandran Plot and Difference Plots}

The Ramachandran Plot is a classic model-based validation tool.  The
Kleywegt Plot (\emph{i.e.} Ramachandran NCS difference Plot) is often
used to examine NCS-related mainchain differences.

Use these tools on this model to find Rotamers with problematic
phi/psi values.  What is the Kleywegt Plot telling you? Is there a
problem? If there is a problme, how might you fix it? (Hint is there
anything in Refinement/Regularization Control (R/RC) that might be of
use?)

\section{NCS and NCS Ghosts}

   Another way of showing NCS differences is using the NCS Ghosts, NCS
   Maps and NCS skipping.  

\subsection{NCS Skipping}

Press \texttt{"P"} to go to the nearest and then \texttt{"O"} to jump
between NCS-related models preserving the relative (NCS-compensating)
view.

\subsection{NCS Ghosts}

\textsf{Draw $\rightarrow$ NCS Ghosts Control\ldots} Then for the
chosen molecule turn on the NCS ghosts.

\subsection{NCS Maps}

NCS maps (\textsf{Calucalate $\rightarrow$ NCS maps}) are useful
addition to the NCS Ghosts because they show the maps corresponding
the the NCS ghost molecules in the context of the NCS master
(typically "A") chain.

 
\section{Sequence Validation}

Sometime people make sequence-based errors in building models.  Coot
has a tool that allows you to compare the sequence in the model with a
reference sequence.

\textsf{Validate $\rightarrow$ Alignment vs. PIR}

Read in the sequence file and press ``OK''.

Is there a sequence problem?  How might you fix it?

\section{Freestyle...}

Use the EDS to download the structure and data/map for \texttt{1BAV}.
Use the above tools to find any issues and correct them.

How about \texttt{1H4P}?  Any issues there?

And how about \texttt{2XDE}? % example of checking by NCS jumping (top
                             % peak), pepflip and LEU 6B.
(Hint: Diff map peaks, NCS jumping).

And how about \texttt{1QW9}?

And \texttt{1QEX}?  Problems?

And how about \texttt{3L0F}? (Easy to spot, non-trivial to
fix.) % example of ligand with incorrect restraints

Another: 3f1l


% 1BJI

% 3eni (new) vs 1m50 (old)

% 3f1l: high res, some model building problems, wrong sidechains, including a leu.

The point of this excercise is not to pick holes in other people's
structures - it is to illustrate (hopefully) that using modern
graphics software (and in particular, Coot) problems in structures can
be readily identified and fixed.  

% It's an opportunity to learn from and improve the software.

\section*{Bottom line:}

\begin{quotation}
  ``At least one person should look at the map\ldots'' 

  -- Dale Tronrud\footnote{I believe}
\end{quotation}

\end{document}
