
\documentclass{book}
\usepackage{a4}
\usepackage{palatino}
%\usepackage{times}
%\usepackage{utopia}
\usepackage{euler}
\usepackage{fancyhdr}
\usepackage{epsf}

\newcommand {\atilde} {$_{\char '176}$} % tilde(~) character

\title{The Coot Reference Manual}
\author{Paul Emsley \\\textsf{\small emsley@ysbl.york.ac.uk}}
%\makeindex  % Not at the moment.  There are no index markups (yet).

\begin{document}
\maketitle
\tableofcontents

\chapter{Acknowledgments}
Paul Emsley is extremely grateful to use the library code of the
following people, without whom Coot could not have been realised:

\begin{trivlist}
\item Kevin Cowtan
\item Eugene Krissinel
\item Stuart McNicholas
\item Raghavendra Chandrashekara
\item Paul Bourke \& Cory Gene Bloyd
\end{trivlist}

Roland Dunbrack \& co-workers for rotamer library data.

Also (for generally useful software used in Coot):

\begin{trivlist}
\item Matteo Frigo \& Steven G. Johnson
\item Gary Houston \& other Guile developers
\item Python developers
\item Gtk+ and GNOME-Canvas developers
\item GNU Scientific Library developers
\item OpenGL developers
\item Janne L\"of
\end{trivlist}

Also those with whom Paul has corresponded about or provided
features and bug fixes and built the software:

\begin{tabular}{ll}
 William G. Scott & Bernhard Lohkamp \\
 Joel Bard  & Ezra Peisach           \\
 Alex Schuettelkopf & Charlie Bond 
\end{tabular}

Not forgetting the testers\footnote{in no particular order}

%\begin{trivlist}
%\item Eleanor J. Dodson
%\item Jan Dohnalek
%\item Karen McLuskey
%\item Bernhard Lohkamp
%\item Aleks Roszak
%\item Florence Vincent
%\item Roberto Steiner
%\item Alex Schuettelkopf
%\item Charlie Bond
%\item Constantina Fotinou
%\item William G. Scott
%\item Adrian Lapthorn
%\end{trivlist}

\begin{tabular}{ll}
Eleanor J. Dodson & Jan Dohnalek \\
Constantina Fotinou & Alex Roszak  \\
Florence Vincent  & Roberto Steiner \\
Karen McLuskey & Adrian Lapthorn   
\end{tabular}

\vspace{5mm}

Those with experience of Quanta, XFit and O will notice similarities
between Coot and those programs, it's fair to say that they have had
considerable influence in the look of Coot, so Paul respectively
thanks for inspiration: Tom Oldfield, Alwyn Jones and Duncan McRee
(and their co-workers).

\chapter{Design Overview}
\section{Why?}
``Why does Coot exist?'' you might ask.  ``Given that other molecular
graphics\footnote{molecular graphics with protein modeling and
  density fitting functions, that is.} programs exist, why write
another?''

Because I like having the source code to programs I use and think that
others feel the same.  Because the other programs don't quite work how
I wanted them to\footnote{and of course, there was no way to fix
  that.}. Because there was the possibility to integrate molecular
graphics into the CCP4 Suite.  

As to why write Coot when CCP4MG was available: that is not how it
happened. Coot\footnote{it was called ``MapView'' at the time.} was
released over a year before CCP4MG was available.  I followed my own
design, toolkit and aesthetic decisions - for good or bad\footnote{for
  example, I was (and remain) less concerned about porting to various
  shades of Microsoft Windows operating systems than the CCP4MG
  developers.}.

\section{Hacker's Guide}

The are several core libraries that are fundamental to Coot:

\begin{itemize}
\item Clipper: Kevin Cowtan's General crystallographic object library
\item mmdb: CCP4's Coordinate Library
\item GTk+: GNU's GUI toolkit.
\end{itemize}

\subsection{GUI}
The GUI is almost entirely built using glade.  Glade writes out its
code in pure C.  This causes a problem.  \texttt{src/interface.h} and
\texttt{src/support.h} both get regenerated in ``C mode'' every time
glade is run.  So, after every time we change the GUI with glade, we
need to run \texttt{post-glade} to introduce the C/C++ linking type
declaration wrapper into these files.

Not all of the GUI is build with glade - there are dynamic elements,
for example the ``Map and Molecule (Display) Control'' window the
frame of which are generated in glade, but the hboxs are filled using
hand-made code (see \texttt{gtk-manual.c}).

\subsection{GUI/State Variables}
The graphics\_info\_t class contains a host of static state variables,
mostly manipulated by GUI element (\emph{e.g} button)
callbacks\footnote{mostly button clicked signals and menu item
  activative signals}. For historical reasons they are initially set
in \texttt{globjects.cc}.  Because the callbacks are written in C by
glade\footnote{the GUI builder}, these variables need a functional
interface to set the variables, and that interface is used by both the
GUI button\footnote{and other GUI elements} callbacks and is exported
to the scripting level.  These function declarations are in
\texttt{c-interface.h}.  All manipulations of graphics\_info\_t's
state variables go via \texttt{c-interface.h}.

Notice that MMDB functions are not allowed in
this interface\footnote{because SWIG chokes on them}. 

\subsection{Scripting}
So, SWIG uses \texttt{c-interface.h} to generate the python/scheme
scripting interface. The scripting language is chosen at
configure-time using either \texttt{--with-guile} or
\texttt{--with-python}.

\section{Validation}
As I write this, a few of us are cobbling together a XML-based system
for validation.  We think that validation data should be presented as
XML data that can be passed between packages and programs.  Either the
program itself will output the data, or we will write a wrapper to
turn the output into the appropriate XML format.  

These XML data will be then available for use in the molecular
graphics and will provide information for a ``Next Unusual Feature''
button.  The library to provide the XML cabability for this is expat,
the same library used in Perl's XML::Parser, Python's XML parser
Pyexpat and Mozilla's XML parser.

\subsection{Example: Temperature Factor Analysis}
Recall that the kurtosis of a distribution, $k$ is given by:

\begin{equation}
  \label{eq:kurtosis}
  k = \frac{\Sigma(X_i - \mu)^4} {N \sigma^4} - 3 
\end{equation}

We calculate the kurtosis for the isotropic temperature factors for
each residue in the molecule and residues with the most leptokurtic
distributions are written out to a file.  The format of the file is
XML.

This is an example of how we expect validation data to be presented to
molecular graphics programs.



\chapter{Refinement and Regularization}

A function that we need for Molecular Graphics is to be able to
regularize (a.k.a ``idealize'') the coordinates of the model.  In
order to do so we need to find the ideal values (also called here
``restraints'', using the Refmac nomenclature).  We have a
multivariable function minimizer that requires the gradients of the
parameters (the coordinates).  Here we describe how to generate the
gradients analytically.  We need the derivatives for the bond lengths,
angles, torsions and planes.

\section{Introduction}

The function that we are trying to minimize for refinement is $S$, where

\begin{displaymath}
  S = S_{bond} + S_{angle} + S_{torsion} + S_{plane} + S_{chiral} + -kS_{map}
\end{displaymath}

For regularization it is:
\begin{displaymath}
  S = S_{bond} + S_{angle} + S_{torsion}
\end{displaymath}



Let's take these 3 parts in turn:

% ------------------------------------------------------------------
%                  Bonds 
% ------------------------------------------------------------------

\section{Bonds}

\begin{displaymath}
  S_{bond} = \sum_{i=1}^{N_{bonds}} {(b_i - b_{0_i})^2}
\end{displaymath}

Where $b_{0_i}$ is the ideal length (from the Refmac dictionary) of
the $i$th bond, $\mathbf{b}_i$ is the bond vector and $b_i$ is its length.

\begin{eqnarray*}
  \label{eq:1}
  \frac{\partial S_i}{\partial x_m} & = & \frac{\partial S_i}{\partial b_i} 
  \frac{\partial b_i}{\partial x_m} \\
   & = & [2(b_i - b_{0_i})]   \frac{\partial b_i}{\partial x_m}
\end{eqnarray*}

\begin{displaymath}
  b_i = \sqrt((x_m-x_k)^2 + (y_m-y_k)^2 + (z_m-z_k)^2)
\end{displaymath}

So: 
\begin{eqnarray*}
\frac{\partial b_i}{\partial x_m} & = & (\frac{1}{2} \frac{1}{b_i}) 2 (x_m - x_k) \\
 &   = & \frac{(x_m - x_k)}{b_i}
\end{eqnarray*}

So: 
\begin{displaymath}
  \frac{\partial S_i}{\partial x_m} = 2[b_i - b_{0_1}] \frac{(x_m - x_k)}{b_i}
\end{displaymath}


% ------------------------------------------------------------------
%                  Angles 
% ------------------------------------------------------------------

\section{Angles}
We are trying to minimise $S_{angle}$, where (for simplicity I ignore
the weights)

\begin{displaymath}
  S_{angle} = \sum_{i=1}^{N_{angles}} {(\theta_i - \theta_{0_i})^2}
\end{displaymath}


Angle $\theta$ contributed to by atoms $k$, $l$ and $m$:

\begin{displaymath}
  \cos \theta = \frac{{\underline a}.{\underline b}}{ab}
\end{displaymath}

\begin{trivlist}
\item where
\item $\underline {a}$ is the bond of atoms $k$ and $l$ $((x_k-x_l), (y_k-y_l), (z_k-z_l))$
\item $\underline {b}$ is the bond of atoms $l$ and $m$  $((x_m-x_l), (y_m-y_l), (z_m-z_l))$
\item Note that the vectors point away from the middle atom $l$.
\end{trivlist}



So: 

\begin{equation}
  \label{eq:1}
  \theta = acos(P)
\end{equation}

where 

\begin{displaymath}
  P = \frac{{\underline a}.{\underline b}}{ab} 
\end{displaymath}

Using the Chain Rule:
\begin{equation}
  \label{eq:2}
  \frac{\partial \theta}{\partial _k} = \frac{\partial \theta}{\partial P} \frac{\partial P}{\partial x_k}
\end{equation}

Given that we are only intereted in $\theta$ in the range $0\rightarrow\pi$:

\begin{equation}
  \label{eq:3}
  \frac{\partial \theta}{\partial P} = -\frac{1}{\sin \theta}
\end{equation}

Let's split up $P$ again using the chain rule: 
\begin{equation}
  \label{eq:4}
  \frac{\partial P}{\partial x_k} = 
  Q\frac{\partial R}{\partial x_k} + R\frac{\partial Q}{\partial x_k}
\end{equation}

where 
\begin{equation}
  \label{eq:5}
  Q =  {\underline a}.{\underline b}
\end{equation}
\begin{equation}
  \label{eq:6}
  R = \frac{1}{ab}
\end{equation}

\subsection{The middle atom}

This is somewhat more tricky than an end atom because the derivatives
of $ab$ and ${\underline a}.{\underline b}$ are not so trivial.  Let's
change the indexing so that we are actually talking about the middle
atom, $l$.

Differentiating (\ref{eq:6}): 

\begin{equation}
  \label{eq:7}
  \frac{\partial R}{\partial x_l} = 
  -\frac{1}{(ab)^2}b\frac{\partial a}{\partial x_l} 
  -\frac{1}{(ab)^2}a\frac{\partial b}{\partial x_l}
\end{equation}

$\frac{\partial a}{\partial x_l}$ is exactly the same as we were using
with bonds:
\begin{displaymath}
  \frac{\partial a}{\partial x_l} = \frac{x_l-x_k}{a}
\end{displaymath}

Similarly:
\begin{displaymath}
  \frac{\partial b}{\partial x_l} = \frac{x_l-x_m}{a}
\end{displaymath}

So substituting those into (\ref{eq:7}):
\begin{displaymath}
  \frac{\partial R}{\partial x_l} = -\frac{x_l-x_k}{a^3b} -\frac{x_l-x_m}{ab^3}
\end{displaymath}

Turning to $Q$, recall (\ref{eq:5}), so: 
\begin{displaymath}
  Q =  
  ((x_k-x_l)(x_m-x_l) + (y_k-y_l)(y_m-y_l) + (z_k-z_l)(z_m-z_l))
\end{displaymath}

Therefore
\begin{displaymath}
   \frac{\partial Q}{\partial x_l} = -(x_k-x_l) -(x_m-x_l)
\end{displaymath}

Substituting all the above into (\ref{eq:4}):
\begin{displaymath}
  \frac{\partial P}{\partial x_l} = ({\underline a}.{\underline b})[-\frac{x_l-x_k}{a^3b} -\frac{x_l-x_m}{ab^3}] + \frac{-(x_k-x_l)-(x_m-x_l)}{ab}
\end{displaymath}

So, combining this and (\ref{eq:3}) into (\ref{eq:2}), we get: 
\begin{displaymath}
  \frac{\partial \theta}{\partial x_l} = -\frac{1}{\sin \theta}  \frac{\partial P}{\partial x_l} 
\end{displaymath}

%\begin{displaymath}
%  \frac{\partial \theta}{\partial x_l} = -\frac{1}{\sin \theta}(({\underline a}.{\underline b})[-\frac{x_l-x_k}{a^3b} -\frac{x_l-x_m}{ab^3}] + \frac{-(x_k-x_l)-(x_m-x_l)}{ab})
%\end{displaymath}






\subsection{An End Atom (Atoms $k$ or $m$)}
This is more simple because there are no cross terms in 
$\frac{\partial R}{\partial x_k}$ and $\frac{\partial Q}{\partial x_k}$.

\begin{displaymath}
  \frac{\partial R}{\partial x_k} = \frac{(x_k-x_l)}{ab}
\end{displaymath}

and 
\begin{displaymath}
  \frac{\partial Q}{\partial x_k} = (x_m-x_l)
\end{displaymath}

So 

\begin{equation}
  \frac{\partial \theta}{\partial x_k} = -\frac{1}{sin\theta} [\frac{(x_l-x_k)}{a^2}cos\theta + \frac{x_m-x_l}{ab}]
\end{equation}


% ------------------------------------------------------------------
%                  Torsions
% ------------------------------------------------------------------

\section{Torsions}
The torsion of 3 vectors (the vectors between one atom and the next in
the torsion angle) is given by:
\begin{equation}
  \label{eq:8}
  \tau(\mathbf{a},\mathbf{b},\mathbf{c}) = \arg(-\mathbf{a}.\mathbf{c}+(\mathbf{a}.\mathbf{b})(\mathbf{b}.\mathbf{c}), \mathbf{a}.(\mathbf{b} \mathbf{\times}\mathbf{c}))
\end{equation}


Let's split the expression up into tractable (for me) portions, the
evaluation of $\theta$ in the program will combine these expressions
starting at the end (the most simple).

\begin{figure}[htbp]
  \centering
  \leavevmode
  \epsfxsize=50mm
%  \epsffile{torsion.eps}
  \caption{Torsion vectors}
  \label{fig:torsion-vectors}
\end{figure}

Obviously: 
\begin{displaymath}
  a_x = P_{2_x}-P_{1_x} , b_x = P_{3_x}-P_{2_x} , c_x = P_{4_x}-P_{3_x}
\end{displaymath}
\begin{displaymath}
  a_y = P_{2_y}-P_{1_y} , b_y = P_{3_y}-P_{2_y} , c_y = P_{4_y}-P_{3_y}
\end{displaymath}
\begin{displaymath}
  a_z = P_{2_z}-P_{1_z} , b_z = P_{3_z}-P_{2_z} , c_z = P_{4_z}-P_{3_z}
\end{displaymath}

Unfortunately, I change the nomenclature because I derived the torsion
terms some time after the angle terms and I had forgotten what I had
previously been using.


\begin{displaymath}
  \theta = \tau(\mathbf{a},\mathbf{b},\mathbf{c}) =  \arctan(D)
\end{displaymath}

where
\begin{displaymath}
  D = \frac{\frac{\mathbf{a}.(\mathbf{b} \mathbf{\times}\mathbf{c})} {b}}{-\mathbf{a}.\mathbf{c}+\frac{(\mathbf{a}.\mathbf{b})(\mathbf{b}.\mathbf{c})}{b^2}}
\end{displaymath}

So

\begin{eqnarray}
  \label{eq:df}
  \frac{\partial \theta}{\partial x_{P_1}} & = & 
  \frac{\partial \theta}{\partial D} \frac{\partial D}{\partial x_{P_1}} \\
  & = & \frac{1}{1+D^2}\frac{\partial D}{\partial x_{P_1}}
\end{eqnarray}

Let
\begin{displaymath}
  E = \frac{\mathbf{a}.(\mathbf{b} \mathbf{\times}\mathbf{c})}{b}
\end{displaymath}
and 
\begin{displaymath}
  F = \frac{1}{-\mathbf{a}.\mathbf{c}+\frac{(\mathbf{a}.\mathbf{b})(\mathbf{b}.\mathbf{c})}{b}}
\end{displaymath}

\begin{equation}
  \label{eq:9}
  F = \frac{1}{G}
\end{equation}

Let
\begin{displaymath}
  G = -\mathbf{a}.\mathbf{c}+\frac{(\mathbf{a}.\mathbf{b})(\mathbf{b}.\mathbf{c})}{b^2}
\end{displaymath}

\begin{displaymath}
  H =  -\mathbf{a}.\mathbf{c}
\end{displaymath}

\begin{displaymath}
  J = \mathbf{a}.\mathbf{b}
\end{displaymath}

\begin{displaymath}
  K = \mathbf{b}.\mathbf{c}
\end{displaymath}

\begin{displaymath}
  L = \frac{1}{b^2}
\end{displaymath}

Differentiating  (\ref{eq:9})
\begin{displaymath}
  \frac{\partial F}{\partial x_{P_1}} = -\frac{1}{G^2}\frac{\partial G}{\partial x_{P_1}}
\end{displaymath}

%So now we have
%
%\begin{displaymath}
%  D = EF
%\end{displaymath}

Substituting for the derivative in (\ref{eq:df}):

\begin{displaymath}
  \frac{\partial \theta}{\partial x_{P_1}} = \frac{1}{1+D^2}[F\frac{\partial E}{\partial x_{P_1}} + E\frac{\partial F}{\partial x_{P_1}}]
\end{displaymath}


Also we have
\begin{displaymath}
  G = H + JKL
\end{displaymath}

Differentiating this: 

\begin{displaymath}
  \frac{\partial G}{\partial x_{P_1}} = \frac{\partial H}{\partial x_{P_1}} + JL\frac{\partial K}{\partial x_{P_1}} + KL\frac{\partial J}{\partial x_{P_1}} + JK\frac{\partial L}{\partial x_{P_1}}
\end{displaymath}

Let's look at the $H$, $J$, $K$ and $L$ derivatives:

\begin{displaymath}
    H = -\mathbf{a}.\mathbf{c} = -a_x c_x - a_y b_y - a_z c_z
\end{displaymath}

\begin{eqnarray*}
  \frac{\partial H}{\partial x_{P_1}} & = & c_x,\\
  \frac{\partial H}{\partial x_{P_2}} & = & -c_x,\\
  \frac{\partial H}{\partial x_{P_3}} & = & a_x,\\
  \frac{\partial H}{\partial x_{P_4}} & = & -a_x,\\
  \frac{\partial K}{\partial x_{P_1}} & = & 0,\\
  \frac{\partial K}{\partial x_{P_2}} & = & -c_x,\\
  \frac{\partial K}{\partial x_{P_3}} & = & c_x + b_x,\\
  \frac{\partial K}{\partial x_{P_4}} & = & b_x,\\
  \frac{\partial J}{\partial x_{P_1}} & = & -b_x,\\
  \frac{\partial J}{\partial x_{P_2}} & = & b_x - a_x,\\
  \frac{\partial J}{\partial x_{P_3}} & = & a_x,\\
  \frac{\partial J}{\partial x_{P_4}} & = & 0
\end{eqnarray*}

The $\frac{\partial b}{\partial x}$ terms are just like the bond
derivatives:

\begin{displaymath}
  \frac{\partial L}{\partial x_{P_1}} = \frac{\partial L}{\partial b} \frac{\partial b}{\partial x_{P_1}}
\end{displaymath}

\emph{i.e. }
\begin{eqnarray*}
  \frac{\partial L}{\partial x_{P_3}} & = &-\frac{2}{b^3} \frac{x_{P_3}-x_{P_2}}{b}\\
  & = &-\frac{2(x_{P_3}-x_{P_2})}{b^4}
\end{eqnarray*}

The derivative with respect to $x_{P_2}$ has the opposite sign.

Notice that $\mathbf{b}$ involves only atoms $P_2$ and $P_3$ so that
the derivates of $L$ with respect to the $P_1$ and $P_4$ coordinates are zero.

\subsection{$\frac{\partial E}{\partial x}$}
For the $\frac{\partial E}{\partial x}$ terms: 

Recall:
\begin{displaymath}
  E = \frac{\mathbf{a}.(\mathbf{b} \mathbf{\times}\mathbf{c})}{b}
\end{displaymath}

Let
\begin{displaymath}
  M = \mathbf{a}.(\mathbf{b} \mathbf{\times}\mathbf{c})
\end{displaymath}

\emph{i.e.}:
\begin{displaymath}
  E = \frac{M}{b}
\end{displaymath}

Differentiating that:
\begin{displaymath}
  \frac{\partial E}{\partial x_{P_3}} = -\frac{M}{b^2} \frac{\partial b}{\partial x_{P_3}} 
  +  \frac{1}{b} \frac{\partial M}{\partial x_{P_3}}
\end{displaymath}

Where, like bonds:
\begin{displaymath}
  \frac{\partial b}{\partial x_{P_3}} = \frac{x_{P_3}-x_{P_2}}{b}
\end{displaymath}
But note again, that the derivative of $b$ is zero for atoms $P_1$ and $P_4$.



\emph{i.e.} for atoms $P_2$ and $P_3$:
\begin{displaymath}
  \frac{\partial E}{\partial x_{P_3}} = -\frac{M(x_{P_3}-x_{P_2})}{b^3} + \frac{1}{b}\frac{\partial M}{\partial x_{P_3}}
\end{displaymath}

but for atoms $P_1$ and $P_4$:
\begin{displaymath}
  \frac{\partial E}{\partial x_{P_1}} =  \frac{1}{b}\frac{\partial M}{\partial x_{P_1}}
\end{displaymath}

\begin{displaymath}
  M = a_x(b_y c_z - b_z c_y) + a_y (b_z c_x - b_x c_z) + a_z (b_x c_y - b_y c_x)
\end{displaymath}

So here are the primitives of $M = \mathbf{a}.(\mathbf{b} \mathbf{\times}\mathbf{c})$

\begin{eqnarray*}
  \frac{\partial M}{\partial x_{P_1}} & = & -(b_y c_z - b_z c_y)\\
  \frac{\partial M}{\partial x_{P_2}} & = & (b_y c_z - b_z c_y) + (a_y c_z - a_z c_y)\\
  \frac{\partial M}{\partial x_{P_3}} & = & (a_z c_y - a_y c_z) + (b_y a_z - b_z a_y)\\
  \frac{\partial M}{\partial x_{P_4}} & = & (a_y b_z - a_z b_y)\\
  \frac{\partial M}{\partial y_{P_1}} & = & -(b_z c_x - b_x c_z)\\
  \frac{\partial M}{\partial y_{P_2}} & = & (b_z c_x - b_x c_z) + (a_z c_x - a_x c_z)\\
  \frac{\partial M}{\partial y_{P_3}} & = & -(a_z c_x - a_x c_z) + (b_z a_x - b_x a_z)\\
  \frac{\partial M}{\partial y_{P_4}} & = & -(b_z a_x - b_x a_z)\\
  \frac{\partial M}{\partial z_{P_1}} & = & -(b_x c_y - b_y c_x)\\
  \frac{\partial M}{\partial z_{P_2}} & = & (b_x c_y - b_y c_x) + (a_x c_y - a_y c_x)\\
  \frac{\partial M}{\partial z_{P_3}} & = & -(a_x c_y - a_y c_x) + (a_y b_x - a_x b_y)\\
  \frac{\partial M}{\partial z_{P_4}} & = & -(a_y b_x - a_x b_y)
\end{eqnarray*}

\subsection{Putting it together}

Combining, we get the following expression for the derivative of
$\theta$ in terms of the primitive derivates:
\begin{displaymath}
  \frac{\partial \theta}{\partial x_{P_1}} = \frac{1}{(1+\tan^2\theta)} \frac{\partial D}{\partial x_{P_1}}
\end{displaymath}

Where 
\begin{displaymath}
  \frac{\partial D}{\partial x_{P_1}} = [F \frac{\partial E}{\partial x_{P_1}} -\frac{E}{G^2} (\frac{\partial H}{\partial x_{P_1}} + JL \frac{\partial K}{\partial x_{P_1}} + KL  \frac{\partial J}{\partial x_{P_1}} + JK  \frac{\partial L}{\partial x_{P_1}})]  
\end{displaymath}









\chapter{Exported Functions}

@section File System Functions 
@subsection @code{(make-directory-maybe dir)}
@cindex @code{(make-directory-maybe dir)}
 
where: 
 @itemize 
     @item dir is a string
 @end itemize


@subsection @code{(set-show-paths-in-display-manager i)}
@cindex @code{(set-show-paths-in-display-manager i)}
 
where: 
 @itemize 
     @item i is an exact integer number
 @end itemize


@subsection @code{(show-paths-in-display-manager-state)}
@cindex @code{(show-paths-in-display-manager-state)}
 
@subsection @code{(add-coordinates-glob-extension ext)}
@cindex @code{(add-coordinates-glob-extension ext)}
 
where: 
 @itemize 
     @item ext is a string
 @end itemize


@subsection @code{(add-data-glob-extension ext)}
@cindex @code{(add-data-glob-extension ext)}
 
where: 
 @itemize 
     @item ext is a string
 @end itemize


@subsection @code{(add-dictionary-glob-extension ext)}
@cindex @code{(add-dictionary-glob-extension ext)}
 
where: 
 @itemize 
     @item ext is a string
 @end itemize


@subsection @code{(add-map-glob-extension ext)}
@cindex @code{(add-map-glob-extension ext)}
 
where: 
 @itemize 
     @item ext is a string
 @end itemize


@subsection @code{(set-sticky-sort-by-date)}
@cindex @code{(set-sticky-sort-by-date)}
 
@subsection @code{(set-filter-fileselection-filenames istate)}
@cindex @code{(set-filter-fileselection-filenames istate)}
 
where: 
 @itemize 
     @item istate is an exact integer number
 @end itemize


@subsection @code{(filter-fileselection-filenames-state)}
@cindex @code{(filter-fileselection-filenames-state)}
 

@section Widget Utilities 
@subsection @code{(info-dialog txt)}
@cindex @code{(info-dialog txt)}
 
where: 
 @itemize 
     @item txt is a string
 @end itemize



@section Widget Utilities 
@subsection @code{(manage-column-selector filename)}
@cindex @code{(manage-column-selector filename)}
 
where: 
 @itemize 
     @item filename is a string
 @end itemize



@section Molecule Info Functions 
@subsection @code{(chain-n-residues chain-id imol)}
@cindex @code{(chain-n-residues chain-id imol)}
 
where: 
 @itemize 
     @item chain-id is a string
     @item imol is an exact integer number
 @end itemize


@subsection @code{(molecule-centre-internal imol iaxis)}
@cindex @code{(molecule-centre-internal imol iaxis)}
 
where: 
 @itemize 
     @item imol is an exact integer number
     @item iaxis is an exact integer number
 @end itemize


@subsection @code{(n-chains imol)}
@cindex @code{(n-chains imol)}
 
where: 
 @itemize 
     @item imol is an exact integer number
 @end itemize


@subsection @code{(is-solvent-chain-p imol chain-id)}
@cindex @code{(is-solvent-chain-p imol chain-id)}
 
where: 
 @itemize 
     @item imol is an exact integer number
     @item chain-id is a string
 @end itemize


@subsection @code{(copy-molecule imol)}
@cindex @code{(copy-molecule imol)}
 
where: 
 @itemize 
     @item imol is an exact integer number
 @end itemize


@subsection @code{(exchange-chain-ids-for-seg-ids imol)}
@cindex @code{(exchange-chain-ids-for-seg-ids imol)}
 
where: 
 @itemize 
     @item imol is an exact integer number
 @end itemize



@section Library and Utility Functions 
@subsection @code{(coot-real-exit retval)}
@cindex @code{(coot-real-exit retval)}
 
where: 
 @itemize 
     @item retval is an exact integer number
 @end itemize


@subsection @code{(first-coords-imol)}
@cindex @code{(first-coords-imol)}
 

@section Graphics Utility Functions 
@subsection @code{(set-do-anti-aliasing state)}
@cindex @code{(set-do-anti-aliasing state)}
 
where: 
 @itemize 
     @item state is an exact integer number
 @end itemize


@subsection @code{(do-anti-aliasing-state)}
@cindex @code{(do-anti-aliasing-state)}
 
@subsection @code{(set-do-GL-lighting state)}
@cindex @code{(set-do-GL-lighting state)}
 
where: 
 @itemize 
     @item state is an exact integer number
 @end itemize


@subsection @code{(do-GL-lighting-state)}
@cindex @code{(do-GL-lighting-state)}
 
@subsection @code{(use-graphics-interface-state)}
@cindex @code{(use-graphics-interface-state)}
 
@subsection @code{(start-graphics-interface)}
@cindex @code{(start-graphics-interface)}
 
@subsection @code{(reset-view)}
@cindex @code{(reset-view)}
 
@subsection @code{(graphics-n-molecules)}
@cindex @code{(graphics-n-molecules)}
 
@subsection @code{(next-map-for-molecule imol)}
@cindex @code{(next-map-for-molecule imol)}
 
where: 
 @itemize 
     @item imol is an exact integer number
 @end itemize


@subsection @code{(toggle-idle-function)}
@cindex @code{(toggle-idle-function)}
 
@subsection @code{(set-idle-function-rotate-angle f)}
@cindex @code{(set-idle-function-rotate-angle f)}
 
where: 
 @itemize 
     @item f is an inexact number
 @end itemize


@subsection @code{(handle-read-draw-molecule filename)}
@cindex @code{(handle-read-draw-molecule filename)}
 
where: 
 @itemize 
     @item filename is a string
 @end itemize


@subsection @code{(read-pdb filename)}
@cindex @code{(read-pdb filename)}
 
where: 
 @itemize 
     @item filename is a string
 @end itemize


@subsection @code{(replace-fragment imol-target imol-fragment atom-selection)}
@cindex @code{(replace-fragment imol-target imol-fragment atom-selection)}
 
where: 
 @itemize 
     @item imol-target is an exact integer number
     @item imol-fragment is an exact integer number
     @item atom-selection is a string
 @end itemize


@subsection @code{(screendump-image filename)}
@cindex @code{(screendump-image filename)}
 
where: 
 @itemize 
     @item filename is a string
 @end itemize



@section Interface Preferences 
@subsection @code{(set-scroll-by-wheel-mouse istate)}
@cindex @code{(set-scroll-by-wheel-mouse istate)}
 
where: 
 @itemize 
     @item istate is an exact integer number
 @end itemize


@subsection @code{(scroll-by-wheel-mouse-state)}
@cindex @code{(scroll-by-wheel-mouse-state)}
 
@subsection @code{(set-default-initial-contour-level-for-map n-sigma)}
@cindex @code{(set-default-initial-contour-level-for-map n-sigma)}
 
where: 
 @itemize 
     @item n-sigma is an inexact number
 @end itemize


@subsection @code{(set-default-initial-contour-level-for-difference-map n-sigma)}
@cindex @code{(set-default-initial-contour-level-for-difference-map n-sigma)}
 
where: 
 @itemize 
     @item n-sigma is an inexact number
 @end itemize


@subsection @code{(print-view-matrix)}
@cindex @code{(print-view-matrix)}
 
@subsection @code{(get-view-matrix-element row col)}
@cindex @code{(get-view-matrix-element row col)}
 
where: 
 @itemize 
     @item row is an exact integer number
     @item col is an exact integer number
 @end itemize


@subsection @code{(get-view-quaternion-internal element)}
@cindex @code{(get-view-quaternion-internal element)}
 
where: 
 @itemize 
     @item element is an exact integer number
 @end itemize


@subsection @code{(set-view-quaternion i j k l)}
@cindex @code{(set-view-quaternion i j k l)}
 
where: 
 @itemize 
     @item i is an inexact number
     @item j is an inexact number
     @item k is an inexact number
     @item l is an inexact number
 @end itemize


@subsection @code{(set-fps-flag t)}
@cindex @code{(set-fps-flag t)}
 
where: 
 @itemize 
     @item t is an exact integer number
 @end itemize


@subsection @code{(get-fps-flag)}
@cindex @code{(get-fps-flag)}
 
@subsection @code{(set-show-origin-marker istate)}
@cindex @code{(set-show-origin-marker istate)}
 
where: 
 @itemize 
     @item istate is an exact integer number
 @end itemize


@subsection @code{(show-origin-marker-state)}
@cindex @code{(show-origin-marker-state)}
 
@subsection @code{(suck-model-fit-dialog)}
@cindex @code{(suck-model-fit-dialog)}
 
@subsection @code{(add-status-bar-text s)}
@cindex @code{(add-status-bar-text s)}
 
where: 
 @itemize 
     @item s is a string
 @end itemize


@subsection @code{(set-model-fit-refine-dialog-stays-on-top istate)}
@cindex @code{(set-model-fit-refine-dialog-stays-on-top istate)}
 
where: 
 @itemize 
     @item istate is an exact integer number
 @end itemize


@subsection @code{(model-fit-refine-dialog-stays-on-top-state)}
@cindex @code{(model-fit-refine-dialog-stays-on-top-state)}
 

@section Mouse Buttons 
@subsection @code{(quanta-buttons)}
@cindex @code{(quanta-buttons)}
 
@subsection @code{(quanta-like-zoom)}
@cindex @code{(quanta-like-zoom)}
 
@subsection @code{(set-control-key-for-rotate state)}
@cindex @code{(set-control-key-for-rotate state)}
 
where: 
 @itemize 
     @item state is an exact integer number
 @end itemize


@subsection @code{(control-key-for-rotate-state)}
@cindex @code{(control-key-for-rotate-state)}
 

@section Cursor Function 
@subsection @code{(normal-cursor)}
@cindex @code{(normal-cursor)}
 
@subsection @code{(fleur-cursor)}
@cindex @code{(fleur-cursor)}
 
@subsection @code{(pick-cursor-maybe)}
@cindex @code{(pick-cursor-maybe)}
 
@subsection @code{(rotate-cursor)}
@cindex @code{(rotate-cursor)}
 
@subsection @code{(set-pick-cursor-index icursor-index)}
@cindex @code{(set-pick-cursor-index icursor-index)}
 
where: 
 @itemize 
     @item icursor-index is an exact integer number
 @end itemize



@section Model/Fit/Refine Functions 
@subsection @code{(post-model-fit-refine-dialog)}
@cindex @code{(post-model-fit-refine-dialog)}
 
@subsection @code{(unset-model-fit-refine-dialog)}
@cindex @code{(unset-model-fit-refine-dialog)}
 
@subsection @code{(unset-refine-params-dialog)}
@cindex @code{(unset-refine-params-dialog)}
 
@subsection @code{(show-select-map-dialog)}
@cindex @code{(show-select-map-dialog)}
 
@subsection @code{(set-model-fit-refine-rotate-translate-zone-label txt)}
@cindex @code{(set-model-fit-refine-rotate-translate-zone-label txt)}
 
where: 
 @itemize 
     @item txt is a string
 @end itemize


@subsection @code{(set-model-fit-refine-place-atom-at-pointer-label txt)}
@cindex @code{(set-model-fit-refine-place-atom-at-pointer-label txt)}
 
where: 
 @itemize 
     @item txt is a string
 @end itemize


@subsection @code{(unset-other-modelling-tools-dialog)}
@cindex @code{(unset-other-modelling-tools-dialog)}
 
@subsection @code{(post-other-modelling-tools-dialog)}
@cindex @code{(post-other-modelling-tools-dialog)}
 

@section Backup Functions 
@subsection @code{(make-backup imol)}
@cindex @code{(make-backup imol)}
 
where: 
 @itemize 
     @item imol is an exact integer number
 @end itemize


@subsection @code{(turn-off-backup imol)}
@cindex @code{(turn-off-backup imol)}
 
where: 
 @itemize 
     @item imol is an exact integer number
 @end itemize


@subsection @code{(turn-on-backup imol)}
@cindex @code{(turn-on-backup imol)}
 
where: 
 @itemize 
     @item imol is an exact integer number
 @end itemize


@subsection @code{(backup-state imol)}
@cindex @code{(backup-state imol)}
 
where: 
 @itemize 
     @item imol is an exact integer number
 @end itemize


@subsection @code{(apply-undo)}
@cindex @code{(apply-undo)}
 
@subsection @code{(apply-redo)}
@cindex @code{(apply-redo)}
 
@subsection @code{(set-have-unsaved-changes imol)}
@cindex @code{(set-have-unsaved-changes imol)}
 
where: 
 @itemize 
     @item imol is an exact integer number
 @end itemize


@subsection @code{(set-undo-molecule imol)}
@cindex @code{(set-undo-molecule imol)}
 
where: 
 @itemize 
     @item imol is an exact integer number
 @end itemize


@subsection @code{(show-set-undo-molecule-chooser)}
@cindex @code{(show-set-undo-molecule-chooser)}
 
@subsection @code{(set-unpathed-backup-file-names state)}
@cindex @code{(set-unpathed-backup-file-names state)}
 
where: 
 @itemize 
     @item state is an exact integer number
 @end itemize


@subsection @code{(unpathed-backup-file-names-state)}
@cindex @code{(unpathed-backup-file-names-state)}
 

@section Recover Session Function 
@subsection @code{(recover-session)}
@cindex @code{(recover-session)}
 

@section Map Functions 
@subsection @code{(calc-phases-generic mtz-file-name)}
@cindex @code{(calc-phases-generic mtz-file-name)}
 
where: 
 @itemize 
     @item mtz-file-name is a string
 @end itemize


@subsection @code{(scroll-wheel-map)}
@cindex @code{(scroll-wheel-map)}
 
@subsection @code{(save-previous-map-colour imol)}
@cindex @code{(save-previous-map-colour imol)}
 
where: 
 @itemize 
     @item imol is an exact integer number
 @end itemize


@subsection @code{(restore-previous-map-colour imol)}
@cindex @code{(restore-previous-map-colour imol)}
 
where: 
 @itemize 
     @item imol is an exact integer number
 @end itemize


@subsection @code{(set-active-map-drag-flag t)}
@cindex @code{(set-active-map-drag-flag t)}
 
where: 
 @itemize 
     @item t is an exact integer number
 @end itemize


@subsection @code{(get-active-map-drag-flag)}
@cindex @code{(get-active-map-drag-flag)}
 
@subsection @code{(set-last-map-colour f1 f2 f3)}
@cindex @code{(set-last-map-colour f1 f2 f3)}
 
where: 
 @itemize 
     @item f1 is an unknown type
     @item f2 is an unknown type
     @item f3 is an unknown type
 @end itemize


@subsection @code{(set-map-colour imol red green blue)}
@cindex @code{(set-map-colour imol red green blue)}
 
where: 
 @itemize 
     @item imol is an exact integer number
     @item red is an inexact number
     @item green is an inexact number
     @item blue is an inexact number
 @end itemize


@subsection @code{( map-no gdouble[4])}
@cindex @code{( map-no gdouble[4])}
 
where: 
 @itemize 
     @item map-no is an exact integer number
     @item gdouble[4] is an unknown type
 @end itemize


@subsection @code{(handle-symmetry-colour-change mol gdouble[4])}
@cindex @code{(handle-symmetry-colour-change mol gdouble[4])}
 
where: 
 @itemize 
     @item mol is an exact integer number
     @item gdouble[4] is an unknown type
 @end itemize


@subsection @code{(set-last-map-sigma-step f)}
@cindex @code{(set-last-map-sigma-step f)}
 
where: 
 @itemize 
     @item f is an inexact number
 @end itemize


@subsection @code{(set-contour-by-sigma-step-by-mol f state imol)}
@cindex @code{(set-contour-by-sigma-step-by-mol f state imol)}
 
where: 
 @itemize 
     @item f is an inexact number
     @item state is an exact integer number
     @item imol is an exact integer number
 @end itemize


@subsection @code{(data-resolution imol)}
@cindex @code{(data-resolution imol)}
 
where: 
 @itemize 
     @item imol is an exact integer number
 @end itemize


@subsection @code{(export-map imol filename)}
@cindex @code{(export-map imol filename)}
 
where: 
 @itemize 
     @item imol is an exact integer number
     @item filename is a string
 @end itemize


@subsection @code{(rotate-map-round-screen-axis-x r-degrees)}
@cindex @code{(rotate-map-round-screen-axis-x r-degrees)}
 
where: 
 @itemize 
     @item r-degrees is an inexact number
 @end itemize


@subsection @code{(rotate-map-round-screen-axis-y r-degrees)}
@cindex @code{(rotate-map-round-screen-axis-y r-degrees)}
 
where: 
 @itemize 
     @item r-degrees is an inexact number
 @end itemize


@subsection @code{(rotate-map-round-screen-axis-z r-degrees)}
@cindex @code{(rotate-map-round-screen-axis-z r-degrees)}
 
where: 
 @itemize 
     @item r-degrees is an inexact number
 @end itemize



@section Density Increment 
@subsection @code{(set-iso-level-increment val)}
@cindex @code{(set-iso-level-increment val)}
 
where: 
 @itemize 
     @item val is an inexact number
 @end itemize


@subsection @code{(set-iso-level-increment-from-text text imol)}
@cindex @code{(set-iso-level-increment-from-text text imol)}
 
where: 
 @itemize 
     @item text is a string
     @item imol is an exact integer number
 @end itemize


@subsection @code{(set-diff-map-iso-level-increment val)}
@cindex @code{(set-diff-map-iso-level-increment val)}
 
where: 
 @itemize 
     @item val is an inexact number
 @end itemize


@subsection @code{(set-diff-map-iso-level-increment-from-text text imol)}
@cindex @code{(set-diff-map-iso-level-increment-from-text text imol)}
 
where: 
 @itemize 
     @item text is a string
     @item imol is an exact integer number
 @end itemize


@subsection @code{(set-map-sampling-rate-text text)}
@cindex @code{(set-map-sampling-rate-text text)}
 
where: 
 @itemize 
     @item text is a string
 @end itemize


@subsection @code{(set-map-sampling-rate r)}
@cindex @code{(set-map-sampling-rate r)}
 
where: 
 @itemize 
     @item r is an inexact number
 @end itemize


@subsection @code{(get-map-sampling-rate)}
@cindex @code{(get-map-sampling-rate)}
 
@subsection @code{(set-scrollable-map imol)}
@cindex @code{(set-scrollable-map imol)}
 
where: 
 @itemize 
     @item imol is an exact integer number
 @end itemize


@subsection @code{(change-contour-level is-increment)}
@cindex @code{(change-contour-level is-increment)}
 
where: 
 @itemize 
     @item is-increment is an exact integer number
 @end itemize


@subsection @code{(set-last-map-contour-level level)}
@cindex @code{(set-last-map-contour-level level)}
 
where: 
 @itemize 
     @item level is an inexact number
 @end itemize


@subsection @code{(set-last-map-contour-level-by-sigma n-sigma)}
@cindex @code{(set-last-map-contour-level-by-sigma n-sigma)}
 
where: 
 @itemize 
     @item n-sigma is an inexact number
 @end itemize


@subsection @code{(set-stop-scroll-diff-map i)}
@cindex @code{(set-stop-scroll-diff-map i)}
 
where: 
 @itemize 
     @item i is an exact integer number
 @end itemize


@subsection @code{(set-stop-scroll-iso-map i)}
@cindex @code{(set-stop-scroll-iso-map i)}
 
where: 
 @itemize 
     @item i is an exact integer number
 @end itemize


@subsection @code{(set-stop-scroll-iso-map-level f)}
@cindex @code{(set-stop-scroll-iso-map-level f)}
 
where: 
 @itemize 
     @item f is an inexact number
 @end itemize


@subsection @code{(set-stop-scroll-diff-map-level f)}
@cindex @code{(set-stop-scroll-diff-map-level f)}
 
where: 
 @itemize 
     @item f is an inexact number
 @end itemize


@subsection @code{(set-residue-density-fit-scale-factor f)}
@cindex @code{(set-residue-density-fit-scale-factor f)}
 
where: 
 @itemize 
     @item f is an inexact number
 @end itemize



@section Density Functions 
@subsection @code{(set-map-line-width w)}
@cindex @code{(set-map-line-width w)}
 
where: 
 @itemize 
     @item w is an exact integer number
 @end itemize


@subsection @code{(map-line-width-state)}
@cindex @code{(map-line-width-state)}
 
@subsection @code{(mtz-file-has-phases-p mtz-file-name)}
@cindex @code{(mtz-file-has-phases-p mtz-file-name)}
 
where: 
 @itemize 
     @item mtz-file-name is a string
 @end itemize


@subsection @code{(is-mtz-file-p filename)}
@cindex @code{(is-mtz-file-p filename)}
 
where: 
 @itemize 
     @item filename is a string
 @end itemize


@subsection @code{(auto-read-make-and-draw-maps filename)}
@cindex @code{(auto-read-make-and-draw-maps filename)}
 
where: 
 @itemize 
     @item filename is a string
 @end itemize


@subsection @code{(set-auto-read-do-difference-map-too i)}
@cindex @code{(set-auto-read-do-difference-map-too i)}
 
where: 
 @itemize 
     @item i is an exact integer number
 @end itemize


@subsection @code{(auto-read-do-difference-map-too-state)}
@cindex @code{(auto-read-do-difference-map-too-state)}
 
@subsection @code{(set-density-size-from-widget text)}
@cindex @code{(set-density-size-from-widget text)}
 
where: 
 @itemize 
     @item text is a string
 @end itemize


@subsection @code{(set-map-radius f)}
@cindex @code{(set-map-radius f)}
 
where: 
 @itemize 
     @item f is an inexact number
 @end itemize


@subsection @code{(set-density-size f)}
@cindex @code{(set-density-size f)}
 
where: 
 @itemize 
     @item f is an inexact number
 @end itemize


@subsection @code{(set-map-radius-slider-max f)}
@cindex @code{(set-map-radius-slider-max f)}
 
where: 
 @itemize 
     @item f is an inexact number
 @end itemize


@subsection @code{(set-display-intro-string str)}
@cindex @code{(set-display-intro-string str)}
 
where: 
 @itemize 
     @item str is a string
 @end itemize


@subsection @code{(set-esoteric-depth-cue istate)}
@cindex @code{(set-esoteric-depth-cue istate)}
 
where: 
 @itemize 
     @item istate is an exact integer number
 @end itemize


@subsection @code{(esoteric-depth-cue-state)}
@cindex @code{(esoteric-depth-cue-state)}
 
@subsection @code{(set-swap-difference-map-colours i)}
@cindex @code{(set-swap-difference-map-colours i)}
 
where: 
 @itemize 
     @item i is an exact integer number
 @end itemize


@subsection @code{(set-map-is-difference-map imol)}
@cindex @code{(set-map-is-difference-map imol)}
 
where: 
 @itemize 
     @item imol is an exact integer number
 @end itemize


@subsection @code{(another-level)}
@cindex @code{(another-level)}
 
@subsection @code{(another-level-from-map-molecule-number imap)}
@cindex @code{(another-level-from-map-molecule-number imap)}
 
where: 
 @itemize 
     @item imap is an exact integer number
 @end itemize


@subsection @code{(residue-density-fit-scale-factor)}
@cindex @code{(residue-density-fit-scale-factor)}
 

@section Parameters from map 
@subsection @code{(mtz-use-weight-for-map imol-map)}
@cindex @code{(mtz-use-weight-for-map imol-map)}
 
where: 
 @itemize 
     @item imol-map is an exact integer number
 @end itemize



@section PDB Functions 
@subsection @code{(write-pdb-file imol file-name)}
@cindex @code{(write-pdb-file imol file-name)}
 
where: 
 @itemize 
     @item imol is an exact integer number
     @item file-name is a string
 @end itemize



@section Refmac Functions 
@subsection @code{(set-refmac-molecule imol)}
@cindex @code{(set-refmac-molecule imol)}
 
where: 
 @itemize 
     @item imol is an exact integer number
 @end itemize


@subsection @code{(set-refmac-counter imol refmac-count)}
@cindex @code{(set-refmac-counter imol refmac-count)}
 
where: 
 @itemize 
     @item imol is an exact integer number
     @item refmac-count is an exact integer number
 @end itemize


@subsection @code{(swap-map-colours imol1 imol2)}
@cindex @code{(swap-map-colours imol1 imol2)}
 
where: 
 @itemize 
     @item imol1 is an exact integer number
     @item imol2 is an exact integer number
 @end itemize


@subsection @code{(set-keep-map-colour-after-refmac istate)}
@cindex @code{(set-keep-map-colour-after-refmac istate)}
 
where: 
 @itemize 
     @item istate is an exact integer number
 @end itemize


@subsection @code{(keep-map-colour-after-refmac-state)}
@cindex @code{(keep-map-colour-after-refmac-state)}
 

@section Symmetry Functions 
@subsection @code{(set-symmetry-size-from-widget text)}
@cindex @code{(set-symmetry-size-from-widget text)}
 
where: 
 @itemize 
     @item text is a string
 @end itemize


@subsection @code{(set-symmetry-size f)}
@cindex @code{(set-symmetry-size f)}
 
where: 
 @itemize 
     @item f is an inexact number
 @end itemize


@subsection @code{(get-show-symmetry)}
@cindex @code{(get-show-symmetry)}
 
@subsection @code{(set-show-symmetry-master state)}
@cindex @code{(set-show-symmetry-master state)}
 
where: 
 @itemize 
     @item state is an exact integer number
 @end itemize


@subsection @code{(set-show-symmetry-molecule mol-no state)}
@cindex @code{(set-show-symmetry-molecule mol-no state)}
 
where: 
 @itemize 
     @item mol-no is an exact integer number
     @item state is an exact integer number
 @end itemize


@subsection @code{(symmetry-as-calphas mol-no state)}
@cindex @code{(symmetry-as-calphas mol-no state)}
 
where: 
 @itemize 
     @item mol-no is an exact integer number
     @item state is an exact integer number
 @end itemize


@subsection @code{(get-symmetry-as-calphas-state imol)}
@cindex @code{(get-symmetry-as-calphas-state imol)}
 
where: 
 @itemize 
     @item imol is an exact integer number
 @end itemize


@subsection @code{(set-symmetry-molecule-rotate-colour-map imol state)}
@cindex @code{(set-symmetry-molecule-rotate-colour-map imol state)}
 
where: 
 @itemize 
     @item imol is an exact integer number
     @item state is an exact integer number
 @end itemize


@subsection @code{(symmetry-molecule-rotate-colour-map-state imol)}
@cindex @code{(symmetry-molecule-rotate-colour-map-state imol)}
 
where: 
 @itemize 
     @item imol is an exact integer number
 @end itemize


@subsection @code{(set-symmetry-colour-by-symop imol state)}
@cindex @code{(set-symmetry-colour-by-symop imol state)}
 
where: 
 @itemize 
     @item imol is an exact integer number
     @item state is an exact integer number
 @end itemize


@subsection @code{(set-symmetry-whole-chain imol state)}
@cindex @code{(set-symmetry-whole-chain imol state)}
 
where: 
 @itemize 
     @item imol is an exact integer number
     @item state is an exact integer number
 @end itemize


@subsection @code{(set-symmetry-atom-labels-expanded state)}
@cindex @code{(set-symmetry-atom-labels-expanded state)}
 
where: 
 @itemize 
     @item state is an exact integer number
 @end itemize


@subsection @code{(has-unit-cell-state imol)}
@cindex @code{(has-unit-cell-state imol)}
 
where: 
 @itemize 
     @item imol is an exact integer number
 @end itemize


@subsection @code{(setup-save-symmetry-coords)}
@cindex @code{(setup-save-symmetry-coords)}
 
@subsection @code{(set-space-group imol spg)}
@cindex @code{(set-space-group imol spg)}
 
where: 
 @itemize 
     @item imol is an exact integer number
     @item spg is a string
 @end itemize


@subsection @code{(set-symmetry-shift-search-size shift)}
@cindex @code{(set-symmetry-shift-search-size shift)}
 
where: 
 @itemize 
     @item shift is an exact integer number
 @end itemize



@section File Selection Functions 
@subsection @code{(clear-refmac-ccp4i-project)}
@cindex @code{(clear-refmac-ccp4i-project)}
 

@section History Functions 
@subsection @code{(print-all-history-in-scheme)}
@cindex @code{(print-all-history-in-scheme)}
 
@subsection @code{(print-all-history-in-python)}
@cindex @code{(print-all-history-in-python)}
 
@subsection @code{(set-console-display-commands-state istate)}
@cindex @code{(set-console-display-commands-state istate)}
 
where: 
 @itemize 
     @item istate is an exact integer number
 @end itemize


@subsection @code{(save-state)}
@cindex @code{(save-state)}
 
@subsection @code{(save-state-file filename)}
@cindex @code{(save-state-file filename)}
 
where: 
 @itemize 
     @item filename is a string
 @end itemize


@subsection @code{(set-save-state-file-name filename)}
@cindex @code{(set-save-state-file-name filename)}
 
where: 
 @itemize 
     @item filename is a string
 @end itemize


@subsection @code{(set-run-state-file-status istat)}
@cindex @code{(set-run-state-file-status istat)}
 
where: 
 @itemize 
     @item istat is an exact integer number
 @end itemize


@subsection @code{(run-state-file)}
@cindex @code{(run-state-file)}
 
@subsection @code{(run-state-file-maybe)}
@cindex @code{(run-state-file-maybe)}
 
@subsection @code{(vt-surface mode)}
@cindex @code{(vt-surface mode)}
 
where: 
 @itemize 
     @item mode is an exact integer number
 @end itemize


@subsection @code{(vt-surface-status)}
@cindex @code{(vt-surface-status)}
 

@section Clipping Functions 
@subsection @code{(do-clipping1-activate)}
@cindex @code{(do-clipping1-activate)}
 
@subsection @code{(set-clipping-back v)}
@cindex @code{(set-clipping-back v)}
 
where: 
 @itemize 
     @item v is an inexact number
 @end itemize


@subsection @code{(set-clipping-front v)}
@cindex @code{(set-clipping-front v)}
 
where: 
 @itemize 
     @item v is an inexact number
 @end itemize



@section Unit Cell 
@subsection @code{(get-show-unit-cell imol)}
@cindex @code{(get-show-unit-cell imol)}
 
where: 
 @itemize 
     @item imol is an exact integer number
 @end itemize


@subsection @code{(set-show-unit-cells-all istate)}
@cindex @code{(set-show-unit-cells-all istate)}
 
where: 
 @itemize 
     @item istate is an exact integer number
 @end itemize


@subsection @code{(set-show-unit-cell imol istate)}
@cindex @code{(set-show-unit-cell imol istate)}
 
where: 
 @itemize 
     @item imol is an exact integer number
     @item istate is an exact integer number
 @end itemize



@section Colour 
@subsection @code{(set-symmetry-colour-merge mol-no v)}
@cindex @code{(set-symmetry-colour-merge mol-no v)}
 
where: 
 @itemize 
     @item mol-no is an exact integer number
     @item v is an inexact number
 @end itemize


@subsection @code{(set-colour-map-rotation-on-read-pdb f)}
@cindex @code{(set-colour-map-rotation-on-read-pdb f)}
 
where: 
 @itemize 
     @item f is an inexact number
 @end itemize


@subsection @code{(set-colour-map-rotation-on-read-pdb-flag i)}
@cindex @code{(set-colour-map-rotation-on-read-pdb-flag i)}
 
where: 
 @itemize 
     @item i is an exact integer number
 @end itemize


@subsection @code{(set-colour-map-rotation-on-read-pdb-c-only-flag i)}
@cindex @code{(set-colour-map-rotation-on-read-pdb-c-only-flag i)}
 
where: 
 @itemize 
     @item i is an exact integer number
 @end itemize


@subsection @code{(set-colour-by-chain imol)}
@cindex @code{(set-colour-by-chain imol)}
 
where: 
 @itemize 
     @item imol is an exact integer number
 @end itemize


@subsection @code{(set-colour-by-molecule imol)}
@cindex @code{(set-colour-by-molecule imol)}
 
where: 
 @itemize 
     @item imol is an exact integer number
 @end itemize


@subsection @code{(set-colour-map-rotation-for-map f)}
@cindex @code{(set-colour-map-rotation-for-map f)}
 
where: 
 @itemize 
     @item f is an inexact number
 @end itemize


@subsection @code{(set-molecule-bonds-colour-map-rotation imol theta)}
@cindex @code{(set-molecule-bonds-colour-map-rotation imol theta)}
 
where: 
 @itemize 
     @item imol is an exact integer number
     @item theta is an inexact number
 @end itemize


@subsection @code{(get-molecule-bonds-colour-map-rotation imol)}
@cindex @code{(get-molecule-bonds-colour-map-rotation imol)}
 
where: 
 @itemize 
     @item imol is an exact integer number
 @end itemize



@section Anisotropic Atoms 
@subsection @code{(get-limit-aniso)}
@cindex @code{(get-limit-aniso)}
 
@subsection @code{(get-show-limit-aniso)}
@cindex @code{(get-show-limit-aniso)}
 
@subsection @code{(get-show-aniso)}
@cindex @code{(get-show-aniso)}
 
@subsection @code{(set-limit-aniso state)}
@cindex @code{(set-limit-aniso state)}
 
where: 
 @itemize 
     @item state is an exact integer number
 @end itemize


@subsection @code{(set-aniso-limit-size-from-widget text)}
@cindex @code{(set-aniso-limit-size-from-widget text)}
 
where: 
 @itemize 
     @item text is a string
 @end itemize


@subsection @code{(set-show-aniso state)}
@cindex @code{(set-show-aniso state)}
 
where: 
 @itemize 
     @item state is an exact integer number
 @end itemize


@subsection @code{(set-aniso-probability f)}
@cindex @code{(set-aniso-probability f)}
 
where: 
 @itemize 
     @item f is an inexact number
 @end itemize


@subsection @code{(get-aniso-probability)}
@cindex @code{(get-aniso-probability)}
 

@section Display Functions 
@subsection @code{(set-graphics-window-size x-size y-size)}
@cindex @code{(set-graphics-window-size x-size y-size)}
 
where: 
 @itemize 
     @item x-size is an exact integer number
     @item y-size is an exact integer number
 @end itemize


@subsection @code{(set-graphics-window-position x-pos y-pos)}
@cindex @code{(set-graphics-window-position x-pos y-pos)}
 
where: 
 @itemize 
     @item x-pos is an exact integer number
     @item y-pos is an exact integer number
 @end itemize


@subsection @code{(store-graphics-window-position x-pos y-pos)}
@cindex @code{(store-graphics-window-position x-pos y-pos)}
 
where: 
 @itemize 
     @item x-pos is an exact integer number
     @item y-pos is an exact integer number
 @end itemize


@subsection @code{(graphics-draw)}
@cindex @code{(graphics-draw)}
 
@subsection @code{(hardware-stereo-mode)}
@cindex @code{(hardware-stereo-mode)}
 
@subsection @code{(stereo-mode-state)}
@cindex @code{(stereo-mode-state)}
 
@subsection @code{(mono-mode)}
@cindex @code{(mono-mode)}
 
@subsection @code{(side-by-side-stereo-mode use-wall-eye-mode)}
@cindex @code{(side-by-side-stereo-mode use-wall-eye-mode)}
 
where: 
 @itemize 
     @item use-wall-eye-mode is an exact integer number
 @end itemize


@subsection @code{(set-hardware-stereo-angle-factor f)}
@cindex @code{(set-hardware-stereo-angle-factor f)}
 
where: 
 @itemize 
     @item f is an inexact number
 @end itemize


@subsection @code{(hardware-stereo-angle-factor-state)}
@cindex @code{(hardware-stereo-angle-factor-state)}
 
@subsection @code{(set-model-fit-refine-dialog-position x-pos y-pos)}
@cindex @code{(set-model-fit-refine-dialog-position x-pos y-pos)}
 
where: 
 @itemize 
     @item x-pos is an exact integer number
     @item y-pos is an exact integer number
 @end itemize


@subsection @code{(set-display-control-dialog-position x-pos y-pos)}
@cindex @code{(set-display-control-dialog-position x-pos y-pos)}
 
where: 
 @itemize 
     @item x-pos is an exact integer number
     @item y-pos is an exact integer number
 @end itemize


@subsection @code{(set-go-to-atom-window-position x-pos y-pos)}
@cindex @code{(set-go-to-atom-window-position x-pos y-pos)}
 
where: 
 @itemize 
     @item x-pos is an exact integer number
     @item y-pos is an exact integer number
 @end itemize


@subsection @code{(set-delete-dialog-position x-pos y-pos)}
@cindex @code{(set-delete-dialog-position x-pos y-pos)}
 
where: 
 @itemize 
     @item x-pos is an exact integer number
     @item y-pos is an exact integer number
 @end itemize


@subsection @code{(set-rotate-translate-dialog-position x-pos y-pos)}
@cindex @code{(set-rotate-translate-dialog-position x-pos y-pos)}
 
where: 
 @itemize 
     @item x-pos is an exact integer number
     @item y-pos is an exact integer number
 @end itemize


@subsection @code{(set-accept-reject-dialog-position x-pos y-pos)}
@cindex @code{(set-accept-reject-dialog-position x-pos y-pos)}
 
where: 
 @itemize 
     @item x-pos is an exact integer number
     @item y-pos is an exact integer number
 @end itemize


@subsection @code{(set-ramachandran-plot-dialog-position x-pos y-pos)}
@cindex @code{(set-ramachandran-plot-dialog-position x-pos y-pos)}
 
where: 
 @itemize 
     @item x-pos is an exact integer number
     @item y-pos is an exact integer number
 @end itemize



@section Smooth Scrolling 
@subsection @code{(set-smooth-scroll-flag v)}
@cindex @code{(set-smooth-scroll-flag v)}
 
where: 
 @itemize 
     @item v is an exact integer number
 @end itemize


@subsection @code{(get-smooth-scroll)}
@cindex @code{(get-smooth-scroll)}
 
@subsection @code{(set-smooth-scroll-steps-str t)}
@cindex @code{(set-smooth-scroll-steps-str t)}
 
where: 
 @itemize 
     @item t is a string
 @end itemize


@subsection @code{(set-smooth-scroll-steps i)}
@cindex @code{(set-smooth-scroll-steps i)}
 
where: 
 @itemize 
     @item i is an exact integer number
 @end itemize


@subsection @code{(set-smooth-scroll-limit-str t)}
@cindex @code{(set-smooth-scroll-limit-str t)}
 
where: 
 @itemize 
     @item t is a string
 @end itemize


@subsection @code{(set-smooth-scroll-limit lim)}
@cindex @code{(set-smooth-scroll-limit lim)}
 
where: 
 @itemize 
     @item lim is an inexact number
 @end itemize



@section Font Size 
@subsection @code{(set-font-size i)}
@cindex @code{(set-font-size i)}
 
where: 
 @itemize 
     @item i is an exact integer number
 @end itemize


@subsection @code{(get-font-size)}
@cindex @code{(get-font-size)}
 

@section Rotation Centre 
@subsection @code{(set-rotation-centre-size-from-widget text)}
@cindex @code{(set-rotation-centre-size-from-widget text)}
 
where: 
 @itemize 
     @item text is an unknown type
 @end itemize


@subsection @code{(set-rotation-centre-size f)}
@cindex @code{(set-rotation-centre-size f)}
 
where: 
 @itemize 
     @item f is an inexact number
 @end itemize


@subsection @code{(recentre-on-read-pdb)}
@cindex @code{(recentre-on-read-pdb)}
 
@subsection @code{(set-recentre-on-read-pdb int)}
@cindex @code{(set-recentre-on-read-pdb int)}
 
where: 
 @itemize 
     @item int is an exact integer number
 @end itemize


@subsection @code{(set-rotation-centre x y z)}
@cindex @code{(set-rotation-centre x y z)}
 
where: 
 @itemize 
     @item x is an inexact number
     @item y is an inexact number
     @item z is an inexact number
 @end itemize


@subsection @code{(set-rotation-centre-internal x y z)}
@cindex @code{(set-rotation-centre-internal x y z)}
 
where: 
 @itemize 
     @item x is an inexact number
     @item y is an inexact number
     @item z is an inexact number
 @end itemize


@subsection @code{(rotation-centre-position axis)}
@cindex @code{(rotation-centre-position axis)}
 
where: 
 @itemize 
     @item axis is an exact integer number
 @end itemize



@section Orthogonal Axes 
@subsection @code{(set-draw-axes i)}
@cindex @code{(set-draw-axes i)}
 
where: 
 @itemize 
     @item i is an exact integer number
 @end itemize



@section Atom Selection Utilities 
@subsection @code{(atom-index imol chain-id iresno atom-id)}
@cindex @code{(atom-index imol chain-id iresno atom-id)}
 
where: 
 @itemize 
     @item imol is an exact integer number
     @item chain-id is a string
     @item iresno is an exact integer number
     @item atom-id is a string
 @end itemize


@subsection @code{(median-temperature-factor imol)}
@cindex @code{(median-temperature-factor imol)}
 
where: 
 @itemize 
     @item imol is an exact integer number
 @end itemize


@subsection @code{(average-temperature-factor imol)}
@cindex @code{(average-temperature-factor imol)}
 
where: 
 @itemize 
     @item imol is an exact integer number
 @end itemize


@subsection @code{(clear-pending-picks)}
@cindex @code{(clear-pending-picks)}
 
@subsection @code{(set-default-temperature-factor-for-new-atoms new-b)}
@cindex @code{(set-default-temperature-factor-for-new-atoms new-b)}
 
where: 
 @itemize 
     @item new-b is an inexact number
 @end itemize


@subsection @code{(default-new-atoms-b-factor)}
@cindex @code{(default-new-atoms-b-factor)}
 

@section Skeletonization 
@subsection @code{(skel-greer-on)}
@cindex @code{(skel-greer-on)}
 
@subsection @code{(skel-greer-off)}
@cindex @code{(skel-greer-off)}
 
@subsection @code{(skel-foadi-on)}
@cindex @code{(skel-foadi-on)}
 
@subsection @code{(skel-foadi-off)}
@cindex @code{(skel-foadi-off)}
 
@subsection @code{(skeletonize-map prune-flag imol)}
@cindex @code{(skeletonize-map prune-flag imol)}
 
where: 
 @itemize 
     @item prune-flag is an exact integer number
     @item imol is an exact integer number
 @end itemize


@subsection @code{(unskeletonize-map imol)}
@cindex @code{(unskeletonize-map imol)}
 
where: 
 @itemize 
     @item imol is an exact integer number
 @end itemize


@subsection @code{(set-initial-map-for-skeletonize)}
@cindex @code{(set-initial-map-for-skeletonize)}
 
@subsection @code{(set-max-skeleton-search-depth v)}
@cindex @code{(set-max-skeleton-search-depth v)}
 
where: 
 @itemize 
     @item v is an exact integer number
 @end itemize


@subsection @code{(set-skeletonization-level-from-widget txt)}
@cindex @code{(set-skeletonization-level-from-widget txt)}
 
where: 
 @itemize 
     @item txt is a string
 @end itemize


@subsection @code{(set-skeleton-box-size-from-widget txt)}
@cindex @code{(set-skeleton-box-size-from-widget txt)}
 
where: 
 @itemize 
     @item txt is a string
 @end itemize


@subsection @code{(set-skeleton-box-size f)}
@cindex @code{(set-skeleton-box-size f)}
 
where: 
 @itemize 
     @item f is an inexact number
 @end itemize



@section Skeleton Colour 
@subsection @code{(handle-skeleton-colour-change mol map-col)}
@cindex @code{(handle-skeleton-colour-change mol map-col)}
 
where: 
 @itemize 
     @item mol is an exact integer number
     @item map-col is an unknown type
 @end itemize


@subsection @code{(set-skeleton-colour imol r g b)}
@cindex @code{(set-skeleton-colour imol r g b)}
 
where: 
 @itemize 
     @item imol is an exact integer number
     @item r is an inexact number
     @item g is an inexact number
     @item b is an inexact number
 @end itemize



@section Read CCP4 Map 
@subsection @code{(handle-read-ccp4-map filename is-diff-map-flag)}
@cindex @code{(handle-read-ccp4-map filename is-diff-map-flag)}
 
where: 
 @itemize 
     @item filename is an unknown type
     @item is-diff-map-flag is an exact integer number
 @end itemize



@section Save Coordinates 
@subsection @code{(save-coordinates imol filename)}
@cindex @code{(save-coordinates imol filename)}
 
where: 
 @itemize 
     @item imol is an exact integer number
     @item filename is a string
 @end itemize


@subsection @code{(set-save-coordinates-in-original-directory i)}
@cindex @code{(set-save-coordinates-in-original-directory i)}
 
where: 
 @itemize 
     @item i is an exact integer number
 @end itemize


@subsection @code{(save-molecule-number-from-option-menu)}
@cindex @code{(save-molecule-number-from-option-menu)}
 
@subsection @code{(set-save-molecule-number imol)}
@cindex @code{(set-save-molecule-number imol)}
 
where: 
 @itemize 
     @item imol is an exact integer number
 @end itemize



@section Read Phases File Functions 
@subsection @code{(possible-cell-symm-for-phs-file)}
@cindex @code{(possible-cell-symm-for-phs-file)}
 

@section Graphics Move 
@subsection @code{(undo-last-move)}
@cindex @code{(undo-last-move)}
 
@subsection @code{(translate-molecule-by imol x y z)}
@cindex @code{(translate-molecule-by imol x y z)}
 
where: 
 @itemize 
     @item imol is an exact integer number
     @item x is an inexact number
     @item y is an inexact number
     @item z is an inexact number
 @end itemize



@section Go To Atom Widget Functions 
@subsection @code{(post-go-to-atom-window)}
@cindex @code{(post-go-to-atom-window)}
 
@subsection @code{(atom-spec-to-atom-index mol chain resno atom-name)}
@cindex @code{(atom-spec-to-atom-index mol chain resno atom-name)}
 
where: 
 @itemize 
     @item mol is an exact integer number
     @item chain is a string
     @item resno is an exact integer number
     @item atom-name is a string
 @end itemize


@subsection @code{(update-go-to-atom-window-on-changed-mol imol)}
@cindex @code{(update-go-to-atom-window-on-changed-mol imol)}
 
where: 
 @itemize 
     @item imol is an exact integer number
 @end itemize


@subsection @code{(update-go-to-atom-window-on-new-mol)}
@cindex @code{(update-go-to-atom-window-on-new-mol)}
 
@subsection @code{(set-go-to-atom-molecule imol)}
@cindex @code{(set-go-to-atom-molecule imol)}
 
where: 
 @itemize 
     @item imol is an exact integer number
 @end itemize


@subsection @code{(unset-go-to-atom-widget)}
@cindex @code{(unset-go-to-atom-widget)}
 

@section AutoBuilding functions (Defunct) 
@subsection @code{(autobuild-ca-on)}
@cindex @code{(autobuild-ca-on)}
 
@subsection @code{(autobuild-ca-off)}
@cindex @code{(autobuild-ca-off)}
 
@subsection @code{(test-fragment)}
@cindex @code{(test-fragment)}
 
@subsection @code{(do-skeleton-prune)}
@cindex @code{(do-skeleton-prune)}
 
@subsection @code{(test-function i j)}
@cindex @code{(test-function i j)}
 
where: 
 @itemize 
     @item i is an exact integer number
     @item j is an exact integer number
 @end itemize



@section Map and Molecule Control 
@subsection @code{(post-display-control-window)}
@cindex @code{(post-display-control-window)}
 
@subsection @code{(add-map-display-control-widgets)}
@cindex @code{(add-map-display-control-widgets)}
 
@subsection @code{(add-mol-display-control-widgets)}
@cindex @code{(add-mol-display-control-widgets)}
 
@subsection @code{(add-map-and-mol-display-control-widgets)}
@cindex @code{(add-map-and-mol-display-control-widgets)}
 
@subsection @code{(reset-graphics-display-control-window)}
@cindex @code{(reset-graphics-display-control-window)}
 
@subsection @code{(toggle-display-map imol imap)}
@cindex @code{(toggle-display-map imol imap)}
 
where: 
 @itemize 
     @item imol is an exact integer number
     @item imap is an exact integer number
 @end itemize


@subsection @code{(toggle-display-mol imol)}
@cindex @code{(toggle-display-mol imol)}
 
where: 
 @itemize 
     @item imol is an exact integer number
 @end itemize


@subsection @code{( imol)}
@cindex @code{( imol)}
 
where: 
 @itemize 
     @item imol is an exact integer number
 @end itemize


@subsection @code{(mol-is-displayed imol)}
@cindex @code{(mol-is-displayed imol)}
 
where: 
 @itemize 
     @item imol is an exact integer number
 @end itemize


@subsection @code{(mol-is-active imol)}
@cindex @code{(mol-is-active imol)}
 
where: 
 @itemize 
     @item imol is an exact integer number
 @end itemize


@subsection @code{(map-is-displayed imol)}
@cindex @code{(map-is-displayed imol)}
 
where: 
 @itemize 
     @item imol is an exact integer number
 @end itemize



@section Merge Molecules 
@subsection @code{(do-merge-molecules-gui)}
@cindex @code{(do-merge-molecules-gui)}
 

@section Mutate Sequence and Loops GUI 

@section Align and Mutate 
@subsection @code{(align-and-mutate imol chain-id fasta-maybe)}
@cindex @code{(align-and-mutate imol chain-id fasta-maybe)}
 
where: 
 @itemize 
     @item imol is an exact integer number
     @item chain-id is a string
     @item fasta-maybe is a string
 @end itemize



@section Renumber Residue Range 
@subsection @code{(change-residue-number imol chain-id current-resno current-inscode new-resno new-inscode)}
@cindex @code{(change-residue-number imol chain-id current-resno current-inscode new-resno new-inscode)}
 
where: 
 @itemize 
     @item imol is an exact integer number
     @item chain-id is a string
     @item current-resno is an exact integer number
     @item current-inscode is a string
     @item new-resno is an exact integer number
     @item new-inscode is a string
 @end itemize



@section Change Chain ID 

@section Scripting 
@subsection @code{(post-scripting-window)}
@cindex @code{(post-scripting-window)}
 
@subsection @code{(run-command-line-scripts)}
@cindex @code{(run-command-line-scripts)}
 
@subsection @code{(set-guile-gui-loaded-flag)}
@cindex @code{(set-guile-gui-loaded-flag)}
 
@subsection @code{(set-found-coot-gui)}
@cindex @code{(set-found-coot-gui)}
 
@subsection @code{(get-monomer three-letter-code)}
@cindex @code{(get-monomer three-letter-code)}
 
where: 
 @itemize 
     @item three-letter-code is a string
 @end itemize


@subsection @code{( filename)}
@cindex @code{( filename)}
 
where: 
 @itemize 
     @item filename is a string
 @end itemize


@subsection @code{( filename)}
@cindex @code{( filename)}
 
where: 
 @itemize 
     @item filename is a string
 @end itemize


@subsection @code{(run-python-script filename)}
@cindex @code{(run-python-script filename)}
 
where: 
 @itemize 
     @item filename is a string
 @end itemize



@section Regularization and Refinement 
@subsection @code{(do-regularize state)}
@cindex @code{(do-regularize state)}
 
where: 
 @itemize 
     @item state is an exact integer number
 @end itemize


@subsection @code{(do-refine state)}
@cindex @code{(do-refine state)}
 
where: 
 @itemize 
     @item state is an exact integer number
 @end itemize


@subsection @code{(add-planar-peptide-restraints)}
@cindex @code{(add-planar-peptide-restraints)}
 
@subsection @code{(remove-planar-peptide-restraints)}
@cindex @code{(remove-planar-peptide-restraints)}
 
@subsection @code{(add-omega-torsion-restriants)}
@cindex @code{(add-omega-torsion-restriants)}
 
@subsection @code{(remove-omega-torsion-restriants)}
@cindex @code{(remove-omega-torsion-restriants)}
 
@subsection @code{(set-refinement-immediate-replacement istate)}
@cindex @code{(set-refinement-immediate-replacement istate)}
 
where: 
 @itemize 
     @item istate is an exact integer number
 @end itemize


@subsection @code{(refinement-immediate-replacement-state)}
@cindex @code{(refinement-immediate-replacement-state)}
 
@subsection @code{(set-residue-selection-flash-frames-number i)}
@cindex @code{(set-residue-selection-flash-frames-number i)}
 
where: 
 @itemize 
     @item i is an exact integer number
 @end itemize


@subsection @code{(accept-regularizement)}
@cindex @code{(accept-regularizement)}
 
@subsection @code{(clear-up-moving-atoms)}
@cindex @code{(clear-up-moving-atoms)}
 
@subsection @code{(clear-moving-atoms-object)}
@cindex @code{(clear-moving-atoms-object)}
 
@subsection @code{(do-peptide-torsions-toggle)}
@cindex @code{(do-peptide-torsions-toggle)}
 
@subsection @code{(set-refine-with-torsion-restraints istate)}
@cindex @code{(set-refine-with-torsion-restraints istate)}
 
where: 
 @itemize 
     @item istate is an exact integer number
 @end itemize


@subsection @code{(set-refine-params-phi-psi-restraints-type restraints-type)}
@cindex @code{(set-refine-params-phi-psi-restraints-type restraints-type)}
 
where: 
 @itemize 
     @item restraints-type is an exact integer number
 @end itemize


@subsection @code{(set-matrix f)}
@cindex @code{(set-matrix f)}
 
where: 
 @itemize 
     @item f is an inexact number
 @end itemize


@subsection @code{(matrix-state)}
@cindex @code{(matrix-state)}
 
@subsection @code{(set-refine-auto-range-step i)}
@cindex @code{(set-refine-auto-range-step i)}
 
where: 
 @itemize 
     @item i is an exact integer number
 @end itemize


@subsection @code{(set-refine-max-residues n)}
@cindex @code{(set-refine-max-residues n)}
 
where: 
 @itemize 
     @item n is an exact integer number
 @end itemize


@subsection @code{(refine-zone-atom-index-define imol ind1 ind2)}
@cindex @code{(refine-zone-atom-index-define imol ind1 ind2)}
 
where: 
 @itemize 
     @item imol is an exact integer number
     @item ind1 is an exact integer number
     @item ind2 is an exact integer number
 @end itemize


@subsection @code{(refine-zone imol chain-id resno1 resno2 altconf)}
@cindex @code{(refine-zone imol chain-id resno1 resno2 altconf)}
 
where: 
 @itemize 
     @item imol is an exact integer number
     @item chain-id is a string
     @item resno1 is an exact integer number
     @item resno2 is an exact integer number
     @item altconf is a string
 @end itemize


@subsection @code{(refine-auto-range imol chain-id resno1 altconf)}
@cindex @code{(refine-auto-range imol chain-id resno1 altconf)}
 
where: 
 @itemize 
     @item imol is an exact integer number
     @item chain-id is a string
     @item resno1 is an exact integer number
     @item altconf is a string
 @end itemize


@subsection @code{(set-dragged-refinement-steps-per-frame v)}
@cindex @code{(set-dragged-refinement-steps-per-frame v)}
 
where: 
 @itemize 
     @item v is an exact integer number
 @end itemize


@subsection @code{(dragged-refinement-steps-per-frame)}
@cindex @code{(dragged-refinement-steps-per-frame)}
 
@subsection @code{(set-refinement-refine-per-frame istate)}
@cindex @code{(set-refinement-refine-per-frame istate)}
 
where: 
 @itemize 
     @item istate is an exact integer number
 @end itemize


@subsection @code{(refinement-refine-per-frame-state)}
@cindex @code{(refinement-refine-per-frame-state)}
 
@subsection @code{(set-fix-chiral-volumes-before-refinement istate)}
@cindex @code{(set-fix-chiral-volumes-before-refinement istate)}
 
where: 
 @itemize 
     @item istate is an exact integer number
 @end itemize


@subsection @code{(check-chiral-volumes imol)}
@cindex @code{(check-chiral-volumes imol)}
 
where: 
 @itemize 
     @item imol is an exact integer number
 @end itemize


@subsection @code{(set-secondary-structure-restraints-type itype)}
@cindex @code{(set-secondary-structure-restraints-type itype)}
 
where: 
 @itemize 
     @item itype is an exact integer number
 @end itemize


@subsection @code{(secondary-structure-restraints-type)}
@cindex @code{(secondary-structure-restraints-type)}
 
@subsection @code{(imol-refinement-map)}
@cindex @code{(imol-refinement-map)}
 
@subsection @code{(set-imol-refinement-map imol)}
@cindex @code{(set-imol-refinement-map imol)}
 
where: 
 @itemize 
     @item imol is an exact integer number
 @end itemize


@subsection @code{(does-residue-exist-p imol chain-id resno inscode)}
@cindex @code{(does-residue-exist-p imol chain-id resno inscode)}
 
where: 
 @itemize 
     @item imol is an exact integer number
     @item chain-id is a string
     @item resno is an exact integer number
     @item inscode is a string
 @end itemize


@subsection @code{(fix-nomenclature-errors imol)}
@cindex @code{(fix-nomenclature-errors imol)}
 
where: 
 @itemize 
     @item imol is an exact integer number
 @end itemize



@section Atom Info 

@section Residue Info 
@subsection @code{(do-residue-info)}
@cindex @code{(do-residue-info)}
 
@subsection @code{( atom-index imol)}
@cindex @code{( atom-index imol)}
 
where: 
 @itemize 
     @item atom-index is an exact integer number
     @item imol is an exact integer number
 @end itemize


@subsection @code{(output-residue-info-as-text atom-index imol)}
@cindex @code{(output-residue-info-as-text atom-index imol)}
 
where: 
 @itemize 
     @item atom-index is an exact integer number
     @item imol is an exact integer number
 @end itemize


@subsection @code{(do-distance-define)}
@cindex @code{(do-distance-define)}
 
@subsection @code{(do-angle-define)}
@cindex @code{(do-angle-define)}
 
@subsection @code{(do-torsion-define)}
@cindex @code{(do-torsion-define)}
 
@subsection @code{(residue-info-apply-all-checkbutton-toggled)}
@cindex @code{(residue-info-apply-all-checkbutton-toggled)}
 
@subsection @code{(clear-residue-info-edit-list)}
@cindex @code{(clear-residue-info-edit-list)}
 
@subsection @code{(unset-residue-info-widget)}
@cindex @code{(unset-residue-info-widget)}
 
@subsection @code{(clear-simple-distances)}
@cindex @code{(clear-simple-distances)}
 
@subsection @code{(clear-last-simple-distance)}
@cindex @code{(clear-last-simple-distance)}
 

@section Residue Environment Functions 

@section Pointer Functions 
@subsection @code{(set-show-pointer-distances istate)}
@cindex @code{(set-show-pointer-distances istate)}
 
where: 
 @itemize 
     @item istate is an exact integer number
 @end itemize



@section Zoom Functions 
@subsection @code{(scale-zoom f)}
@cindex @code{(scale-zoom f)}
 
where: 
 @itemize 
     @item f is an inexact number
 @end itemize


@subsection @code{(scale-zoom-internal f)}
@cindex @code{(scale-zoom-internal f)}
 
where: 
 @itemize 
     @item f is an inexact number
 @end itemize


@subsection @code{(zoom-factor)}
@cindex @code{(zoom-factor)}
 
@subsection @code{(set-smooth-scroll-do-zoom i)}
@cindex @code{(set-smooth-scroll-do-zoom i)}
 
where: 
 @itemize 
     @item i is an exact integer number
 @end itemize


@subsection @code{(smooth-scroll-do-zoom)}
@cindex @code{(smooth-scroll-do-zoom)}
 
@subsection @code{(smooth-scroll-zoom-limit)}
@cindex @code{(smooth-scroll-zoom-limit)}
 
@subsection @code{(set-smooth-scroll-zoom-limit f)}
@cindex @code{(set-smooth-scroll-zoom-limit f)}
 
where: 
 @itemize 
     @item f is an inexact number
 @end itemize


@subsection @code{(handle-cns-data-file filename)}
@cindex @code{(handle-cns-data-file filename)}
 
where: 
 @itemize 
     @item filename is a string
 @end itemize



@section mmCIF Functions 
@subsection @code{(auto-read-cif-data-with-phases filename)}
@cindex @code{(auto-read-cif-data-with-phases filename)}
 
where: 
 @itemize 
     @item filename is a string
 @end itemize


@subsection @code{(read-cif-data-with-phases-sigmaa filename)}
@cindex @code{(read-cif-data-with-phases-sigmaa filename)}
 
where: 
 @itemize 
     @item filename is a string
 @end itemize


@subsection @code{(read-cif-data-with-phases-diff-sigmaa filename)}
@cindex @code{(read-cif-data-with-phases-diff-sigmaa filename)}
 
where: 
 @itemize 
     @item filename is a string
 @end itemize


@subsection @code{(read-cif-data filename imol-coords)}
@cindex @code{(read-cif-data filename imol-coords)}
 
where: 
 @itemize 
     @item filename is a string
     @item imol-coords is an exact integer number
 @end itemize


@subsection @code{(read-cif-data-2fofc-map filename imol-coords)}
@cindex @code{(read-cif-data-2fofc-map filename imol-coords)}
 
where: 
 @itemize 
     @item filename is a string
     @item imol-coords is an exact integer number
 @end itemize


@subsection @code{(read-cif-data-fofc-map filename imol-coords)}
@cindex @code{(read-cif-data-fofc-map filename imol-coords)}
 
where: 
 @itemize 
     @item filename is a string
     @item imol-coords is an exact integer number
 @end itemize


@subsection @code{(read-cif-data-with-phases-fo-fc filename)}
@cindex @code{(read-cif-data-with-phases-fo-fc filename)}
 
where: 
 @itemize 
     @item filename is a string
 @end itemize


@subsection @code{(read-cif-data-with-phases-2fo-fc filename)}
@cindex @code{(read-cif-data-with-phases-2fo-fc filename)}
 
where: 
 @itemize 
     @item filename is a string
 @end itemize


@subsection @code{(read-cif-data-with-phases-fo-alpha-calc filename)}
@cindex @code{(read-cif-data-with-phases-fo-alpha-calc filename)}
 
where: 
 @itemize 
     @item filename is a string
 @end itemize


@subsection @code{(handle-cif-dictionary filename)}
@cindex @code{(handle-cif-dictionary filename)}
 
where: 
 @itemize 
     @item filename is a string
 @end itemize


@subsection @code{(read-cif-dictionary filename)}
@cindex @code{(read-cif-dictionary filename)}
 
where: 
 @itemize 
     @item filename is a string
 @end itemize


@subsection @code{(write-connectivity monomer-name filename)}
@cindex @code{(write-connectivity monomer-name filename)}
 
where: 
 @itemize 
     @item monomer-name is an unknown type
     @item filename is a string
 @end itemize


@subsection @code{(import-all-refmac-cifs)}
@cindex @code{(import-all-refmac-cifs)}
 

@section SHELXL Functions 
@subsection @code{(read-shelx-ins-file filename)}
@cindex @code{(read-shelx-ins-file filename)}
 
where: 
 @itemize 
     @item filename is a string
 @end itemize


@subsection @code{(write-shelx-ins-file imol filename)}
@cindex @code{(write-shelx-ins-file imol filename)}
 
where: 
 @itemize 
     @item imol is an exact integer number
     @item filename is a string
 @end itemize


@subsection @code{(handle-shelx-fcf-file-internal filename)}
@cindex @code{(handle-shelx-fcf-file-internal filename)}
 
where: 
 @itemize 
     @item filename is a string
 @end itemize



@section Validation Functions 
@subsection @code{(deviant-geometry imol)}
@cindex @code{(deviant-geometry imol)}
 
where: 
 @itemize 
     @item imol is an exact integer number
 @end itemize


@subsection @code{(is-valid-model-molecule imol)}
@cindex @code{(is-valid-model-molecule imol)}
 
where: 
 @itemize 
     @item imol is an exact integer number
 @end itemize


@subsection @code{(is-valid-map-molecule imol)}
@cindex @code{(is-valid-map-molecule imol)}
 
where: 
 @itemize 
     @item imol is an exact integer number
 @end itemize


@subsection @code{(difference-map-peaks imol imol-coords level do-positive-level-flag do-negative-level-flag)}
@cindex @code{(difference-map-peaks imol imol-coords level do-positive-level-flag do-negative-level-flag)}
 
where: 
 @itemize 
     @item imol is an exact integer number
     @item imol-coords is an exact integer number
     @item level is an inexact number
     @item do-positive-level-flag is an exact integer number
     @item do-negative-level-flag is an exact integer number
 @end itemize


@subsection @code{(clear-diff-map-peaks)}
@cindex @code{(clear-diff-map-peaks)}
 
@subsection @code{(gln-asn-b-factor-outliers imol)}
@cindex @code{(gln-asn-b-factor-outliers imol)}
 
where: 
 @itemize 
     @item imol is an exact integer number
 @end itemize



@section Ramachandran Plot Functions 
@subsection @code{(do-ramachandran-plot imol)}
@cindex @code{(do-ramachandran-plot imol)}
 
where: 
 @itemize 
     @item imol is an exact integer number
 @end itemize


@subsection @code{(add-on-rama-choices)}
@cindex @code{(add-on-rama-choices)}
 
@subsection @code{(set-moving-atoms phi psi)}
@cindex @code{(set-moving-atoms phi psi)}
 
where: 
 @itemize 
     @item phi is an unknown type
     @item psi is an unknown type
 @end itemize


@subsection @code{(accept-phi-psi-moving-atoms)}
@cindex @code{(accept-phi-psi-moving-atoms)}
 
@subsection @code{(setup-edit-phi-psi state)}
@cindex @code{(setup-edit-phi-psi state)}
 
where: 
 @itemize 
     @item state is an exact integer number
 @end itemize


@subsection @code{(destroy-edit-backbone-rama-plot)}
@cindex @code{(destroy-edit-backbone-rama-plot)}
 
@subsection @code{(ramachandran-plot-differences imol1 imol2)}
@cindex @code{(ramachandran-plot-differences imol1 imol2)}
 
where: 
 @itemize 
     @item imol1 is an exact integer number
     @item imol2 is an exact integer number
 @end itemize


@subsection @code{(do-sequence-view imol)}
@cindex @code{(do-sequence-view imol)}
 
where: 
 @itemize 
     @item imol is an exact integer number
 @end itemize


@subsection @code{(add-on-sequence-view-choices)}
@cindex @code{(add-on-sequence-view-choices)}
 
@subsection @code{(change-peptide-carbonyl-by angle)}
@cindex @code{(change-peptide-carbonyl-by angle)}
 
where: 
 @itemize 
     @item angle is an unknown type
 @end itemize


@subsection @code{(change-peptide-peptide-by angle)}
@cindex @code{(change-peptide-peptide-by angle)}
 
where: 
 @itemize 
     @item angle is an unknown type
 @end itemize


@subsection @code{(execute-setup-backbone-torsion-edit imol atom-index)}
@cindex @code{(execute-setup-backbone-torsion-edit imol atom-index)}
 
where: 
 @itemize 
     @item imol is an exact integer number
     @item atom-index is an exact integer number
 @end itemize


@subsection @code{(setup-backbone-torsion-edit state)}
@cindex @code{(setup-backbone-torsion-edit state)}
 
where: 
 @itemize 
     @item state is an exact integer number
 @end itemize


@subsection @code{(set-backbone-torsion-peptide-button-start-pos ix iy)}
@cindex @code{(set-backbone-torsion-peptide-button-start-pos ix iy)}
 
where: 
 @itemize 
     @item ix is an exact integer number
     @item iy is an exact integer number
 @end itemize


@subsection @code{(change-peptide-peptide-by-current-button-pos ix iy)}
@cindex @code{(change-peptide-peptide-by-current-button-pos ix iy)}
 
where: 
 @itemize 
     @item ix is an exact integer number
     @item iy is an exact integer number
 @end itemize


@subsection @code{(set-backbone-torsion-carbonyl-button-start-pos ix iy)}
@cindex @code{(set-backbone-torsion-carbonyl-button-start-pos ix iy)}
 
where: 
 @itemize 
     @item ix is an exact integer number
     @item iy is an exact integer number
 @end itemize


@subsection @code{(change-peptide-carbonyl-by-current-button-pos ix iy)}
@cindex @code{(change-peptide-carbonyl-by-current-button-pos ix iy)}
 
where: 
 @itemize 
     @item ix is an exact integer number
     @item iy is an exact integer number
 @end itemize



@section Atom Labelling 
@subsection @code{(add-atom-label imol chain-id iresno atom-id)}
@cindex @code{(add-atom-label imol chain-id iresno atom-id)}
 
where: 
 @itemize 
     @item imol is an exact integer number
     @item chain-id is a string
     @item iresno is an exact integer number
     @item atom-id is a string
 @end itemize


@subsection @code{(remove-atom-label imol chain-id iresno atom-id)}
@cindex @code{(remove-atom-label imol chain-id iresno atom-id)}
 
where: 
 @itemize 
     @item imol is an exact integer number
     @item chain-id is a string
     @item iresno is an exact integer number
     @item atom-id is a string
 @end itemize


@subsection @code{(remove-all-atom-labels)}
@cindex @code{(remove-all-atom-labels)}
 
@subsection @code{(set-label-on-recentre-flag i)}
@cindex @code{(set-label-on-recentre-flag i)}
 
where: 
 @itemize 
     @item i is an exact integer number
 @end itemize


@subsection @code{(centre-atom-label-status)}
@cindex @code{(centre-atom-label-status)}
 
@subsection @code{(set-brief-atom-labels istat)}
@cindex @code{(set-brief-atom-labels istat)}
 
where: 
 @itemize 
     @item istat is an exact integer number
 @end itemize


@subsection @code{(brief-atom-labels-state)}
@cindex @code{(brief-atom-labels-state)}
 

@section Screen Rotation 
@subsection @code{(rotate-y-scene nsteps stepsize)}
@cindex @code{(rotate-y-scene nsteps stepsize)}
 
where: 
 @itemize 
     @item nsteps is an exact integer number
     @item stepsize is an inexact number
 @end itemize


@subsection @code{(rotate-x-scene nsteps stepsize)}
@cindex @code{(rotate-x-scene nsteps stepsize)}
 
where: 
 @itemize 
     @item nsteps is an exact integer number
     @item stepsize is an inexact number
 @end itemize


@subsection @code{(rotate-z-scene nsteps stepsize)}
@cindex @code{(rotate-z-scene nsteps stepsize)}
 
where: 
 @itemize 
     @item nsteps is an exact integer number
     @item stepsize is an inexact number
 @end itemize



@section Background Colour 
@subsection @code{(set-background-colour red green blue)}
@cindex @code{(set-background-colour red green blue)}
 
where: 
 @itemize 
     @item red is an unknown type
     @item green is an unknown type
     @item blue is an unknown type
 @end itemize


@subsection @code{(background-is-black-p)}
@cindex @code{(background-is-black-p)}
 

@section Ligand Fitting Functions 
@subsection @code{(set-ligand-acceptable-fit-fraction f)}
@cindex @code{(set-ligand-acceptable-fit-fraction f)}
 
where: 
 @itemize 
     @item f is an inexact number
 @end itemize


@subsection @code{(set-ligand-cluster-sigma-level f)}
@cindex @code{(set-ligand-cluster-sigma-level f)}
 
where: 
 @itemize 
     @item f is an inexact number
 @end itemize


@subsection @code{(set-ligand-flexible-ligand-n-samples i)}
@cindex @code{(set-ligand-flexible-ligand-n-samples i)}
 
where: 
 @itemize 
     @item i is an exact integer number
 @end itemize


@subsection @code{(set-ligand-verbose-reporting i)}
@cindex @code{(set-ligand-verbose-reporting i)}
 
where: 
 @itemize 
     @item i is an exact integer number
 @end itemize


@subsection @code{(set-find-ligand-n-top-ligands n)}
@cindex @code{(set-find-ligand-n-top-ligands n)}
 
where: 
 @itemize 
     @item n is an exact integer number
 @end itemize


@subsection @code{(set-find-ligand-mask-waters istate)}
@cindex @code{(set-find-ligand-mask-waters istate)}
 
where: 
 @itemize 
     @item istate is an exact integer number
 @end itemize


@subsection @code{(set-ligand-search-protein-molecule imol)}
@cindex @code{(set-ligand-search-protein-molecule imol)}
 
where: 
 @itemize 
     @item imol is an exact integer number
 @end itemize


@subsection @code{(set-ligand-search-map-molecule imol-map)}
@cindex @code{(set-ligand-search-map-molecule imol-map)}
 
where: 
 @itemize 
     @item imol-map is an exact integer number
 @end itemize


@subsection @code{(add-ligand-search-ligand-molecule imol-ligand)}
@cindex @code{(add-ligand-search-ligand-molecule imol-ligand)}
 
where: 
 @itemize 
     @item imol-ligand is an exact integer number
 @end itemize


@subsection @code{(add-ligand-search-wiggly-ligand-molecule imol-ligand)}
@cindex @code{(add-ligand-search-wiggly-ligand-molecule imol-ligand)}
 
where: 
 @itemize 
     @item imol-ligand is an exact integer number
 @end itemize


@subsection @code{(execute-ligand-search)}
@cindex @code{(execute-ligand-search)}
 
@subsection @code{(ligand-expert)}
@cindex @code{(ligand-expert)}
 
@subsection @code{(do-find-ligands-dialog)}
@cindex @code{(do-find-ligands-dialog)}
 

@section Water Fitting Functions 
@subsection @code{(renumber-waters imol)}
@cindex @code{(renumber-waters imol)}
 
where: 
 @itemize 
     @item imol is an exact integer number
 @end itemize


@subsection @code{(set-value-for-find-waters-sigma-cut-off f)}
@cindex @code{(set-value-for-find-waters-sigma-cut-off f)}
 
where: 
 @itemize 
     @item f is an inexact number
 @end itemize


@subsection @code{(set-ligand-water-spherical-variance-limit f)}
@cindex @code{(set-ligand-water-spherical-variance-limit f)}
 
where: 
 @itemize 
     @item f is an inexact number
 @end itemize


@subsection @code{(set-ligand-water-to-protein-distance-limits f1 f2)}
@cindex @code{(set-ligand-water-to-protein-distance-limits f1 f2)}
 
where: 
 @itemize 
     @item f1 is an inexact number
     @item f2 is an inexact number
 @end itemize


@subsection @code{(set-ligand-water-n-cycles i)}
@cindex @code{(set-ligand-water-n-cycles i)}
 
where: 
 @itemize 
     @item i is an exact integer number
 @end itemize


@subsection @code{(set-write-peaksearched-waters)}
@cindex @code{(set-write-peaksearched-waters)}
 
@subsection @code{(execute-find-blobs imol-model imol-for-map cut-off interactive-flag)}
@cindex @code{(execute-find-blobs imol-model imol-for-map cut-off interactive-flag)}
 
where: 
 @itemize 
     @item imol-model is an exact integer number
     @item imol-for-map is an exact integer number
     @item cut-off is an inexact number
     @item interactive-flag is an exact integer number
 @end itemize



@section Bond Representation 
@subsection @code{(set-default-bond-thickness t)}
@cindex @code{(set-default-bond-thickness t)}
 
where: 
 @itemize 
     @item t is an exact integer number
 @end itemize


@subsection @code{(set-bond-thickness imol t)}
@cindex @code{(set-bond-thickness imol t)}
 
where: 
 @itemize 
     @item imol is an exact integer number
     @item t is an inexact number
 @end itemize


@subsection @code{(set-bond-thickness-intermediate-atoms t)}
@cindex @code{(set-bond-thickness-intermediate-atoms t)}
 
where: 
 @itemize 
     @item t is an inexact number
 @end itemize


@subsection @code{(set-unbonded-atom-star-size f)}
@cindex @code{(set-unbonded-atom-star-size f)}
 
where: 
 @itemize 
     @item f is an inexact number
 @end itemize


@subsection @code{(set-draw-zero-occ-markers status)}
@cindex @code{(set-draw-zero-occ-markers status)}
 
where: 
 @itemize 
     @item status is an exact integer number
 @end itemize


@subsection @code{(set-draw-hydrogens imol istat)}
@cindex @code{(set-draw-hydrogens imol istat)}
 
where: 
 @itemize 
     @item imol is an exact integer number
     @item istat is an exact integer number
 @end itemize


@subsection @code{( imol)}
@cindex @code{( imol)}
 
where: 
 @itemize 
     @item imol is an exact integer number
 @end itemize


@subsection @code{( imol)}
@cindex @code{( imol)}
 
where: 
 @itemize 
     @item imol is an exact integer number
 @end itemize


@subsection @code{(graphics-to-bonds-no-waters-representation imol)}
@cindex @code{(graphics-to-bonds-no-waters-representation imol)}
 
where: 
 @itemize 
     @item imol is an exact integer number
 @end itemize


@subsection @code{(graphics-to-bonds-representation mol)}
@cindex @code{(graphics-to-bonds-representation mol)}
 
where: 
 @itemize 
     @item mol is an exact integer number
 @end itemize


@subsection @code{(graphics-to-ca-plus-ligands-sec-struct-representation imol)}
@cindex @code{(graphics-to-ca-plus-ligands-sec-struct-representation imol)}
 
where: 
 @itemize 
     @item imol is an exact integer number
 @end itemize


@subsection @code{(graphics-to-sec-struct-bonds-representation imol)}
@cindex @code{(graphics-to-sec-struct-bonds-representation imol)}
 
where: 
 @itemize 
     @item imol is an exact integer number
 @end itemize


@subsection @code{(graphics-to-rainbow-representation imol)}
@cindex @code{(graphics-to-rainbow-representation imol)}
 
where: 
 @itemize 
     @item imol is an exact integer number
 @end itemize


@subsection @code{(graphics-to-b-factor-representation imol)}
@cindex @code{(graphics-to-b-factor-representation imol)}
 
where: 
 @itemize 
     @item imol is an exact integer number
 @end itemize


@subsection @code{(graphics-to-occupancy-represenation imol)}
@cindex @code{(graphics-to-occupancy-represenation imol)}
 
where: 
 @itemize 
     @item imol is an exact integer number
 @end itemize


@subsection @code{(graphics-molecule-bond-type imol)}
@cindex @code{(graphics-molecule-bond-type imol)}
 
where: 
 @itemize 
     @item imol is an exact integer number
 @end itemize


@subsection @code{(clear-ball-and-stick imol)}
@cindex @code{(clear-ball-and-stick imol)}
 
where: 
 @itemize 
     @item imol is an exact integer number
 @end itemize


@subsection @code{(clear-dots imol dots-handle)}
@cindex @code{(clear-dots imol dots-handle)}
 
where: 
 @itemize 
     @item imol is an exact integer number
     @item dots-handle is an exact integer number
 @end itemize


@subsection @code{(n-dots-sets imol)}
@cindex @code{(n-dots-sets imol)}
 
where: 
 @itemize 
     @item imol is an exact integer number
 @end itemize



@section Pep-flip 
@subsection @code{(do-pepflip state)}
@cindex @code{(do-pepflip state)}
 
where: 
 @itemize 
     @item state is an exact integer number
 @end itemize


@subsection @code{(pepflip ires chain-id imol)}
@cindex @code{(pepflip ires chain-id imol)}
 
where: 
 @itemize 
     @item ires is an exact integer number
     @item chain-id is a string
     @item imol is an exact integer number
 @end itemize



@section Rigid Body Refinement 
@subsection @code{(do-rigid-body-refine state)}
@cindex @code{(do-rigid-body-refine state)}
 
where: 
 @itemize 
     @item state is an exact integer number
 @end itemize


@subsection @code{(execute-rigid-body-refine auto-range-flag)}
@cindex @code{(execute-rigid-body-refine auto-range-flag)}
 
where: 
 @itemize 
     @item auto-range-flag is an exact integer number
 @end itemize


@subsection @code{(set-rigid-body-fit-acceptable-fit-fraction f)}
@cindex @code{(set-rigid-body-fit-acceptable-fit-fraction f)}
 
where: 
 @itemize 
     @item f is an inexact number
 @end itemize



@section Dynamic Map 
@subsection @code{(toggle-dynamic-map-display-size)}
@cindex @code{(toggle-dynamic-map-display-size)}
 
@subsection @code{(toggle-dynamic-map-sampling)}
@cindex @code{(toggle-dynamic-map-sampling)}
 
@subsection @code{(set-dynamic-map-size-display-on)}
@cindex @code{(set-dynamic-map-size-display-on)}
 
@subsection @code{(set-dynamic-map-size-display-off)}
@cindex @code{(set-dynamic-map-size-display-off)}
 
@subsection @code{(set-dynamic-map-sampling-on)}
@cindex @code{(set-dynamic-map-sampling-on)}
 
@subsection @code{(set-dynamic-map-sampling-off)}
@cindex @code{(set-dynamic-map-sampling-off)}
 
@subsection @code{(set-dynamic-map-zoom-offset i)}
@cindex @code{(set-dynamic-map-zoom-offset i)}
 
where: 
 @itemize 
     @item i is an exact integer number
 @end itemize



@section Add Terminal Residue Functions 
@subsection @code{(do-add-terminal-residue state)}
@cindex @code{(do-add-terminal-residue state)}
 
where: 
 @itemize 
     @item state is an exact integer number
 @end itemize


@subsection @code{(set-add-terminal-residue-n-phi-psi-trials n)}
@cindex @code{(set-add-terminal-residue-n-phi-psi-trials n)}
 
where: 
 @itemize 
     @item n is an exact integer number
 @end itemize


@subsection @code{(set-add-terminal-residue-add-other-residue-flag i)}
@cindex @code{(set-add-terminal-residue-add-other-residue-flag i)}
 
where: 
 @itemize 
     @item i is an exact integer number
 @end itemize


@subsection @code{(set-terminal-residue-do-rigid-body-refine v)}
@cindex @code{(set-terminal-residue-do-rigid-body-refine v)}
 
where: 
 @itemize 
     @item v is an exact integer number
 @end itemize


@subsection @code{(add-terminal-residue-immediate-addition-state)}
@cindex @code{(add-terminal-residue-immediate-addition-state)}
 
@subsection @code{(set-add-terminal-residue-immediate-addition i)}
@cindex @code{(set-add-terminal-residue-immediate-addition i)}
 
where: 
 @itemize 
     @item i is an exact integer number
 @end itemize


@subsection @code{(set-add-terminal-residue-default-residue-type type)}
@cindex @code{(set-add-terminal-residue-default-residue-type type)}
 
where: 
 @itemize 
     @item type is a string
 @end itemize


@subsection @code{(set-add-terminal-residue-do-post-refine istat)}
@cindex @code{(set-add-terminal-residue-do-post-refine istat)}
 
where: 
 @itemize 
     @item istat is an exact integer number
 @end itemize



@section Delete Residues 
@subsection @code{(delete-atom-by-atom-index imol index do-delete-dialog)}
@cindex @code{(delete-atom-by-atom-index imol index do-delete-dialog)}
 
where: 
 @itemize 
     @item imol is an exact integer number
     @item index is an exact integer number
     @item do-delete-dialog is an exact integer number
 @end itemize


@subsection @code{(delete-residue-by-atom-index imol index do-delete-dialog)}
@cindex @code{(delete-residue-by-atom-index imol index do-delete-dialog)}
 
where: 
 @itemize 
     @item imol is an exact integer number
     @item index is an exact integer number
     @item do-delete-dialog is an exact integer number
 @end itemize


@subsection @code{(delete-residue-hydrogens-by-atom-index imol index do-delete-dialog)}
@cindex @code{(delete-residue-hydrogens-by-atom-index imol index do-delete-dialog)}
 
where: 
 @itemize 
     @item imol is an exact integer number
     @item index is an exact integer number
     @item do-delete-dialog is an exact integer number
 @end itemize


@subsection @code{(delete-residue-range imol chain-id resno-start end-resno)}
@cindex @code{(delete-residue-range imol chain-id resno-start end-resno)}
 
where: 
 @itemize 
     @item imol is an exact integer number
     @item chain-id is a string
     @item resno-start is an exact integer number
     @item end-resno is an exact integer number
 @end itemize


@subsection @code{(delete-residue imol chain-id resno inscode)}
@cindex @code{(delete-residue imol chain-id resno inscode)}
 
where: 
 @itemize 
     @item imol is an exact integer number
     @item chain-id is a string
     @item resno is an exact integer number
     @item inscode is a string
 @end itemize


@subsection @code{(delete-residue-with-altconf imol chain-id resno inscode altloc)}
@cindex @code{(delete-residue-with-altconf imol chain-id resno inscode altloc)}
 
where: 
 @itemize 
     @item imol is an exact integer number
     @item chain-id is a string
     @item resno is an exact integer number
     @item inscode is a string
     @item altloc is a string
 @end itemize


@subsection @code{(delete-residue-hydrogens imol chain-id resno inscode altloc)}
@cindex @code{(delete-residue-hydrogens imol chain-id resno inscode altloc)}
 
where: 
 @itemize 
     @item imol is an exact integer number
     @item chain-id is a string
     @item resno is an exact integer number
     @item inscode is a string
     @item altloc is a string
 @end itemize


@subsection @code{(delete-atom imol chain-id resno at-name altloc)}
@cindex @code{(delete-atom imol chain-id resno at-name altloc)}
 
where: 
 @itemize 
     @item imol is an exact integer number
     @item chain-id is a string
     @item resno is an exact integer number
     @item at-name is a string
     @item altloc is a string
 @end itemize


@subsection @code{(delete-residue-sidechain imol chain-id resno ins-code)}
@cindex @code{(delete-residue-sidechain imol chain-id resno ins-code)}
 
where: 
 @itemize 
     @item imol is an exact integer number
     @item chain-id is a string
     @item resno is an exact integer number
     @item ins-code is an unknown type
 @end itemize


@subsection @code{(set-delete-atom-mode)}
@cindex @code{(set-delete-atom-mode)}
 
@subsection @code{(set-delete-residue-mode)}
@cindex @code{(set-delete-residue-mode)}
 
@subsection @code{(set-delete-residue-zone-mode)}
@cindex @code{(set-delete-residue-zone-mode)}
 
@subsection @code{(set-delete-residue-hydrogens-mode)}
@cindex @code{(set-delete-residue-hydrogens-mode)}
 
@subsection @code{(set-delete-water-mode)}
@cindex @code{(set-delete-water-mode)}
 
@subsection @code{(set-delete-sidechain-mode)}
@cindex @code{(set-delete-sidechain-mode)}
 
@subsection @code{(delete-item-mode-is-atom-p)}
@cindex @code{(delete-item-mode-is-atom-p)}
 
@subsection @code{(delete-item-mode-is-residue-p)}
@cindex @code{(delete-item-mode-is-residue-p)}
 
@subsection @code{(delete-item-mode-is-water-p)}
@cindex @code{(delete-item-mode-is-water-p)}
 
@subsection @code{(delete-item-mode-is-sidechain-p)}
@cindex @code{(delete-item-mode-is-sidechain-p)}
 
@subsection @code{(clear-pending-delete-item)}
@cindex @code{(clear-pending-delete-item)}
 
@subsection @code{(set-keep-delete-item-active-state istate)}
@cindex @code{(set-keep-delete-item-active-state istate)}
 
where: 
 @itemize 
     @item istate is an exact integer number
 @end itemize



@section Rotate/Translate Buttons 
@subsection @code{(do-rot-trans-setup state)}
@cindex @code{(do-rot-trans-setup state)}
 
where: 
 @itemize 
     @item state is an exact integer number
 @end itemize


@subsection @code{(rot-trans-reset-previous)}
@cindex @code{(rot-trans-reset-previous)}
 
@subsection @code{(do-cis-trans-conversion-setup istate)}
@cindex @code{(do-cis-trans-conversion-setup istate)}
 
where: 
 @itemize 
     @item istate is an exact integer number
 @end itemize



@section Mainchain Building Functions 
@subsection @code{(do-db-main state)}
@cindex @code{(do-db-main state)}
 
where: 
 @itemize 
     @item state is an exact integer number
 @end itemize



@section Close Molecule FUnctions 
@subsection @code{(close-molecule imol)}
@cindex @code{(close-molecule imol)}
 
where: 
 @itemize 
     @item imol is an exact integer number
 @end itemize



@section Rotatmer Functions 
@subsection @code{(setup-rotamers state)}
@cindex @code{(setup-rotamers state)}
 
where: 
 @itemize 
     @item state is an exact integer number
 @end itemize


@subsection @code{(do-rotamers atom-index imol)}
@cindex @code{(do-rotamers atom-index imol)}
 
where: 
 @itemize 
     @item atom-index is an exact integer number
     @item imol is an exact integer number
 @end itemize


@subsection @code{(set-rotamer-lowest-probability f)}
@cindex @code{(set-rotamer-lowest-probability f)}
 
where: 
 @itemize 
     @item f is an inexact number
 @end itemize


@subsection @code{(set-rotamer-check-clashes i)}
@cindex @code{(set-rotamer-check-clashes i)}
 
where: 
 @itemize 
     @item i is an exact integer number
 @end itemize


@subsection @code{(set-auto-fit-best-rotamer-clash-flag i)}
@cindex @code{(set-auto-fit-best-rotamer-clash-flag i)}
 
where: 
 @itemize 
     @item i is an exact integer number
 @end itemize


@subsection @code{(setup-auto-fit-rotamer state)}
@cindex @code{(setup-auto-fit-rotamer state)}
 
where: 
 @itemize 
     @item state is an exact integer number
 @end itemize


@subsection @code{(fill-partial-residues imol)}
@cindex @code{(fill-partial-residues imol)}
 
where: 
 @itemize 
     @item imol is an exact integer number
 @end itemize


@subsection @code{(setup-180-degree-flip state)}
@cindex @code{(setup-180-degree-flip state)}
 
where: 
 @itemize 
     @item state is an exact integer number
 @end itemize



@section Mutate Functions 
@subsection @code{(setup-mutate state)}
@cindex @code{(setup-mutate state)}
 
where: 
 @itemize 
     @item state is an exact integer number
 @end itemize


@subsection @code{(setup-mutate-auto-fit state)}
@cindex @code{(setup-mutate-auto-fit state)}
 
where: 
 @itemize 
     @item state is an exact integer number
 @end itemize


@subsection @code{(do-mutation type is-stub-flag)}
@cindex @code{(do-mutation type is-stub-flag)}
 
where: 
 @itemize 
     @item type is a string
     @item is-stub-flag is an exact integer number
 @end itemize


@subsection @code{(progressive-residues-in-chain-check chain-id imol)}
@cindex @code{(progressive-residues-in-chain-check chain-id imol)}
 
where: 
 @itemize 
     @item chain-id is a string
     @item imol is an exact integer number
 @end itemize


@subsection @code{(mutate ires chain-id imol target-res-type)}
@cindex @code{(mutate ires chain-id imol target-res-type)}
 
where: 
 @itemize 
     @item ires is an exact integer number
     @item chain-id is a string
     @item imol is an exact integer number
     @item target-res-type is a string
 @end itemize


@subsection @code{(set-mutate-auto-fit-do-post-refine istate)}
@cindex @code{(set-mutate-auto-fit-do-post-refine istate)}
 
where: 
 @itemize 
     @item istate is an exact integer number
 @end itemize


@subsection @code{(mutate-auto-fit-do-post-refine-state)}
@cindex @code{(mutate-auto-fit-do-post-refine-state)}
 
@subsection @code{(do-base-mutation type)}
@cindex @code{(do-base-mutation type)}
 
where: 
 @itemize 
     @item type is a string
 @end itemize


@subsection @code{(set-residue-type-chooser-stub-state istat)}
@cindex @code{(set-residue-type-chooser-stub-state istat)}
 
where: 
 @itemize 
     @item istat is an exact integer number
 @end itemize



@section Alternative Conformation 
@subsection @code{(alt-conf-split-type-number)}
@cindex @code{(alt-conf-split-type-number)}
 
@subsection @code{(set-add-alt-conf-split-type-number i)}
@cindex @code{(set-add-alt-conf-split-type-number i)}
 
where: 
 @itemize 
     @item i is an exact integer number
 @end itemize


@subsection @code{(unset-add-alt-conf-dialog)}
@cindex @code{(unset-add-alt-conf-dialog)}
 
@subsection @code{(unset-add-alt-conf-define)}
@cindex @code{(unset-add-alt-conf-define)}
 
@subsection @code{(altconf)}
@cindex @code{(altconf)}
 
@subsection @code{(set-add-alt-conf-new-atoms-occupancy f)}
@cindex @code{(set-add-alt-conf-new-atoms-occupancy f)}
 
where: 
 @itemize 
     @item f is an inexact number
 @end itemize


@subsection @code{(set-show-alt-conf-intermediate-atoms i)}
@cindex @code{(set-show-alt-conf-intermediate-atoms i)}
 
where: 
 @itemize 
     @item i is an exact integer number
 @end itemize


@subsection @code{(show-alt-conf-intermediate-atoms-state)}
@cindex @code{(show-alt-conf-intermediate-atoms-state)}
 
@subsection @code{(zero-occupancy-residue-range imol chain-id ires1 ires2)}
@cindex @code{(zero-occupancy-residue-range imol chain-id ires1 ires2)}
 
where: 
 @itemize 
     @item imol is an exact integer number
     @item chain-id is a string
     @item ires1 is an exact integer number
     @item ires2 is an exact integer number
 @end itemize


@subsection @code{(fill-occupancy-residue-range imol chain-id ires1 ires2)}
@cindex @code{(fill-occupancy-residue-range imol chain-id ires1 ires2)}
 
where: 
 @itemize 
     @item imol is an exact integer number
     @item chain-id is a string
     @item ires1 is an exact integer number
     @item ires2 is an exact integer number
 @end itemize



@section Pointer Atom Functions 
@subsection @code{(place-atom-at-pointer)}
@cindex @code{(place-atom-at-pointer)}
 
@subsection @code{(place-typed-atom-at-pointer type)}
@cindex @code{(place-typed-atom-at-pointer type)}
 
where: 
 @itemize 
     @item type is a string
 @end itemize


@subsection @code{(set-pointer-atom-is-dummy i)}
@cindex @code{(set-pointer-atom-is-dummy i)}
 
where: 
 @itemize 
     @item i is an exact integer number
 @end itemize


@subsection @code{(display-where-is-pointer)}
@cindex @code{(display-where-is-pointer)}
 

@section Baton Build Functions 
@subsection @code{(set-baton-mode i)}
@cindex @code{(set-baton-mode i)}
 
where: 
 @itemize 
     @item i is an exact integer number
 @end itemize


@subsection @code{(set-draw-baton i)}
@cindex @code{(set-draw-baton i)}
 
where: 
 @itemize 
     @item i is an exact integer number
 @end itemize


@subsection @code{(accept-baton-position)}
@cindex @code{(accept-baton-position)}
 
@subsection @code{(baton-try-another)}
@cindex @code{(baton-try-another)}
 
@subsection @code{(shorten-baton)}
@cindex @code{(shorten-baton)}
 
@subsection @code{(lengthen-baton)}
@cindex @code{(lengthen-baton)}
 
@subsection @code{(baton-build-delete-last-residue)}
@cindex @code{(baton-build-delete-last-residue)}
 
@subsection @code{(set-baton-build-params istart-resno chain-id backwards)}
@cindex @code{(set-baton-build-params istart-resno chain-id backwards)}
 
where: 
 @itemize 
     @item istart-resno is an exact integer number
     @item chain-id is a string
     @item backwards is a string
 @end itemize



@section Post-Baton Functions 
@subsection @code{(reverse-direction-of-fragment imol chain-id resno)}
@cindex @code{(reverse-direction-of-fragment imol chain-id resno)}
 
where: 
 @itemize 
     @item imol is an exact integer number
     @item chain-id is a string
     @item resno is an exact integer number
 @end itemize


@subsection @code{(setup-reverse-direction i)}
@cindex @code{(setup-reverse-direction i)}
 
where: 
 @itemize 
     @item i is an exact integer number
 @end itemize



@section Terminal OXT Atom 
@subsection @code{(add-OXT-to-residue imol reso insertion-code chain-id)}
@cindex @code{(add-OXT-to-residue imol reso insertion-code chain-id)}
 
where: 
 @itemize 
     @item imol is an exact integer number
     @item reso is an exact integer number
     @item insertion-code is a string
     @item chain-id is a string
 @end itemize



@section Crosshairs 
@subsection @code{(set-draw-crosshairs i)}
@cindex @code{(set-draw-crosshairs i)}
 
where: 
 @itemize 
     @item i is an exact integer number
 @end itemize


@subsection @code{(draw-crosshairs-state)}
@cindex @code{(draw-crosshairs-state)}
 

@section Edit Chi Angles 
@subsection @code{(setup-edit-chi-angles state)}
@cindex @code{(setup-edit-chi-angles state)}
 
where: 
 @itemize 
     @item state is an exact integer number
 @end itemize


@subsection @code{(set-find-hydrogen-torsion state)}
@cindex @code{(set-find-hydrogen-torsion state)}
 
where: 
 @itemize 
     @item state is an exact integer number
 @end itemize


@subsection @code{(set-graphics-edit-current-chi ichi)}
@cindex @code{(set-graphics-edit-current-chi ichi)}
 
where: 
 @itemize 
     @item ichi is an exact integer number
 @end itemize


@subsection @code{(unset-moving-atom-move-chis)}
@cindex @code{(unset-moving-atom-move-chis)}
 
@subsection @code{(set-show-chi-angle-bond imode)}
@cindex @code{(set-show-chi-angle-bond imode)}
 
where: 
 @itemize 
     @item imode is an exact integer number
 @end itemize



@section Masks 
@subsection @code{(mask-map-by-molecule map-mol-no coord-mol-no invert-flag)}
@cindex @code{(mask-map-by-molecule map-mol-no coord-mol-no invert-flag)}
 
where: 
 @itemize 
     @item map-mol-no is an exact integer number
     @item coord-mol-no is an exact integer number
     @item invert-flag is an exact integer number
 @end itemize


@subsection @code{(mask-map-by-atom-selection map-mol-no coords-mol-no mmdb-atom-selection invert-flag)}
@cindex @code{(mask-map-by-atom-selection map-mol-no coords-mol-no mmdb-atom-selection invert-flag)}
 
where: 
 @itemize 
     @item map-mol-no is an exact integer number
     @item coords-mol-no is an exact integer number
     @item mmdb-atom-selection is a string
     @item invert-flag is an exact integer number
 @end itemize


@subsection @code{(set-map-mask-atom-radius rad)}
@cindex @code{(set-map-mask-atom-radius rad)}
 
where: 
 @itemize 
     @item rad is an inexact number
 @end itemize



@section Check Waters Interface 
@subsection @code{(set-check-waters-b-factor-limit f)}
@cindex @code{(set-check-waters-b-factor-limit f)}
 
where: 
 @itemize 
     @item f is an inexact number
 @end itemize


@subsection @code{(set-check-waters-map-sigma-limit f)}
@cindex @code{(set-check-waters-map-sigma-limit f)}
 
where: 
 @itemize 
     @item f is an inexact number
 @end itemize


@subsection @code{(set-check-waters-min-dist-limit f)}
@cindex @code{(set-check-waters-min-dist-limit f)}
 
where: 
 @itemize 
     @item f is an inexact number
 @end itemize


@subsection @code{(set-check-waters-max-dist-limit f)}
@cindex @code{(set-check-waters-max-dist-limit f)}
 
where: 
 @itemize 
     @item f is an inexact number
 @end itemize


@subsection @code{(check-waters-by-difference-map-sigma-level-state)}
@cindex @code{(check-waters-by-difference-map-sigma-level-state)}
 
@subsection @code{(set-check-waters-by-difference-map-sigma-level f)}
@cindex @code{(set-check-waters-by-difference-map-sigma-level f)}
 
where: 
 @itemize 
     @item f is an inexact number
 @end itemize



@section Least-Squares matching 
@subsection @code{(clear-lsq-matches)}
@cindex @code{(clear-lsq-matches)}
 
@subsection @code{(apply-lsq-matches imol-reference imol-moving)}
@cindex @code{(apply-lsq-matches imol-reference imol-moving)}
 
where: 
 @itemize 
     @item imol-reference is an exact integer number
     @item imol-moving is an exact integer number
 @end itemize



@section Least-Squares plane interface 
@subsection @code{(setup-lsq-deviation state)}
@cindex @code{(setup-lsq-deviation state)}
 
where: 
 @itemize 
     @item state is an exact integer number
 @end itemize


@subsection @code{(setup-lsq-plane-define state)}
@cindex @code{(setup-lsq-plane-define state)}
 
where: 
 @itemize 
     @item state is an exact integer number
 @end itemize


@subsection @code{(unset-lsq-plane-dialog)}
@cindex @code{(unset-lsq-plane-dialog)}
 
@subsection @code{(remove-last-lsq-plane-atom)}
@cindex @code{(remove-last-lsq-plane-atom)}
 

@section Trim 
@subsection @code{(raster3d rd3-filename)}
@cindex @code{(raster3d rd3-filename)}
 
where: 
 @itemize 
     @item rd3-filename is a string
 @end itemize


@subsection @code{(povray filename)}
@cindex @code{(povray filename)}
 
where: 
 @itemize 
     @item filename is a string
 @end itemize


@subsection @code{(make-image-raster3d filename)}
@cindex @code{(make-image-raster3d filename)}
 
where: 
 @itemize 
     @item filename is a string
 @end itemize


@subsection @code{(make-image-povray filename)}
@cindex @code{(make-image-povray filename)}
 
where: 
 @itemize 
     @item filename is a string
 @end itemize


@subsection @code{(set-raster3d-bond-thickness f)}
@cindex @code{(set-raster3d-bond-thickness f)}
 
where: 
 @itemize 
     @item f is an inexact number
 @end itemize


@subsection @code{(set-raster3d-density-thickness f)}
@cindex @code{(set-raster3d-density-thickness f)}
 
where: 
 @itemize 
     @item f is an inexact number
 @end itemize


@subsection @code{(set-renderer-show-atoms istate)}
@cindex @code{(set-renderer-show-atoms istate)}
 
where: 
 @itemize 
     @item istate is an exact integer number
 @end itemize


@subsection @code{(raster-screen-shot)}
@cindex @code{(raster-screen-shot)}
 
@subsection @code{(citation-notice-off)}
@cindex @code{(citation-notice-off)}
 

@section Superposition (SSM) 
@subsection @code{(superpose imol1 imol2 move-imol2-flag)}
@cindex @code{(superpose imol1 imol2 move-imol2-flag)}
 
where: 
 @itemize 
     @item imol1 is an exact integer number
     @item imol2 is an exact integer number
     @item move-imol2-flag is an exact integer number
 @end itemize



@section NCS 
@subsection @code{(set-draw-ncs-ghosts imol istate)}
@cindex @code{(set-draw-ncs-ghosts imol istate)}
 
where: 
 @itemize 
     @item imol is an exact integer number
     @item istate is an exact integer number
 @end itemize


@subsection @code{(set-ncs-ghost-bond-thickness imol f)}
@cindex @code{(set-ncs-ghost-bond-thickness imol f)}
 
where: 
 @itemize 
     @item imol is an exact integer number
     @item f is an inexact number
 @end itemize


@subsection @code{(ncs-update-ghosts imol)}
@cindex @code{(ncs-update-ghosts imol)}
 
where: 
 @itemize 
     @item imol is an exact integer number
 @end itemize


@subsection @code{(make-dynamically-transformed-ncs-maps imol-model imol-map)}
@cindex @code{(make-dynamically-transformed-ncs-maps imol-model imol-map)}
 
where: 
 @itemize 
     @item imol-model is an exact integer number
     @item imol-map is an exact integer number
 @end itemize


@subsection @code{(make-ncs-ghosts-maybe imol)}
@cindex @code{(make-ncs-ghosts-maybe imol)}
 
where: 
 @itemize 
     @item imol is an exact integer number
 @end itemize


@subsection @code{(show-strict-ncs-state imol)}
@cindex @code{(show-strict-ncs-state imol)}
 
where: 
 @itemize 
     @item imol is an exact integer number
 @end itemize


@subsection @code{(set-show-strict-ncs imol state)}
@cindex @code{(set-show-strict-ncs imol state)}
 
where: 
 @itemize 
     @item imol is an exact integer number
     @item state is an exact integer number
 @end itemize


@subsection @code{(set-ncs-homology-level flev)}
@cindex @code{(set-ncs-homology-level flev)}
 
where: 
 @itemize 
     @item flev is an inexact number
 @end itemize


@subsection @code{(copy-chain imol from-chain to-chain)}
@cindex @code{(copy-chain imol from-chain to-chain)}
 
where: 
 @itemize 
     @item imol is an exact integer number
     @item from-chain is a string
     @item to-chain is a string
 @end itemize


@subsection @code{(copy-from-ncs-master-to-others imol chain-id)}
@cindex @code{(copy-from-ncs-master-to-others imol chain-id)}
 
where: 
 @itemize 
     @item imol is an exact integer number
     @item chain-id is a string
 @end itemize


@subsection @code{(ncs-control-change-ncs-master-to-chain imol ichain)}
@cindex @code{(ncs-control-change-ncs-master-to-chain imol ichain)}
 
where: 
 @itemize 
     @item imol is an exact integer number
     @item ichain is an exact integer number
 @end itemize


@subsection @code{(ncs-control-display-chain imol ichain state)}
@cindex @code{(ncs-control-display-chain imol ichain state)}
 
where: 
 @itemize 
     @item imol is an exact integer number
     @item ichain is an exact integer number
     @item state is an exact integer number
 @end itemize


@subsection @code{(place-helix-here)}
@cindex @code{(place-helix-here)}
 
@subsection @code{(new-molecule-by-residue-type-selection imol residue-type)}
@cindex @code{(new-molecule-by-residue-type-selection imol residue-type)}
 
where: 
 @itemize 
     @item imol is an exact integer number
     @item residue-type is a string
 @end itemize


@subsection @code{(new-molecule-by-atom-selection imol atom-selection)}
@cindex @code{(new-molecule-by-atom-selection imol atom-selection)}
 
where: 
 @itemize 
     @item imol is an exact integer number
     @item atom-selection is an unknown type
 @end itemize



@section Miguel's orientation axes matrix 

@section RNA/DNA 

@section Sequence (Assignment) 
@subsection @code{(print-sequence-chain imol chain-id)}
@cindex @code{(print-sequence-chain imol chain-id)}
 
where: 
 @itemize 
     @item imol is an exact integer number
     @item chain-id is a string
 @end itemize


@subsection @code{(assign-fasta-sequence imol chain-id-in seq)}
@cindex @code{(assign-fasta-sequence imol chain-id-in seq)}
 
where: 
 @itemize 
     @item imol is an exact integer number
     @item chain-id-in is a string
     @item seq is a string
 @end itemize


@subsection @code{(assign-sequence imol-model imol-map chain-id)}
@cindex @code{(assign-sequence imol-model imol-map chain-id)}
 
where: 
 @itemize 
     @item imol-model is an exact integer number
     @item imol-map is an exact integer number
     @item chain-id is a string
 @end itemize



@section Surfaces 
@subsection @code{(do-surface imol istate)}
@cindex @code{(do-surface imol istate)}
 
where: 
 @itemize 
     @item imol is an exact integer number
     @item istate is an exact integer number
 @end itemize


@subsection @code{(fffear-search imol-model imol-map)}
@cindex @code{(fffear-search imol-model imol-map)}
 
where: 
 @itemize 
     @item imol-model is an exact integer number
     @item imol-map is an exact integer number
 @end itemize


@subsection @code{(set-fffear-angular-resolution f)}
@cindex @code{(set-fffear-angular-resolution f)}
 
where: 
 @itemize 
     @item f is an inexact number
 @end itemize


@subsection @code{(fffear-angular-resolution)}
@cindex @code{(fffear-angular-resolution)}
 

@section Remote Control 
@subsection @code{(make-socket-listener-maybe)}
@cindex @code{(make-socket-listener-maybe)}
 
@subsection @code{(set-coot-listener-socket-state-internal sock-state)}
@cindex @code{(set-coot-listener-socket-state-internal sock-state)}
 
where: 
 @itemize 
     @item sock-state is an exact integer number
 @end itemize



@section Display Lists for Maps 
@subsection @code{(set-display-lists-for-maps i)}
@cindex @code{(set-display-lists-for-maps i)}
 
where: 
 @itemize 
     @item i is an exact integer number
 @end itemize



@section Preferences 
@subsection @code{(preferences)}
@cindex @code{(preferences)}
 
@subsection @code{(clear-preferences)}
@cindex @code{(clear-preferences)}
 
@subsection @code{(set-mark-cis-peptides-as-bad istate)}
@cindex @code{(set-mark-cis-peptides-as-bad istate)}
 
where: 
 @itemize 
     @item istate is an exact integer number
 @end itemize


@subsection @code{(show-mark-cis-peptides-as-bad-state)}
@cindex @code{(show-mark-cis-peptides-as-bad-state)}
 
@subsection @code{(browser-url url)}
@cindex @code{(browser-url url)}
 
where: 
 @itemize 
     @item url is a string
 @end itemize


@subsection @code{(set-browser-interface browser)}
@cindex @code{(set-browser-interface browser)}
 
where: 
 @itemize 
     @item browser is a string
 @end itemize


@subsection @code{(handle-online-coot-search-request entry-text)}
@cindex @code{(handle-online-coot-search-request entry-text)}
 
where: 
 @itemize 
     @item entry-text is a string
 @end itemize


@subsection @code{(new-generic-object-number objname)}
@cindex @code{(new-generic-object-number objname)}
 
where: 
 @itemize 
     @item objname is a string
 @end itemize


@subsection @code{(to-generic-object-add-display-list-handle object-number display-list-id)}
@cindex @code{(to-generic-object-add-display-list-handle object-number display-list-id)}
 
where: 
 @itemize 
     @item object-number is an exact integer number
     @item display-list-id is an exact integer number
 @end itemize


@subsection @code{(set-display-generic-object object-number istate)}
@cindex @code{(set-display-generic-object object-number istate)}
 
where: 
 @itemize 
     @item object-number is an exact integer number
     @item istate is an exact integer number
 @end itemize


@subsection @code{(generic-object-is-displayed-p object-number)}
@cindex @code{(generic-object-is-displayed-p object-number)}
 
where: 
 @itemize 
     @item object-number is an exact integer number
 @end itemize


@subsection @code{(generic-object-index name)}
@cindex @code{(generic-object-index name)}
 
where: 
 @itemize 
     @item name is a string
 @end itemize


@subsection @code{(number-of-generic-objects)}
@cindex @code{(number-of-generic-objects)}
 
@subsection @code{(generic-object-info)}
@cindex @code{(generic-object-info)}
 
@subsection @code{(generic-object-has-objects-p obj-no)}
@cindex @code{(generic-object-has-objects-p obj-no)}
 
where: 
 @itemize 
     @item obj-no is an exact integer number
 @end itemize


@subsection @code{(close-generic-object object-number)}
@cindex @code{(close-generic-object object-number)}
 
where: 
 @itemize 
     @item object-number is an exact integer number
 @end itemize


@subsection @code{(is-closed-generic-object-p object-number)}
@cindex @code{(is-closed-generic-object-p object-number)}
 
where: 
 @itemize 
     @item object-number is an exact integer number
 @end itemize


@subsection @code{(generic-objects-gui-wrapper)}
@cindex @code{(generic-objects-gui-wrapper)}
 
@subsection @code{(handle-read-draw-probe-dots dots-file)}
@cindex @code{(handle-read-draw-probe-dots dots-file)}
 
where: 
 @itemize 
     @item dots-file is a string
 @end itemize


@subsection @code{(handle-read-draw-probe-dots-unformatted dots-file imol show-clash-gui-flag)}
@cindex @code{(handle-read-draw-probe-dots-unformatted dots-file imol show-clash-gui-flag)}
 
where: 
 @itemize 
     @item dots-file is a string
     @item imol is an exact integer number
     @item show-clash-gui-flag is an exact integer number
 @end itemize


@subsection @code{(set-do-probe-dots-on-rotamers-and-chis state)}
@cindex @code{(set-do-probe-dots-on-rotamers-and-chis state)}
 
where: 
 @itemize 
     @item state is an exact integer number
 @end itemize


@subsection @code{(do-probe-dots-on-rotamers-and-chis-state)}
@cindex @code{(do-probe-dots-on-rotamers-and-chis-state)}
 
@subsection @code{(set-do-probe-dots-post-refine state)}
@cindex @code{(set-do-probe-dots-post-refine state)}
 
where: 
 @itemize 
     @item state is an exact integer number
 @end itemize


@subsection @code{(do-probe-dots-post-refine-state)}
@cindex @code{(do-probe-dots-post-refine-state)}
 
@subsection @code{(set-interactive-probe-dots-molprobity-radius r)}
@cindex @code{(set-interactive-probe-dots-molprobity-radius r)}
 
where: 
 @itemize 
     @item r is an inexact number
 @end itemize


@subsection @code{(interactive-probe-dots-molprobity-radius)}
@cindex @code{(interactive-probe-dots-molprobity-radius)}
 
@subsection @code{(probe-available-p)}
@cindex @code{(probe-available-p)}
 
@subsection @code{(set-dti-stereo-mode state)}
@cindex @code{(set-dti-stereo-mode state)}
 
where: 
 @itemize 
     @item state is an exact integer number
 @end itemize


@subsection @code{(do-smiles-gui)}
@cindex @code{(do-smiles-gui)}
 

@section Fun 
@subsection @code{(do-tw)}
@cindex @code{(do-tw)}
 
@subsection @code{(place-text text x y z size)}
@cindex @code{(place-text text x y z size)}
 
where: 
 @itemize 
     @item text is an unknown type
     @item x is an inexact number
     @item y is an inexact number
     @item z is an inexact number
     @item size is an exact integer number
 @end itemize


@subsection @code{(remove-text text-handle)}
@cindex @code{(remove-text text-handle)}
 
where: 
 @itemize 
     @item text-handle is an exact integer number
 @end itemize




%
\documentclass{book}
\usepackage{a4}
\usepackage{palatino}
%\usepackage{times}
%\usepackage{utopia}
\usepackage{euler}
\usepackage{fancyhdr}
\usepackage{epsf}

\newcommand {\atilde} {$_{\char '176}$} % tilde(~) character

\title{The Coot Reference Manual}
\author{Paul Emsley \\\textsf{\small emsley@ysbl.york.ac.uk}}
%\makeindex  % Not at the moment.  There are no index markups (yet).

\begin{document}
\maketitle
\tableofcontents

\chapter{Acknowledgments}
Paul Emsley is extremely grateful to use the library code of the
following people, without whom Coot could not have been realised:

\begin{trivlist}
\item Kevin Cowtan
\item Eugene Krissinel
\item Stuart McNicholas
\item Raghavendra Chandrashekara
\item Paul Bourke \& Cory Gene Bloyd
\end{trivlist}

Roland Dunbrack \& co-workers for rotamer library data.

Also (for generally useful software used in Coot):

\begin{trivlist}
\item Matteo Frigo \& Steven G. Johnson
\item Gary Houston \& other Guile developers
\item Python developers
\item Gtk+ and GNOME-Canvas developers
\item GNU Scientific Library developers
\item OpenGL developers
\item Janne L\"of
\end{trivlist}

Also those with whom Paul has corresponded about or provided
features and bug fixes and built the software:

\begin{tabular}{ll}
 William G. Scott & Bernhard Lohkamp \\
 Joel Bard  & Ezra Peisach           \\
 Alex Schuettelkopf & Charlie Bond 
\end{tabular}

Not forgetting the testers\footnote{in no particular order}

%\begin{trivlist}
%\item Eleanor J. Dodson
%\item Jan Dohnalek
%\item Karen McLuskey
%\item Bernhard Lohkamp
%\item Aleks Roszak
%\item Florence Vincent
%\item Roberto Steiner
%\item Alex Schuettelkopf
%\item Charlie Bond
%\item Constantina Fotinou
%\item William G. Scott
%\item Adrian Lapthorn
%\end{trivlist}

\begin{tabular}{ll}
Eleanor J. Dodson & Jan Dohnalek \\
Constantina Fotinou & Alex Roszak  \\
Florence Vincent  & Roberto Steiner \\
Karen McLuskey & Adrian Lapthorn   
\end{tabular}

\vspace{5mm}

Those with experience of Quanta, XFit and O will notice similarities
between Coot and those programs, it's fair to say that they have had
considerable influence in the look of Coot, so Paul respectively
thanks for inspiration: Tom Oldfield, Alwyn Jones and Duncan McRee
(and their co-workers).

\chapter{Design Overview}
\section{Why?}
``Why does Coot exist?'' you might ask.  ``Given that other molecular
graphics\footnote{molecular graphics with protein modeling and
  density fitting functions, that is.} programs exist, why write
another?''

Because I like having the source code to programs I use and think that
others feel the same.  Because the other programs don't quite work how
I wanted them to\footnote{and of course, there was no way to fix
  that.}. Because there was the possibility to integrate molecular
graphics into the CCP4 Suite.  

As to why write Coot when CCP4MG was available: that is not how it
happened. Coot\footnote{it was called ``MapView'' at the time.} was
released over a year before CCP4MG was available.  I followed my own
design, toolkit and aesthetic decisions - for good or bad\footnote{for
  example, I was (and remain) less concerned about porting to various
  shades of Microsoft Windows operating systems than the CCP4MG
  developers.}.

\section{Hacker's Guide}

The are several core libraries that are fundamental to Coot:

\begin{itemize}
\item Clipper: Kevin Cowtan's General crystallographic object library
\item mmdb: CCP4's Coordinate Library
\item GTk+: GNU's GUI toolkit.
\end{itemize}

\subsection{GUI}
The GUI is almost entirely built using glade.  Glade writes out its
code in pure C.  This causes a problem.  \texttt{src/interface.h} and
\texttt{src/support.h} both get regenerated in ``C mode'' every time
glade is run.  So, after every time we change the GUI with glade, we
need to run \texttt{post-glade} to introduce the C/C++ linking type
declaration wrapper into these files.

Not all of the GUI is build with glade - there are dynamic elements,
for example the ``Map and Molecule (Display) Control'' window the
frame of which are generated in glade, but the hboxs are filled using
hand-made code (see \texttt{gtk-manual.c}).

\subsection{GUI/State Variables}
The graphics\_info\_t class contains a host of static state variables,
mostly manipulated by GUI element (\emph{e.g} button)
callbacks\footnote{mostly button clicked signals and menu item
  activative signals}. For historical reasons they are initially set
in \texttt{globjects.cc}.  Because the callbacks are written in C by
glade\footnote{the GUI builder}, these variables need a functional
interface to set the variables, and that interface is used by both the
GUI button\footnote{and other GUI elements} callbacks and is exported
to the scripting level.  These function declarations are in
\texttt{c-interface.h}.  All manipulations of graphics\_info\_t's
state variables go via \texttt{c-interface.h}.

Notice that MMDB functions are not allowed in
this interface\footnote{because SWIG chokes on them}. 

\subsection{Scripting}
So, SWIG uses \texttt{c-interface.h} to generate the python/scheme
scripting interface. The scripting language is chosen at
configure-time using either \texttt{--with-guile} or
\texttt{--with-python}.

\section{Validation}
As I write this, a few of us are cobbling together a XML-based system
for validation.  We think that validation data should be presented as
XML data that can be passed between packages and programs.  Either the
program itself will output the data, or we will write a wrapper to
turn the output into the appropriate XML format.  

These XML data will be then available for use in the molecular
graphics and will provide information for a ``Next Unusual Feature''
button.  The library to provide the XML cabability for this is expat,
the same library used in Perl's XML::Parser, Python's XML parser
Pyexpat and Mozilla's XML parser.

\subsection{Example: Temperature Factor Analysis}
Recall that the kurtosis of a distribution, $k$ is given by:

\begin{equation}
  \label{eq:kurtosis}
  k = \frac{\Sigma(X_i - \mu)^4} {N \sigma^4} - 3 
\end{equation}

We calculate the kurtosis for the isotropic temperature factors for
each residue in the molecule and residues with the most leptokurtic
distributions are written out to a file.  The format of the file is
XML.

This is an example of how we expect validation data to be presented to
molecular graphics programs.



\chapter{Refinement and Regularization}

A function that we need for Molecular Graphics is to be able to
regularize (a.k.a ``idealize'') the coordinates of the model.  In
order to do so we need to find the ideal values (also called here
``restraints'', using the Refmac nomenclature).  We have a
multivariable function minimizer that requires the gradients of the
parameters (the coordinates).  Here we describe how to generate the
gradients analytically.  We need the derivatives for the bond lengths,
angles, torsions and planes.

\section{Introduction}

The function that we are trying to minimize for refinement is $S$, where

\begin{displaymath}
  S = S_{bond} + S_{angle} + S_{torsion} + S_{plane} + S_{chiral} + -kS_{map}
\end{displaymath}

For regularization it is:
\begin{displaymath}
  S = S_{bond} + S_{angle} + S_{torsion}
\end{displaymath}



Let's take these 3 parts in turn:

% ------------------------------------------------------------------
%                  Bonds 
% ------------------------------------------------------------------

\section{Bonds}

\begin{displaymath}
  S_{bond} = \sum_{i=1}^{N_{bonds}} {(b_i - b_{0_i})^2}
\end{displaymath}

Where $b_{0_i}$ is the ideal length (from the Refmac dictionary) of
the $i$th bond, $\mathbf{b}_i$ is the bond vector and $b_i$ is its length.

\begin{eqnarray*}
  \label{eq:1}
  \frac{\partial S_i}{\partial x_m} & = & \frac{\partial S_i}{\partial b_i} 
  \frac{\partial b_i}{\partial x_m} \\
   & = & [2(b_i - b_{0_i})]   \frac{\partial b_i}{\partial x_m}
\end{eqnarray*}

\begin{displaymath}
  b_i = \sqrt((x_m-x_k)^2 + (y_m-y_k)^2 + (z_m-z_k)^2)
\end{displaymath}

So: 
\begin{eqnarray*}
\frac{\partial b_i}{\partial x_m} & = & (\frac{1}{2} \frac{1}{b_i}) 2 (x_m - x_k) \\
 &   = & \frac{(x_m - x_k)}{b_i}
\end{eqnarray*}

So: 
\begin{displaymath}
  \frac{\partial S_i}{\partial x_m} = 2[b_i - b_{0_1}] \frac{(x_m - x_k)}{b_i}
\end{displaymath}


% ------------------------------------------------------------------
%                  Angles 
% ------------------------------------------------------------------

\section{Angles}
We are trying to minimise $S_{angle}$, where (for simplicity I ignore
the weights)

\begin{displaymath}
  S_{angle} = \sum_{i=1}^{N_{angles}} {(\theta_i - \theta_{0_i})^2}
\end{displaymath}


Angle $\theta$ contributed to by atoms $k$, $l$ and $m$:

\begin{displaymath}
  \cos \theta = \frac{{\underline a}.{\underline b}}{ab}
\end{displaymath}

\begin{trivlist}
\item where
\item $\underline {a}$ is the bond of atoms $k$ and $l$ $((x_k-x_l), (y_k-y_l), (z_k-z_l))$
\item $\underline {b}$ is the bond of atoms $l$ and $m$  $((x_m-x_l), (y_m-y_l), (z_m-z_l))$
\item Note that the vectors point away from the middle atom $l$.
\end{trivlist}



So: 

\begin{equation}
  \label{eq:1}
  \theta = acos(P)
\end{equation}

where 

\begin{displaymath}
  P = \frac{{\underline a}.{\underline b}}{ab} 
\end{displaymath}

Using the Chain Rule:
\begin{equation}
  \label{eq:2}
  \frac{\partial \theta}{\partial _k} = \frac{\partial \theta}{\partial P} \frac{\partial P}{\partial x_k}
\end{equation}

Given that we are only intereted in $\theta$ in the range $0\rightarrow\pi$:

\begin{equation}
  \label{eq:3}
  \frac{\partial \theta}{\partial P} = -\frac{1}{\sin \theta}
\end{equation}

Let's split up $P$ again using the chain rule: 
\begin{equation}
  \label{eq:4}
  \frac{\partial P}{\partial x_k} = 
  Q\frac{\partial R}{\partial x_k} + R\frac{\partial Q}{\partial x_k}
\end{equation}

where 
\begin{equation}
  \label{eq:5}
  Q =  {\underline a}.{\underline b}
\end{equation}
\begin{equation}
  \label{eq:6}
  R = \frac{1}{ab}
\end{equation}

\subsection{The middle atom}

This is somewhat more tricky than an end atom because the derivatives
of $ab$ and ${\underline a}.{\underline b}$ are not so trivial.  Let's
change the indexing so that we are actually talking about the middle
atom, $l$.

Differentiating (\ref{eq:6}): 

\begin{equation}
  \label{eq:7}
  \frac{\partial R}{\partial x_l} = 
  -\frac{1}{(ab)^2}b\frac{\partial a}{\partial x_l} 
  -\frac{1}{(ab)^2}a\frac{\partial b}{\partial x_l}
\end{equation}

$\frac{\partial a}{\partial x_l}$ is exactly the same as we were using
with bonds:
\begin{displaymath}
  \frac{\partial a}{\partial x_l} = \frac{x_l-x_k}{a}
\end{displaymath}

Similarly:
\begin{displaymath}
  \frac{\partial b}{\partial x_l} = \frac{x_l-x_m}{a}
\end{displaymath}

So substituting those into (\ref{eq:7}):
\begin{displaymath}
  \frac{\partial R}{\partial x_l} = -\frac{x_l-x_k}{a^3b} -\frac{x_l-x_m}{ab^3}
\end{displaymath}

Turning to $Q$, recall (\ref{eq:5}), so: 
\begin{displaymath}
  Q =  
  ((x_k-x_l)(x_m-x_l) + (y_k-y_l)(y_m-y_l) + (z_k-z_l)(z_m-z_l))
\end{displaymath}

Therefore
\begin{displaymath}
   \frac{\partial Q}{\partial x_l} = -(x_k-x_l) -(x_m-x_l)
\end{displaymath}

Substituting all the above into (\ref{eq:4}):
\begin{displaymath}
  \frac{\partial P}{\partial x_l} = ({\underline a}.{\underline b})[-\frac{x_l-x_k}{a^3b} -\frac{x_l-x_m}{ab^3}] + \frac{-(x_k-x_l)-(x_m-x_l)}{ab}
\end{displaymath}

So, combining this and (\ref{eq:3}) into (\ref{eq:2}), we get: 
\begin{displaymath}
  \frac{\partial \theta}{\partial x_l} = -\frac{1}{\sin \theta}  \frac{\partial P}{\partial x_l} 
\end{displaymath}

%\begin{displaymath}
%  \frac{\partial \theta}{\partial x_l} = -\frac{1}{\sin \theta}(({\underline a}.{\underline b})[-\frac{x_l-x_k}{a^3b} -\frac{x_l-x_m}{ab^3}] + \frac{-(x_k-x_l)-(x_m-x_l)}{ab})
%\end{displaymath}






\subsection{An End Atom (Atoms $k$ or $m$)}
This is more simple because there are no cross terms in 
$\frac{\partial R}{\partial x_k}$ and $\frac{\partial Q}{\partial x_k}$.

\begin{displaymath}
  \frac{\partial R}{\partial x_k} = \frac{(x_k-x_l)}{ab}
\end{displaymath}

and 
\begin{displaymath}
  \frac{\partial Q}{\partial x_k} = (x_m-x_l)
\end{displaymath}

So 

\begin{equation}
  \frac{\partial \theta}{\partial x_k} = -\frac{1}{sin\theta} [\frac{(x_l-x_k)}{a^2}cos\theta + \frac{x_m-x_l}{ab}]
\end{equation}


% ------------------------------------------------------------------
%                  Torsions
% ------------------------------------------------------------------

\section{Torsions}
The torsion of 3 vectors (the vectors between one atom and the next in
the torsion angle) is given by:
\begin{equation}
  \label{eq:8}
  \tau(\mathbf{a},\mathbf{b},\mathbf{c}) = \arg(-\mathbf{a}.\mathbf{c}+(\mathbf{a}.\mathbf{b})(\mathbf{b}.\mathbf{c}), \mathbf{a}.(\mathbf{b} \mathbf{\times}\mathbf{c}))
\end{equation}


Let's split the expression up into tractable (for me) portions, the
evaluation of $\theta$ in the program will combine these expressions
starting at the end (the most simple).

\begin{figure}[htbp]
  \centering
  \leavevmode
  \epsfxsize=50mm
%  \epsffile{torsion.eps}
  \caption{Torsion vectors}
  \label{fig:torsion-vectors}
\end{figure}

Obviously: 
\begin{displaymath}
  a_x = P_{2_x}-P_{1_x} , b_x = P_{3_x}-P_{2_x} , c_x = P_{4_x}-P_{3_x}
\end{displaymath}
\begin{displaymath}
  a_y = P_{2_y}-P_{1_y} , b_y = P_{3_y}-P_{2_y} , c_y = P_{4_y}-P_{3_y}
\end{displaymath}
\begin{displaymath}
  a_z = P_{2_z}-P_{1_z} , b_z = P_{3_z}-P_{2_z} , c_z = P_{4_z}-P_{3_z}
\end{displaymath}

Unfortunately, I change the nomenclature because I derived the torsion
terms some time after the angle terms and I had forgotten what I had
previously been using.


\begin{displaymath}
  \theta = \tau(\mathbf{a},\mathbf{b},\mathbf{c}) =  \arctan(D)
\end{displaymath}

where
\begin{displaymath}
  D = \frac{\frac{\mathbf{a}.(\mathbf{b} \mathbf{\times}\mathbf{c})} {b}}{-\mathbf{a}.\mathbf{c}+\frac{(\mathbf{a}.\mathbf{b})(\mathbf{b}.\mathbf{c})}{b^2}}
\end{displaymath}

So

\begin{eqnarray}
  \label{eq:df}
  \frac{\partial \theta}{\partial x_{P_1}} & = & 
  \frac{\partial \theta}{\partial D} \frac{\partial D}{\partial x_{P_1}} \\
  & = & \frac{1}{1+D^2}\frac{\partial D}{\partial x_{P_1}}
\end{eqnarray}

Let
\begin{displaymath}
  E = \frac{\mathbf{a}.(\mathbf{b} \mathbf{\times}\mathbf{c})}{b}
\end{displaymath}
and 
\begin{displaymath}
  F = \frac{1}{-\mathbf{a}.\mathbf{c}+\frac{(\mathbf{a}.\mathbf{b})(\mathbf{b}.\mathbf{c})}{b}}
\end{displaymath}

\begin{equation}
  \label{eq:9}
  F = \frac{1}{G}
\end{equation}

Let
\begin{displaymath}
  G = -\mathbf{a}.\mathbf{c}+\frac{(\mathbf{a}.\mathbf{b})(\mathbf{b}.\mathbf{c})}{b^2}
\end{displaymath}

\begin{displaymath}
  H =  -\mathbf{a}.\mathbf{c}
\end{displaymath}

\begin{displaymath}
  J = \mathbf{a}.\mathbf{b}
\end{displaymath}

\begin{displaymath}
  K = \mathbf{b}.\mathbf{c}
\end{displaymath}

\begin{displaymath}
  L = \frac{1}{b^2}
\end{displaymath}

Differentiating  (\ref{eq:9})
\begin{displaymath}
  \frac{\partial F}{\partial x_{P_1}} = -\frac{1}{G^2}\frac{\partial G}{\partial x_{P_1}}
\end{displaymath}

%So now we have
%
%\begin{displaymath}
%  D = EF
%\end{displaymath}

Substituting for the derivative in (\ref{eq:df}):

\begin{displaymath}
  \frac{\partial \theta}{\partial x_{P_1}} = \frac{1}{1+D^2}[F\frac{\partial E}{\partial x_{P_1}} + E\frac{\partial F}{\partial x_{P_1}}]
\end{displaymath}


Also we have
\begin{displaymath}
  G = H + JKL
\end{displaymath}

Differentiating this: 

\begin{displaymath}
  \frac{\partial G}{\partial x_{P_1}} = \frac{\partial H}{\partial x_{P_1}} + JL\frac{\partial K}{\partial x_{P_1}} + KL\frac{\partial J}{\partial x_{P_1}} + JK\frac{\partial L}{\partial x_{P_1}}
\end{displaymath}

Let's look at the $H$, $J$, $K$ and $L$ derivatives:

\begin{displaymath}
    H = -\mathbf{a}.\mathbf{c} = -a_x c_x - a_y b_y - a_z c_z
\end{displaymath}

\begin{eqnarray*}
  \frac{\partial H}{\partial x_{P_1}} & = & c_x,\\
  \frac{\partial H}{\partial x_{P_2}} & = & -c_x,\\
  \frac{\partial H}{\partial x_{P_3}} & = & a_x,\\
  \frac{\partial H}{\partial x_{P_4}} & = & -a_x,\\
  \frac{\partial K}{\partial x_{P_1}} & = & 0,\\
  \frac{\partial K}{\partial x_{P_2}} & = & -c_x,\\
  \frac{\partial K}{\partial x_{P_3}} & = & c_x + b_x,\\
  \frac{\partial K}{\partial x_{P_4}} & = & b_x,\\
  \frac{\partial J}{\partial x_{P_1}} & = & -b_x,\\
  \frac{\partial J}{\partial x_{P_2}} & = & b_x - a_x,\\
  \frac{\partial J}{\partial x_{P_3}} & = & a_x,\\
  \frac{\partial J}{\partial x_{P_4}} & = & 0
\end{eqnarray*}

The $\frac{\partial b}{\partial x}$ terms are just like the bond
derivatives:

\begin{displaymath}
  \frac{\partial L}{\partial x_{P_1}} = \frac{\partial L}{\partial b} \frac{\partial b}{\partial x_{P_1}}
\end{displaymath}

\emph{i.e. }
\begin{eqnarray*}
  \frac{\partial L}{\partial x_{P_3}} & = &-\frac{2}{b^3} \frac{x_{P_3}-x_{P_2}}{b}\\
  & = &-\frac{2(x_{P_3}-x_{P_2})}{b^4}
\end{eqnarray*}

The derivative with respect to $x_{P_2}$ has the opposite sign.

Notice that $\mathbf{b}$ involves only atoms $P_2$ and $P_3$ so that
the derivates of $L$ with respect to the $P_1$ and $P_4$ coordinates are zero.

\subsection{$\frac{\partial E}{\partial x}$}
For the $\frac{\partial E}{\partial x}$ terms: 

Recall:
\begin{displaymath}
  E = \frac{\mathbf{a}.(\mathbf{b} \mathbf{\times}\mathbf{c})}{b}
\end{displaymath}

Let
\begin{displaymath}
  M = \mathbf{a}.(\mathbf{b} \mathbf{\times}\mathbf{c})
\end{displaymath}

\emph{i.e.}:
\begin{displaymath}
  E = \frac{M}{b}
\end{displaymath}

Differentiating that:
\begin{displaymath}
  \frac{\partial E}{\partial x_{P_3}} = -\frac{M}{b^2} \frac{\partial b}{\partial x_{P_3}} 
  +  \frac{1}{b} \frac{\partial M}{\partial x_{P_3}}
\end{displaymath}

Where, like bonds:
\begin{displaymath}
  \frac{\partial b}{\partial x_{P_3}} = \frac{x_{P_3}-x_{P_2}}{b}
\end{displaymath}
But note again, that the derivative of $b$ is zero for atoms $P_1$ and $P_4$.



\emph{i.e.} for atoms $P_2$ and $P_3$:
\begin{displaymath}
  \frac{\partial E}{\partial x_{P_3}} = -\frac{M(x_{P_3}-x_{P_2})}{b^3} + \frac{1}{b}\frac{\partial M}{\partial x_{P_3}}
\end{displaymath}

but for atoms $P_1$ and $P_4$:
\begin{displaymath}
  \frac{\partial E}{\partial x_{P_1}} =  \frac{1}{b}\frac{\partial M}{\partial x_{P_1}}
\end{displaymath}

\begin{displaymath}
  M = a_x(b_y c_z - b_z c_y) + a_y (b_z c_x - b_x c_z) + a_z (b_x c_y - b_y c_x)
\end{displaymath}

So here are the primitives of $M = \mathbf{a}.(\mathbf{b} \mathbf{\times}\mathbf{c})$

\begin{eqnarray*}
  \frac{\partial M}{\partial x_{P_1}} & = & -(b_y c_z - b_z c_y)\\
  \frac{\partial M}{\partial x_{P_2}} & = & (b_y c_z - b_z c_y) + (a_y c_z - a_z c_y)\\
  \frac{\partial M}{\partial x_{P_3}} & = & (a_z c_y - a_y c_z) + (b_y a_z - b_z a_y)\\
  \frac{\partial M}{\partial x_{P_4}} & = & (a_y b_z - a_z b_y)\\
  \frac{\partial M}{\partial y_{P_1}} & = & -(b_z c_x - b_x c_z)\\
  \frac{\partial M}{\partial y_{P_2}} & = & (b_z c_x - b_x c_z) + (a_z c_x - a_x c_z)\\
  \frac{\partial M}{\partial y_{P_3}} & = & -(a_z c_x - a_x c_z) + (b_z a_x - b_x a_z)\\
  \frac{\partial M}{\partial y_{P_4}} & = & -(b_z a_x - b_x a_z)\\
  \frac{\partial M}{\partial z_{P_1}} & = & -(b_x c_y - b_y c_x)\\
  \frac{\partial M}{\partial z_{P_2}} & = & (b_x c_y - b_y c_x) + (a_x c_y - a_y c_x)\\
  \frac{\partial M}{\partial z_{P_3}} & = & -(a_x c_y - a_y c_x) + (a_y b_x - a_x b_y)\\
  \frac{\partial M}{\partial z_{P_4}} & = & -(a_y b_x - a_x b_y)
\end{eqnarray*}

\subsection{Putting it together}

Combining, we get the following expression for the derivative of
$\theta$ in terms of the primitive derivates:
\begin{displaymath}
  \frac{\partial \theta}{\partial x_{P_1}} = \frac{1}{(1+\tan^2\theta)} \frac{\partial D}{\partial x_{P_1}}
\end{displaymath}

Where 
\begin{displaymath}
  \frac{\partial D}{\partial x_{P_1}} = [F \frac{\partial E}{\partial x_{P_1}} -\frac{E}{G^2} (\frac{\partial H}{\partial x_{P_1}} + JL \frac{\partial K}{\partial x_{P_1}} + KL  \frac{\partial J}{\partial x_{P_1}} + JK  \frac{\partial L}{\partial x_{P_1}})]  
\end{displaymath}









\chapter{Exported Functions}

@section File System Functions 
@subsection @code{(make-directory-maybe dir)}
@cindex @code{(make-directory-maybe dir)}
 
where: 
 @itemize 
     @item dir is a string
 @end itemize


@subsection @code{(set-show-paths-in-display-manager i)}
@cindex @code{(set-show-paths-in-display-manager i)}
 
where: 
 @itemize 
     @item i is an exact integer number
 @end itemize


@subsection @code{(show-paths-in-display-manager-state)}
@cindex @code{(show-paths-in-display-manager-state)}
 
@subsection @code{(add-coordinates-glob-extension ext)}
@cindex @code{(add-coordinates-glob-extension ext)}
 
where: 
 @itemize 
     @item ext is a string
 @end itemize


@subsection @code{(add-data-glob-extension ext)}
@cindex @code{(add-data-glob-extension ext)}
 
where: 
 @itemize 
     @item ext is a string
 @end itemize


@subsection @code{(add-dictionary-glob-extension ext)}
@cindex @code{(add-dictionary-glob-extension ext)}
 
where: 
 @itemize 
     @item ext is a string
 @end itemize


@subsection @code{(add-map-glob-extension ext)}
@cindex @code{(add-map-glob-extension ext)}
 
where: 
 @itemize 
     @item ext is a string
 @end itemize


@subsection @code{(set-sticky-sort-by-date)}
@cindex @code{(set-sticky-sort-by-date)}
 
@subsection @code{(set-filter-fileselection-filenames istate)}
@cindex @code{(set-filter-fileselection-filenames istate)}
 
where: 
 @itemize 
     @item istate is an exact integer number
 @end itemize


@subsection @code{(filter-fileselection-filenames-state)}
@cindex @code{(filter-fileselection-filenames-state)}
 

@section Widget Utilities 
@subsection @code{(info-dialog txt)}
@cindex @code{(info-dialog txt)}
 
where: 
 @itemize 
     @item txt is a string
 @end itemize



@section Widget Utilities 
@subsection @code{(manage-column-selector filename)}
@cindex @code{(manage-column-selector filename)}
 
where: 
 @itemize 
     @item filename is a string
 @end itemize



@section Molecule Info Functions 
@subsection @code{(chain-n-residues chain-id imol)}
@cindex @code{(chain-n-residues chain-id imol)}
 
where: 
 @itemize 
     @item chain-id is a string
     @item imol is an exact integer number
 @end itemize


@subsection @code{(molecule-centre-internal imol iaxis)}
@cindex @code{(molecule-centre-internal imol iaxis)}
 
where: 
 @itemize 
     @item imol is an exact integer number
     @item iaxis is an exact integer number
 @end itemize


@subsection @code{(n-chains imol)}
@cindex @code{(n-chains imol)}
 
where: 
 @itemize 
     @item imol is an exact integer number
 @end itemize


@subsection @code{(is-solvent-chain-p imol chain-id)}
@cindex @code{(is-solvent-chain-p imol chain-id)}
 
where: 
 @itemize 
     @item imol is an exact integer number
     @item chain-id is a string
 @end itemize


@subsection @code{(copy-molecule imol)}
@cindex @code{(copy-molecule imol)}
 
where: 
 @itemize 
     @item imol is an exact integer number
 @end itemize


@subsection @code{(exchange-chain-ids-for-seg-ids imol)}
@cindex @code{(exchange-chain-ids-for-seg-ids imol)}
 
where: 
 @itemize 
     @item imol is an exact integer number
 @end itemize



@section Library and Utility Functions 
@subsection @code{(coot-real-exit retval)}
@cindex @code{(coot-real-exit retval)}
 
where: 
 @itemize 
     @item retval is an exact integer number
 @end itemize


@subsection @code{(first-coords-imol)}
@cindex @code{(first-coords-imol)}
 

@section Graphics Utility Functions 
@subsection @code{(set-do-anti-aliasing state)}
@cindex @code{(set-do-anti-aliasing state)}
 
where: 
 @itemize 
     @item state is an exact integer number
 @end itemize


@subsection @code{(do-anti-aliasing-state)}
@cindex @code{(do-anti-aliasing-state)}
 
@subsection @code{(set-do-GL-lighting state)}
@cindex @code{(set-do-GL-lighting state)}
 
where: 
 @itemize 
     @item state is an exact integer number
 @end itemize


@subsection @code{(do-GL-lighting-state)}
@cindex @code{(do-GL-lighting-state)}
 
@subsection @code{(use-graphics-interface-state)}
@cindex @code{(use-graphics-interface-state)}
 
@subsection @code{(start-graphics-interface)}
@cindex @code{(start-graphics-interface)}
 
@subsection @code{(reset-view)}
@cindex @code{(reset-view)}
 
@subsection @code{(graphics-n-molecules)}
@cindex @code{(graphics-n-molecules)}
 
@subsection @code{(next-map-for-molecule imol)}
@cindex @code{(next-map-for-molecule imol)}
 
where: 
 @itemize 
     @item imol is an exact integer number
 @end itemize


@subsection @code{(toggle-idle-function)}
@cindex @code{(toggle-idle-function)}
 
@subsection @code{(set-idle-function-rotate-angle f)}
@cindex @code{(set-idle-function-rotate-angle f)}
 
where: 
 @itemize 
     @item f is an inexact number
 @end itemize


@subsection @code{(handle-read-draw-molecule filename)}
@cindex @code{(handle-read-draw-molecule filename)}
 
where: 
 @itemize 
     @item filename is a string
 @end itemize


@subsection @code{(read-pdb filename)}
@cindex @code{(read-pdb filename)}
 
where: 
 @itemize 
     @item filename is a string
 @end itemize


@subsection @code{(replace-fragment imol-target imol-fragment atom-selection)}
@cindex @code{(replace-fragment imol-target imol-fragment atom-selection)}
 
where: 
 @itemize 
     @item imol-target is an exact integer number
     @item imol-fragment is an exact integer number
     @item atom-selection is a string
 @end itemize


@subsection @code{(screendump-image filename)}
@cindex @code{(screendump-image filename)}
 
where: 
 @itemize 
     @item filename is a string
 @end itemize



@section Interface Preferences 
@subsection @code{(set-scroll-by-wheel-mouse istate)}
@cindex @code{(set-scroll-by-wheel-mouse istate)}
 
where: 
 @itemize 
     @item istate is an exact integer number
 @end itemize


@subsection @code{(scroll-by-wheel-mouse-state)}
@cindex @code{(scroll-by-wheel-mouse-state)}
 
@subsection @code{(set-default-initial-contour-level-for-map n-sigma)}
@cindex @code{(set-default-initial-contour-level-for-map n-sigma)}
 
where: 
 @itemize 
     @item n-sigma is an inexact number
 @end itemize


@subsection @code{(set-default-initial-contour-level-for-difference-map n-sigma)}
@cindex @code{(set-default-initial-contour-level-for-difference-map n-sigma)}
 
where: 
 @itemize 
     @item n-sigma is an inexact number
 @end itemize


@subsection @code{(print-view-matrix)}
@cindex @code{(print-view-matrix)}
 
@subsection @code{(get-view-matrix-element row col)}
@cindex @code{(get-view-matrix-element row col)}
 
where: 
 @itemize 
     @item row is an exact integer number
     @item col is an exact integer number
 @end itemize


@subsection @code{(get-view-quaternion-internal element)}
@cindex @code{(get-view-quaternion-internal element)}
 
where: 
 @itemize 
     @item element is an exact integer number
 @end itemize


@subsection @code{(set-view-quaternion i j k l)}
@cindex @code{(set-view-quaternion i j k l)}
 
where: 
 @itemize 
     @item i is an inexact number
     @item j is an inexact number
     @item k is an inexact number
     @item l is an inexact number
 @end itemize


@subsection @code{(set-fps-flag t)}
@cindex @code{(set-fps-flag t)}
 
where: 
 @itemize 
     @item t is an exact integer number
 @end itemize


@subsection @code{(get-fps-flag)}
@cindex @code{(get-fps-flag)}
 
@subsection @code{(set-show-origin-marker istate)}
@cindex @code{(set-show-origin-marker istate)}
 
where: 
 @itemize 
     @item istate is an exact integer number
 @end itemize


@subsection @code{(show-origin-marker-state)}
@cindex @code{(show-origin-marker-state)}
 
@subsection @code{(suck-model-fit-dialog)}
@cindex @code{(suck-model-fit-dialog)}
 
@subsection @code{(add-status-bar-text s)}
@cindex @code{(add-status-bar-text s)}
 
where: 
 @itemize 
     @item s is a string
 @end itemize


@subsection @code{(set-model-fit-refine-dialog-stays-on-top istate)}
@cindex @code{(set-model-fit-refine-dialog-stays-on-top istate)}
 
where: 
 @itemize 
     @item istate is an exact integer number
 @end itemize


@subsection @code{(model-fit-refine-dialog-stays-on-top-state)}
@cindex @code{(model-fit-refine-dialog-stays-on-top-state)}
 

@section Mouse Buttons 
@subsection @code{(quanta-buttons)}
@cindex @code{(quanta-buttons)}
 
@subsection @code{(quanta-like-zoom)}
@cindex @code{(quanta-like-zoom)}
 
@subsection @code{(set-control-key-for-rotate state)}
@cindex @code{(set-control-key-for-rotate state)}
 
where: 
 @itemize 
     @item state is an exact integer number
 @end itemize


@subsection @code{(control-key-for-rotate-state)}
@cindex @code{(control-key-for-rotate-state)}
 

@section Cursor Function 
@subsection @code{(normal-cursor)}
@cindex @code{(normal-cursor)}
 
@subsection @code{(fleur-cursor)}
@cindex @code{(fleur-cursor)}
 
@subsection @code{(pick-cursor-maybe)}
@cindex @code{(pick-cursor-maybe)}
 
@subsection @code{(rotate-cursor)}
@cindex @code{(rotate-cursor)}
 
@subsection @code{(set-pick-cursor-index icursor-index)}
@cindex @code{(set-pick-cursor-index icursor-index)}
 
where: 
 @itemize 
     @item icursor-index is an exact integer number
 @end itemize



@section Model/Fit/Refine Functions 
@subsection @code{(post-model-fit-refine-dialog)}
@cindex @code{(post-model-fit-refine-dialog)}
 
@subsection @code{(unset-model-fit-refine-dialog)}
@cindex @code{(unset-model-fit-refine-dialog)}
 
@subsection @code{(unset-refine-params-dialog)}
@cindex @code{(unset-refine-params-dialog)}
 
@subsection @code{(show-select-map-dialog)}
@cindex @code{(show-select-map-dialog)}
 
@subsection @code{(set-model-fit-refine-rotate-translate-zone-label txt)}
@cindex @code{(set-model-fit-refine-rotate-translate-zone-label txt)}
 
where: 
 @itemize 
     @item txt is a string
 @end itemize


@subsection @code{(set-model-fit-refine-place-atom-at-pointer-label txt)}
@cindex @code{(set-model-fit-refine-place-atom-at-pointer-label txt)}
 
where: 
 @itemize 
     @item txt is a string
 @end itemize


@subsection @code{(unset-other-modelling-tools-dialog)}
@cindex @code{(unset-other-modelling-tools-dialog)}
 
@subsection @code{(post-other-modelling-tools-dialog)}
@cindex @code{(post-other-modelling-tools-dialog)}
 

@section Backup Functions 
@subsection @code{(make-backup imol)}
@cindex @code{(make-backup imol)}
 
where: 
 @itemize 
     @item imol is an exact integer number
 @end itemize


@subsection @code{(turn-off-backup imol)}
@cindex @code{(turn-off-backup imol)}
 
where: 
 @itemize 
     @item imol is an exact integer number
 @end itemize


@subsection @code{(turn-on-backup imol)}
@cindex @code{(turn-on-backup imol)}
 
where: 
 @itemize 
     @item imol is an exact integer number
 @end itemize


@subsection @code{(backup-state imol)}
@cindex @code{(backup-state imol)}
 
where: 
 @itemize 
     @item imol is an exact integer number
 @end itemize


@subsection @code{(apply-undo)}
@cindex @code{(apply-undo)}
 
@subsection @code{(apply-redo)}
@cindex @code{(apply-redo)}
 
@subsection @code{(set-have-unsaved-changes imol)}
@cindex @code{(set-have-unsaved-changes imol)}
 
where: 
 @itemize 
     @item imol is an exact integer number
 @end itemize


@subsection @code{(set-undo-molecule imol)}
@cindex @code{(set-undo-molecule imol)}
 
where: 
 @itemize 
     @item imol is an exact integer number
 @end itemize


@subsection @code{(show-set-undo-molecule-chooser)}
@cindex @code{(show-set-undo-molecule-chooser)}
 
@subsection @code{(set-unpathed-backup-file-names state)}
@cindex @code{(set-unpathed-backup-file-names state)}
 
where: 
 @itemize 
     @item state is an exact integer number
 @end itemize


@subsection @code{(unpathed-backup-file-names-state)}
@cindex @code{(unpathed-backup-file-names-state)}
 

@section Recover Session Function 
@subsection @code{(recover-session)}
@cindex @code{(recover-session)}
 

@section Map Functions 
@subsection @code{(calc-phases-generic mtz-file-name)}
@cindex @code{(calc-phases-generic mtz-file-name)}
 
where: 
 @itemize 
     @item mtz-file-name is a string
 @end itemize


@subsection @code{(scroll-wheel-map)}
@cindex @code{(scroll-wheel-map)}
 
@subsection @code{(save-previous-map-colour imol)}
@cindex @code{(save-previous-map-colour imol)}
 
where: 
 @itemize 
     @item imol is an exact integer number
 @end itemize


@subsection @code{(restore-previous-map-colour imol)}
@cindex @code{(restore-previous-map-colour imol)}
 
where: 
 @itemize 
     @item imol is an exact integer number
 @end itemize


@subsection @code{(set-active-map-drag-flag t)}
@cindex @code{(set-active-map-drag-flag t)}
 
where: 
 @itemize 
     @item t is an exact integer number
 @end itemize


@subsection @code{(get-active-map-drag-flag)}
@cindex @code{(get-active-map-drag-flag)}
 
@subsection @code{(set-last-map-colour f1 f2 f3)}
@cindex @code{(set-last-map-colour f1 f2 f3)}
 
where: 
 @itemize 
     @item f1 is an unknown type
     @item f2 is an unknown type
     @item f3 is an unknown type
 @end itemize


@subsection @code{(set-map-colour imol red green blue)}
@cindex @code{(set-map-colour imol red green blue)}
 
where: 
 @itemize 
     @item imol is an exact integer number
     @item red is an inexact number
     @item green is an inexact number
     @item blue is an inexact number
 @end itemize


@subsection @code{( map-no gdouble[4])}
@cindex @code{( map-no gdouble[4])}
 
where: 
 @itemize 
     @item map-no is an exact integer number
     @item gdouble[4] is an unknown type
 @end itemize


@subsection @code{(handle-symmetry-colour-change mol gdouble[4])}
@cindex @code{(handle-symmetry-colour-change mol gdouble[4])}
 
where: 
 @itemize 
     @item mol is an exact integer number
     @item gdouble[4] is an unknown type
 @end itemize


@subsection @code{(set-last-map-sigma-step f)}
@cindex @code{(set-last-map-sigma-step f)}
 
where: 
 @itemize 
     @item f is an inexact number
 @end itemize


@subsection @code{(set-contour-by-sigma-step-by-mol f state imol)}
@cindex @code{(set-contour-by-sigma-step-by-mol f state imol)}
 
where: 
 @itemize 
     @item f is an inexact number
     @item state is an exact integer number
     @item imol is an exact integer number
 @end itemize


@subsection @code{(data-resolution imol)}
@cindex @code{(data-resolution imol)}
 
where: 
 @itemize 
     @item imol is an exact integer number
 @end itemize


@subsection @code{(export-map imol filename)}
@cindex @code{(export-map imol filename)}
 
where: 
 @itemize 
     @item imol is an exact integer number
     @item filename is a string
 @end itemize


@subsection @code{(rotate-map-round-screen-axis-x r-degrees)}
@cindex @code{(rotate-map-round-screen-axis-x r-degrees)}
 
where: 
 @itemize 
     @item r-degrees is an inexact number
 @end itemize


@subsection @code{(rotate-map-round-screen-axis-y r-degrees)}
@cindex @code{(rotate-map-round-screen-axis-y r-degrees)}
 
where: 
 @itemize 
     @item r-degrees is an inexact number
 @end itemize


@subsection @code{(rotate-map-round-screen-axis-z r-degrees)}
@cindex @code{(rotate-map-round-screen-axis-z r-degrees)}
 
where: 
 @itemize 
     @item r-degrees is an inexact number
 @end itemize



@section Density Increment 
@subsection @code{(set-iso-level-increment val)}
@cindex @code{(set-iso-level-increment val)}
 
where: 
 @itemize 
     @item val is an inexact number
 @end itemize


@subsection @code{(set-iso-level-increment-from-text text imol)}
@cindex @code{(set-iso-level-increment-from-text text imol)}
 
where: 
 @itemize 
     @item text is a string
     @item imol is an exact integer number
 @end itemize


@subsection @code{(set-diff-map-iso-level-increment val)}
@cindex @code{(set-diff-map-iso-level-increment val)}
 
where: 
 @itemize 
     @item val is an inexact number
 @end itemize


@subsection @code{(set-diff-map-iso-level-increment-from-text text imol)}
@cindex @code{(set-diff-map-iso-level-increment-from-text text imol)}
 
where: 
 @itemize 
     @item text is a string
     @item imol is an exact integer number
 @end itemize


@subsection @code{(set-map-sampling-rate-text text)}
@cindex @code{(set-map-sampling-rate-text text)}
 
where: 
 @itemize 
     @item text is a string
 @end itemize


@subsection @code{(set-map-sampling-rate r)}
@cindex @code{(set-map-sampling-rate r)}
 
where: 
 @itemize 
     @item r is an inexact number
 @end itemize


@subsection @code{(get-map-sampling-rate)}
@cindex @code{(get-map-sampling-rate)}
 
@subsection @code{(set-scrollable-map imol)}
@cindex @code{(set-scrollable-map imol)}
 
where: 
 @itemize 
     @item imol is an exact integer number
 @end itemize


@subsection @code{(change-contour-level is-increment)}
@cindex @code{(change-contour-level is-increment)}
 
where: 
 @itemize 
     @item is-increment is an exact integer number
 @end itemize


@subsection @code{(set-last-map-contour-level level)}
@cindex @code{(set-last-map-contour-level level)}
 
where: 
 @itemize 
     @item level is an inexact number
 @end itemize


@subsection @code{(set-last-map-contour-level-by-sigma n-sigma)}
@cindex @code{(set-last-map-contour-level-by-sigma n-sigma)}
 
where: 
 @itemize 
     @item n-sigma is an inexact number
 @end itemize


@subsection @code{(set-stop-scroll-diff-map i)}
@cindex @code{(set-stop-scroll-diff-map i)}
 
where: 
 @itemize 
     @item i is an exact integer number
 @end itemize


@subsection @code{(set-stop-scroll-iso-map i)}
@cindex @code{(set-stop-scroll-iso-map i)}
 
where: 
 @itemize 
     @item i is an exact integer number
 @end itemize


@subsection @code{(set-stop-scroll-iso-map-level f)}
@cindex @code{(set-stop-scroll-iso-map-level f)}
 
where: 
 @itemize 
     @item f is an inexact number
 @end itemize


@subsection @code{(set-stop-scroll-diff-map-level f)}
@cindex @code{(set-stop-scroll-diff-map-level f)}
 
where: 
 @itemize 
     @item f is an inexact number
 @end itemize


@subsection @code{(set-residue-density-fit-scale-factor f)}
@cindex @code{(set-residue-density-fit-scale-factor f)}
 
where: 
 @itemize 
     @item f is an inexact number
 @end itemize



@section Density Functions 
@subsection @code{(set-map-line-width w)}
@cindex @code{(set-map-line-width w)}
 
where: 
 @itemize 
     @item w is an exact integer number
 @end itemize


@subsection @code{(map-line-width-state)}
@cindex @code{(map-line-width-state)}
 
@subsection @code{(mtz-file-has-phases-p mtz-file-name)}
@cindex @code{(mtz-file-has-phases-p mtz-file-name)}
 
where: 
 @itemize 
     @item mtz-file-name is a string
 @end itemize


@subsection @code{(is-mtz-file-p filename)}
@cindex @code{(is-mtz-file-p filename)}
 
where: 
 @itemize 
     @item filename is a string
 @end itemize


@subsection @code{(auto-read-make-and-draw-maps filename)}
@cindex @code{(auto-read-make-and-draw-maps filename)}
 
where: 
 @itemize 
     @item filename is a string
 @end itemize


@subsection @code{(set-auto-read-do-difference-map-too i)}
@cindex @code{(set-auto-read-do-difference-map-too i)}
 
where: 
 @itemize 
     @item i is an exact integer number
 @end itemize


@subsection @code{(auto-read-do-difference-map-too-state)}
@cindex @code{(auto-read-do-difference-map-too-state)}
 
@subsection @code{(set-density-size-from-widget text)}
@cindex @code{(set-density-size-from-widget text)}
 
where: 
 @itemize 
     @item text is a string
 @end itemize


@subsection @code{(set-map-radius f)}
@cindex @code{(set-map-radius f)}
 
where: 
 @itemize 
     @item f is an inexact number
 @end itemize


@subsection @code{(set-density-size f)}
@cindex @code{(set-density-size f)}
 
where: 
 @itemize 
     @item f is an inexact number
 @end itemize


@subsection @code{(set-map-radius-slider-max f)}
@cindex @code{(set-map-radius-slider-max f)}
 
where: 
 @itemize 
     @item f is an inexact number
 @end itemize


@subsection @code{(set-display-intro-string str)}
@cindex @code{(set-display-intro-string str)}
 
where: 
 @itemize 
     @item str is a string
 @end itemize


@subsection @code{(set-esoteric-depth-cue istate)}
@cindex @code{(set-esoteric-depth-cue istate)}
 
where: 
 @itemize 
     @item istate is an exact integer number
 @end itemize


@subsection @code{(esoteric-depth-cue-state)}
@cindex @code{(esoteric-depth-cue-state)}
 
@subsection @code{(set-swap-difference-map-colours i)}
@cindex @code{(set-swap-difference-map-colours i)}
 
where: 
 @itemize 
     @item i is an exact integer number
 @end itemize


@subsection @code{(set-map-is-difference-map imol)}
@cindex @code{(set-map-is-difference-map imol)}
 
where: 
 @itemize 
     @item imol is an exact integer number
 @end itemize


@subsection @code{(another-level)}
@cindex @code{(another-level)}
 
@subsection @code{(another-level-from-map-molecule-number imap)}
@cindex @code{(another-level-from-map-molecule-number imap)}
 
where: 
 @itemize 
     @item imap is an exact integer number
 @end itemize


@subsection @code{(residue-density-fit-scale-factor)}
@cindex @code{(residue-density-fit-scale-factor)}
 

@section Parameters from map 
@subsection @code{(mtz-use-weight-for-map imol-map)}
@cindex @code{(mtz-use-weight-for-map imol-map)}
 
where: 
 @itemize 
     @item imol-map is an exact integer number
 @end itemize



@section PDB Functions 
@subsection @code{(write-pdb-file imol file-name)}
@cindex @code{(write-pdb-file imol file-name)}
 
where: 
 @itemize 
     @item imol is an exact integer number
     @item file-name is a string
 @end itemize



@section Refmac Functions 
@subsection @code{(set-refmac-molecule imol)}
@cindex @code{(set-refmac-molecule imol)}
 
where: 
 @itemize 
     @item imol is an exact integer number
 @end itemize


@subsection @code{(set-refmac-counter imol refmac-count)}
@cindex @code{(set-refmac-counter imol refmac-count)}
 
where: 
 @itemize 
     @item imol is an exact integer number
     @item refmac-count is an exact integer number
 @end itemize


@subsection @code{(swap-map-colours imol1 imol2)}
@cindex @code{(swap-map-colours imol1 imol2)}
 
where: 
 @itemize 
     @item imol1 is an exact integer number
     @item imol2 is an exact integer number
 @end itemize


@subsection @code{(set-keep-map-colour-after-refmac istate)}
@cindex @code{(set-keep-map-colour-after-refmac istate)}
 
where: 
 @itemize 
     @item istate is an exact integer number
 @end itemize


@subsection @code{(keep-map-colour-after-refmac-state)}
@cindex @code{(keep-map-colour-after-refmac-state)}
 

@section Symmetry Functions 
@subsection @code{(set-symmetry-size-from-widget text)}
@cindex @code{(set-symmetry-size-from-widget text)}
 
where: 
 @itemize 
     @item text is a string
 @end itemize


@subsection @code{(set-symmetry-size f)}
@cindex @code{(set-symmetry-size f)}
 
where: 
 @itemize 
     @item f is an inexact number
 @end itemize


@subsection @code{(get-show-symmetry)}
@cindex @code{(get-show-symmetry)}
 
@subsection @code{(set-show-symmetry-master state)}
@cindex @code{(set-show-symmetry-master state)}
 
where: 
 @itemize 
     @item state is an exact integer number
 @end itemize


@subsection @code{(set-show-symmetry-molecule mol-no state)}
@cindex @code{(set-show-symmetry-molecule mol-no state)}
 
where: 
 @itemize 
     @item mol-no is an exact integer number
     @item state is an exact integer number
 @end itemize


@subsection @code{(symmetry-as-calphas mol-no state)}
@cindex @code{(symmetry-as-calphas mol-no state)}
 
where: 
 @itemize 
     @item mol-no is an exact integer number
     @item state is an exact integer number
 @end itemize


@subsection @code{(get-symmetry-as-calphas-state imol)}
@cindex @code{(get-symmetry-as-calphas-state imol)}
 
where: 
 @itemize 
     @item imol is an exact integer number
 @end itemize


@subsection @code{(set-symmetry-molecule-rotate-colour-map imol state)}
@cindex @code{(set-symmetry-molecule-rotate-colour-map imol state)}
 
where: 
 @itemize 
     @item imol is an exact integer number
     @item state is an exact integer number
 @end itemize


@subsection @code{(symmetry-molecule-rotate-colour-map-state imol)}
@cindex @code{(symmetry-molecule-rotate-colour-map-state imol)}
 
where: 
 @itemize 
     @item imol is an exact integer number
 @end itemize


@subsection @code{(set-symmetry-colour-by-symop imol state)}
@cindex @code{(set-symmetry-colour-by-symop imol state)}
 
where: 
 @itemize 
     @item imol is an exact integer number
     @item state is an exact integer number
 @end itemize


@subsection @code{(set-symmetry-whole-chain imol state)}
@cindex @code{(set-symmetry-whole-chain imol state)}
 
where: 
 @itemize 
     @item imol is an exact integer number
     @item state is an exact integer number
 @end itemize


@subsection @code{(set-symmetry-atom-labels-expanded state)}
@cindex @code{(set-symmetry-atom-labels-expanded state)}
 
where: 
 @itemize 
     @item state is an exact integer number
 @end itemize


@subsection @code{(has-unit-cell-state imol)}
@cindex @code{(has-unit-cell-state imol)}
 
where: 
 @itemize 
     @item imol is an exact integer number
 @end itemize


@subsection @code{(setup-save-symmetry-coords)}
@cindex @code{(setup-save-symmetry-coords)}
 
@subsection @code{(set-space-group imol spg)}
@cindex @code{(set-space-group imol spg)}
 
where: 
 @itemize 
     @item imol is an exact integer number
     @item spg is a string
 @end itemize


@subsection @code{(set-symmetry-shift-search-size shift)}
@cindex @code{(set-symmetry-shift-search-size shift)}
 
where: 
 @itemize 
     @item shift is an exact integer number
 @end itemize



@section File Selection Functions 
@subsection @code{(clear-refmac-ccp4i-project)}
@cindex @code{(clear-refmac-ccp4i-project)}
 

@section History Functions 
@subsection @code{(print-all-history-in-scheme)}
@cindex @code{(print-all-history-in-scheme)}
 
@subsection @code{(print-all-history-in-python)}
@cindex @code{(print-all-history-in-python)}
 
@subsection @code{(set-console-display-commands-state istate)}
@cindex @code{(set-console-display-commands-state istate)}
 
where: 
 @itemize 
     @item istate is an exact integer number
 @end itemize


@subsection @code{(save-state)}
@cindex @code{(save-state)}
 
@subsection @code{(save-state-file filename)}
@cindex @code{(save-state-file filename)}
 
where: 
 @itemize 
     @item filename is a string
 @end itemize


@subsection @code{(set-save-state-file-name filename)}
@cindex @code{(set-save-state-file-name filename)}
 
where: 
 @itemize 
     @item filename is a string
 @end itemize


@subsection @code{(set-run-state-file-status istat)}
@cindex @code{(set-run-state-file-status istat)}
 
where: 
 @itemize 
     @item istat is an exact integer number
 @end itemize


@subsection @code{(run-state-file)}
@cindex @code{(run-state-file)}
 
@subsection @code{(run-state-file-maybe)}
@cindex @code{(run-state-file-maybe)}
 
@subsection @code{(vt-surface mode)}
@cindex @code{(vt-surface mode)}
 
where: 
 @itemize 
     @item mode is an exact integer number
 @end itemize


@subsection @code{(vt-surface-status)}
@cindex @code{(vt-surface-status)}
 

@section Clipping Functions 
@subsection @code{(do-clipping1-activate)}
@cindex @code{(do-clipping1-activate)}
 
@subsection @code{(set-clipping-back v)}
@cindex @code{(set-clipping-back v)}
 
where: 
 @itemize 
     @item v is an inexact number
 @end itemize


@subsection @code{(set-clipping-front v)}
@cindex @code{(set-clipping-front v)}
 
where: 
 @itemize 
     @item v is an inexact number
 @end itemize



@section Unit Cell 
@subsection @code{(get-show-unit-cell imol)}
@cindex @code{(get-show-unit-cell imol)}
 
where: 
 @itemize 
     @item imol is an exact integer number
 @end itemize


@subsection @code{(set-show-unit-cells-all istate)}
@cindex @code{(set-show-unit-cells-all istate)}
 
where: 
 @itemize 
     @item istate is an exact integer number
 @end itemize


@subsection @code{(set-show-unit-cell imol istate)}
@cindex @code{(set-show-unit-cell imol istate)}
 
where: 
 @itemize 
     @item imol is an exact integer number
     @item istate is an exact integer number
 @end itemize



@section Colour 
@subsection @code{(set-symmetry-colour-merge mol-no v)}
@cindex @code{(set-symmetry-colour-merge mol-no v)}
 
where: 
 @itemize 
     @item mol-no is an exact integer number
     @item v is an inexact number
 @end itemize


@subsection @code{(set-colour-map-rotation-on-read-pdb f)}
@cindex @code{(set-colour-map-rotation-on-read-pdb f)}
 
where: 
 @itemize 
     @item f is an inexact number
 @end itemize


@subsection @code{(set-colour-map-rotation-on-read-pdb-flag i)}
@cindex @code{(set-colour-map-rotation-on-read-pdb-flag i)}
 
where: 
 @itemize 
     @item i is an exact integer number
 @end itemize


@subsection @code{(set-colour-map-rotation-on-read-pdb-c-only-flag i)}
@cindex @code{(set-colour-map-rotation-on-read-pdb-c-only-flag i)}
 
where: 
 @itemize 
     @item i is an exact integer number
 @end itemize


@subsection @code{(set-colour-by-chain imol)}
@cindex @code{(set-colour-by-chain imol)}
 
where: 
 @itemize 
     @item imol is an exact integer number
 @end itemize


@subsection @code{(set-colour-by-molecule imol)}
@cindex @code{(set-colour-by-molecule imol)}
 
where: 
 @itemize 
     @item imol is an exact integer number
 @end itemize


@subsection @code{(set-colour-map-rotation-for-map f)}
@cindex @code{(set-colour-map-rotation-for-map f)}
 
where: 
 @itemize 
     @item f is an inexact number
 @end itemize


@subsection @code{(set-molecule-bonds-colour-map-rotation imol theta)}
@cindex @code{(set-molecule-bonds-colour-map-rotation imol theta)}
 
where: 
 @itemize 
     @item imol is an exact integer number
     @item theta is an inexact number
 @end itemize


@subsection @code{(get-molecule-bonds-colour-map-rotation imol)}
@cindex @code{(get-molecule-bonds-colour-map-rotation imol)}
 
where: 
 @itemize 
     @item imol is an exact integer number
 @end itemize



@section Anisotropic Atoms 
@subsection @code{(get-limit-aniso)}
@cindex @code{(get-limit-aniso)}
 
@subsection @code{(get-show-limit-aniso)}
@cindex @code{(get-show-limit-aniso)}
 
@subsection @code{(get-show-aniso)}
@cindex @code{(get-show-aniso)}
 
@subsection @code{(set-limit-aniso state)}
@cindex @code{(set-limit-aniso state)}
 
where: 
 @itemize 
     @item state is an exact integer number
 @end itemize


@subsection @code{(set-aniso-limit-size-from-widget text)}
@cindex @code{(set-aniso-limit-size-from-widget text)}
 
where: 
 @itemize 
     @item text is a string
 @end itemize


@subsection @code{(set-show-aniso state)}
@cindex @code{(set-show-aniso state)}
 
where: 
 @itemize 
     @item state is an exact integer number
 @end itemize


@subsection @code{(set-aniso-probability f)}
@cindex @code{(set-aniso-probability f)}
 
where: 
 @itemize 
     @item f is an inexact number
 @end itemize


@subsection @code{(get-aniso-probability)}
@cindex @code{(get-aniso-probability)}
 

@section Display Functions 
@subsection @code{(set-graphics-window-size x-size y-size)}
@cindex @code{(set-graphics-window-size x-size y-size)}
 
where: 
 @itemize 
     @item x-size is an exact integer number
     @item y-size is an exact integer number
 @end itemize


@subsection @code{(set-graphics-window-position x-pos y-pos)}
@cindex @code{(set-graphics-window-position x-pos y-pos)}
 
where: 
 @itemize 
     @item x-pos is an exact integer number
     @item y-pos is an exact integer number
 @end itemize


@subsection @code{(store-graphics-window-position x-pos y-pos)}
@cindex @code{(store-graphics-window-position x-pos y-pos)}
 
where: 
 @itemize 
     @item x-pos is an exact integer number
     @item y-pos is an exact integer number
 @end itemize


@subsection @code{(graphics-draw)}
@cindex @code{(graphics-draw)}
 
@subsection @code{(hardware-stereo-mode)}
@cindex @code{(hardware-stereo-mode)}
 
@subsection @code{(stereo-mode-state)}
@cindex @code{(stereo-mode-state)}
 
@subsection @code{(mono-mode)}
@cindex @code{(mono-mode)}
 
@subsection @code{(side-by-side-stereo-mode use-wall-eye-mode)}
@cindex @code{(side-by-side-stereo-mode use-wall-eye-mode)}
 
where: 
 @itemize 
     @item use-wall-eye-mode is an exact integer number
 @end itemize


@subsection @code{(set-hardware-stereo-angle-factor f)}
@cindex @code{(set-hardware-stereo-angle-factor f)}
 
where: 
 @itemize 
     @item f is an inexact number
 @end itemize


@subsection @code{(hardware-stereo-angle-factor-state)}
@cindex @code{(hardware-stereo-angle-factor-state)}
 
@subsection @code{(set-model-fit-refine-dialog-position x-pos y-pos)}
@cindex @code{(set-model-fit-refine-dialog-position x-pos y-pos)}
 
where: 
 @itemize 
     @item x-pos is an exact integer number
     @item y-pos is an exact integer number
 @end itemize


@subsection @code{(set-display-control-dialog-position x-pos y-pos)}
@cindex @code{(set-display-control-dialog-position x-pos y-pos)}
 
where: 
 @itemize 
     @item x-pos is an exact integer number
     @item y-pos is an exact integer number
 @end itemize


@subsection @code{(set-go-to-atom-window-position x-pos y-pos)}
@cindex @code{(set-go-to-atom-window-position x-pos y-pos)}
 
where: 
 @itemize 
     @item x-pos is an exact integer number
     @item y-pos is an exact integer number
 @end itemize


@subsection @code{(set-delete-dialog-position x-pos y-pos)}
@cindex @code{(set-delete-dialog-position x-pos y-pos)}
 
where: 
 @itemize 
     @item x-pos is an exact integer number
     @item y-pos is an exact integer number
 @end itemize


@subsection @code{(set-rotate-translate-dialog-position x-pos y-pos)}
@cindex @code{(set-rotate-translate-dialog-position x-pos y-pos)}
 
where: 
 @itemize 
     @item x-pos is an exact integer number
     @item y-pos is an exact integer number
 @end itemize


@subsection @code{(set-accept-reject-dialog-position x-pos y-pos)}
@cindex @code{(set-accept-reject-dialog-position x-pos y-pos)}
 
where: 
 @itemize 
     @item x-pos is an exact integer number
     @item y-pos is an exact integer number
 @end itemize


@subsection @code{(set-ramachandran-plot-dialog-position x-pos y-pos)}
@cindex @code{(set-ramachandran-plot-dialog-position x-pos y-pos)}
 
where: 
 @itemize 
     @item x-pos is an exact integer number
     @item y-pos is an exact integer number
 @end itemize



@section Smooth Scrolling 
@subsection @code{(set-smooth-scroll-flag v)}
@cindex @code{(set-smooth-scroll-flag v)}
 
where: 
 @itemize 
     @item v is an exact integer number
 @end itemize


@subsection @code{(get-smooth-scroll)}
@cindex @code{(get-smooth-scroll)}
 
@subsection @code{(set-smooth-scroll-steps-str t)}
@cindex @code{(set-smooth-scroll-steps-str t)}
 
where: 
 @itemize 
     @item t is a string
 @end itemize


@subsection @code{(set-smooth-scroll-steps i)}
@cindex @code{(set-smooth-scroll-steps i)}
 
where: 
 @itemize 
     @item i is an exact integer number
 @end itemize


@subsection @code{(set-smooth-scroll-limit-str t)}
@cindex @code{(set-smooth-scroll-limit-str t)}
 
where: 
 @itemize 
     @item t is a string
 @end itemize


@subsection @code{(set-smooth-scroll-limit lim)}
@cindex @code{(set-smooth-scroll-limit lim)}
 
where: 
 @itemize 
     @item lim is an inexact number
 @end itemize



@section Font Size 
@subsection @code{(set-font-size i)}
@cindex @code{(set-font-size i)}
 
where: 
 @itemize 
     @item i is an exact integer number
 @end itemize


@subsection @code{(get-font-size)}
@cindex @code{(get-font-size)}
 

@section Rotation Centre 
@subsection @code{(set-rotation-centre-size-from-widget text)}
@cindex @code{(set-rotation-centre-size-from-widget text)}
 
where: 
 @itemize 
     @item text is an unknown type
 @end itemize


@subsection @code{(set-rotation-centre-size f)}
@cindex @code{(set-rotation-centre-size f)}
 
where: 
 @itemize 
     @item f is an inexact number
 @end itemize


@subsection @code{(recentre-on-read-pdb)}
@cindex @code{(recentre-on-read-pdb)}
 
@subsection @code{(set-recentre-on-read-pdb int)}
@cindex @code{(set-recentre-on-read-pdb int)}
 
where: 
 @itemize 
     @item int is an exact integer number
 @end itemize


@subsection @code{(set-rotation-centre x y z)}
@cindex @code{(set-rotation-centre x y z)}
 
where: 
 @itemize 
     @item x is an inexact number
     @item y is an inexact number
     @item z is an inexact number
 @end itemize


@subsection @code{(set-rotation-centre-internal x y z)}
@cindex @code{(set-rotation-centre-internal x y z)}
 
where: 
 @itemize 
     @item x is an inexact number
     @item y is an inexact number
     @item z is an inexact number
 @end itemize


@subsection @code{(rotation-centre-position axis)}
@cindex @code{(rotation-centre-position axis)}
 
where: 
 @itemize 
     @item axis is an exact integer number
 @end itemize



@section Orthogonal Axes 
@subsection @code{(set-draw-axes i)}
@cindex @code{(set-draw-axes i)}
 
where: 
 @itemize 
     @item i is an exact integer number
 @end itemize



@section Atom Selection Utilities 
@subsection @code{(atom-index imol chain-id iresno atom-id)}
@cindex @code{(atom-index imol chain-id iresno atom-id)}
 
where: 
 @itemize 
     @item imol is an exact integer number
     @item chain-id is a string
     @item iresno is an exact integer number
     @item atom-id is a string
 @end itemize


@subsection @code{(median-temperature-factor imol)}
@cindex @code{(median-temperature-factor imol)}
 
where: 
 @itemize 
     @item imol is an exact integer number
 @end itemize


@subsection @code{(average-temperature-factor imol)}
@cindex @code{(average-temperature-factor imol)}
 
where: 
 @itemize 
     @item imol is an exact integer number
 @end itemize


@subsection @code{(clear-pending-picks)}
@cindex @code{(clear-pending-picks)}
 
@subsection @code{(set-default-temperature-factor-for-new-atoms new-b)}
@cindex @code{(set-default-temperature-factor-for-new-atoms new-b)}
 
where: 
 @itemize 
     @item new-b is an inexact number
 @end itemize


@subsection @code{(default-new-atoms-b-factor)}
@cindex @code{(default-new-atoms-b-factor)}
 

@section Skeletonization 
@subsection @code{(skel-greer-on)}
@cindex @code{(skel-greer-on)}
 
@subsection @code{(skel-greer-off)}
@cindex @code{(skel-greer-off)}
 
@subsection @code{(skel-foadi-on)}
@cindex @code{(skel-foadi-on)}
 
@subsection @code{(skel-foadi-off)}
@cindex @code{(skel-foadi-off)}
 
@subsection @code{(skeletonize-map prune-flag imol)}
@cindex @code{(skeletonize-map prune-flag imol)}
 
where: 
 @itemize 
     @item prune-flag is an exact integer number
     @item imol is an exact integer number
 @end itemize


@subsection @code{(unskeletonize-map imol)}
@cindex @code{(unskeletonize-map imol)}
 
where: 
 @itemize 
     @item imol is an exact integer number
 @end itemize


@subsection @code{(set-initial-map-for-skeletonize)}
@cindex @code{(set-initial-map-for-skeletonize)}
 
@subsection @code{(set-max-skeleton-search-depth v)}
@cindex @code{(set-max-skeleton-search-depth v)}
 
where: 
 @itemize 
     @item v is an exact integer number
 @end itemize


@subsection @code{(set-skeletonization-level-from-widget txt)}
@cindex @code{(set-skeletonization-level-from-widget txt)}
 
where: 
 @itemize 
     @item txt is a string
 @end itemize


@subsection @code{(set-skeleton-box-size-from-widget txt)}
@cindex @code{(set-skeleton-box-size-from-widget txt)}
 
where: 
 @itemize 
     @item txt is a string
 @end itemize


@subsection @code{(set-skeleton-box-size f)}
@cindex @code{(set-skeleton-box-size f)}
 
where: 
 @itemize 
     @item f is an inexact number
 @end itemize



@section Skeleton Colour 
@subsection @code{(handle-skeleton-colour-change mol map-col)}
@cindex @code{(handle-skeleton-colour-change mol map-col)}
 
where: 
 @itemize 
     @item mol is an exact integer number
     @item map-col is an unknown type
 @end itemize


@subsection @code{(set-skeleton-colour imol r g b)}
@cindex @code{(set-skeleton-colour imol r g b)}
 
where: 
 @itemize 
     @item imol is an exact integer number
     @item r is an inexact number
     @item g is an inexact number
     @item b is an inexact number
 @end itemize



@section Read CCP4 Map 
@subsection @code{(handle-read-ccp4-map filename is-diff-map-flag)}
@cindex @code{(handle-read-ccp4-map filename is-diff-map-flag)}
 
where: 
 @itemize 
     @item filename is an unknown type
     @item is-diff-map-flag is an exact integer number
 @end itemize



@section Save Coordinates 
@subsection @code{(save-coordinates imol filename)}
@cindex @code{(save-coordinates imol filename)}
 
where: 
 @itemize 
     @item imol is an exact integer number
     @item filename is a string
 @end itemize


@subsection @code{(set-save-coordinates-in-original-directory i)}
@cindex @code{(set-save-coordinates-in-original-directory i)}
 
where: 
 @itemize 
     @item i is an exact integer number
 @end itemize


@subsection @code{(save-molecule-number-from-option-menu)}
@cindex @code{(save-molecule-number-from-option-menu)}
 
@subsection @code{(set-save-molecule-number imol)}
@cindex @code{(set-save-molecule-number imol)}
 
where: 
 @itemize 
     @item imol is an exact integer number
 @end itemize



@section Read Phases File Functions 
@subsection @code{(possible-cell-symm-for-phs-file)}
@cindex @code{(possible-cell-symm-for-phs-file)}
 

@section Graphics Move 
@subsection @code{(undo-last-move)}
@cindex @code{(undo-last-move)}
 
@subsection @code{(translate-molecule-by imol x y z)}
@cindex @code{(translate-molecule-by imol x y z)}
 
where: 
 @itemize 
     @item imol is an exact integer number
     @item x is an inexact number
     @item y is an inexact number
     @item z is an inexact number
 @end itemize



@section Go To Atom Widget Functions 
@subsection @code{(post-go-to-atom-window)}
@cindex @code{(post-go-to-atom-window)}
 
@subsection @code{(atom-spec-to-atom-index mol chain resno atom-name)}
@cindex @code{(atom-spec-to-atom-index mol chain resno atom-name)}
 
where: 
 @itemize 
     @item mol is an exact integer number
     @item chain is a string
     @item resno is an exact integer number
     @item atom-name is a string
 @end itemize


@subsection @code{(update-go-to-atom-window-on-changed-mol imol)}
@cindex @code{(update-go-to-atom-window-on-changed-mol imol)}
 
where: 
 @itemize 
     @item imol is an exact integer number
 @end itemize


@subsection @code{(update-go-to-atom-window-on-new-mol)}
@cindex @code{(update-go-to-atom-window-on-new-mol)}
 
@subsection @code{(set-go-to-atom-molecule imol)}
@cindex @code{(set-go-to-atom-molecule imol)}
 
where: 
 @itemize 
     @item imol is an exact integer number
 @end itemize


@subsection @code{(unset-go-to-atom-widget)}
@cindex @code{(unset-go-to-atom-widget)}
 

@section AutoBuilding functions (Defunct) 
@subsection @code{(autobuild-ca-on)}
@cindex @code{(autobuild-ca-on)}
 
@subsection @code{(autobuild-ca-off)}
@cindex @code{(autobuild-ca-off)}
 
@subsection @code{(test-fragment)}
@cindex @code{(test-fragment)}
 
@subsection @code{(do-skeleton-prune)}
@cindex @code{(do-skeleton-prune)}
 
@subsection @code{(test-function i j)}
@cindex @code{(test-function i j)}
 
where: 
 @itemize 
     @item i is an exact integer number
     @item j is an exact integer number
 @end itemize



@section Map and Molecule Control 
@subsection @code{(post-display-control-window)}
@cindex @code{(post-display-control-window)}
 
@subsection @code{(add-map-display-control-widgets)}
@cindex @code{(add-map-display-control-widgets)}
 
@subsection @code{(add-mol-display-control-widgets)}
@cindex @code{(add-mol-display-control-widgets)}
 
@subsection @code{(add-map-and-mol-display-control-widgets)}
@cindex @code{(add-map-and-mol-display-control-widgets)}
 
@subsection @code{(reset-graphics-display-control-window)}
@cindex @code{(reset-graphics-display-control-window)}
 
@subsection @code{(toggle-display-map imol imap)}
@cindex @code{(toggle-display-map imol imap)}
 
where: 
 @itemize 
     @item imol is an exact integer number
     @item imap is an exact integer number
 @end itemize


@subsection @code{(toggle-display-mol imol)}
@cindex @code{(toggle-display-mol imol)}
 
where: 
 @itemize 
     @item imol is an exact integer number
 @end itemize


@subsection @code{( imol)}
@cindex @code{( imol)}
 
where: 
 @itemize 
     @item imol is an exact integer number
 @end itemize


@subsection @code{(mol-is-displayed imol)}
@cindex @code{(mol-is-displayed imol)}
 
where: 
 @itemize 
     @item imol is an exact integer number
 @end itemize


@subsection @code{(mol-is-active imol)}
@cindex @code{(mol-is-active imol)}
 
where: 
 @itemize 
     @item imol is an exact integer number
 @end itemize


@subsection @code{(map-is-displayed imol)}
@cindex @code{(map-is-displayed imol)}
 
where: 
 @itemize 
     @item imol is an exact integer number
 @end itemize



@section Merge Molecules 
@subsection @code{(do-merge-molecules-gui)}
@cindex @code{(do-merge-molecules-gui)}
 

@section Mutate Sequence and Loops GUI 

@section Align and Mutate 
@subsection @code{(align-and-mutate imol chain-id fasta-maybe)}
@cindex @code{(align-and-mutate imol chain-id fasta-maybe)}
 
where: 
 @itemize 
     @item imol is an exact integer number
     @item chain-id is a string
     @item fasta-maybe is a string
 @end itemize



@section Renumber Residue Range 
@subsection @code{(change-residue-number imol chain-id current-resno current-inscode new-resno new-inscode)}
@cindex @code{(change-residue-number imol chain-id current-resno current-inscode new-resno new-inscode)}
 
where: 
 @itemize 
     @item imol is an exact integer number
     @item chain-id is a string
     @item current-resno is an exact integer number
     @item current-inscode is a string
     @item new-resno is an exact integer number
     @item new-inscode is a string
 @end itemize



@section Change Chain ID 

@section Scripting 
@subsection @code{(post-scripting-window)}
@cindex @code{(post-scripting-window)}
 
@subsection @code{(run-command-line-scripts)}
@cindex @code{(run-command-line-scripts)}
 
@subsection @code{(set-guile-gui-loaded-flag)}
@cindex @code{(set-guile-gui-loaded-flag)}
 
@subsection @code{(set-found-coot-gui)}
@cindex @code{(set-found-coot-gui)}
 
@subsection @code{(get-monomer three-letter-code)}
@cindex @code{(get-monomer three-letter-code)}
 
where: 
 @itemize 
     @item three-letter-code is a string
 @end itemize


@subsection @code{( filename)}
@cindex @code{( filename)}
 
where: 
 @itemize 
     @item filename is a string
 @end itemize


@subsection @code{( filename)}
@cindex @code{( filename)}
 
where: 
 @itemize 
     @item filename is a string
 @end itemize


@subsection @code{(run-python-script filename)}
@cindex @code{(run-python-script filename)}
 
where: 
 @itemize 
     @item filename is a string
 @end itemize



@section Regularization and Refinement 
@subsection @code{(do-regularize state)}
@cindex @code{(do-regularize state)}
 
where: 
 @itemize 
     @item state is an exact integer number
 @end itemize


@subsection @code{(do-refine state)}
@cindex @code{(do-refine state)}
 
where: 
 @itemize 
     @item state is an exact integer number
 @end itemize


@subsection @code{(add-planar-peptide-restraints)}
@cindex @code{(add-planar-peptide-restraints)}
 
@subsection @code{(remove-planar-peptide-restraints)}
@cindex @code{(remove-planar-peptide-restraints)}
 
@subsection @code{(add-omega-torsion-restriants)}
@cindex @code{(add-omega-torsion-restriants)}
 
@subsection @code{(remove-omega-torsion-restriants)}
@cindex @code{(remove-omega-torsion-restriants)}
 
@subsection @code{(set-refinement-immediate-replacement istate)}
@cindex @code{(set-refinement-immediate-replacement istate)}
 
where: 
 @itemize 
     @item istate is an exact integer number
 @end itemize


@subsection @code{(refinement-immediate-replacement-state)}
@cindex @code{(refinement-immediate-replacement-state)}
 
@subsection @code{(set-residue-selection-flash-frames-number i)}
@cindex @code{(set-residue-selection-flash-frames-number i)}
 
where: 
 @itemize 
     @item i is an exact integer number
 @end itemize


@subsection @code{(accept-regularizement)}
@cindex @code{(accept-regularizement)}
 
@subsection @code{(clear-up-moving-atoms)}
@cindex @code{(clear-up-moving-atoms)}
 
@subsection @code{(clear-moving-atoms-object)}
@cindex @code{(clear-moving-atoms-object)}
 
@subsection @code{(do-peptide-torsions-toggle)}
@cindex @code{(do-peptide-torsions-toggle)}
 
@subsection @code{(set-refine-with-torsion-restraints istate)}
@cindex @code{(set-refine-with-torsion-restraints istate)}
 
where: 
 @itemize 
     @item istate is an exact integer number
 @end itemize


@subsection @code{(set-refine-params-phi-psi-restraints-type restraints-type)}
@cindex @code{(set-refine-params-phi-psi-restraints-type restraints-type)}
 
where: 
 @itemize 
     @item restraints-type is an exact integer number
 @end itemize


@subsection @code{(set-matrix f)}
@cindex @code{(set-matrix f)}
 
where: 
 @itemize 
     @item f is an inexact number
 @end itemize


@subsection @code{(matrix-state)}
@cindex @code{(matrix-state)}
 
@subsection @code{(set-refine-auto-range-step i)}
@cindex @code{(set-refine-auto-range-step i)}
 
where: 
 @itemize 
     @item i is an exact integer number
 @end itemize


@subsection @code{(set-refine-max-residues n)}
@cindex @code{(set-refine-max-residues n)}
 
where: 
 @itemize 
     @item n is an exact integer number
 @end itemize


@subsection @code{(refine-zone-atom-index-define imol ind1 ind2)}
@cindex @code{(refine-zone-atom-index-define imol ind1 ind2)}
 
where: 
 @itemize 
     @item imol is an exact integer number
     @item ind1 is an exact integer number
     @item ind2 is an exact integer number
 @end itemize


@subsection @code{(refine-zone imol chain-id resno1 resno2 altconf)}
@cindex @code{(refine-zone imol chain-id resno1 resno2 altconf)}
 
where: 
 @itemize 
     @item imol is an exact integer number
     @item chain-id is a string
     @item resno1 is an exact integer number
     @item resno2 is an exact integer number
     @item altconf is a string
 @end itemize


@subsection @code{(refine-auto-range imol chain-id resno1 altconf)}
@cindex @code{(refine-auto-range imol chain-id resno1 altconf)}
 
where: 
 @itemize 
     @item imol is an exact integer number
     @item chain-id is a string
     @item resno1 is an exact integer number
     @item altconf is a string
 @end itemize


@subsection @code{(set-dragged-refinement-steps-per-frame v)}
@cindex @code{(set-dragged-refinement-steps-per-frame v)}
 
where: 
 @itemize 
     @item v is an exact integer number
 @end itemize


@subsection @code{(dragged-refinement-steps-per-frame)}
@cindex @code{(dragged-refinement-steps-per-frame)}
 
@subsection @code{(set-refinement-refine-per-frame istate)}
@cindex @code{(set-refinement-refine-per-frame istate)}
 
where: 
 @itemize 
     @item istate is an exact integer number
 @end itemize


@subsection @code{(refinement-refine-per-frame-state)}
@cindex @code{(refinement-refine-per-frame-state)}
 
@subsection @code{(set-fix-chiral-volumes-before-refinement istate)}
@cindex @code{(set-fix-chiral-volumes-before-refinement istate)}
 
where: 
 @itemize 
     @item istate is an exact integer number
 @end itemize


@subsection @code{(check-chiral-volumes imol)}
@cindex @code{(check-chiral-volumes imol)}
 
where: 
 @itemize 
     @item imol is an exact integer number
 @end itemize


@subsection @code{(set-secondary-structure-restraints-type itype)}
@cindex @code{(set-secondary-structure-restraints-type itype)}
 
where: 
 @itemize 
     @item itype is an exact integer number
 @end itemize


@subsection @code{(secondary-structure-restraints-type)}
@cindex @code{(secondary-structure-restraints-type)}
 
@subsection @code{(imol-refinement-map)}
@cindex @code{(imol-refinement-map)}
 
@subsection @code{(set-imol-refinement-map imol)}
@cindex @code{(set-imol-refinement-map imol)}
 
where: 
 @itemize 
     @item imol is an exact integer number
 @end itemize


@subsection @code{(does-residue-exist-p imol chain-id resno inscode)}
@cindex @code{(does-residue-exist-p imol chain-id resno inscode)}
 
where: 
 @itemize 
     @item imol is an exact integer number
     @item chain-id is a string
     @item resno is an exact integer number
     @item inscode is a string
 @end itemize


@subsection @code{(fix-nomenclature-errors imol)}
@cindex @code{(fix-nomenclature-errors imol)}
 
where: 
 @itemize 
     @item imol is an exact integer number
 @end itemize



@section Atom Info 

@section Residue Info 
@subsection @code{(do-residue-info)}
@cindex @code{(do-residue-info)}
 
@subsection @code{( atom-index imol)}
@cindex @code{( atom-index imol)}
 
where: 
 @itemize 
     @item atom-index is an exact integer number
     @item imol is an exact integer number
 @end itemize


@subsection @code{(output-residue-info-as-text atom-index imol)}
@cindex @code{(output-residue-info-as-text atom-index imol)}
 
where: 
 @itemize 
     @item atom-index is an exact integer number
     @item imol is an exact integer number
 @end itemize


@subsection @code{(do-distance-define)}
@cindex @code{(do-distance-define)}
 
@subsection @code{(do-angle-define)}
@cindex @code{(do-angle-define)}
 
@subsection @code{(do-torsion-define)}
@cindex @code{(do-torsion-define)}
 
@subsection @code{(residue-info-apply-all-checkbutton-toggled)}
@cindex @code{(residue-info-apply-all-checkbutton-toggled)}
 
@subsection @code{(clear-residue-info-edit-list)}
@cindex @code{(clear-residue-info-edit-list)}
 
@subsection @code{(unset-residue-info-widget)}
@cindex @code{(unset-residue-info-widget)}
 
@subsection @code{(clear-simple-distances)}
@cindex @code{(clear-simple-distances)}
 
@subsection @code{(clear-last-simple-distance)}
@cindex @code{(clear-last-simple-distance)}
 

@section Residue Environment Functions 

@section Pointer Functions 
@subsection @code{(set-show-pointer-distances istate)}
@cindex @code{(set-show-pointer-distances istate)}
 
where: 
 @itemize 
     @item istate is an exact integer number
 @end itemize



@section Zoom Functions 
@subsection @code{(scale-zoom f)}
@cindex @code{(scale-zoom f)}
 
where: 
 @itemize 
     @item f is an inexact number
 @end itemize


@subsection @code{(scale-zoom-internal f)}
@cindex @code{(scale-zoom-internal f)}
 
where: 
 @itemize 
     @item f is an inexact number
 @end itemize


@subsection @code{(zoom-factor)}
@cindex @code{(zoom-factor)}
 
@subsection @code{(set-smooth-scroll-do-zoom i)}
@cindex @code{(set-smooth-scroll-do-zoom i)}
 
where: 
 @itemize 
     @item i is an exact integer number
 @end itemize


@subsection @code{(smooth-scroll-do-zoom)}
@cindex @code{(smooth-scroll-do-zoom)}
 
@subsection @code{(smooth-scroll-zoom-limit)}
@cindex @code{(smooth-scroll-zoom-limit)}
 
@subsection @code{(set-smooth-scroll-zoom-limit f)}
@cindex @code{(set-smooth-scroll-zoom-limit f)}
 
where: 
 @itemize 
     @item f is an inexact number
 @end itemize


@subsection @code{(handle-cns-data-file filename)}
@cindex @code{(handle-cns-data-file filename)}
 
where: 
 @itemize 
     @item filename is a string
 @end itemize



@section mmCIF Functions 
@subsection @code{(auto-read-cif-data-with-phases filename)}
@cindex @code{(auto-read-cif-data-with-phases filename)}
 
where: 
 @itemize 
     @item filename is a string
 @end itemize


@subsection @code{(read-cif-data-with-phases-sigmaa filename)}
@cindex @code{(read-cif-data-with-phases-sigmaa filename)}
 
where: 
 @itemize 
     @item filename is a string
 @end itemize


@subsection @code{(read-cif-data-with-phases-diff-sigmaa filename)}
@cindex @code{(read-cif-data-with-phases-diff-sigmaa filename)}
 
where: 
 @itemize 
     @item filename is a string
 @end itemize


@subsection @code{(read-cif-data filename imol-coords)}
@cindex @code{(read-cif-data filename imol-coords)}
 
where: 
 @itemize 
     @item filename is a string
     @item imol-coords is an exact integer number
 @end itemize


@subsection @code{(read-cif-data-2fofc-map filename imol-coords)}
@cindex @code{(read-cif-data-2fofc-map filename imol-coords)}
 
where: 
 @itemize 
     @item filename is a string
     @item imol-coords is an exact integer number
 @end itemize


@subsection @code{(read-cif-data-fofc-map filename imol-coords)}
@cindex @code{(read-cif-data-fofc-map filename imol-coords)}
 
where: 
 @itemize 
     @item filename is a string
     @item imol-coords is an exact integer number
 @end itemize


@subsection @code{(read-cif-data-with-phases-fo-fc filename)}
@cindex @code{(read-cif-data-with-phases-fo-fc filename)}
 
where: 
 @itemize 
     @item filename is a string
 @end itemize


@subsection @code{(read-cif-data-with-phases-2fo-fc filename)}
@cindex @code{(read-cif-data-with-phases-2fo-fc filename)}
 
where: 
 @itemize 
     @item filename is a string
 @end itemize


@subsection @code{(read-cif-data-with-phases-fo-alpha-calc filename)}
@cindex @code{(read-cif-data-with-phases-fo-alpha-calc filename)}
 
where: 
 @itemize 
     @item filename is a string
 @end itemize


@subsection @code{(handle-cif-dictionary filename)}
@cindex @code{(handle-cif-dictionary filename)}
 
where: 
 @itemize 
     @item filename is a string
 @end itemize


@subsection @code{(read-cif-dictionary filename)}
@cindex @code{(read-cif-dictionary filename)}
 
where: 
 @itemize 
     @item filename is a string
 @end itemize


@subsection @code{(write-connectivity monomer-name filename)}
@cindex @code{(write-connectivity monomer-name filename)}
 
where: 
 @itemize 
     @item monomer-name is an unknown type
     @item filename is a string
 @end itemize


@subsection @code{(import-all-refmac-cifs)}
@cindex @code{(import-all-refmac-cifs)}
 

@section SHELXL Functions 
@subsection @code{(read-shelx-ins-file filename)}
@cindex @code{(read-shelx-ins-file filename)}
 
where: 
 @itemize 
     @item filename is a string
 @end itemize


@subsection @code{(write-shelx-ins-file imol filename)}
@cindex @code{(write-shelx-ins-file imol filename)}
 
where: 
 @itemize 
     @item imol is an exact integer number
     @item filename is a string
 @end itemize


@subsection @code{(handle-shelx-fcf-file-internal filename)}
@cindex @code{(handle-shelx-fcf-file-internal filename)}
 
where: 
 @itemize 
     @item filename is a string
 @end itemize



@section Validation Functions 
@subsection @code{(deviant-geometry imol)}
@cindex @code{(deviant-geometry imol)}
 
where: 
 @itemize 
     @item imol is an exact integer number
 @end itemize


@subsection @code{(is-valid-model-molecule imol)}
@cindex @code{(is-valid-model-molecule imol)}
 
where: 
 @itemize 
     @item imol is an exact integer number
 @end itemize


@subsection @code{(is-valid-map-molecule imol)}
@cindex @code{(is-valid-map-molecule imol)}
 
where: 
 @itemize 
     @item imol is an exact integer number
 @end itemize


@subsection @code{(difference-map-peaks imol imol-coords level do-positive-level-flag do-negative-level-flag)}
@cindex @code{(difference-map-peaks imol imol-coords level do-positive-level-flag do-negative-level-flag)}
 
where: 
 @itemize 
     @item imol is an exact integer number
     @item imol-coords is an exact integer number
     @item level is an inexact number
     @item do-positive-level-flag is an exact integer number
     @item do-negative-level-flag is an exact integer number
 @end itemize


@subsection @code{(clear-diff-map-peaks)}
@cindex @code{(clear-diff-map-peaks)}
 
@subsection @code{(gln-asn-b-factor-outliers imol)}
@cindex @code{(gln-asn-b-factor-outliers imol)}
 
where: 
 @itemize 
     @item imol is an exact integer number
 @end itemize



@section Ramachandran Plot Functions 
@subsection @code{(do-ramachandran-plot imol)}
@cindex @code{(do-ramachandran-plot imol)}
 
where: 
 @itemize 
     @item imol is an exact integer number
 @end itemize


@subsection @code{(add-on-rama-choices)}
@cindex @code{(add-on-rama-choices)}
 
@subsection @code{(set-moving-atoms phi psi)}
@cindex @code{(set-moving-atoms phi psi)}
 
where: 
 @itemize 
     @item phi is an unknown type
     @item psi is an unknown type
 @end itemize


@subsection @code{(accept-phi-psi-moving-atoms)}
@cindex @code{(accept-phi-psi-moving-atoms)}
 
@subsection @code{(setup-edit-phi-psi state)}
@cindex @code{(setup-edit-phi-psi state)}
 
where: 
 @itemize 
     @item state is an exact integer number
 @end itemize


@subsection @code{(destroy-edit-backbone-rama-plot)}
@cindex @code{(destroy-edit-backbone-rama-plot)}
 
@subsection @code{(ramachandran-plot-differences imol1 imol2)}
@cindex @code{(ramachandran-plot-differences imol1 imol2)}
 
where: 
 @itemize 
     @item imol1 is an exact integer number
     @item imol2 is an exact integer number
 @end itemize


@subsection @code{(do-sequence-view imol)}
@cindex @code{(do-sequence-view imol)}
 
where: 
 @itemize 
     @item imol is an exact integer number
 @end itemize


@subsection @code{(add-on-sequence-view-choices)}
@cindex @code{(add-on-sequence-view-choices)}
 
@subsection @code{(change-peptide-carbonyl-by angle)}
@cindex @code{(change-peptide-carbonyl-by angle)}
 
where: 
 @itemize 
     @item angle is an unknown type
 @end itemize


@subsection @code{(change-peptide-peptide-by angle)}
@cindex @code{(change-peptide-peptide-by angle)}
 
where: 
 @itemize 
     @item angle is an unknown type
 @end itemize


@subsection @code{(execute-setup-backbone-torsion-edit imol atom-index)}
@cindex @code{(execute-setup-backbone-torsion-edit imol atom-index)}
 
where: 
 @itemize 
     @item imol is an exact integer number
     @item atom-index is an exact integer number
 @end itemize


@subsection @code{(setup-backbone-torsion-edit state)}
@cindex @code{(setup-backbone-torsion-edit state)}
 
where: 
 @itemize 
     @item state is an exact integer number
 @end itemize


@subsection @code{(set-backbone-torsion-peptide-button-start-pos ix iy)}
@cindex @code{(set-backbone-torsion-peptide-button-start-pos ix iy)}
 
where: 
 @itemize 
     @item ix is an exact integer number
     @item iy is an exact integer number
 @end itemize


@subsection @code{(change-peptide-peptide-by-current-button-pos ix iy)}
@cindex @code{(change-peptide-peptide-by-current-button-pos ix iy)}
 
where: 
 @itemize 
     @item ix is an exact integer number
     @item iy is an exact integer number
 @end itemize


@subsection @code{(set-backbone-torsion-carbonyl-button-start-pos ix iy)}
@cindex @code{(set-backbone-torsion-carbonyl-button-start-pos ix iy)}
 
where: 
 @itemize 
     @item ix is an exact integer number
     @item iy is an exact integer number
 @end itemize


@subsection @code{(change-peptide-carbonyl-by-current-button-pos ix iy)}
@cindex @code{(change-peptide-carbonyl-by-current-button-pos ix iy)}
 
where: 
 @itemize 
     @item ix is an exact integer number
     @item iy is an exact integer number
 @end itemize



@section Atom Labelling 
@subsection @code{(add-atom-label imol chain-id iresno atom-id)}
@cindex @code{(add-atom-label imol chain-id iresno atom-id)}
 
where: 
 @itemize 
     @item imol is an exact integer number
     @item chain-id is a string
     @item iresno is an exact integer number
     @item atom-id is a string
 @end itemize


@subsection @code{(remove-atom-label imol chain-id iresno atom-id)}
@cindex @code{(remove-atom-label imol chain-id iresno atom-id)}
 
where: 
 @itemize 
     @item imol is an exact integer number
     @item chain-id is a string
     @item iresno is an exact integer number
     @item atom-id is a string
 @end itemize


@subsection @code{(remove-all-atom-labels)}
@cindex @code{(remove-all-atom-labels)}
 
@subsection @code{(set-label-on-recentre-flag i)}
@cindex @code{(set-label-on-recentre-flag i)}
 
where: 
 @itemize 
     @item i is an exact integer number
 @end itemize


@subsection @code{(centre-atom-label-status)}
@cindex @code{(centre-atom-label-status)}
 
@subsection @code{(set-brief-atom-labels istat)}
@cindex @code{(set-brief-atom-labels istat)}
 
where: 
 @itemize 
     @item istat is an exact integer number
 @end itemize


@subsection @code{(brief-atom-labels-state)}
@cindex @code{(brief-atom-labels-state)}
 

@section Screen Rotation 
@subsection @code{(rotate-y-scene nsteps stepsize)}
@cindex @code{(rotate-y-scene nsteps stepsize)}
 
where: 
 @itemize 
     @item nsteps is an exact integer number
     @item stepsize is an inexact number
 @end itemize


@subsection @code{(rotate-x-scene nsteps stepsize)}
@cindex @code{(rotate-x-scene nsteps stepsize)}
 
where: 
 @itemize 
     @item nsteps is an exact integer number
     @item stepsize is an inexact number
 @end itemize


@subsection @code{(rotate-z-scene nsteps stepsize)}
@cindex @code{(rotate-z-scene nsteps stepsize)}
 
where: 
 @itemize 
     @item nsteps is an exact integer number
     @item stepsize is an inexact number
 @end itemize



@section Background Colour 
@subsection @code{(set-background-colour red green blue)}
@cindex @code{(set-background-colour red green blue)}
 
where: 
 @itemize 
     @item red is an unknown type
     @item green is an unknown type
     @item blue is an unknown type
 @end itemize


@subsection @code{(background-is-black-p)}
@cindex @code{(background-is-black-p)}
 

@section Ligand Fitting Functions 
@subsection @code{(set-ligand-acceptable-fit-fraction f)}
@cindex @code{(set-ligand-acceptable-fit-fraction f)}
 
where: 
 @itemize 
     @item f is an inexact number
 @end itemize


@subsection @code{(set-ligand-cluster-sigma-level f)}
@cindex @code{(set-ligand-cluster-sigma-level f)}
 
where: 
 @itemize 
     @item f is an inexact number
 @end itemize


@subsection @code{(set-ligand-flexible-ligand-n-samples i)}
@cindex @code{(set-ligand-flexible-ligand-n-samples i)}
 
where: 
 @itemize 
     @item i is an exact integer number
 @end itemize


@subsection @code{(set-ligand-verbose-reporting i)}
@cindex @code{(set-ligand-verbose-reporting i)}
 
where: 
 @itemize 
     @item i is an exact integer number
 @end itemize


@subsection @code{(set-find-ligand-n-top-ligands n)}
@cindex @code{(set-find-ligand-n-top-ligands n)}
 
where: 
 @itemize 
     @item n is an exact integer number
 @end itemize


@subsection @code{(set-find-ligand-mask-waters istate)}
@cindex @code{(set-find-ligand-mask-waters istate)}
 
where: 
 @itemize 
     @item istate is an exact integer number
 @end itemize


@subsection @code{(set-ligand-search-protein-molecule imol)}
@cindex @code{(set-ligand-search-protein-molecule imol)}
 
where: 
 @itemize 
     @item imol is an exact integer number
 @end itemize


@subsection @code{(set-ligand-search-map-molecule imol-map)}
@cindex @code{(set-ligand-search-map-molecule imol-map)}
 
where: 
 @itemize 
     @item imol-map is an exact integer number
 @end itemize


@subsection @code{(add-ligand-search-ligand-molecule imol-ligand)}
@cindex @code{(add-ligand-search-ligand-molecule imol-ligand)}
 
where: 
 @itemize 
     @item imol-ligand is an exact integer number
 @end itemize


@subsection @code{(add-ligand-search-wiggly-ligand-molecule imol-ligand)}
@cindex @code{(add-ligand-search-wiggly-ligand-molecule imol-ligand)}
 
where: 
 @itemize 
     @item imol-ligand is an exact integer number
 @end itemize


@subsection @code{(execute-ligand-search)}
@cindex @code{(execute-ligand-search)}
 
@subsection @code{(ligand-expert)}
@cindex @code{(ligand-expert)}
 
@subsection @code{(do-find-ligands-dialog)}
@cindex @code{(do-find-ligands-dialog)}
 

@section Water Fitting Functions 
@subsection @code{(renumber-waters imol)}
@cindex @code{(renumber-waters imol)}
 
where: 
 @itemize 
     @item imol is an exact integer number
 @end itemize


@subsection @code{(set-value-for-find-waters-sigma-cut-off f)}
@cindex @code{(set-value-for-find-waters-sigma-cut-off f)}
 
where: 
 @itemize 
     @item f is an inexact number
 @end itemize


@subsection @code{(set-ligand-water-spherical-variance-limit f)}
@cindex @code{(set-ligand-water-spherical-variance-limit f)}
 
where: 
 @itemize 
     @item f is an inexact number
 @end itemize


@subsection @code{(set-ligand-water-to-protein-distance-limits f1 f2)}
@cindex @code{(set-ligand-water-to-protein-distance-limits f1 f2)}
 
where: 
 @itemize 
     @item f1 is an inexact number
     @item f2 is an inexact number
 @end itemize


@subsection @code{(set-ligand-water-n-cycles i)}
@cindex @code{(set-ligand-water-n-cycles i)}
 
where: 
 @itemize 
     @item i is an exact integer number
 @end itemize


@subsection @code{(set-write-peaksearched-waters)}
@cindex @code{(set-write-peaksearched-waters)}
 
@subsection @code{(execute-find-blobs imol-model imol-for-map cut-off interactive-flag)}
@cindex @code{(execute-find-blobs imol-model imol-for-map cut-off interactive-flag)}
 
where: 
 @itemize 
     @item imol-model is an exact integer number
     @item imol-for-map is an exact integer number
     @item cut-off is an inexact number
     @item interactive-flag is an exact integer number
 @end itemize



@section Bond Representation 
@subsection @code{(set-default-bond-thickness t)}
@cindex @code{(set-default-bond-thickness t)}
 
where: 
 @itemize 
     @item t is an exact integer number
 @end itemize


@subsection @code{(set-bond-thickness imol t)}
@cindex @code{(set-bond-thickness imol t)}
 
where: 
 @itemize 
     @item imol is an exact integer number
     @item t is an inexact number
 @end itemize


@subsection @code{(set-bond-thickness-intermediate-atoms t)}
@cindex @code{(set-bond-thickness-intermediate-atoms t)}
 
where: 
 @itemize 
     @item t is an inexact number
 @end itemize


@subsection @code{(set-unbonded-atom-star-size f)}
@cindex @code{(set-unbonded-atom-star-size f)}
 
where: 
 @itemize 
     @item f is an inexact number
 @end itemize


@subsection @code{(set-draw-zero-occ-markers status)}
@cindex @code{(set-draw-zero-occ-markers status)}
 
where: 
 @itemize 
     @item status is an exact integer number
 @end itemize


@subsection @code{(set-draw-hydrogens imol istat)}
@cindex @code{(set-draw-hydrogens imol istat)}
 
where: 
 @itemize 
     @item imol is an exact integer number
     @item istat is an exact integer number
 @end itemize


@subsection @code{( imol)}
@cindex @code{( imol)}
 
where: 
 @itemize 
     @item imol is an exact integer number
 @end itemize


@subsection @code{( imol)}
@cindex @code{( imol)}
 
where: 
 @itemize 
     @item imol is an exact integer number
 @end itemize


@subsection @code{(graphics-to-bonds-no-waters-representation imol)}
@cindex @code{(graphics-to-bonds-no-waters-representation imol)}
 
where: 
 @itemize 
     @item imol is an exact integer number
 @end itemize


@subsection @code{(graphics-to-bonds-representation mol)}
@cindex @code{(graphics-to-bonds-representation mol)}
 
where: 
 @itemize 
     @item mol is an exact integer number
 @end itemize


@subsection @code{(graphics-to-ca-plus-ligands-sec-struct-representation imol)}
@cindex @code{(graphics-to-ca-plus-ligands-sec-struct-representation imol)}
 
where: 
 @itemize 
     @item imol is an exact integer number
 @end itemize


@subsection @code{(graphics-to-sec-struct-bonds-representation imol)}
@cindex @code{(graphics-to-sec-struct-bonds-representation imol)}
 
where: 
 @itemize 
     @item imol is an exact integer number
 @end itemize


@subsection @code{(graphics-to-rainbow-representation imol)}
@cindex @code{(graphics-to-rainbow-representation imol)}
 
where: 
 @itemize 
     @item imol is an exact integer number
 @end itemize


@subsection @code{(graphics-to-b-factor-representation imol)}
@cindex @code{(graphics-to-b-factor-representation imol)}
 
where: 
 @itemize 
     @item imol is an exact integer number
 @end itemize


@subsection @code{(graphics-to-occupancy-represenation imol)}
@cindex @code{(graphics-to-occupancy-represenation imol)}
 
where: 
 @itemize 
     @item imol is an exact integer number
 @end itemize


@subsection @code{(graphics-molecule-bond-type imol)}
@cindex @code{(graphics-molecule-bond-type imol)}
 
where: 
 @itemize 
     @item imol is an exact integer number
 @end itemize


@subsection @code{(clear-ball-and-stick imol)}
@cindex @code{(clear-ball-and-stick imol)}
 
where: 
 @itemize 
     @item imol is an exact integer number
 @end itemize


@subsection @code{(clear-dots imol dots-handle)}
@cindex @code{(clear-dots imol dots-handle)}
 
where: 
 @itemize 
     @item imol is an exact integer number
     @item dots-handle is an exact integer number
 @end itemize


@subsection @code{(n-dots-sets imol)}
@cindex @code{(n-dots-sets imol)}
 
where: 
 @itemize 
     @item imol is an exact integer number
 @end itemize



@section Pep-flip 
@subsection @code{(do-pepflip state)}
@cindex @code{(do-pepflip state)}
 
where: 
 @itemize 
     @item state is an exact integer number
 @end itemize


@subsection @code{(pepflip ires chain-id imol)}
@cindex @code{(pepflip ires chain-id imol)}
 
where: 
 @itemize 
     @item ires is an exact integer number
     @item chain-id is a string
     @item imol is an exact integer number
 @end itemize



@section Rigid Body Refinement 
@subsection @code{(do-rigid-body-refine state)}
@cindex @code{(do-rigid-body-refine state)}
 
where: 
 @itemize 
     @item state is an exact integer number
 @end itemize


@subsection @code{(execute-rigid-body-refine auto-range-flag)}
@cindex @code{(execute-rigid-body-refine auto-range-flag)}
 
where: 
 @itemize 
     @item auto-range-flag is an exact integer number
 @end itemize


@subsection @code{(set-rigid-body-fit-acceptable-fit-fraction f)}
@cindex @code{(set-rigid-body-fit-acceptable-fit-fraction f)}
 
where: 
 @itemize 
     @item f is an inexact number
 @end itemize



@section Dynamic Map 
@subsection @code{(toggle-dynamic-map-display-size)}
@cindex @code{(toggle-dynamic-map-display-size)}
 
@subsection @code{(toggle-dynamic-map-sampling)}
@cindex @code{(toggle-dynamic-map-sampling)}
 
@subsection @code{(set-dynamic-map-size-display-on)}
@cindex @code{(set-dynamic-map-size-display-on)}
 
@subsection @code{(set-dynamic-map-size-display-off)}
@cindex @code{(set-dynamic-map-size-display-off)}
 
@subsection @code{(set-dynamic-map-sampling-on)}
@cindex @code{(set-dynamic-map-sampling-on)}
 
@subsection @code{(set-dynamic-map-sampling-off)}
@cindex @code{(set-dynamic-map-sampling-off)}
 
@subsection @code{(set-dynamic-map-zoom-offset i)}
@cindex @code{(set-dynamic-map-zoom-offset i)}
 
where: 
 @itemize 
     @item i is an exact integer number
 @end itemize



@section Add Terminal Residue Functions 
@subsection @code{(do-add-terminal-residue state)}
@cindex @code{(do-add-terminal-residue state)}
 
where: 
 @itemize 
     @item state is an exact integer number
 @end itemize


@subsection @code{(set-add-terminal-residue-n-phi-psi-trials n)}
@cindex @code{(set-add-terminal-residue-n-phi-psi-trials n)}
 
where: 
 @itemize 
     @item n is an exact integer number
 @end itemize


@subsection @code{(set-add-terminal-residue-add-other-residue-flag i)}
@cindex @code{(set-add-terminal-residue-add-other-residue-flag i)}
 
where: 
 @itemize 
     @item i is an exact integer number
 @end itemize


@subsection @code{(set-terminal-residue-do-rigid-body-refine v)}
@cindex @code{(set-terminal-residue-do-rigid-body-refine v)}
 
where: 
 @itemize 
     @item v is an exact integer number
 @end itemize


@subsection @code{(add-terminal-residue-immediate-addition-state)}
@cindex @code{(add-terminal-residue-immediate-addition-state)}
 
@subsection @code{(set-add-terminal-residue-immediate-addition i)}
@cindex @code{(set-add-terminal-residue-immediate-addition i)}
 
where: 
 @itemize 
     @item i is an exact integer number
 @end itemize


@subsection @code{(set-add-terminal-residue-default-residue-type type)}
@cindex @code{(set-add-terminal-residue-default-residue-type type)}
 
where: 
 @itemize 
     @item type is a string
 @end itemize


@subsection @code{(set-add-terminal-residue-do-post-refine istat)}
@cindex @code{(set-add-terminal-residue-do-post-refine istat)}
 
where: 
 @itemize 
     @item istat is an exact integer number
 @end itemize



@section Delete Residues 
@subsection @code{(delete-atom-by-atom-index imol index do-delete-dialog)}
@cindex @code{(delete-atom-by-atom-index imol index do-delete-dialog)}
 
where: 
 @itemize 
     @item imol is an exact integer number
     @item index is an exact integer number
     @item do-delete-dialog is an exact integer number
 @end itemize


@subsection @code{(delete-residue-by-atom-index imol index do-delete-dialog)}
@cindex @code{(delete-residue-by-atom-index imol index do-delete-dialog)}
 
where: 
 @itemize 
     @item imol is an exact integer number
     @item index is an exact integer number
     @item do-delete-dialog is an exact integer number
 @end itemize


@subsection @code{(delete-residue-hydrogens-by-atom-index imol index do-delete-dialog)}
@cindex @code{(delete-residue-hydrogens-by-atom-index imol index do-delete-dialog)}
 
where: 
 @itemize 
     @item imol is an exact integer number
     @item index is an exact integer number
     @item do-delete-dialog is an exact integer number
 @end itemize


@subsection @code{(delete-residue-range imol chain-id resno-start end-resno)}
@cindex @code{(delete-residue-range imol chain-id resno-start end-resno)}
 
where: 
 @itemize 
     @item imol is an exact integer number
     @item chain-id is a string
     @item resno-start is an exact integer number
     @item end-resno is an exact integer number
 @end itemize


@subsection @code{(delete-residue imol chain-id resno inscode)}
@cindex @code{(delete-residue imol chain-id resno inscode)}
 
where: 
 @itemize 
     @item imol is an exact integer number
     @item chain-id is a string
     @item resno is an exact integer number
     @item inscode is a string
 @end itemize


@subsection @code{(delete-residue-with-altconf imol chain-id resno inscode altloc)}
@cindex @code{(delete-residue-with-altconf imol chain-id resno inscode altloc)}
 
where: 
 @itemize 
     @item imol is an exact integer number
     @item chain-id is a string
     @item resno is an exact integer number
     @item inscode is a string
     @item altloc is a string
 @end itemize


@subsection @code{(delete-residue-hydrogens imol chain-id resno inscode altloc)}
@cindex @code{(delete-residue-hydrogens imol chain-id resno inscode altloc)}
 
where: 
 @itemize 
     @item imol is an exact integer number
     @item chain-id is a string
     @item resno is an exact integer number
     @item inscode is a string
     @item altloc is a string
 @end itemize


@subsection @code{(delete-atom imol chain-id resno at-name altloc)}
@cindex @code{(delete-atom imol chain-id resno at-name altloc)}
 
where: 
 @itemize 
     @item imol is an exact integer number
     @item chain-id is a string
     @item resno is an exact integer number
     @item at-name is a string
     @item altloc is a string
 @end itemize


@subsection @code{(delete-residue-sidechain imol chain-id resno ins-code)}
@cindex @code{(delete-residue-sidechain imol chain-id resno ins-code)}
 
where: 
 @itemize 
     @item imol is an exact integer number
     @item chain-id is a string
     @item resno is an exact integer number
     @item ins-code is an unknown type
 @end itemize


@subsection @code{(set-delete-atom-mode)}
@cindex @code{(set-delete-atom-mode)}
 
@subsection @code{(set-delete-residue-mode)}
@cindex @code{(set-delete-residue-mode)}
 
@subsection @code{(set-delete-residue-zone-mode)}
@cindex @code{(set-delete-residue-zone-mode)}
 
@subsection @code{(set-delete-residue-hydrogens-mode)}
@cindex @code{(set-delete-residue-hydrogens-mode)}
 
@subsection @code{(set-delete-water-mode)}
@cindex @code{(set-delete-water-mode)}
 
@subsection @code{(set-delete-sidechain-mode)}
@cindex @code{(set-delete-sidechain-mode)}
 
@subsection @code{(delete-item-mode-is-atom-p)}
@cindex @code{(delete-item-mode-is-atom-p)}
 
@subsection @code{(delete-item-mode-is-residue-p)}
@cindex @code{(delete-item-mode-is-residue-p)}
 
@subsection @code{(delete-item-mode-is-water-p)}
@cindex @code{(delete-item-mode-is-water-p)}
 
@subsection @code{(delete-item-mode-is-sidechain-p)}
@cindex @code{(delete-item-mode-is-sidechain-p)}
 
@subsection @code{(clear-pending-delete-item)}
@cindex @code{(clear-pending-delete-item)}
 
@subsection @code{(set-keep-delete-item-active-state istate)}
@cindex @code{(set-keep-delete-item-active-state istate)}
 
where: 
 @itemize 
     @item istate is an exact integer number
 @end itemize



@section Rotate/Translate Buttons 
@subsection @code{(do-rot-trans-setup state)}
@cindex @code{(do-rot-trans-setup state)}
 
where: 
 @itemize 
     @item state is an exact integer number
 @end itemize


@subsection @code{(rot-trans-reset-previous)}
@cindex @code{(rot-trans-reset-previous)}
 
@subsection @code{(do-cis-trans-conversion-setup istate)}
@cindex @code{(do-cis-trans-conversion-setup istate)}
 
where: 
 @itemize 
     @item istate is an exact integer number
 @end itemize



@section Mainchain Building Functions 
@subsection @code{(do-db-main state)}
@cindex @code{(do-db-main state)}
 
where: 
 @itemize 
     @item state is an exact integer number
 @end itemize



@section Close Molecule FUnctions 
@subsection @code{(close-molecule imol)}
@cindex @code{(close-molecule imol)}
 
where: 
 @itemize 
     @item imol is an exact integer number
 @end itemize



@section Rotatmer Functions 
@subsection @code{(setup-rotamers state)}
@cindex @code{(setup-rotamers state)}
 
where: 
 @itemize 
     @item state is an exact integer number
 @end itemize


@subsection @code{(do-rotamers atom-index imol)}
@cindex @code{(do-rotamers atom-index imol)}
 
where: 
 @itemize 
     @item atom-index is an exact integer number
     @item imol is an exact integer number
 @end itemize


@subsection @code{(set-rotamer-lowest-probability f)}
@cindex @code{(set-rotamer-lowest-probability f)}
 
where: 
 @itemize 
     @item f is an inexact number
 @end itemize


@subsection @code{(set-rotamer-check-clashes i)}
@cindex @code{(set-rotamer-check-clashes i)}
 
where: 
 @itemize 
     @item i is an exact integer number
 @end itemize


@subsection @code{(set-auto-fit-best-rotamer-clash-flag i)}
@cindex @code{(set-auto-fit-best-rotamer-clash-flag i)}
 
where: 
 @itemize 
     @item i is an exact integer number
 @end itemize


@subsection @code{(setup-auto-fit-rotamer state)}
@cindex @code{(setup-auto-fit-rotamer state)}
 
where: 
 @itemize 
     @item state is an exact integer number
 @end itemize


@subsection @code{(fill-partial-residues imol)}
@cindex @code{(fill-partial-residues imol)}
 
where: 
 @itemize 
     @item imol is an exact integer number
 @end itemize


@subsection @code{(setup-180-degree-flip state)}
@cindex @code{(setup-180-degree-flip state)}
 
where: 
 @itemize 
     @item state is an exact integer number
 @end itemize



@section Mutate Functions 
@subsection @code{(setup-mutate state)}
@cindex @code{(setup-mutate state)}
 
where: 
 @itemize 
     @item state is an exact integer number
 @end itemize


@subsection @code{(setup-mutate-auto-fit state)}
@cindex @code{(setup-mutate-auto-fit state)}
 
where: 
 @itemize 
     @item state is an exact integer number
 @end itemize


@subsection @code{(do-mutation type is-stub-flag)}
@cindex @code{(do-mutation type is-stub-flag)}
 
where: 
 @itemize 
     @item type is a string
     @item is-stub-flag is an exact integer number
 @end itemize


@subsection @code{(progressive-residues-in-chain-check chain-id imol)}
@cindex @code{(progressive-residues-in-chain-check chain-id imol)}
 
where: 
 @itemize 
     @item chain-id is a string
     @item imol is an exact integer number
 @end itemize


@subsection @code{(mutate ires chain-id imol target-res-type)}
@cindex @code{(mutate ires chain-id imol target-res-type)}
 
where: 
 @itemize 
     @item ires is an exact integer number
     @item chain-id is a string
     @item imol is an exact integer number
     @item target-res-type is a string
 @end itemize


@subsection @code{(set-mutate-auto-fit-do-post-refine istate)}
@cindex @code{(set-mutate-auto-fit-do-post-refine istate)}
 
where: 
 @itemize 
     @item istate is an exact integer number
 @end itemize


@subsection @code{(mutate-auto-fit-do-post-refine-state)}
@cindex @code{(mutate-auto-fit-do-post-refine-state)}
 
@subsection @code{(do-base-mutation type)}
@cindex @code{(do-base-mutation type)}
 
where: 
 @itemize 
     @item type is a string
 @end itemize


@subsection @code{(set-residue-type-chooser-stub-state istat)}
@cindex @code{(set-residue-type-chooser-stub-state istat)}
 
where: 
 @itemize 
     @item istat is an exact integer number
 @end itemize



@section Alternative Conformation 
@subsection @code{(alt-conf-split-type-number)}
@cindex @code{(alt-conf-split-type-number)}
 
@subsection @code{(set-add-alt-conf-split-type-number i)}
@cindex @code{(set-add-alt-conf-split-type-number i)}
 
where: 
 @itemize 
     @item i is an exact integer number
 @end itemize


@subsection @code{(unset-add-alt-conf-dialog)}
@cindex @code{(unset-add-alt-conf-dialog)}
 
@subsection @code{(unset-add-alt-conf-define)}
@cindex @code{(unset-add-alt-conf-define)}
 
@subsection @code{(altconf)}
@cindex @code{(altconf)}
 
@subsection @code{(set-add-alt-conf-new-atoms-occupancy f)}
@cindex @code{(set-add-alt-conf-new-atoms-occupancy f)}
 
where: 
 @itemize 
     @item f is an inexact number
 @end itemize


@subsection @code{(set-show-alt-conf-intermediate-atoms i)}
@cindex @code{(set-show-alt-conf-intermediate-atoms i)}
 
where: 
 @itemize 
     @item i is an exact integer number
 @end itemize


@subsection @code{(show-alt-conf-intermediate-atoms-state)}
@cindex @code{(show-alt-conf-intermediate-atoms-state)}
 
@subsection @code{(zero-occupancy-residue-range imol chain-id ires1 ires2)}
@cindex @code{(zero-occupancy-residue-range imol chain-id ires1 ires2)}
 
where: 
 @itemize 
     @item imol is an exact integer number
     @item chain-id is a string
     @item ires1 is an exact integer number
     @item ires2 is an exact integer number
 @end itemize


@subsection @code{(fill-occupancy-residue-range imol chain-id ires1 ires2)}
@cindex @code{(fill-occupancy-residue-range imol chain-id ires1 ires2)}
 
where: 
 @itemize 
     @item imol is an exact integer number
     @item chain-id is a string
     @item ires1 is an exact integer number
     @item ires2 is an exact integer number
 @end itemize



@section Pointer Atom Functions 
@subsection @code{(place-atom-at-pointer)}
@cindex @code{(place-atom-at-pointer)}
 
@subsection @code{(place-typed-atom-at-pointer type)}
@cindex @code{(place-typed-atom-at-pointer type)}
 
where: 
 @itemize 
     @item type is a string
 @end itemize


@subsection @code{(set-pointer-atom-is-dummy i)}
@cindex @code{(set-pointer-atom-is-dummy i)}
 
where: 
 @itemize 
     @item i is an exact integer number
 @end itemize


@subsection @code{(display-where-is-pointer)}
@cindex @code{(display-where-is-pointer)}
 

@section Baton Build Functions 
@subsection @code{(set-baton-mode i)}
@cindex @code{(set-baton-mode i)}
 
where: 
 @itemize 
     @item i is an exact integer number
 @end itemize


@subsection @code{(set-draw-baton i)}
@cindex @code{(set-draw-baton i)}
 
where: 
 @itemize 
     @item i is an exact integer number
 @end itemize


@subsection @code{(accept-baton-position)}
@cindex @code{(accept-baton-position)}
 
@subsection @code{(baton-try-another)}
@cindex @code{(baton-try-another)}
 
@subsection @code{(shorten-baton)}
@cindex @code{(shorten-baton)}
 
@subsection @code{(lengthen-baton)}
@cindex @code{(lengthen-baton)}
 
@subsection @code{(baton-build-delete-last-residue)}
@cindex @code{(baton-build-delete-last-residue)}
 
@subsection @code{(set-baton-build-params istart-resno chain-id backwards)}
@cindex @code{(set-baton-build-params istart-resno chain-id backwards)}
 
where: 
 @itemize 
     @item istart-resno is an exact integer number
     @item chain-id is a string
     @item backwards is a string
 @end itemize



@section Post-Baton Functions 
@subsection @code{(reverse-direction-of-fragment imol chain-id resno)}
@cindex @code{(reverse-direction-of-fragment imol chain-id resno)}
 
where: 
 @itemize 
     @item imol is an exact integer number
     @item chain-id is a string
     @item resno is an exact integer number
 @end itemize


@subsection @code{(setup-reverse-direction i)}
@cindex @code{(setup-reverse-direction i)}
 
where: 
 @itemize 
     @item i is an exact integer number
 @end itemize



@section Terminal OXT Atom 
@subsection @code{(add-OXT-to-residue imol reso insertion-code chain-id)}
@cindex @code{(add-OXT-to-residue imol reso insertion-code chain-id)}
 
where: 
 @itemize 
     @item imol is an exact integer number
     @item reso is an exact integer number
     @item insertion-code is a string
     @item chain-id is a string
 @end itemize



@section Crosshairs 
@subsection @code{(set-draw-crosshairs i)}
@cindex @code{(set-draw-crosshairs i)}
 
where: 
 @itemize 
     @item i is an exact integer number
 @end itemize


@subsection @code{(draw-crosshairs-state)}
@cindex @code{(draw-crosshairs-state)}
 

@section Edit Chi Angles 
@subsection @code{(setup-edit-chi-angles state)}
@cindex @code{(setup-edit-chi-angles state)}
 
where: 
 @itemize 
     @item state is an exact integer number
 @end itemize


@subsection @code{(set-find-hydrogen-torsion state)}
@cindex @code{(set-find-hydrogen-torsion state)}
 
where: 
 @itemize 
     @item state is an exact integer number
 @end itemize


@subsection @code{(set-graphics-edit-current-chi ichi)}
@cindex @code{(set-graphics-edit-current-chi ichi)}
 
where: 
 @itemize 
     @item ichi is an exact integer number
 @end itemize


@subsection @code{(unset-moving-atom-move-chis)}
@cindex @code{(unset-moving-atom-move-chis)}
 
@subsection @code{(set-show-chi-angle-bond imode)}
@cindex @code{(set-show-chi-angle-bond imode)}
 
where: 
 @itemize 
     @item imode is an exact integer number
 @end itemize



@section Masks 
@subsection @code{(mask-map-by-molecule map-mol-no coord-mol-no invert-flag)}
@cindex @code{(mask-map-by-molecule map-mol-no coord-mol-no invert-flag)}
 
where: 
 @itemize 
     @item map-mol-no is an exact integer number
     @item coord-mol-no is an exact integer number
     @item invert-flag is an exact integer number
 @end itemize


@subsection @code{(mask-map-by-atom-selection map-mol-no coords-mol-no mmdb-atom-selection invert-flag)}
@cindex @code{(mask-map-by-atom-selection map-mol-no coords-mol-no mmdb-atom-selection invert-flag)}
 
where: 
 @itemize 
     @item map-mol-no is an exact integer number
     @item coords-mol-no is an exact integer number
     @item mmdb-atom-selection is a string
     @item invert-flag is an exact integer number
 @end itemize


@subsection @code{(set-map-mask-atom-radius rad)}
@cindex @code{(set-map-mask-atom-radius rad)}
 
where: 
 @itemize 
     @item rad is an inexact number
 @end itemize



@section Check Waters Interface 
@subsection @code{(set-check-waters-b-factor-limit f)}
@cindex @code{(set-check-waters-b-factor-limit f)}
 
where: 
 @itemize 
     @item f is an inexact number
 @end itemize


@subsection @code{(set-check-waters-map-sigma-limit f)}
@cindex @code{(set-check-waters-map-sigma-limit f)}
 
where: 
 @itemize 
     @item f is an inexact number
 @end itemize


@subsection @code{(set-check-waters-min-dist-limit f)}
@cindex @code{(set-check-waters-min-dist-limit f)}
 
where: 
 @itemize 
     @item f is an inexact number
 @end itemize


@subsection @code{(set-check-waters-max-dist-limit f)}
@cindex @code{(set-check-waters-max-dist-limit f)}
 
where: 
 @itemize 
     @item f is an inexact number
 @end itemize


@subsection @code{(check-waters-by-difference-map-sigma-level-state)}
@cindex @code{(check-waters-by-difference-map-sigma-level-state)}
 
@subsection @code{(set-check-waters-by-difference-map-sigma-level f)}
@cindex @code{(set-check-waters-by-difference-map-sigma-level f)}
 
where: 
 @itemize 
     @item f is an inexact number
 @end itemize



@section Least-Squares matching 
@subsection @code{(clear-lsq-matches)}
@cindex @code{(clear-lsq-matches)}
 
@subsection @code{(apply-lsq-matches imol-reference imol-moving)}
@cindex @code{(apply-lsq-matches imol-reference imol-moving)}
 
where: 
 @itemize 
     @item imol-reference is an exact integer number
     @item imol-moving is an exact integer number
 @end itemize



@section Least-Squares plane interface 
@subsection @code{(setup-lsq-deviation state)}
@cindex @code{(setup-lsq-deviation state)}
 
where: 
 @itemize 
     @item state is an exact integer number
 @end itemize


@subsection @code{(setup-lsq-plane-define state)}
@cindex @code{(setup-lsq-plane-define state)}
 
where: 
 @itemize 
     @item state is an exact integer number
 @end itemize


@subsection @code{(unset-lsq-plane-dialog)}
@cindex @code{(unset-lsq-plane-dialog)}
 
@subsection @code{(remove-last-lsq-plane-atom)}
@cindex @code{(remove-last-lsq-plane-atom)}
 

@section Trim 
@subsection @code{(raster3d rd3-filename)}
@cindex @code{(raster3d rd3-filename)}
 
where: 
 @itemize 
     @item rd3-filename is a string
 @end itemize


@subsection @code{(povray filename)}
@cindex @code{(povray filename)}
 
where: 
 @itemize 
     @item filename is a string
 @end itemize


@subsection @code{(make-image-raster3d filename)}
@cindex @code{(make-image-raster3d filename)}
 
where: 
 @itemize 
     @item filename is a string
 @end itemize


@subsection @code{(make-image-povray filename)}
@cindex @code{(make-image-povray filename)}
 
where: 
 @itemize 
     @item filename is a string
 @end itemize


@subsection @code{(set-raster3d-bond-thickness f)}
@cindex @code{(set-raster3d-bond-thickness f)}
 
where: 
 @itemize 
     @item f is an inexact number
 @end itemize


@subsection @code{(set-raster3d-density-thickness f)}
@cindex @code{(set-raster3d-density-thickness f)}
 
where: 
 @itemize 
     @item f is an inexact number
 @end itemize


@subsection @code{(set-renderer-show-atoms istate)}
@cindex @code{(set-renderer-show-atoms istate)}
 
where: 
 @itemize 
     @item istate is an exact integer number
 @end itemize


@subsection @code{(raster-screen-shot)}
@cindex @code{(raster-screen-shot)}
 
@subsection @code{(citation-notice-off)}
@cindex @code{(citation-notice-off)}
 

@section Superposition (SSM) 
@subsection @code{(superpose imol1 imol2 move-imol2-flag)}
@cindex @code{(superpose imol1 imol2 move-imol2-flag)}
 
where: 
 @itemize 
     @item imol1 is an exact integer number
     @item imol2 is an exact integer number
     @item move-imol2-flag is an exact integer number
 @end itemize



@section NCS 
@subsection @code{(set-draw-ncs-ghosts imol istate)}
@cindex @code{(set-draw-ncs-ghosts imol istate)}
 
where: 
 @itemize 
     @item imol is an exact integer number
     @item istate is an exact integer number
 @end itemize


@subsection @code{(set-ncs-ghost-bond-thickness imol f)}
@cindex @code{(set-ncs-ghost-bond-thickness imol f)}
 
where: 
 @itemize 
     @item imol is an exact integer number
     @item f is an inexact number
 @end itemize


@subsection @code{(ncs-update-ghosts imol)}
@cindex @code{(ncs-update-ghosts imol)}
 
where: 
 @itemize 
     @item imol is an exact integer number
 @end itemize


@subsection @code{(make-dynamically-transformed-ncs-maps imol-model imol-map)}
@cindex @code{(make-dynamically-transformed-ncs-maps imol-model imol-map)}
 
where: 
 @itemize 
     @item imol-model is an exact integer number
     @item imol-map is an exact integer number
 @end itemize


@subsection @code{(make-ncs-ghosts-maybe imol)}
@cindex @code{(make-ncs-ghosts-maybe imol)}
 
where: 
 @itemize 
     @item imol is an exact integer number
 @end itemize


@subsection @code{(show-strict-ncs-state imol)}
@cindex @code{(show-strict-ncs-state imol)}
 
where: 
 @itemize 
     @item imol is an exact integer number
 @end itemize


@subsection @code{(set-show-strict-ncs imol state)}
@cindex @code{(set-show-strict-ncs imol state)}
 
where: 
 @itemize 
     @item imol is an exact integer number
     @item state is an exact integer number
 @end itemize


@subsection @code{(set-ncs-homology-level flev)}
@cindex @code{(set-ncs-homology-level flev)}
 
where: 
 @itemize 
     @item flev is an inexact number
 @end itemize


@subsection @code{(copy-chain imol from-chain to-chain)}
@cindex @code{(copy-chain imol from-chain to-chain)}
 
where: 
 @itemize 
     @item imol is an exact integer number
     @item from-chain is a string
     @item to-chain is a string
 @end itemize


@subsection @code{(copy-from-ncs-master-to-others imol chain-id)}
@cindex @code{(copy-from-ncs-master-to-others imol chain-id)}
 
where: 
 @itemize 
     @item imol is an exact integer number
     @item chain-id is a string
 @end itemize


@subsection @code{(ncs-control-change-ncs-master-to-chain imol ichain)}
@cindex @code{(ncs-control-change-ncs-master-to-chain imol ichain)}
 
where: 
 @itemize 
     @item imol is an exact integer number
     @item ichain is an exact integer number
 @end itemize


@subsection @code{(ncs-control-display-chain imol ichain state)}
@cindex @code{(ncs-control-display-chain imol ichain state)}
 
where: 
 @itemize 
     @item imol is an exact integer number
     @item ichain is an exact integer number
     @item state is an exact integer number
 @end itemize


@subsection @code{(place-helix-here)}
@cindex @code{(place-helix-here)}
 
@subsection @code{(new-molecule-by-residue-type-selection imol residue-type)}
@cindex @code{(new-molecule-by-residue-type-selection imol residue-type)}
 
where: 
 @itemize 
     @item imol is an exact integer number
     @item residue-type is a string
 @end itemize


@subsection @code{(new-molecule-by-atom-selection imol atom-selection)}
@cindex @code{(new-molecule-by-atom-selection imol atom-selection)}
 
where: 
 @itemize 
     @item imol is an exact integer number
     @item atom-selection is an unknown type
 @end itemize



@section Miguel's orientation axes matrix 

@section RNA/DNA 

@section Sequence (Assignment) 
@subsection @code{(print-sequence-chain imol chain-id)}
@cindex @code{(print-sequence-chain imol chain-id)}
 
where: 
 @itemize 
     @item imol is an exact integer number
     @item chain-id is a string
 @end itemize


@subsection @code{(assign-fasta-sequence imol chain-id-in seq)}
@cindex @code{(assign-fasta-sequence imol chain-id-in seq)}
 
where: 
 @itemize 
     @item imol is an exact integer number
     @item chain-id-in is a string
     @item seq is a string
 @end itemize


@subsection @code{(assign-sequence imol-model imol-map chain-id)}
@cindex @code{(assign-sequence imol-model imol-map chain-id)}
 
where: 
 @itemize 
     @item imol-model is an exact integer number
     @item imol-map is an exact integer number
     @item chain-id is a string
 @end itemize



@section Surfaces 
@subsection @code{(do-surface imol istate)}
@cindex @code{(do-surface imol istate)}
 
where: 
 @itemize 
     @item imol is an exact integer number
     @item istate is an exact integer number
 @end itemize


@subsection @code{(fffear-search imol-model imol-map)}
@cindex @code{(fffear-search imol-model imol-map)}
 
where: 
 @itemize 
     @item imol-model is an exact integer number
     @item imol-map is an exact integer number
 @end itemize


@subsection @code{(set-fffear-angular-resolution f)}
@cindex @code{(set-fffear-angular-resolution f)}
 
where: 
 @itemize 
     @item f is an inexact number
 @end itemize


@subsection @code{(fffear-angular-resolution)}
@cindex @code{(fffear-angular-resolution)}
 

@section Remote Control 
@subsection @code{(make-socket-listener-maybe)}
@cindex @code{(make-socket-listener-maybe)}
 
@subsection @code{(set-coot-listener-socket-state-internal sock-state)}
@cindex @code{(set-coot-listener-socket-state-internal sock-state)}
 
where: 
 @itemize 
     @item sock-state is an exact integer number
 @end itemize



@section Display Lists for Maps 
@subsection @code{(set-display-lists-for-maps i)}
@cindex @code{(set-display-lists-for-maps i)}
 
where: 
 @itemize 
     @item i is an exact integer number
 @end itemize



@section Preferences 
@subsection @code{(preferences)}
@cindex @code{(preferences)}
 
@subsection @code{(clear-preferences)}
@cindex @code{(clear-preferences)}
 
@subsection @code{(set-mark-cis-peptides-as-bad istate)}
@cindex @code{(set-mark-cis-peptides-as-bad istate)}
 
where: 
 @itemize 
     @item istate is an exact integer number
 @end itemize


@subsection @code{(show-mark-cis-peptides-as-bad-state)}
@cindex @code{(show-mark-cis-peptides-as-bad-state)}
 
@subsection @code{(browser-url url)}
@cindex @code{(browser-url url)}
 
where: 
 @itemize 
     @item url is a string
 @end itemize


@subsection @code{(set-browser-interface browser)}
@cindex @code{(set-browser-interface browser)}
 
where: 
 @itemize 
     @item browser is a string
 @end itemize


@subsection @code{(handle-online-coot-search-request entry-text)}
@cindex @code{(handle-online-coot-search-request entry-text)}
 
where: 
 @itemize 
     @item entry-text is a string
 @end itemize


@subsection @code{(new-generic-object-number objname)}
@cindex @code{(new-generic-object-number objname)}
 
where: 
 @itemize 
     @item objname is a string
 @end itemize


@subsection @code{(to-generic-object-add-display-list-handle object-number display-list-id)}
@cindex @code{(to-generic-object-add-display-list-handle object-number display-list-id)}
 
where: 
 @itemize 
     @item object-number is an exact integer number
     @item display-list-id is an exact integer number
 @end itemize


@subsection @code{(set-display-generic-object object-number istate)}
@cindex @code{(set-display-generic-object object-number istate)}
 
where: 
 @itemize 
     @item object-number is an exact integer number
     @item istate is an exact integer number
 @end itemize


@subsection @code{(generic-object-is-displayed-p object-number)}
@cindex @code{(generic-object-is-displayed-p object-number)}
 
where: 
 @itemize 
     @item object-number is an exact integer number
 @end itemize


@subsection @code{(generic-object-index name)}
@cindex @code{(generic-object-index name)}
 
where: 
 @itemize 
     @item name is a string
 @end itemize


@subsection @code{(number-of-generic-objects)}
@cindex @code{(number-of-generic-objects)}
 
@subsection @code{(generic-object-info)}
@cindex @code{(generic-object-info)}
 
@subsection @code{(generic-object-has-objects-p obj-no)}
@cindex @code{(generic-object-has-objects-p obj-no)}
 
where: 
 @itemize 
     @item obj-no is an exact integer number
 @end itemize


@subsection @code{(close-generic-object object-number)}
@cindex @code{(close-generic-object object-number)}
 
where: 
 @itemize 
     @item object-number is an exact integer number
 @end itemize


@subsection @code{(is-closed-generic-object-p object-number)}
@cindex @code{(is-closed-generic-object-p object-number)}
 
where: 
 @itemize 
     @item object-number is an exact integer number
 @end itemize


@subsection @code{(generic-objects-gui-wrapper)}
@cindex @code{(generic-objects-gui-wrapper)}
 
@subsection @code{(handle-read-draw-probe-dots dots-file)}
@cindex @code{(handle-read-draw-probe-dots dots-file)}
 
where: 
 @itemize 
     @item dots-file is a string
 @end itemize


@subsection @code{(handle-read-draw-probe-dots-unformatted dots-file imol show-clash-gui-flag)}
@cindex @code{(handle-read-draw-probe-dots-unformatted dots-file imol show-clash-gui-flag)}
 
where: 
 @itemize 
     @item dots-file is a string
     @item imol is an exact integer number
     @item show-clash-gui-flag is an exact integer number
 @end itemize


@subsection @code{(set-do-probe-dots-on-rotamers-and-chis state)}
@cindex @code{(set-do-probe-dots-on-rotamers-and-chis state)}
 
where: 
 @itemize 
     @item state is an exact integer number
 @end itemize


@subsection @code{(do-probe-dots-on-rotamers-and-chis-state)}
@cindex @code{(do-probe-dots-on-rotamers-and-chis-state)}
 
@subsection @code{(set-do-probe-dots-post-refine state)}
@cindex @code{(set-do-probe-dots-post-refine state)}
 
where: 
 @itemize 
     @item state is an exact integer number
 @end itemize


@subsection @code{(do-probe-dots-post-refine-state)}
@cindex @code{(do-probe-dots-post-refine-state)}
 
@subsection @code{(set-interactive-probe-dots-molprobity-radius r)}
@cindex @code{(set-interactive-probe-dots-molprobity-radius r)}
 
where: 
 @itemize 
     @item r is an inexact number
 @end itemize


@subsection @code{(interactive-probe-dots-molprobity-radius)}
@cindex @code{(interactive-probe-dots-molprobity-radius)}
 
@subsection @code{(probe-available-p)}
@cindex @code{(probe-available-p)}
 
@subsection @code{(set-dti-stereo-mode state)}
@cindex @code{(set-dti-stereo-mode state)}
 
where: 
 @itemize 
     @item state is an exact integer number
 @end itemize


@subsection @code{(do-smiles-gui)}
@cindex @code{(do-smiles-gui)}
 

@section Fun 
@subsection @code{(do-tw)}
@cindex @code{(do-tw)}
 
@subsection @code{(place-text text x y z size)}
@cindex @code{(place-text text x y z size)}
 
where: 
 @itemize 
     @item text is an unknown type
     @item x is an inexact number
     @item y is an inexact number
     @item z is an inexact number
     @item size is an exact integer number
 @end itemize


@subsection @code{(remove-text text-handle)}
@cindex @code{(remove-text text-handle)}
 
where: 
 @itemize 
     @item text-handle is an exact integer number
 @end itemize




%
\documentclass{book}
\usepackage{a4}
\usepackage{palatino}
%\usepackage{times}
%\usepackage{utopia}
\usepackage{euler}
\usepackage{fancyhdr}
\usepackage{epsf}

\newcommand {\atilde} {$_{\char '176}$} % tilde(~) character

\title{The Coot Reference Manual}
\author{Paul Emsley \\\textsf{\small emsley@ysbl.york.ac.uk}}
%\makeindex  % Not at the moment.  There are no index markups (yet).

\begin{document}
\maketitle
\tableofcontents

\chapter{Acknowledgments}
Paul Emsley is extremely grateful to use the library code of the
following people, without whom Coot could not have been realised:

\begin{trivlist}
\item Kevin Cowtan
\item Eugene Krissinel
\item Stuart McNicholas
\item Raghavendra Chandrashekara
\item Paul Bourke \& Cory Gene Bloyd
\end{trivlist}

Roland Dunbrack \& co-workers for rotamer library data.

Also (for generally useful software used in Coot):

\begin{trivlist}
\item Matteo Frigo \& Steven G. Johnson
\item Gary Houston \& other Guile developers
\item Python developers
\item Gtk+ and GNOME-Canvas developers
\item GNU Scientific Library developers
\item OpenGL developers
\item Janne L\"of
\end{trivlist}

Also those with whom Paul has corresponded about or provided
features and bug fixes and built the software:

\begin{tabular}{ll}
 William G. Scott & Bernhard Lohkamp \\
 Joel Bard  & Ezra Peisach           \\
 Alex Schuettelkopf & Charlie Bond 
\end{tabular}

Not forgetting the testers\footnote{in no particular order}

%\begin{trivlist}
%\item Eleanor J. Dodson
%\item Jan Dohnalek
%\item Karen McLuskey
%\item Bernhard Lohkamp
%\item Aleks Roszak
%\item Florence Vincent
%\item Roberto Steiner
%\item Alex Schuettelkopf
%\item Charlie Bond
%\item Constantina Fotinou
%\item William G. Scott
%\item Adrian Lapthorn
%\end{trivlist}

\begin{tabular}{ll}
Eleanor J. Dodson & Jan Dohnalek \\
Constantina Fotinou & Alex Roszak  \\
Florence Vincent  & Roberto Steiner \\
Karen McLuskey & Adrian Lapthorn   
\end{tabular}

\vspace{5mm}

Those with experience of Quanta, XFit and O will notice similarities
between Coot and those programs, it's fair to say that they have had
considerable influence in the look of Coot, so Paul respectively
thanks for inspiration: Tom Oldfield, Alwyn Jones and Duncan McRee
(and their co-workers).

\chapter{Design Overview}
\section{Why?}
``Why does Coot exist?'' you might ask.  ``Given that other molecular
graphics\footnote{molecular graphics with protein modeling and
  density fitting functions, that is.} programs exist, why write
another?''

Because I like having the source code to programs I use and think that
others feel the same.  Because the other programs don't quite work how
I wanted them to\footnote{and of course, there was no way to fix
  that.}. Because there was the possibility to integrate molecular
graphics into the CCP4 Suite.  

As to why write Coot when CCP4MG was available: that is not how it
happened. Coot\footnote{it was called ``MapView'' at the time.} was
released over a year before CCP4MG was available.  I followed my own
design, toolkit and aesthetic decisions - for good or bad\footnote{for
  example, I was (and remain) less concerned about porting to various
  shades of Microsoft Windows operating systems than the CCP4MG
  developers.}.

\section{Hacker's Guide}

The are several core libraries that are fundamental to Coot:

\begin{itemize}
\item Clipper: Kevin Cowtan's General crystallographic object library
\item mmdb: CCP4's Coordinate Library
\item GTk+: GNU's GUI toolkit.
\end{itemize}

\subsection{GUI}
The GUI is almost entirely built using glade.  Glade writes out its
code in pure C.  This causes a problem.  \texttt{src/interface.h} and
\texttt{src/support.h} both get regenerated in ``C mode'' every time
glade is run.  So, after every time we change the GUI with glade, we
need to run \texttt{post-glade} to introduce the C/C++ linking type
declaration wrapper into these files.

Not all of the GUI is build with glade - there are dynamic elements,
for example the ``Map and Molecule (Display) Control'' window the
frame of which are generated in glade, but the hboxs are filled using
hand-made code (see \texttt{gtk-manual.c}).

\subsection{GUI/State Variables}
The graphics\_info\_t class contains a host of static state variables,
mostly manipulated by GUI element (\emph{e.g} button)
callbacks\footnote{mostly button clicked signals and menu item
  activative signals}. For historical reasons they are initially set
in \texttt{globjects.cc}.  Because the callbacks are written in C by
glade\footnote{the GUI builder}, these variables need a functional
interface to set the variables, and that interface is used by both the
GUI button\footnote{and other GUI elements} callbacks and is exported
to the scripting level.  These function declarations are in
\texttt{c-interface.h}.  All manipulations of graphics\_info\_t's
state variables go via \texttt{c-interface.h}.

Notice that MMDB functions are not allowed in
this interface\footnote{because SWIG chokes on them}. 

\subsection{Scripting}
So, SWIG uses \texttt{c-interface.h} to generate the python/scheme
scripting interface. The scripting language is chosen at
configure-time using either \texttt{--with-guile} or
\texttt{--with-python}.

\section{Validation}
As I write this, a few of us are cobbling together a XML-based system
for validation.  We think that validation data should be presented as
XML data that can be passed between packages and programs.  Either the
program itself will output the data, or we will write a wrapper to
turn the output into the appropriate XML format.  

These XML data will be then available for use in the molecular
graphics and will provide information for a ``Next Unusual Feature''
button.  The library to provide the XML cabability for this is expat,
the same library used in Perl's XML::Parser, Python's XML parser
Pyexpat and Mozilla's XML parser.

\subsection{Example: Temperature Factor Analysis}
Recall that the kurtosis of a distribution, $k$ is given by:

\begin{equation}
  \label{eq:kurtosis}
  k = \frac{\Sigma(X_i - \mu)^4} {N \sigma^4} - 3 
\end{equation}

We calculate the kurtosis for the isotropic temperature factors for
each residue in the molecule and residues with the most leptokurtic
distributions are written out to a file.  The format of the file is
XML.

This is an example of how we expect validation data to be presented to
molecular graphics programs.



\chapter{Refinement and Regularization}

A function that we need for Molecular Graphics is to be able to
regularize (a.k.a ``idealize'') the coordinates of the model.  In
order to do so we need to find the ideal values (also called here
``restraints'', using the Refmac nomenclature).  We have a
multivariable function minimizer that requires the gradients of the
parameters (the coordinates).  Here we describe how to generate the
gradients analytically.  We need the derivatives for the bond lengths,
angles, torsions and planes.

\section{Introduction}

The function that we are trying to minimize for refinement is $S$, where

\begin{displaymath}
  S = S_{bond} + S_{angle} + S_{torsion} + S_{plane} + S_{chiral} + -kS_{map}
\end{displaymath}

For regularization it is:
\begin{displaymath}
  S = S_{bond} + S_{angle} + S_{torsion}
\end{displaymath}



Let's take these 3 parts in turn:

% ------------------------------------------------------------------
%                  Bonds 
% ------------------------------------------------------------------

\section{Bonds}

\begin{displaymath}
  S_{bond} = \sum_{i=1}^{N_{bonds}} {(b_i - b_{0_i})^2}
\end{displaymath}

Where $b_{0_i}$ is the ideal length (from the Refmac dictionary) of
the $i$th bond, $\mathbf{b}_i$ is the bond vector and $b_i$ is its length.

\begin{eqnarray*}
  \label{eq:1}
  \frac{\partial S_i}{\partial x_m} & = & \frac{\partial S_i}{\partial b_i} 
  \frac{\partial b_i}{\partial x_m} \\
   & = & [2(b_i - b_{0_i})]   \frac{\partial b_i}{\partial x_m}
\end{eqnarray*}

\begin{displaymath}
  b_i = \sqrt((x_m-x_k)^2 + (y_m-y_k)^2 + (z_m-z_k)^2)
\end{displaymath}

So: 
\begin{eqnarray*}
\frac{\partial b_i}{\partial x_m} & = & (\frac{1}{2} \frac{1}{b_i}) 2 (x_m - x_k) \\
 &   = & \frac{(x_m - x_k)}{b_i}
\end{eqnarray*}

So: 
\begin{displaymath}
  \frac{\partial S_i}{\partial x_m} = 2[b_i - b_{0_1}] \frac{(x_m - x_k)}{b_i}
\end{displaymath}


% ------------------------------------------------------------------
%                  Angles 
% ------------------------------------------------------------------

\section{Angles}
We are trying to minimise $S_{angle}$, where (for simplicity I ignore
the weights)

\begin{displaymath}
  S_{angle} = \sum_{i=1}^{N_{angles}} {(\theta_i - \theta_{0_i})^2}
\end{displaymath}


Angle $\theta$ contributed to by atoms $k$, $l$ and $m$:

\begin{displaymath}
  \cos \theta = \frac{{\underline a}.{\underline b}}{ab}
\end{displaymath}

\begin{trivlist}
\item where
\item $\underline {a}$ is the bond of atoms $k$ and $l$ $((x_k-x_l), (y_k-y_l), (z_k-z_l))$
\item $\underline {b}$ is the bond of atoms $l$ and $m$  $((x_m-x_l), (y_m-y_l), (z_m-z_l))$
\item Note that the vectors point away from the middle atom $l$.
\end{trivlist}



So: 

\begin{equation}
  \label{eq:1}
  \theta = acos(P)
\end{equation}

where 

\begin{displaymath}
  P = \frac{{\underline a}.{\underline b}}{ab} 
\end{displaymath}

Using the Chain Rule:
\begin{equation}
  \label{eq:2}
  \frac{\partial \theta}{\partial _k} = \frac{\partial \theta}{\partial P} \frac{\partial P}{\partial x_k}
\end{equation}

Given that we are only intereted in $\theta$ in the range $0\rightarrow\pi$:

\begin{equation}
  \label{eq:3}
  \frac{\partial \theta}{\partial P} = -\frac{1}{\sin \theta}
\end{equation}

Let's split up $P$ again using the chain rule: 
\begin{equation}
  \label{eq:4}
  \frac{\partial P}{\partial x_k} = 
  Q\frac{\partial R}{\partial x_k} + R\frac{\partial Q}{\partial x_k}
\end{equation}

where 
\begin{equation}
  \label{eq:5}
  Q =  {\underline a}.{\underline b}
\end{equation}
\begin{equation}
  \label{eq:6}
  R = \frac{1}{ab}
\end{equation}

\subsection{The middle atom}

This is somewhat more tricky than an end atom because the derivatives
of $ab$ and ${\underline a}.{\underline b}$ are not so trivial.  Let's
change the indexing so that we are actually talking about the middle
atom, $l$.

Differentiating (\ref{eq:6}): 

\begin{equation}
  \label{eq:7}
  \frac{\partial R}{\partial x_l} = 
  -\frac{1}{(ab)^2}b\frac{\partial a}{\partial x_l} 
  -\frac{1}{(ab)^2}a\frac{\partial b}{\partial x_l}
\end{equation}

$\frac{\partial a}{\partial x_l}$ is exactly the same as we were using
with bonds:
\begin{displaymath}
  \frac{\partial a}{\partial x_l} = \frac{x_l-x_k}{a}
\end{displaymath}

Similarly:
\begin{displaymath}
  \frac{\partial b}{\partial x_l} = \frac{x_l-x_m}{a}
\end{displaymath}

So substituting those into (\ref{eq:7}):
\begin{displaymath}
  \frac{\partial R}{\partial x_l} = -\frac{x_l-x_k}{a^3b} -\frac{x_l-x_m}{ab^3}
\end{displaymath}

Turning to $Q$, recall (\ref{eq:5}), so: 
\begin{displaymath}
  Q =  
  ((x_k-x_l)(x_m-x_l) + (y_k-y_l)(y_m-y_l) + (z_k-z_l)(z_m-z_l))
\end{displaymath}

Therefore
\begin{displaymath}
   \frac{\partial Q}{\partial x_l} = -(x_k-x_l) -(x_m-x_l)
\end{displaymath}

Substituting all the above into (\ref{eq:4}):
\begin{displaymath}
  \frac{\partial P}{\partial x_l} = ({\underline a}.{\underline b})[-\frac{x_l-x_k}{a^3b} -\frac{x_l-x_m}{ab^3}] + \frac{-(x_k-x_l)-(x_m-x_l)}{ab}
\end{displaymath}

So, combining this and (\ref{eq:3}) into (\ref{eq:2}), we get: 
\begin{displaymath}
  \frac{\partial \theta}{\partial x_l} = -\frac{1}{\sin \theta}  \frac{\partial P}{\partial x_l} 
\end{displaymath}

%\begin{displaymath}
%  \frac{\partial \theta}{\partial x_l} = -\frac{1}{\sin \theta}(({\underline a}.{\underline b})[-\frac{x_l-x_k}{a^3b} -\frac{x_l-x_m}{ab^3}] + \frac{-(x_k-x_l)-(x_m-x_l)}{ab})
%\end{displaymath}






\subsection{An End Atom (Atoms $k$ or $m$)}
This is more simple because there are no cross terms in 
$\frac{\partial R}{\partial x_k}$ and $\frac{\partial Q}{\partial x_k}$.

\begin{displaymath}
  \frac{\partial R}{\partial x_k} = \frac{(x_k-x_l)}{ab}
\end{displaymath}

and 
\begin{displaymath}
  \frac{\partial Q}{\partial x_k} = (x_m-x_l)
\end{displaymath}

So 

\begin{equation}
  \frac{\partial \theta}{\partial x_k} = -\frac{1}{sin\theta} [\frac{(x_l-x_k)}{a^2}cos\theta + \frac{x_m-x_l}{ab}]
\end{equation}


% ------------------------------------------------------------------
%                  Torsions
% ------------------------------------------------------------------

\section{Torsions}
The torsion of 3 vectors (the vectors between one atom and the next in
the torsion angle) is given by:
\begin{equation}
  \label{eq:8}
  \tau(\mathbf{a},\mathbf{b},\mathbf{c}) = \arg(-\mathbf{a}.\mathbf{c}+(\mathbf{a}.\mathbf{b})(\mathbf{b}.\mathbf{c}), \mathbf{a}.(\mathbf{b} \mathbf{\times}\mathbf{c}))
\end{equation}


Let's split the expression up into tractable (for me) portions, the
evaluation of $\theta$ in the program will combine these expressions
starting at the end (the most simple).

\begin{figure}[htbp]
  \centering
  \leavevmode
  \epsfxsize=50mm
%  \epsffile{torsion.eps}
  \caption{Torsion vectors}
  \label{fig:torsion-vectors}
\end{figure}

Obviously: 
\begin{displaymath}
  a_x = P_{2_x}-P_{1_x} , b_x = P_{3_x}-P_{2_x} , c_x = P_{4_x}-P_{3_x}
\end{displaymath}
\begin{displaymath}
  a_y = P_{2_y}-P_{1_y} , b_y = P_{3_y}-P_{2_y} , c_y = P_{4_y}-P_{3_y}
\end{displaymath}
\begin{displaymath}
  a_z = P_{2_z}-P_{1_z} , b_z = P_{3_z}-P_{2_z} , c_z = P_{4_z}-P_{3_z}
\end{displaymath}

Unfortunately, I change the nomenclature because I derived the torsion
terms some time after the angle terms and I had forgotten what I had
previously been using.


\begin{displaymath}
  \theta = \tau(\mathbf{a},\mathbf{b},\mathbf{c}) =  \arctan(D)
\end{displaymath}

where
\begin{displaymath}
  D = \frac{\frac{\mathbf{a}.(\mathbf{b} \mathbf{\times}\mathbf{c})} {b}}{-\mathbf{a}.\mathbf{c}+\frac{(\mathbf{a}.\mathbf{b})(\mathbf{b}.\mathbf{c})}{b^2}}
\end{displaymath}

So

\begin{eqnarray}
  \label{eq:df}
  \frac{\partial \theta}{\partial x_{P_1}} & = & 
  \frac{\partial \theta}{\partial D} \frac{\partial D}{\partial x_{P_1}} \\
  & = & \frac{1}{1+D^2}\frac{\partial D}{\partial x_{P_1}}
\end{eqnarray}

Let
\begin{displaymath}
  E = \frac{\mathbf{a}.(\mathbf{b} \mathbf{\times}\mathbf{c})}{b}
\end{displaymath}
and 
\begin{displaymath}
  F = \frac{1}{-\mathbf{a}.\mathbf{c}+\frac{(\mathbf{a}.\mathbf{b})(\mathbf{b}.\mathbf{c})}{b}}
\end{displaymath}

\begin{equation}
  \label{eq:9}
  F = \frac{1}{G}
\end{equation}

Let
\begin{displaymath}
  G = -\mathbf{a}.\mathbf{c}+\frac{(\mathbf{a}.\mathbf{b})(\mathbf{b}.\mathbf{c})}{b^2}
\end{displaymath}

\begin{displaymath}
  H =  -\mathbf{a}.\mathbf{c}
\end{displaymath}

\begin{displaymath}
  J = \mathbf{a}.\mathbf{b}
\end{displaymath}

\begin{displaymath}
  K = \mathbf{b}.\mathbf{c}
\end{displaymath}

\begin{displaymath}
  L = \frac{1}{b^2}
\end{displaymath}

Differentiating  (\ref{eq:9})
\begin{displaymath}
  \frac{\partial F}{\partial x_{P_1}} = -\frac{1}{G^2}\frac{\partial G}{\partial x_{P_1}}
\end{displaymath}

%So now we have
%
%\begin{displaymath}
%  D = EF
%\end{displaymath}

Substituting for the derivative in (\ref{eq:df}):

\begin{displaymath}
  \frac{\partial \theta}{\partial x_{P_1}} = \frac{1}{1+D^2}[F\frac{\partial E}{\partial x_{P_1}} + E\frac{\partial F}{\partial x_{P_1}}]
\end{displaymath}


Also we have
\begin{displaymath}
  G = H + JKL
\end{displaymath}

Differentiating this: 

\begin{displaymath}
  \frac{\partial G}{\partial x_{P_1}} = \frac{\partial H}{\partial x_{P_1}} + JL\frac{\partial K}{\partial x_{P_1}} + KL\frac{\partial J}{\partial x_{P_1}} + JK\frac{\partial L}{\partial x_{P_1}}
\end{displaymath}

Let's look at the $H$, $J$, $K$ and $L$ derivatives:

\begin{displaymath}
    H = -\mathbf{a}.\mathbf{c} = -a_x c_x - a_y b_y - a_z c_z
\end{displaymath}

\begin{eqnarray*}
  \frac{\partial H}{\partial x_{P_1}} & = & c_x,\\
  \frac{\partial H}{\partial x_{P_2}} & = & -c_x,\\
  \frac{\partial H}{\partial x_{P_3}} & = & a_x,\\
  \frac{\partial H}{\partial x_{P_4}} & = & -a_x,\\
  \frac{\partial K}{\partial x_{P_1}} & = & 0,\\
  \frac{\partial K}{\partial x_{P_2}} & = & -c_x,\\
  \frac{\partial K}{\partial x_{P_3}} & = & c_x + b_x,\\
  \frac{\partial K}{\partial x_{P_4}} & = & b_x,\\
  \frac{\partial J}{\partial x_{P_1}} & = & -b_x,\\
  \frac{\partial J}{\partial x_{P_2}} & = & b_x - a_x,\\
  \frac{\partial J}{\partial x_{P_3}} & = & a_x,\\
  \frac{\partial J}{\partial x_{P_4}} & = & 0
\end{eqnarray*}

The $\frac{\partial b}{\partial x}$ terms are just like the bond
derivatives:

\begin{displaymath}
  \frac{\partial L}{\partial x_{P_1}} = \frac{\partial L}{\partial b} \frac{\partial b}{\partial x_{P_1}}
\end{displaymath}

\emph{i.e. }
\begin{eqnarray*}
  \frac{\partial L}{\partial x_{P_3}} & = &-\frac{2}{b^3} \frac{x_{P_3}-x_{P_2}}{b}\\
  & = &-\frac{2(x_{P_3}-x_{P_2})}{b^4}
\end{eqnarray*}

The derivative with respect to $x_{P_2}$ has the opposite sign.

Notice that $\mathbf{b}$ involves only atoms $P_2$ and $P_3$ so that
the derivates of $L$ with respect to the $P_1$ and $P_4$ coordinates are zero.

\subsection{$\frac{\partial E}{\partial x}$}
For the $\frac{\partial E}{\partial x}$ terms: 

Recall:
\begin{displaymath}
  E = \frac{\mathbf{a}.(\mathbf{b} \mathbf{\times}\mathbf{c})}{b}
\end{displaymath}

Let
\begin{displaymath}
  M = \mathbf{a}.(\mathbf{b} \mathbf{\times}\mathbf{c})
\end{displaymath}

\emph{i.e.}:
\begin{displaymath}
  E = \frac{M}{b}
\end{displaymath}

Differentiating that:
\begin{displaymath}
  \frac{\partial E}{\partial x_{P_3}} = -\frac{M}{b^2} \frac{\partial b}{\partial x_{P_3}} 
  +  \frac{1}{b} \frac{\partial M}{\partial x_{P_3}}
\end{displaymath}

Where, like bonds:
\begin{displaymath}
  \frac{\partial b}{\partial x_{P_3}} = \frac{x_{P_3}-x_{P_2}}{b}
\end{displaymath}
But note again, that the derivative of $b$ is zero for atoms $P_1$ and $P_4$.



\emph{i.e.} for atoms $P_2$ and $P_3$:
\begin{displaymath}
  \frac{\partial E}{\partial x_{P_3}} = -\frac{M(x_{P_3}-x_{P_2})}{b^3} + \frac{1}{b}\frac{\partial M}{\partial x_{P_3}}
\end{displaymath}

but for atoms $P_1$ and $P_4$:
\begin{displaymath}
  \frac{\partial E}{\partial x_{P_1}} =  \frac{1}{b}\frac{\partial M}{\partial x_{P_1}}
\end{displaymath}

\begin{displaymath}
  M = a_x(b_y c_z - b_z c_y) + a_y (b_z c_x - b_x c_z) + a_z (b_x c_y - b_y c_x)
\end{displaymath}

So here are the primitives of $M = \mathbf{a}.(\mathbf{b} \mathbf{\times}\mathbf{c})$

\begin{eqnarray*}
  \frac{\partial M}{\partial x_{P_1}} & = & -(b_y c_z - b_z c_y)\\
  \frac{\partial M}{\partial x_{P_2}} & = & (b_y c_z - b_z c_y) + (a_y c_z - a_z c_y)\\
  \frac{\partial M}{\partial x_{P_3}} & = & (a_z c_y - a_y c_z) + (b_y a_z - b_z a_y)\\
  \frac{\partial M}{\partial x_{P_4}} & = & (a_y b_z - a_z b_y)\\
  \frac{\partial M}{\partial y_{P_1}} & = & -(b_z c_x - b_x c_z)\\
  \frac{\partial M}{\partial y_{P_2}} & = & (b_z c_x - b_x c_z) + (a_z c_x - a_x c_z)\\
  \frac{\partial M}{\partial y_{P_3}} & = & -(a_z c_x - a_x c_z) + (b_z a_x - b_x a_z)\\
  \frac{\partial M}{\partial y_{P_4}} & = & -(b_z a_x - b_x a_z)\\
  \frac{\partial M}{\partial z_{P_1}} & = & -(b_x c_y - b_y c_x)\\
  \frac{\partial M}{\partial z_{P_2}} & = & (b_x c_y - b_y c_x) + (a_x c_y - a_y c_x)\\
  \frac{\partial M}{\partial z_{P_3}} & = & -(a_x c_y - a_y c_x) + (a_y b_x - a_x b_y)\\
  \frac{\partial M}{\partial z_{P_4}} & = & -(a_y b_x - a_x b_y)
\end{eqnarray*}

\subsection{Putting it together}

Combining, we get the following expression for the derivative of
$\theta$ in terms of the primitive derivates:
\begin{displaymath}
  \frac{\partial \theta}{\partial x_{P_1}} = \frac{1}{(1+\tan^2\theta)} \frac{\partial D}{\partial x_{P_1}}
\end{displaymath}

Where 
\begin{displaymath}
  \frac{\partial D}{\partial x_{P_1}} = [F \frac{\partial E}{\partial x_{P_1}} -\frac{E}{G^2} (\frac{\partial H}{\partial x_{P_1}} + JL \frac{\partial K}{\partial x_{P_1}} + KL  \frac{\partial J}{\partial x_{P_1}} + JK  \frac{\partial L}{\partial x_{P_1}})]  
\end{displaymath}









\chapter{Exported Functions}

@section File System Functions 
@subsection @code{(make-directory-maybe dir)}
@cindex @code{(make-directory-maybe dir)}
 
where: 
 @itemize 
     @item dir is a string
 @end itemize


@subsection @code{(set-show-paths-in-display-manager i)}
@cindex @code{(set-show-paths-in-display-manager i)}
 
where: 
 @itemize 
     @item i is an exact integer number
 @end itemize


@subsection @code{(show-paths-in-display-manager-state)}
@cindex @code{(show-paths-in-display-manager-state)}
 
@subsection @code{(add-coordinates-glob-extension ext)}
@cindex @code{(add-coordinates-glob-extension ext)}
 
where: 
 @itemize 
     @item ext is a string
 @end itemize


@subsection @code{(add-data-glob-extension ext)}
@cindex @code{(add-data-glob-extension ext)}
 
where: 
 @itemize 
     @item ext is a string
 @end itemize


@subsection @code{(add-dictionary-glob-extension ext)}
@cindex @code{(add-dictionary-glob-extension ext)}
 
where: 
 @itemize 
     @item ext is a string
 @end itemize


@subsection @code{(add-map-glob-extension ext)}
@cindex @code{(add-map-glob-extension ext)}
 
where: 
 @itemize 
     @item ext is a string
 @end itemize


@subsection @code{(set-sticky-sort-by-date)}
@cindex @code{(set-sticky-sort-by-date)}
 
@subsection @code{(set-filter-fileselection-filenames istate)}
@cindex @code{(set-filter-fileselection-filenames istate)}
 
where: 
 @itemize 
     @item istate is an exact integer number
 @end itemize


@subsection @code{(filter-fileselection-filenames-state)}
@cindex @code{(filter-fileselection-filenames-state)}
 

@section Widget Utilities 
@subsection @code{(info-dialog txt)}
@cindex @code{(info-dialog txt)}
 
where: 
 @itemize 
     @item txt is a string
 @end itemize



@section Widget Utilities 
@subsection @code{(manage-column-selector filename)}
@cindex @code{(manage-column-selector filename)}
 
where: 
 @itemize 
     @item filename is a string
 @end itemize



@section Molecule Info Functions 
@subsection @code{(chain-n-residues chain-id imol)}
@cindex @code{(chain-n-residues chain-id imol)}
 
where: 
 @itemize 
     @item chain-id is a string
     @item imol is an exact integer number
 @end itemize


@subsection @code{(molecule-centre-internal imol iaxis)}
@cindex @code{(molecule-centre-internal imol iaxis)}
 
where: 
 @itemize 
     @item imol is an exact integer number
     @item iaxis is an exact integer number
 @end itemize


@subsection @code{(n-chains imol)}
@cindex @code{(n-chains imol)}
 
where: 
 @itemize 
     @item imol is an exact integer number
 @end itemize


@subsection @code{(is-solvent-chain-p imol chain-id)}
@cindex @code{(is-solvent-chain-p imol chain-id)}
 
where: 
 @itemize 
     @item imol is an exact integer number
     @item chain-id is a string
 @end itemize


@subsection @code{(copy-molecule imol)}
@cindex @code{(copy-molecule imol)}
 
where: 
 @itemize 
     @item imol is an exact integer number
 @end itemize


@subsection @code{(exchange-chain-ids-for-seg-ids imol)}
@cindex @code{(exchange-chain-ids-for-seg-ids imol)}
 
where: 
 @itemize 
     @item imol is an exact integer number
 @end itemize



@section Library and Utility Functions 
@subsection @code{(coot-real-exit retval)}
@cindex @code{(coot-real-exit retval)}
 
where: 
 @itemize 
     @item retval is an exact integer number
 @end itemize


@subsection @code{(first-coords-imol)}
@cindex @code{(first-coords-imol)}
 

@section Graphics Utility Functions 
@subsection @code{(set-do-anti-aliasing state)}
@cindex @code{(set-do-anti-aliasing state)}
 
where: 
 @itemize 
     @item state is an exact integer number
 @end itemize


@subsection @code{(do-anti-aliasing-state)}
@cindex @code{(do-anti-aliasing-state)}
 
@subsection @code{(set-do-GL-lighting state)}
@cindex @code{(set-do-GL-lighting state)}
 
where: 
 @itemize 
     @item state is an exact integer number
 @end itemize


@subsection @code{(do-GL-lighting-state)}
@cindex @code{(do-GL-lighting-state)}
 
@subsection @code{(use-graphics-interface-state)}
@cindex @code{(use-graphics-interface-state)}
 
@subsection @code{(start-graphics-interface)}
@cindex @code{(start-graphics-interface)}
 
@subsection @code{(reset-view)}
@cindex @code{(reset-view)}
 
@subsection @code{(graphics-n-molecules)}
@cindex @code{(graphics-n-molecules)}
 
@subsection @code{(next-map-for-molecule imol)}
@cindex @code{(next-map-for-molecule imol)}
 
where: 
 @itemize 
     @item imol is an exact integer number
 @end itemize


@subsection @code{(toggle-idle-function)}
@cindex @code{(toggle-idle-function)}
 
@subsection @code{(set-idle-function-rotate-angle f)}
@cindex @code{(set-idle-function-rotate-angle f)}
 
where: 
 @itemize 
     @item f is an inexact number
 @end itemize


@subsection @code{(handle-read-draw-molecule filename)}
@cindex @code{(handle-read-draw-molecule filename)}
 
where: 
 @itemize 
     @item filename is a string
 @end itemize


@subsection @code{(read-pdb filename)}
@cindex @code{(read-pdb filename)}
 
where: 
 @itemize 
     @item filename is a string
 @end itemize


@subsection @code{(replace-fragment imol-target imol-fragment atom-selection)}
@cindex @code{(replace-fragment imol-target imol-fragment atom-selection)}
 
where: 
 @itemize 
     @item imol-target is an exact integer number
     @item imol-fragment is an exact integer number
     @item atom-selection is a string
 @end itemize


@subsection @code{(screendump-image filename)}
@cindex @code{(screendump-image filename)}
 
where: 
 @itemize 
     @item filename is a string
 @end itemize



@section Interface Preferences 
@subsection @code{(set-scroll-by-wheel-mouse istate)}
@cindex @code{(set-scroll-by-wheel-mouse istate)}
 
where: 
 @itemize 
     @item istate is an exact integer number
 @end itemize


@subsection @code{(scroll-by-wheel-mouse-state)}
@cindex @code{(scroll-by-wheel-mouse-state)}
 
@subsection @code{(set-default-initial-contour-level-for-map n-sigma)}
@cindex @code{(set-default-initial-contour-level-for-map n-sigma)}
 
where: 
 @itemize 
     @item n-sigma is an inexact number
 @end itemize


@subsection @code{(set-default-initial-contour-level-for-difference-map n-sigma)}
@cindex @code{(set-default-initial-contour-level-for-difference-map n-sigma)}
 
where: 
 @itemize 
     @item n-sigma is an inexact number
 @end itemize


@subsection @code{(print-view-matrix)}
@cindex @code{(print-view-matrix)}
 
@subsection @code{(get-view-matrix-element row col)}
@cindex @code{(get-view-matrix-element row col)}
 
where: 
 @itemize 
     @item row is an exact integer number
     @item col is an exact integer number
 @end itemize


@subsection @code{(get-view-quaternion-internal element)}
@cindex @code{(get-view-quaternion-internal element)}
 
where: 
 @itemize 
     @item element is an exact integer number
 @end itemize


@subsection @code{(set-view-quaternion i j k l)}
@cindex @code{(set-view-quaternion i j k l)}
 
where: 
 @itemize 
     @item i is an inexact number
     @item j is an inexact number
     @item k is an inexact number
     @item l is an inexact number
 @end itemize


@subsection @code{(set-fps-flag t)}
@cindex @code{(set-fps-flag t)}
 
where: 
 @itemize 
     @item t is an exact integer number
 @end itemize


@subsection @code{(get-fps-flag)}
@cindex @code{(get-fps-flag)}
 
@subsection @code{(set-show-origin-marker istate)}
@cindex @code{(set-show-origin-marker istate)}
 
where: 
 @itemize 
     @item istate is an exact integer number
 @end itemize


@subsection @code{(show-origin-marker-state)}
@cindex @code{(show-origin-marker-state)}
 
@subsection @code{(suck-model-fit-dialog)}
@cindex @code{(suck-model-fit-dialog)}
 
@subsection @code{(add-status-bar-text s)}
@cindex @code{(add-status-bar-text s)}
 
where: 
 @itemize 
     @item s is a string
 @end itemize


@subsection @code{(set-model-fit-refine-dialog-stays-on-top istate)}
@cindex @code{(set-model-fit-refine-dialog-stays-on-top istate)}
 
where: 
 @itemize 
     @item istate is an exact integer number
 @end itemize


@subsection @code{(model-fit-refine-dialog-stays-on-top-state)}
@cindex @code{(model-fit-refine-dialog-stays-on-top-state)}
 

@section Mouse Buttons 
@subsection @code{(quanta-buttons)}
@cindex @code{(quanta-buttons)}
 
@subsection @code{(quanta-like-zoom)}
@cindex @code{(quanta-like-zoom)}
 
@subsection @code{(set-control-key-for-rotate state)}
@cindex @code{(set-control-key-for-rotate state)}
 
where: 
 @itemize 
     @item state is an exact integer number
 @end itemize


@subsection @code{(control-key-for-rotate-state)}
@cindex @code{(control-key-for-rotate-state)}
 

@section Cursor Function 
@subsection @code{(normal-cursor)}
@cindex @code{(normal-cursor)}
 
@subsection @code{(fleur-cursor)}
@cindex @code{(fleur-cursor)}
 
@subsection @code{(pick-cursor-maybe)}
@cindex @code{(pick-cursor-maybe)}
 
@subsection @code{(rotate-cursor)}
@cindex @code{(rotate-cursor)}
 
@subsection @code{(set-pick-cursor-index icursor-index)}
@cindex @code{(set-pick-cursor-index icursor-index)}
 
where: 
 @itemize 
     @item icursor-index is an exact integer number
 @end itemize



@section Model/Fit/Refine Functions 
@subsection @code{(post-model-fit-refine-dialog)}
@cindex @code{(post-model-fit-refine-dialog)}
 
@subsection @code{(unset-model-fit-refine-dialog)}
@cindex @code{(unset-model-fit-refine-dialog)}
 
@subsection @code{(unset-refine-params-dialog)}
@cindex @code{(unset-refine-params-dialog)}
 
@subsection @code{(show-select-map-dialog)}
@cindex @code{(show-select-map-dialog)}
 
@subsection @code{(set-model-fit-refine-rotate-translate-zone-label txt)}
@cindex @code{(set-model-fit-refine-rotate-translate-zone-label txt)}
 
where: 
 @itemize 
     @item txt is a string
 @end itemize


@subsection @code{(set-model-fit-refine-place-atom-at-pointer-label txt)}
@cindex @code{(set-model-fit-refine-place-atom-at-pointer-label txt)}
 
where: 
 @itemize 
     @item txt is a string
 @end itemize


@subsection @code{(unset-other-modelling-tools-dialog)}
@cindex @code{(unset-other-modelling-tools-dialog)}
 
@subsection @code{(post-other-modelling-tools-dialog)}
@cindex @code{(post-other-modelling-tools-dialog)}
 

@section Backup Functions 
@subsection @code{(make-backup imol)}
@cindex @code{(make-backup imol)}
 
where: 
 @itemize 
     @item imol is an exact integer number
 @end itemize


@subsection @code{(turn-off-backup imol)}
@cindex @code{(turn-off-backup imol)}
 
where: 
 @itemize 
     @item imol is an exact integer number
 @end itemize


@subsection @code{(turn-on-backup imol)}
@cindex @code{(turn-on-backup imol)}
 
where: 
 @itemize 
     @item imol is an exact integer number
 @end itemize


@subsection @code{(backup-state imol)}
@cindex @code{(backup-state imol)}
 
where: 
 @itemize 
     @item imol is an exact integer number
 @end itemize


@subsection @code{(apply-undo)}
@cindex @code{(apply-undo)}
 
@subsection @code{(apply-redo)}
@cindex @code{(apply-redo)}
 
@subsection @code{(set-have-unsaved-changes imol)}
@cindex @code{(set-have-unsaved-changes imol)}
 
where: 
 @itemize 
     @item imol is an exact integer number
 @end itemize


@subsection @code{(set-undo-molecule imol)}
@cindex @code{(set-undo-molecule imol)}
 
where: 
 @itemize 
     @item imol is an exact integer number
 @end itemize


@subsection @code{(show-set-undo-molecule-chooser)}
@cindex @code{(show-set-undo-molecule-chooser)}
 
@subsection @code{(set-unpathed-backup-file-names state)}
@cindex @code{(set-unpathed-backup-file-names state)}
 
where: 
 @itemize 
     @item state is an exact integer number
 @end itemize


@subsection @code{(unpathed-backup-file-names-state)}
@cindex @code{(unpathed-backup-file-names-state)}
 

@section Recover Session Function 
@subsection @code{(recover-session)}
@cindex @code{(recover-session)}
 

@section Map Functions 
@subsection @code{(calc-phases-generic mtz-file-name)}
@cindex @code{(calc-phases-generic mtz-file-name)}
 
where: 
 @itemize 
     @item mtz-file-name is a string
 @end itemize


@subsection @code{(scroll-wheel-map)}
@cindex @code{(scroll-wheel-map)}
 
@subsection @code{(save-previous-map-colour imol)}
@cindex @code{(save-previous-map-colour imol)}
 
where: 
 @itemize 
     @item imol is an exact integer number
 @end itemize


@subsection @code{(restore-previous-map-colour imol)}
@cindex @code{(restore-previous-map-colour imol)}
 
where: 
 @itemize 
     @item imol is an exact integer number
 @end itemize


@subsection @code{(set-active-map-drag-flag t)}
@cindex @code{(set-active-map-drag-flag t)}
 
where: 
 @itemize 
     @item t is an exact integer number
 @end itemize


@subsection @code{(get-active-map-drag-flag)}
@cindex @code{(get-active-map-drag-flag)}
 
@subsection @code{(set-last-map-colour f1 f2 f3)}
@cindex @code{(set-last-map-colour f1 f2 f3)}
 
where: 
 @itemize 
     @item f1 is an unknown type
     @item f2 is an unknown type
     @item f3 is an unknown type
 @end itemize


@subsection @code{(set-map-colour imol red green blue)}
@cindex @code{(set-map-colour imol red green blue)}
 
where: 
 @itemize 
     @item imol is an exact integer number
     @item red is an inexact number
     @item green is an inexact number
     @item blue is an inexact number
 @end itemize


@subsection @code{( map-no gdouble[4])}
@cindex @code{( map-no gdouble[4])}
 
where: 
 @itemize 
     @item map-no is an exact integer number
     @item gdouble[4] is an unknown type
 @end itemize


@subsection @code{(handle-symmetry-colour-change mol gdouble[4])}
@cindex @code{(handle-symmetry-colour-change mol gdouble[4])}
 
where: 
 @itemize 
     @item mol is an exact integer number
     @item gdouble[4] is an unknown type
 @end itemize


@subsection @code{(set-last-map-sigma-step f)}
@cindex @code{(set-last-map-sigma-step f)}
 
where: 
 @itemize 
     @item f is an inexact number
 @end itemize


@subsection @code{(set-contour-by-sigma-step-by-mol f state imol)}
@cindex @code{(set-contour-by-sigma-step-by-mol f state imol)}
 
where: 
 @itemize 
     @item f is an inexact number
     @item state is an exact integer number
     @item imol is an exact integer number
 @end itemize


@subsection @code{(data-resolution imol)}
@cindex @code{(data-resolution imol)}
 
where: 
 @itemize 
     @item imol is an exact integer number
 @end itemize


@subsection @code{(export-map imol filename)}
@cindex @code{(export-map imol filename)}
 
where: 
 @itemize 
     @item imol is an exact integer number
     @item filename is a string
 @end itemize


@subsection @code{(rotate-map-round-screen-axis-x r-degrees)}
@cindex @code{(rotate-map-round-screen-axis-x r-degrees)}
 
where: 
 @itemize 
     @item r-degrees is an inexact number
 @end itemize


@subsection @code{(rotate-map-round-screen-axis-y r-degrees)}
@cindex @code{(rotate-map-round-screen-axis-y r-degrees)}
 
where: 
 @itemize 
     @item r-degrees is an inexact number
 @end itemize


@subsection @code{(rotate-map-round-screen-axis-z r-degrees)}
@cindex @code{(rotate-map-round-screen-axis-z r-degrees)}
 
where: 
 @itemize 
     @item r-degrees is an inexact number
 @end itemize



@section Density Increment 
@subsection @code{(set-iso-level-increment val)}
@cindex @code{(set-iso-level-increment val)}
 
where: 
 @itemize 
     @item val is an inexact number
 @end itemize


@subsection @code{(set-iso-level-increment-from-text text imol)}
@cindex @code{(set-iso-level-increment-from-text text imol)}
 
where: 
 @itemize 
     @item text is a string
     @item imol is an exact integer number
 @end itemize


@subsection @code{(set-diff-map-iso-level-increment val)}
@cindex @code{(set-diff-map-iso-level-increment val)}
 
where: 
 @itemize 
     @item val is an inexact number
 @end itemize


@subsection @code{(set-diff-map-iso-level-increment-from-text text imol)}
@cindex @code{(set-diff-map-iso-level-increment-from-text text imol)}
 
where: 
 @itemize 
     @item text is a string
     @item imol is an exact integer number
 @end itemize


@subsection @code{(set-map-sampling-rate-text text)}
@cindex @code{(set-map-sampling-rate-text text)}
 
where: 
 @itemize 
     @item text is a string
 @end itemize


@subsection @code{(set-map-sampling-rate r)}
@cindex @code{(set-map-sampling-rate r)}
 
where: 
 @itemize 
     @item r is an inexact number
 @end itemize


@subsection @code{(get-map-sampling-rate)}
@cindex @code{(get-map-sampling-rate)}
 
@subsection @code{(set-scrollable-map imol)}
@cindex @code{(set-scrollable-map imol)}
 
where: 
 @itemize 
     @item imol is an exact integer number
 @end itemize


@subsection @code{(change-contour-level is-increment)}
@cindex @code{(change-contour-level is-increment)}
 
where: 
 @itemize 
     @item is-increment is an exact integer number
 @end itemize


@subsection @code{(set-last-map-contour-level level)}
@cindex @code{(set-last-map-contour-level level)}
 
where: 
 @itemize 
     @item level is an inexact number
 @end itemize


@subsection @code{(set-last-map-contour-level-by-sigma n-sigma)}
@cindex @code{(set-last-map-contour-level-by-sigma n-sigma)}
 
where: 
 @itemize 
     @item n-sigma is an inexact number
 @end itemize


@subsection @code{(set-stop-scroll-diff-map i)}
@cindex @code{(set-stop-scroll-diff-map i)}
 
where: 
 @itemize 
     @item i is an exact integer number
 @end itemize


@subsection @code{(set-stop-scroll-iso-map i)}
@cindex @code{(set-stop-scroll-iso-map i)}
 
where: 
 @itemize 
     @item i is an exact integer number
 @end itemize


@subsection @code{(set-stop-scroll-iso-map-level f)}
@cindex @code{(set-stop-scroll-iso-map-level f)}
 
where: 
 @itemize 
     @item f is an inexact number
 @end itemize


@subsection @code{(set-stop-scroll-diff-map-level f)}
@cindex @code{(set-stop-scroll-diff-map-level f)}
 
where: 
 @itemize 
     @item f is an inexact number
 @end itemize


@subsection @code{(set-residue-density-fit-scale-factor f)}
@cindex @code{(set-residue-density-fit-scale-factor f)}
 
where: 
 @itemize 
     @item f is an inexact number
 @end itemize



@section Density Functions 
@subsection @code{(set-map-line-width w)}
@cindex @code{(set-map-line-width w)}
 
where: 
 @itemize 
     @item w is an exact integer number
 @end itemize


@subsection @code{(map-line-width-state)}
@cindex @code{(map-line-width-state)}
 
@subsection @code{(mtz-file-has-phases-p mtz-file-name)}
@cindex @code{(mtz-file-has-phases-p mtz-file-name)}
 
where: 
 @itemize 
     @item mtz-file-name is a string
 @end itemize


@subsection @code{(is-mtz-file-p filename)}
@cindex @code{(is-mtz-file-p filename)}
 
where: 
 @itemize 
     @item filename is a string
 @end itemize


@subsection @code{(auto-read-make-and-draw-maps filename)}
@cindex @code{(auto-read-make-and-draw-maps filename)}
 
where: 
 @itemize 
     @item filename is a string
 @end itemize


@subsection @code{(set-auto-read-do-difference-map-too i)}
@cindex @code{(set-auto-read-do-difference-map-too i)}
 
where: 
 @itemize 
     @item i is an exact integer number
 @end itemize


@subsection @code{(auto-read-do-difference-map-too-state)}
@cindex @code{(auto-read-do-difference-map-too-state)}
 
@subsection @code{(set-density-size-from-widget text)}
@cindex @code{(set-density-size-from-widget text)}
 
where: 
 @itemize 
     @item text is a string
 @end itemize


@subsection @code{(set-map-radius f)}
@cindex @code{(set-map-radius f)}
 
where: 
 @itemize 
     @item f is an inexact number
 @end itemize


@subsection @code{(set-density-size f)}
@cindex @code{(set-density-size f)}
 
where: 
 @itemize 
     @item f is an inexact number
 @end itemize


@subsection @code{(set-map-radius-slider-max f)}
@cindex @code{(set-map-radius-slider-max f)}
 
where: 
 @itemize 
     @item f is an inexact number
 @end itemize


@subsection @code{(set-display-intro-string str)}
@cindex @code{(set-display-intro-string str)}
 
where: 
 @itemize 
     @item str is a string
 @end itemize


@subsection @code{(set-esoteric-depth-cue istate)}
@cindex @code{(set-esoteric-depth-cue istate)}
 
where: 
 @itemize 
     @item istate is an exact integer number
 @end itemize


@subsection @code{(esoteric-depth-cue-state)}
@cindex @code{(esoteric-depth-cue-state)}
 
@subsection @code{(set-swap-difference-map-colours i)}
@cindex @code{(set-swap-difference-map-colours i)}
 
where: 
 @itemize 
     @item i is an exact integer number
 @end itemize


@subsection @code{(set-map-is-difference-map imol)}
@cindex @code{(set-map-is-difference-map imol)}
 
where: 
 @itemize 
     @item imol is an exact integer number
 @end itemize


@subsection @code{(another-level)}
@cindex @code{(another-level)}
 
@subsection @code{(another-level-from-map-molecule-number imap)}
@cindex @code{(another-level-from-map-molecule-number imap)}
 
where: 
 @itemize 
     @item imap is an exact integer number
 @end itemize


@subsection @code{(residue-density-fit-scale-factor)}
@cindex @code{(residue-density-fit-scale-factor)}
 

@section Parameters from map 
@subsection @code{(mtz-use-weight-for-map imol-map)}
@cindex @code{(mtz-use-weight-for-map imol-map)}
 
where: 
 @itemize 
     @item imol-map is an exact integer number
 @end itemize



@section PDB Functions 
@subsection @code{(write-pdb-file imol file-name)}
@cindex @code{(write-pdb-file imol file-name)}
 
where: 
 @itemize 
     @item imol is an exact integer number
     @item file-name is a string
 @end itemize



@section Refmac Functions 
@subsection @code{(set-refmac-molecule imol)}
@cindex @code{(set-refmac-molecule imol)}
 
where: 
 @itemize 
     @item imol is an exact integer number
 @end itemize


@subsection @code{(set-refmac-counter imol refmac-count)}
@cindex @code{(set-refmac-counter imol refmac-count)}
 
where: 
 @itemize 
     @item imol is an exact integer number
     @item refmac-count is an exact integer number
 @end itemize


@subsection @code{(swap-map-colours imol1 imol2)}
@cindex @code{(swap-map-colours imol1 imol2)}
 
where: 
 @itemize 
     @item imol1 is an exact integer number
     @item imol2 is an exact integer number
 @end itemize


@subsection @code{(set-keep-map-colour-after-refmac istate)}
@cindex @code{(set-keep-map-colour-after-refmac istate)}
 
where: 
 @itemize 
     @item istate is an exact integer number
 @end itemize


@subsection @code{(keep-map-colour-after-refmac-state)}
@cindex @code{(keep-map-colour-after-refmac-state)}
 

@section Symmetry Functions 
@subsection @code{(set-symmetry-size-from-widget text)}
@cindex @code{(set-symmetry-size-from-widget text)}
 
where: 
 @itemize 
     @item text is a string
 @end itemize


@subsection @code{(set-symmetry-size f)}
@cindex @code{(set-symmetry-size f)}
 
where: 
 @itemize 
     @item f is an inexact number
 @end itemize


@subsection @code{(get-show-symmetry)}
@cindex @code{(get-show-symmetry)}
 
@subsection @code{(set-show-symmetry-master state)}
@cindex @code{(set-show-symmetry-master state)}
 
where: 
 @itemize 
     @item state is an exact integer number
 @end itemize


@subsection @code{(set-show-symmetry-molecule mol-no state)}
@cindex @code{(set-show-symmetry-molecule mol-no state)}
 
where: 
 @itemize 
     @item mol-no is an exact integer number
     @item state is an exact integer number
 @end itemize


@subsection @code{(symmetry-as-calphas mol-no state)}
@cindex @code{(symmetry-as-calphas mol-no state)}
 
where: 
 @itemize 
     @item mol-no is an exact integer number
     @item state is an exact integer number
 @end itemize


@subsection @code{(get-symmetry-as-calphas-state imol)}
@cindex @code{(get-symmetry-as-calphas-state imol)}
 
where: 
 @itemize 
     @item imol is an exact integer number
 @end itemize


@subsection @code{(set-symmetry-molecule-rotate-colour-map imol state)}
@cindex @code{(set-symmetry-molecule-rotate-colour-map imol state)}
 
where: 
 @itemize 
     @item imol is an exact integer number
     @item state is an exact integer number
 @end itemize


@subsection @code{(symmetry-molecule-rotate-colour-map-state imol)}
@cindex @code{(symmetry-molecule-rotate-colour-map-state imol)}
 
where: 
 @itemize 
     @item imol is an exact integer number
 @end itemize


@subsection @code{(set-symmetry-colour-by-symop imol state)}
@cindex @code{(set-symmetry-colour-by-symop imol state)}
 
where: 
 @itemize 
     @item imol is an exact integer number
     @item state is an exact integer number
 @end itemize


@subsection @code{(set-symmetry-whole-chain imol state)}
@cindex @code{(set-symmetry-whole-chain imol state)}
 
where: 
 @itemize 
     @item imol is an exact integer number
     @item state is an exact integer number
 @end itemize


@subsection @code{(set-symmetry-atom-labels-expanded state)}
@cindex @code{(set-symmetry-atom-labels-expanded state)}
 
where: 
 @itemize 
     @item state is an exact integer number
 @end itemize


@subsection @code{(has-unit-cell-state imol)}
@cindex @code{(has-unit-cell-state imol)}
 
where: 
 @itemize 
     @item imol is an exact integer number
 @end itemize


@subsection @code{(setup-save-symmetry-coords)}
@cindex @code{(setup-save-symmetry-coords)}
 
@subsection @code{(set-space-group imol spg)}
@cindex @code{(set-space-group imol spg)}
 
where: 
 @itemize 
     @item imol is an exact integer number
     @item spg is a string
 @end itemize


@subsection @code{(set-symmetry-shift-search-size shift)}
@cindex @code{(set-symmetry-shift-search-size shift)}
 
where: 
 @itemize 
     @item shift is an exact integer number
 @end itemize



@section File Selection Functions 
@subsection @code{(clear-refmac-ccp4i-project)}
@cindex @code{(clear-refmac-ccp4i-project)}
 

@section History Functions 
@subsection @code{(print-all-history-in-scheme)}
@cindex @code{(print-all-history-in-scheme)}
 
@subsection @code{(print-all-history-in-python)}
@cindex @code{(print-all-history-in-python)}
 
@subsection @code{(set-console-display-commands-state istate)}
@cindex @code{(set-console-display-commands-state istate)}
 
where: 
 @itemize 
     @item istate is an exact integer number
 @end itemize


@subsection @code{(save-state)}
@cindex @code{(save-state)}
 
@subsection @code{(save-state-file filename)}
@cindex @code{(save-state-file filename)}
 
where: 
 @itemize 
     @item filename is a string
 @end itemize


@subsection @code{(set-save-state-file-name filename)}
@cindex @code{(set-save-state-file-name filename)}
 
where: 
 @itemize 
     @item filename is a string
 @end itemize


@subsection @code{(set-run-state-file-status istat)}
@cindex @code{(set-run-state-file-status istat)}
 
where: 
 @itemize 
     @item istat is an exact integer number
 @end itemize


@subsection @code{(run-state-file)}
@cindex @code{(run-state-file)}
 
@subsection @code{(run-state-file-maybe)}
@cindex @code{(run-state-file-maybe)}
 
@subsection @code{(vt-surface mode)}
@cindex @code{(vt-surface mode)}
 
where: 
 @itemize 
     @item mode is an exact integer number
 @end itemize


@subsection @code{(vt-surface-status)}
@cindex @code{(vt-surface-status)}
 

@section Clipping Functions 
@subsection @code{(do-clipping1-activate)}
@cindex @code{(do-clipping1-activate)}
 
@subsection @code{(set-clipping-back v)}
@cindex @code{(set-clipping-back v)}
 
where: 
 @itemize 
     @item v is an inexact number
 @end itemize


@subsection @code{(set-clipping-front v)}
@cindex @code{(set-clipping-front v)}
 
where: 
 @itemize 
     @item v is an inexact number
 @end itemize



@section Unit Cell 
@subsection @code{(get-show-unit-cell imol)}
@cindex @code{(get-show-unit-cell imol)}
 
where: 
 @itemize 
     @item imol is an exact integer number
 @end itemize


@subsection @code{(set-show-unit-cells-all istate)}
@cindex @code{(set-show-unit-cells-all istate)}
 
where: 
 @itemize 
     @item istate is an exact integer number
 @end itemize


@subsection @code{(set-show-unit-cell imol istate)}
@cindex @code{(set-show-unit-cell imol istate)}
 
where: 
 @itemize 
     @item imol is an exact integer number
     @item istate is an exact integer number
 @end itemize



@section Colour 
@subsection @code{(set-symmetry-colour-merge mol-no v)}
@cindex @code{(set-symmetry-colour-merge mol-no v)}
 
where: 
 @itemize 
     @item mol-no is an exact integer number
     @item v is an inexact number
 @end itemize


@subsection @code{(set-colour-map-rotation-on-read-pdb f)}
@cindex @code{(set-colour-map-rotation-on-read-pdb f)}
 
where: 
 @itemize 
     @item f is an inexact number
 @end itemize


@subsection @code{(set-colour-map-rotation-on-read-pdb-flag i)}
@cindex @code{(set-colour-map-rotation-on-read-pdb-flag i)}
 
where: 
 @itemize 
     @item i is an exact integer number
 @end itemize


@subsection @code{(set-colour-map-rotation-on-read-pdb-c-only-flag i)}
@cindex @code{(set-colour-map-rotation-on-read-pdb-c-only-flag i)}
 
where: 
 @itemize 
     @item i is an exact integer number
 @end itemize


@subsection @code{(set-colour-by-chain imol)}
@cindex @code{(set-colour-by-chain imol)}
 
where: 
 @itemize 
     @item imol is an exact integer number
 @end itemize


@subsection @code{(set-colour-by-molecule imol)}
@cindex @code{(set-colour-by-molecule imol)}
 
where: 
 @itemize 
     @item imol is an exact integer number
 @end itemize


@subsection @code{(set-colour-map-rotation-for-map f)}
@cindex @code{(set-colour-map-rotation-for-map f)}
 
where: 
 @itemize 
     @item f is an inexact number
 @end itemize


@subsection @code{(set-molecule-bonds-colour-map-rotation imol theta)}
@cindex @code{(set-molecule-bonds-colour-map-rotation imol theta)}
 
where: 
 @itemize 
     @item imol is an exact integer number
     @item theta is an inexact number
 @end itemize


@subsection @code{(get-molecule-bonds-colour-map-rotation imol)}
@cindex @code{(get-molecule-bonds-colour-map-rotation imol)}
 
where: 
 @itemize 
     @item imol is an exact integer number
 @end itemize



@section Anisotropic Atoms 
@subsection @code{(get-limit-aniso)}
@cindex @code{(get-limit-aniso)}
 
@subsection @code{(get-show-limit-aniso)}
@cindex @code{(get-show-limit-aniso)}
 
@subsection @code{(get-show-aniso)}
@cindex @code{(get-show-aniso)}
 
@subsection @code{(set-limit-aniso state)}
@cindex @code{(set-limit-aniso state)}
 
where: 
 @itemize 
     @item state is an exact integer number
 @end itemize


@subsection @code{(set-aniso-limit-size-from-widget text)}
@cindex @code{(set-aniso-limit-size-from-widget text)}
 
where: 
 @itemize 
     @item text is a string
 @end itemize


@subsection @code{(set-show-aniso state)}
@cindex @code{(set-show-aniso state)}
 
where: 
 @itemize 
     @item state is an exact integer number
 @end itemize


@subsection @code{(set-aniso-probability f)}
@cindex @code{(set-aniso-probability f)}
 
where: 
 @itemize 
     @item f is an inexact number
 @end itemize


@subsection @code{(get-aniso-probability)}
@cindex @code{(get-aniso-probability)}
 

@section Display Functions 
@subsection @code{(set-graphics-window-size x-size y-size)}
@cindex @code{(set-graphics-window-size x-size y-size)}
 
where: 
 @itemize 
     @item x-size is an exact integer number
     @item y-size is an exact integer number
 @end itemize


@subsection @code{(set-graphics-window-position x-pos y-pos)}
@cindex @code{(set-graphics-window-position x-pos y-pos)}
 
where: 
 @itemize 
     @item x-pos is an exact integer number
     @item y-pos is an exact integer number
 @end itemize


@subsection @code{(store-graphics-window-position x-pos y-pos)}
@cindex @code{(store-graphics-window-position x-pos y-pos)}
 
where: 
 @itemize 
     @item x-pos is an exact integer number
     @item y-pos is an exact integer number
 @end itemize


@subsection @code{(graphics-draw)}
@cindex @code{(graphics-draw)}
 
@subsection @code{(hardware-stereo-mode)}
@cindex @code{(hardware-stereo-mode)}
 
@subsection @code{(stereo-mode-state)}
@cindex @code{(stereo-mode-state)}
 
@subsection @code{(mono-mode)}
@cindex @code{(mono-mode)}
 
@subsection @code{(side-by-side-stereo-mode use-wall-eye-mode)}
@cindex @code{(side-by-side-stereo-mode use-wall-eye-mode)}
 
where: 
 @itemize 
     @item use-wall-eye-mode is an exact integer number
 @end itemize


@subsection @code{(set-hardware-stereo-angle-factor f)}
@cindex @code{(set-hardware-stereo-angle-factor f)}
 
where: 
 @itemize 
     @item f is an inexact number
 @end itemize


@subsection @code{(hardware-stereo-angle-factor-state)}
@cindex @code{(hardware-stereo-angle-factor-state)}
 
@subsection @code{(set-model-fit-refine-dialog-position x-pos y-pos)}
@cindex @code{(set-model-fit-refine-dialog-position x-pos y-pos)}
 
where: 
 @itemize 
     @item x-pos is an exact integer number
     @item y-pos is an exact integer number
 @end itemize


@subsection @code{(set-display-control-dialog-position x-pos y-pos)}
@cindex @code{(set-display-control-dialog-position x-pos y-pos)}
 
where: 
 @itemize 
     @item x-pos is an exact integer number
     @item y-pos is an exact integer number
 @end itemize


@subsection @code{(set-go-to-atom-window-position x-pos y-pos)}
@cindex @code{(set-go-to-atom-window-position x-pos y-pos)}
 
where: 
 @itemize 
     @item x-pos is an exact integer number
     @item y-pos is an exact integer number
 @end itemize


@subsection @code{(set-delete-dialog-position x-pos y-pos)}
@cindex @code{(set-delete-dialog-position x-pos y-pos)}
 
where: 
 @itemize 
     @item x-pos is an exact integer number
     @item y-pos is an exact integer number
 @end itemize


@subsection @code{(set-rotate-translate-dialog-position x-pos y-pos)}
@cindex @code{(set-rotate-translate-dialog-position x-pos y-pos)}
 
where: 
 @itemize 
     @item x-pos is an exact integer number
     @item y-pos is an exact integer number
 @end itemize


@subsection @code{(set-accept-reject-dialog-position x-pos y-pos)}
@cindex @code{(set-accept-reject-dialog-position x-pos y-pos)}
 
where: 
 @itemize 
     @item x-pos is an exact integer number
     @item y-pos is an exact integer number
 @end itemize


@subsection @code{(set-ramachandran-plot-dialog-position x-pos y-pos)}
@cindex @code{(set-ramachandran-plot-dialog-position x-pos y-pos)}
 
where: 
 @itemize 
     @item x-pos is an exact integer number
     @item y-pos is an exact integer number
 @end itemize



@section Smooth Scrolling 
@subsection @code{(set-smooth-scroll-flag v)}
@cindex @code{(set-smooth-scroll-flag v)}
 
where: 
 @itemize 
     @item v is an exact integer number
 @end itemize


@subsection @code{(get-smooth-scroll)}
@cindex @code{(get-smooth-scroll)}
 
@subsection @code{(set-smooth-scroll-steps-str t)}
@cindex @code{(set-smooth-scroll-steps-str t)}
 
where: 
 @itemize 
     @item t is a string
 @end itemize


@subsection @code{(set-smooth-scroll-steps i)}
@cindex @code{(set-smooth-scroll-steps i)}
 
where: 
 @itemize 
     @item i is an exact integer number
 @end itemize


@subsection @code{(set-smooth-scroll-limit-str t)}
@cindex @code{(set-smooth-scroll-limit-str t)}
 
where: 
 @itemize 
     @item t is a string
 @end itemize


@subsection @code{(set-smooth-scroll-limit lim)}
@cindex @code{(set-smooth-scroll-limit lim)}
 
where: 
 @itemize 
     @item lim is an inexact number
 @end itemize



@section Font Size 
@subsection @code{(set-font-size i)}
@cindex @code{(set-font-size i)}
 
where: 
 @itemize 
     @item i is an exact integer number
 @end itemize


@subsection @code{(get-font-size)}
@cindex @code{(get-font-size)}
 

@section Rotation Centre 
@subsection @code{(set-rotation-centre-size-from-widget text)}
@cindex @code{(set-rotation-centre-size-from-widget text)}
 
where: 
 @itemize 
     @item text is an unknown type
 @end itemize


@subsection @code{(set-rotation-centre-size f)}
@cindex @code{(set-rotation-centre-size f)}
 
where: 
 @itemize 
     @item f is an inexact number
 @end itemize


@subsection @code{(recentre-on-read-pdb)}
@cindex @code{(recentre-on-read-pdb)}
 
@subsection @code{(set-recentre-on-read-pdb int)}
@cindex @code{(set-recentre-on-read-pdb int)}
 
where: 
 @itemize 
     @item int is an exact integer number
 @end itemize


@subsection @code{(set-rotation-centre x y z)}
@cindex @code{(set-rotation-centre x y z)}
 
where: 
 @itemize 
     @item x is an inexact number
     @item y is an inexact number
     @item z is an inexact number
 @end itemize


@subsection @code{(set-rotation-centre-internal x y z)}
@cindex @code{(set-rotation-centre-internal x y z)}
 
where: 
 @itemize 
     @item x is an inexact number
     @item y is an inexact number
     @item z is an inexact number
 @end itemize


@subsection @code{(rotation-centre-position axis)}
@cindex @code{(rotation-centre-position axis)}
 
where: 
 @itemize 
     @item axis is an exact integer number
 @end itemize



@section Orthogonal Axes 
@subsection @code{(set-draw-axes i)}
@cindex @code{(set-draw-axes i)}
 
where: 
 @itemize 
     @item i is an exact integer number
 @end itemize



@section Atom Selection Utilities 
@subsection @code{(atom-index imol chain-id iresno atom-id)}
@cindex @code{(atom-index imol chain-id iresno atom-id)}
 
where: 
 @itemize 
     @item imol is an exact integer number
     @item chain-id is a string
     @item iresno is an exact integer number
     @item atom-id is a string
 @end itemize


@subsection @code{(median-temperature-factor imol)}
@cindex @code{(median-temperature-factor imol)}
 
where: 
 @itemize 
     @item imol is an exact integer number
 @end itemize


@subsection @code{(average-temperature-factor imol)}
@cindex @code{(average-temperature-factor imol)}
 
where: 
 @itemize 
     @item imol is an exact integer number
 @end itemize


@subsection @code{(clear-pending-picks)}
@cindex @code{(clear-pending-picks)}
 
@subsection @code{(set-default-temperature-factor-for-new-atoms new-b)}
@cindex @code{(set-default-temperature-factor-for-new-atoms new-b)}
 
where: 
 @itemize 
     @item new-b is an inexact number
 @end itemize


@subsection @code{(default-new-atoms-b-factor)}
@cindex @code{(default-new-atoms-b-factor)}
 

@section Skeletonization 
@subsection @code{(skel-greer-on)}
@cindex @code{(skel-greer-on)}
 
@subsection @code{(skel-greer-off)}
@cindex @code{(skel-greer-off)}
 
@subsection @code{(skel-foadi-on)}
@cindex @code{(skel-foadi-on)}
 
@subsection @code{(skel-foadi-off)}
@cindex @code{(skel-foadi-off)}
 
@subsection @code{(skeletonize-map prune-flag imol)}
@cindex @code{(skeletonize-map prune-flag imol)}
 
where: 
 @itemize 
     @item prune-flag is an exact integer number
     @item imol is an exact integer number
 @end itemize


@subsection @code{(unskeletonize-map imol)}
@cindex @code{(unskeletonize-map imol)}
 
where: 
 @itemize 
     @item imol is an exact integer number
 @end itemize


@subsection @code{(set-initial-map-for-skeletonize)}
@cindex @code{(set-initial-map-for-skeletonize)}
 
@subsection @code{(set-max-skeleton-search-depth v)}
@cindex @code{(set-max-skeleton-search-depth v)}
 
where: 
 @itemize 
     @item v is an exact integer number
 @end itemize


@subsection @code{(set-skeletonization-level-from-widget txt)}
@cindex @code{(set-skeletonization-level-from-widget txt)}
 
where: 
 @itemize 
     @item txt is a string
 @end itemize


@subsection @code{(set-skeleton-box-size-from-widget txt)}
@cindex @code{(set-skeleton-box-size-from-widget txt)}
 
where: 
 @itemize 
     @item txt is a string
 @end itemize


@subsection @code{(set-skeleton-box-size f)}
@cindex @code{(set-skeleton-box-size f)}
 
where: 
 @itemize 
     @item f is an inexact number
 @end itemize



@section Skeleton Colour 
@subsection @code{(handle-skeleton-colour-change mol map-col)}
@cindex @code{(handle-skeleton-colour-change mol map-col)}
 
where: 
 @itemize 
     @item mol is an exact integer number
     @item map-col is an unknown type
 @end itemize


@subsection @code{(set-skeleton-colour imol r g b)}
@cindex @code{(set-skeleton-colour imol r g b)}
 
where: 
 @itemize 
     @item imol is an exact integer number
     @item r is an inexact number
     @item g is an inexact number
     @item b is an inexact number
 @end itemize



@section Read CCP4 Map 
@subsection @code{(handle-read-ccp4-map filename is-diff-map-flag)}
@cindex @code{(handle-read-ccp4-map filename is-diff-map-flag)}
 
where: 
 @itemize 
     @item filename is an unknown type
     @item is-diff-map-flag is an exact integer number
 @end itemize



@section Save Coordinates 
@subsection @code{(save-coordinates imol filename)}
@cindex @code{(save-coordinates imol filename)}
 
where: 
 @itemize 
     @item imol is an exact integer number
     @item filename is a string
 @end itemize


@subsection @code{(set-save-coordinates-in-original-directory i)}
@cindex @code{(set-save-coordinates-in-original-directory i)}
 
where: 
 @itemize 
     @item i is an exact integer number
 @end itemize


@subsection @code{(save-molecule-number-from-option-menu)}
@cindex @code{(save-molecule-number-from-option-menu)}
 
@subsection @code{(set-save-molecule-number imol)}
@cindex @code{(set-save-molecule-number imol)}
 
where: 
 @itemize 
     @item imol is an exact integer number
 @end itemize



@section Read Phases File Functions 
@subsection @code{(possible-cell-symm-for-phs-file)}
@cindex @code{(possible-cell-symm-for-phs-file)}
 

@section Graphics Move 
@subsection @code{(undo-last-move)}
@cindex @code{(undo-last-move)}
 
@subsection @code{(translate-molecule-by imol x y z)}
@cindex @code{(translate-molecule-by imol x y z)}
 
where: 
 @itemize 
     @item imol is an exact integer number
     @item x is an inexact number
     @item y is an inexact number
     @item z is an inexact number
 @end itemize



@section Go To Atom Widget Functions 
@subsection @code{(post-go-to-atom-window)}
@cindex @code{(post-go-to-atom-window)}
 
@subsection @code{(atom-spec-to-atom-index mol chain resno atom-name)}
@cindex @code{(atom-spec-to-atom-index mol chain resno atom-name)}
 
where: 
 @itemize 
     @item mol is an exact integer number
     @item chain is a string
     @item resno is an exact integer number
     @item atom-name is a string
 @end itemize


@subsection @code{(update-go-to-atom-window-on-changed-mol imol)}
@cindex @code{(update-go-to-atom-window-on-changed-mol imol)}
 
where: 
 @itemize 
     @item imol is an exact integer number
 @end itemize


@subsection @code{(update-go-to-atom-window-on-new-mol)}
@cindex @code{(update-go-to-atom-window-on-new-mol)}
 
@subsection @code{(set-go-to-atom-molecule imol)}
@cindex @code{(set-go-to-atom-molecule imol)}
 
where: 
 @itemize 
     @item imol is an exact integer number
 @end itemize


@subsection @code{(unset-go-to-atom-widget)}
@cindex @code{(unset-go-to-atom-widget)}
 

@section AutoBuilding functions (Defunct) 
@subsection @code{(autobuild-ca-on)}
@cindex @code{(autobuild-ca-on)}
 
@subsection @code{(autobuild-ca-off)}
@cindex @code{(autobuild-ca-off)}
 
@subsection @code{(test-fragment)}
@cindex @code{(test-fragment)}
 
@subsection @code{(do-skeleton-prune)}
@cindex @code{(do-skeleton-prune)}
 
@subsection @code{(test-function i j)}
@cindex @code{(test-function i j)}
 
where: 
 @itemize 
     @item i is an exact integer number
     @item j is an exact integer number
 @end itemize



@section Map and Molecule Control 
@subsection @code{(post-display-control-window)}
@cindex @code{(post-display-control-window)}
 
@subsection @code{(add-map-display-control-widgets)}
@cindex @code{(add-map-display-control-widgets)}
 
@subsection @code{(add-mol-display-control-widgets)}
@cindex @code{(add-mol-display-control-widgets)}
 
@subsection @code{(add-map-and-mol-display-control-widgets)}
@cindex @code{(add-map-and-mol-display-control-widgets)}
 
@subsection @code{(reset-graphics-display-control-window)}
@cindex @code{(reset-graphics-display-control-window)}
 
@subsection @code{(toggle-display-map imol imap)}
@cindex @code{(toggle-display-map imol imap)}
 
where: 
 @itemize 
     @item imol is an exact integer number
     @item imap is an exact integer number
 @end itemize


@subsection @code{(toggle-display-mol imol)}
@cindex @code{(toggle-display-mol imol)}
 
where: 
 @itemize 
     @item imol is an exact integer number
 @end itemize


@subsection @code{( imol)}
@cindex @code{( imol)}
 
where: 
 @itemize 
     @item imol is an exact integer number
 @end itemize


@subsection @code{(mol-is-displayed imol)}
@cindex @code{(mol-is-displayed imol)}
 
where: 
 @itemize 
     @item imol is an exact integer number
 @end itemize


@subsection @code{(mol-is-active imol)}
@cindex @code{(mol-is-active imol)}
 
where: 
 @itemize 
     @item imol is an exact integer number
 @end itemize


@subsection @code{(map-is-displayed imol)}
@cindex @code{(map-is-displayed imol)}
 
where: 
 @itemize 
     @item imol is an exact integer number
 @end itemize



@section Merge Molecules 
@subsection @code{(do-merge-molecules-gui)}
@cindex @code{(do-merge-molecules-gui)}
 

@section Mutate Sequence and Loops GUI 

@section Align and Mutate 
@subsection @code{(align-and-mutate imol chain-id fasta-maybe)}
@cindex @code{(align-and-mutate imol chain-id fasta-maybe)}
 
where: 
 @itemize 
     @item imol is an exact integer number
     @item chain-id is a string
     @item fasta-maybe is a string
 @end itemize



@section Renumber Residue Range 
@subsection @code{(change-residue-number imol chain-id current-resno current-inscode new-resno new-inscode)}
@cindex @code{(change-residue-number imol chain-id current-resno current-inscode new-resno new-inscode)}
 
where: 
 @itemize 
     @item imol is an exact integer number
     @item chain-id is a string
     @item current-resno is an exact integer number
     @item current-inscode is a string
     @item new-resno is an exact integer number
     @item new-inscode is a string
 @end itemize



@section Change Chain ID 

@section Scripting 
@subsection @code{(post-scripting-window)}
@cindex @code{(post-scripting-window)}
 
@subsection @code{(run-command-line-scripts)}
@cindex @code{(run-command-line-scripts)}
 
@subsection @code{(set-guile-gui-loaded-flag)}
@cindex @code{(set-guile-gui-loaded-flag)}
 
@subsection @code{(set-found-coot-gui)}
@cindex @code{(set-found-coot-gui)}
 
@subsection @code{(get-monomer three-letter-code)}
@cindex @code{(get-monomer three-letter-code)}
 
where: 
 @itemize 
     @item three-letter-code is a string
 @end itemize


@subsection @code{( filename)}
@cindex @code{( filename)}
 
where: 
 @itemize 
     @item filename is a string
 @end itemize


@subsection @code{( filename)}
@cindex @code{( filename)}
 
where: 
 @itemize 
     @item filename is a string
 @end itemize


@subsection @code{(run-python-script filename)}
@cindex @code{(run-python-script filename)}
 
where: 
 @itemize 
     @item filename is a string
 @end itemize



@section Regularization and Refinement 
@subsection @code{(do-regularize state)}
@cindex @code{(do-regularize state)}
 
where: 
 @itemize 
     @item state is an exact integer number
 @end itemize


@subsection @code{(do-refine state)}
@cindex @code{(do-refine state)}
 
where: 
 @itemize 
     @item state is an exact integer number
 @end itemize


@subsection @code{(add-planar-peptide-restraints)}
@cindex @code{(add-planar-peptide-restraints)}
 
@subsection @code{(remove-planar-peptide-restraints)}
@cindex @code{(remove-planar-peptide-restraints)}
 
@subsection @code{(add-omega-torsion-restriants)}
@cindex @code{(add-omega-torsion-restriants)}
 
@subsection @code{(remove-omega-torsion-restriants)}
@cindex @code{(remove-omega-torsion-restriants)}
 
@subsection @code{(set-refinement-immediate-replacement istate)}
@cindex @code{(set-refinement-immediate-replacement istate)}
 
where: 
 @itemize 
     @item istate is an exact integer number
 @end itemize


@subsection @code{(refinement-immediate-replacement-state)}
@cindex @code{(refinement-immediate-replacement-state)}
 
@subsection @code{(set-residue-selection-flash-frames-number i)}
@cindex @code{(set-residue-selection-flash-frames-number i)}
 
where: 
 @itemize 
     @item i is an exact integer number
 @end itemize


@subsection @code{(accept-regularizement)}
@cindex @code{(accept-regularizement)}
 
@subsection @code{(clear-up-moving-atoms)}
@cindex @code{(clear-up-moving-atoms)}
 
@subsection @code{(clear-moving-atoms-object)}
@cindex @code{(clear-moving-atoms-object)}
 
@subsection @code{(do-peptide-torsions-toggle)}
@cindex @code{(do-peptide-torsions-toggle)}
 
@subsection @code{(set-refine-with-torsion-restraints istate)}
@cindex @code{(set-refine-with-torsion-restraints istate)}
 
where: 
 @itemize 
     @item istate is an exact integer number
 @end itemize


@subsection @code{(set-refine-params-phi-psi-restraints-type restraints-type)}
@cindex @code{(set-refine-params-phi-psi-restraints-type restraints-type)}
 
where: 
 @itemize 
     @item restraints-type is an exact integer number
 @end itemize


@subsection @code{(set-matrix f)}
@cindex @code{(set-matrix f)}
 
where: 
 @itemize 
     @item f is an inexact number
 @end itemize


@subsection @code{(matrix-state)}
@cindex @code{(matrix-state)}
 
@subsection @code{(set-refine-auto-range-step i)}
@cindex @code{(set-refine-auto-range-step i)}
 
where: 
 @itemize 
     @item i is an exact integer number
 @end itemize


@subsection @code{(set-refine-max-residues n)}
@cindex @code{(set-refine-max-residues n)}
 
where: 
 @itemize 
     @item n is an exact integer number
 @end itemize


@subsection @code{(refine-zone-atom-index-define imol ind1 ind2)}
@cindex @code{(refine-zone-atom-index-define imol ind1 ind2)}
 
where: 
 @itemize 
     @item imol is an exact integer number
     @item ind1 is an exact integer number
     @item ind2 is an exact integer number
 @end itemize


@subsection @code{(refine-zone imol chain-id resno1 resno2 altconf)}
@cindex @code{(refine-zone imol chain-id resno1 resno2 altconf)}
 
where: 
 @itemize 
     @item imol is an exact integer number
     @item chain-id is a string
     @item resno1 is an exact integer number
     @item resno2 is an exact integer number
     @item altconf is a string
 @end itemize


@subsection @code{(refine-auto-range imol chain-id resno1 altconf)}
@cindex @code{(refine-auto-range imol chain-id resno1 altconf)}
 
where: 
 @itemize 
     @item imol is an exact integer number
     @item chain-id is a string
     @item resno1 is an exact integer number
     @item altconf is a string
 @end itemize


@subsection @code{(set-dragged-refinement-steps-per-frame v)}
@cindex @code{(set-dragged-refinement-steps-per-frame v)}
 
where: 
 @itemize 
     @item v is an exact integer number
 @end itemize


@subsection @code{(dragged-refinement-steps-per-frame)}
@cindex @code{(dragged-refinement-steps-per-frame)}
 
@subsection @code{(set-refinement-refine-per-frame istate)}
@cindex @code{(set-refinement-refine-per-frame istate)}
 
where: 
 @itemize 
     @item istate is an exact integer number
 @end itemize


@subsection @code{(refinement-refine-per-frame-state)}
@cindex @code{(refinement-refine-per-frame-state)}
 
@subsection @code{(set-fix-chiral-volumes-before-refinement istate)}
@cindex @code{(set-fix-chiral-volumes-before-refinement istate)}
 
where: 
 @itemize 
     @item istate is an exact integer number
 @end itemize


@subsection @code{(check-chiral-volumes imol)}
@cindex @code{(check-chiral-volumes imol)}
 
where: 
 @itemize 
     @item imol is an exact integer number
 @end itemize


@subsection @code{(set-secondary-structure-restraints-type itype)}
@cindex @code{(set-secondary-structure-restraints-type itype)}
 
where: 
 @itemize 
     @item itype is an exact integer number
 @end itemize


@subsection @code{(secondary-structure-restraints-type)}
@cindex @code{(secondary-structure-restraints-type)}
 
@subsection @code{(imol-refinement-map)}
@cindex @code{(imol-refinement-map)}
 
@subsection @code{(set-imol-refinement-map imol)}
@cindex @code{(set-imol-refinement-map imol)}
 
where: 
 @itemize 
     @item imol is an exact integer number
 @end itemize


@subsection @code{(does-residue-exist-p imol chain-id resno inscode)}
@cindex @code{(does-residue-exist-p imol chain-id resno inscode)}
 
where: 
 @itemize 
     @item imol is an exact integer number
     @item chain-id is a string
     @item resno is an exact integer number
     @item inscode is a string
 @end itemize


@subsection @code{(fix-nomenclature-errors imol)}
@cindex @code{(fix-nomenclature-errors imol)}
 
where: 
 @itemize 
     @item imol is an exact integer number
 @end itemize



@section Atom Info 

@section Residue Info 
@subsection @code{(do-residue-info)}
@cindex @code{(do-residue-info)}
 
@subsection @code{( atom-index imol)}
@cindex @code{( atom-index imol)}
 
where: 
 @itemize 
     @item atom-index is an exact integer number
     @item imol is an exact integer number
 @end itemize


@subsection @code{(output-residue-info-as-text atom-index imol)}
@cindex @code{(output-residue-info-as-text atom-index imol)}
 
where: 
 @itemize 
     @item atom-index is an exact integer number
     @item imol is an exact integer number
 @end itemize


@subsection @code{(do-distance-define)}
@cindex @code{(do-distance-define)}
 
@subsection @code{(do-angle-define)}
@cindex @code{(do-angle-define)}
 
@subsection @code{(do-torsion-define)}
@cindex @code{(do-torsion-define)}
 
@subsection @code{(residue-info-apply-all-checkbutton-toggled)}
@cindex @code{(residue-info-apply-all-checkbutton-toggled)}
 
@subsection @code{(clear-residue-info-edit-list)}
@cindex @code{(clear-residue-info-edit-list)}
 
@subsection @code{(unset-residue-info-widget)}
@cindex @code{(unset-residue-info-widget)}
 
@subsection @code{(clear-simple-distances)}
@cindex @code{(clear-simple-distances)}
 
@subsection @code{(clear-last-simple-distance)}
@cindex @code{(clear-last-simple-distance)}
 

@section Residue Environment Functions 

@section Pointer Functions 
@subsection @code{(set-show-pointer-distances istate)}
@cindex @code{(set-show-pointer-distances istate)}
 
where: 
 @itemize 
     @item istate is an exact integer number
 @end itemize



@section Zoom Functions 
@subsection @code{(scale-zoom f)}
@cindex @code{(scale-zoom f)}
 
where: 
 @itemize 
     @item f is an inexact number
 @end itemize


@subsection @code{(scale-zoom-internal f)}
@cindex @code{(scale-zoom-internal f)}
 
where: 
 @itemize 
     @item f is an inexact number
 @end itemize


@subsection @code{(zoom-factor)}
@cindex @code{(zoom-factor)}
 
@subsection @code{(set-smooth-scroll-do-zoom i)}
@cindex @code{(set-smooth-scroll-do-zoom i)}
 
where: 
 @itemize 
     @item i is an exact integer number
 @end itemize


@subsection @code{(smooth-scroll-do-zoom)}
@cindex @code{(smooth-scroll-do-zoom)}
 
@subsection @code{(smooth-scroll-zoom-limit)}
@cindex @code{(smooth-scroll-zoom-limit)}
 
@subsection @code{(set-smooth-scroll-zoom-limit f)}
@cindex @code{(set-smooth-scroll-zoom-limit f)}
 
where: 
 @itemize 
     @item f is an inexact number
 @end itemize


@subsection @code{(handle-cns-data-file filename)}
@cindex @code{(handle-cns-data-file filename)}
 
where: 
 @itemize 
     @item filename is a string
 @end itemize



@section mmCIF Functions 
@subsection @code{(auto-read-cif-data-with-phases filename)}
@cindex @code{(auto-read-cif-data-with-phases filename)}
 
where: 
 @itemize 
     @item filename is a string
 @end itemize


@subsection @code{(read-cif-data-with-phases-sigmaa filename)}
@cindex @code{(read-cif-data-with-phases-sigmaa filename)}
 
where: 
 @itemize 
     @item filename is a string
 @end itemize


@subsection @code{(read-cif-data-with-phases-diff-sigmaa filename)}
@cindex @code{(read-cif-data-with-phases-diff-sigmaa filename)}
 
where: 
 @itemize 
     @item filename is a string
 @end itemize


@subsection @code{(read-cif-data filename imol-coords)}
@cindex @code{(read-cif-data filename imol-coords)}
 
where: 
 @itemize 
     @item filename is a string
     @item imol-coords is an exact integer number
 @end itemize


@subsection @code{(read-cif-data-2fofc-map filename imol-coords)}
@cindex @code{(read-cif-data-2fofc-map filename imol-coords)}
 
where: 
 @itemize 
     @item filename is a string
     @item imol-coords is an exact integer number
 @end itemize


@subsection @code{(read-cif-data-fofc-map filename imol-coords)}
@cindex @code{(read-cif-data-fofc-map filename imol-coords)}
 
where: 
 @itemize 
     @item filename is a string
     @item imol-coords is an exact integer number
 @end itemize


@subsection @code{(read-cif-data-with-phases-fo-fc filename)}
@cindex @code{(read-cif-data-with-phases-fo-fc filename)}
 
where: 
 @itemize 
     @item filename is a string
 @end itemize


@subsection @code{(read-cif-data-with-phases-2fo-fc filename)}
@cindex @code{(read-cif-data-with-phases-2fo-fc filename)}
 
where: 
 @itemize 
     @item filename is a string
 @end itemize


@subsection @code{(read-cif-data-with-phases-fo-alpha-calc filename)}
@cindex @code{(read-cif-data-with-phases-fo-alpha-calc filename)}
 
where: 
 @itemize 
     @item filename is a string
 @end itemize


@subsection @code{(handle-cif-dictionary filename)}
@cindex @code{(handle-cif-dictionary filename)}
 
where: 
 @itemize 
     @item filename is a string
 @end itemize


@subsection @code{(read-cif-dictionary filename)}
@cindex @code{(read-cif-dictionary filename)}
 
where: 
 @itemize 
     @item filename is a string
 @end itemize


@subsection @code{(write-connectivity monomer-name filename)}
@cindex @code{(write-connectivity monomer-name filename)}
 
where: 
 @itemize 
     @item monomer-name is an unknown type
     @item filename is a string
 @end itemize


@subsection @code{(import-all-refmac-cifs)}
@cindex @code{(import-all-refmac-cifs)}
 

@section SHELXL Functions 
@subsection @code{(read-shelx-ins-file filename)}
@cindex @code{(read-shelx-ins-file filename)}
 
where: 
 @itemize 
     @item filename is a string
 @end itemize


@subsection @code{(write-shelx-ins-file imol filename)}
@cindex @code{(write-shelx-ins-file imol filename)}
 
where: 
 @itemize 
     @item imol is an exact integer number
     @item filename is a string
 @end itemize


@subsection @code{(handle-shelx-fcf-file-internal filename)}
@cindex @code{(handle-shelx-fcf-file-internal filename)}
 
where: 
 @itemize 
     @item filename is a string
 @end itemize



@section Validation Functions 
@subsection @code{(deviant-geometry imol)}
@cindex @code{(deviant-geometry imol)}
 
where: 
 @itemize 
     @item imol is an exact integer number
 @end itemize


@subsection @code{(is-valid-model-molecule imol)}
@cindex @code{(is-valid-model-molecule imol)}
 
where: 
 @itemize 
     @item imol is an exact integer number
 @end itemize


@subsection @code{(is-valid-map-molecule imol)}
@cindex @code{(is-valid-map-molecule imol)}
 
where: 
 @itemize 
     @item imol is an exact integer number
 @end itemize


@subsection @code{(difference-map-peaks imol imol-coords level do-positive-level-flag do-negative-level-flag)}
@cindex @code{(difference-map-peaks imol imol-coords level do-positive-level-flag do-negative-level-flag)}
 
where: 
 @itemize 
     @item imol is an exact integer number
     @item imol-coords is an exact integer number
     @item level is an inexact number
     @item do-positive-level-flag is an exact integer number
     @item do-negative-level-flag is an exact integer number
 @end itemize


@subsection @code{(clear-diff-map-peaks)}
@cindex @code{(clear-diff-map-peaks)}
 
@subsection @code{(gln-asn-b-factor-outliers imol)}
@cindex @code{(gln-asn-b-factor-outliers imol)}
 
where: 
 @itemize 
     @item imol is an exact integer number
 @end itemize



@section Ramachandran Plot Functions 
@subsection @code{(do-ramachandran-plot imol)}
@cindex @code{(do-ramachandran-plot imol)}
 
where: 
 @itemize 
     @item imol is an exact integer number
 @end itemize


@subsection @code{(add-on-rama-choices)}
@cindex @code{(add-on-rama-choices)}
 
@subsection @code{(set-moving-atoms phi psi)}
@cindex @code{(set-moving-atoms phi psi)}
 
where: 
 @itemize 
     @item phi is an unknown type
     @item psi is an unknown type
 @end itemize


@subsection @code{(accept-phi-psi-moving-atoms)}
@cindex @code{(accept-phi-psi-moving-atoms)}
 
@subsection @code{(setup-edit-phi-psi state)}
@cindex @code{(setup-edit-phi-psi state)}
 
where: 
 @itemize 
     @item state is an exact integer number
 @end itemize


@subsection @code{(destroy-edit-backbone-rama-plot)}
@cindex @code{(destroy-edit-backbone-rama-plot)}
 
@subsection @code{(ramachandran-plot-differences imol1 imol2)}
@cindex @code{(ramachandran-plot-differences imol1 imol2)}
 
where: 
 @itemize 
     @item imol1 is an exact integer number
     @item imol2 is an exact integer number
 @end itemize


@subsection @code{(do-sequence-view imol)}
@cindex @code{(do-sequence-view imol)}
 
where: 
 @itemize 
     @item imol is an exact integer number
 @end itemize


@subsection @code{(add-on-sequence-view-choices)}
@cindex @code{(add-on-sequence-view-choices)}
 
@subsection @code{(change-peptide-carbonyl-by angle)}
@cindex @code{(change-peptide-carbonyl-by angle)}
 
where: 
 @itemize 
     @item angle is an unknown type
 @end itemize


@subsection @code{(change-peptide-peptide-by angle)}
@cindex @code{(change-peptide-peptide-by angle)}
 
where: 
 @itemize 
     @item angle is an unknown type
 @end itemize


@subsection @code{(execute-setup-backbone-torsion-edit imol atom-index)}
@cindex @code{(execute-setup-backbone-torsion-edit imol atom-index)}
 
where: 
 @itemize 
     @item imol is an exact integer number
     @item atom-index is an exact integer number
 @end itemize


@subsection @code{(setup-backbone-torsion-edit state)}
@cindex @code{(setup-backbone-torsion-edit state)}
 
where: 
 @itemize 
     @item state is an exact integer number
 @end itemize


@subsection @code{(set-backbone-torsion-peptide-button-start-pos ix iy)}
@cindex @code{(set-backbone-torsion-peptide-button-start-pos ix iy)}
 
where: 
 @itemize 
     @item ix is an exact integer number
     @item iy is an exact integer number
 @end itemize


@subsection @code{(change-peptide-peptide-by-current-button-pos ix iy)}
@cindex @code{(change-peptide-peptide-by-current-button-pos ix iy)}
 
where: 
 @itemize 
     @item ix is an exact integer number
     @item iy is an exact integer number
 @end itemize


@subsection @code{(set-backbone-torsion-carbonyl-button-start-pos ix iy)}
@cindex @code{(set-backbone-torsion-carbonyl-button-start-pos ix iy)}
 
where: 
 @itemize 
     @item ix is an exact integer number
     @item iy is an exact integer number
 @end itemize


@subsection @code{(change-peptide-carbonyl-by-current-button-pos ix iy)}
@cindex @code{(change-peptide-carbonyl-by-current-button-pos ix iy)}
 
where: 
 @itemize 
     @item ix is an exact integer number
     @item iy is an exact integer number
 @end itemize



@section Atom Labelling 
@subsection @code{(add-atom-label imol chain-id iresno atom-id)}
@cindex @code{(add-atom-label imol chain-id iresno atom-id)}
 
where: 
 @itemize 
     @item imol is an exact integer number
     @item chain-id is a string
     @item iresno is an exact integer number
     @item atom-id is a string
 @end itemize


@subsection @code{(remove-atom-label imol chain-id iresno atom-id)}
@cindex @code{(remove-atom-label imol chain-id iresno atom-id)}
 
where: 
 @itemize 
     @item imol is an exact integer number
     @item chain-id is a string
     @item iresno is an exact integer number
     @item atom-id is a string
 @end itemize


@subsection @code{(remove-all-atom-labels)}
@cindex @code{(remove-all-atom-labels)}
 
@subsection @code{(set-label-on-recentre-flag i)}
@cindex @code{(set-label-on-recentre-flag i)}
 
where: 
 @itemize 
     @item i is an exact integer number
 @end itemize


@subsection @code{(centre-atom-label-status)}
@cindex @code{(centre-atom-label-status)}
 
@subsection @code{(set-brief-atom-labels istat)}
@cindex @code{(set-brief-atom-labels istat)}
 
where: 
 @itemize 
     @item istat is an exact integer number
 @end itemize


@subsection @code{(brief-atom-labels-state)}
@cindex @code{(brief-atom-labels-state)}
 

@section Screen Rotation 
@subsection @code{(rotate-y-scene nsteps stepsize)}
@cindex @code{(rotate-y-scene nsteps stepsize)}
 
where: 
 @itemize 
     @item nsteps is an exact integer number
     @item stepsize is an inexact number
 @end itemize


@subsection @code{(rotate-x-scene nsteps stepsize)}
@cindex @code{(rotate-x-scene nsteps stepsize)}
 
where: 
 @itemize 
     @item nsteps is an exact integer number
     @item stepsize is an inexact number
 @end itemize


@subsection @code{(rotate-z-scene nsteps stepsize)}
@cindex @code{(rotate-z-scene nsteps stepsize)}
 
where: 
 @itemize 
     @item nsteps is an exact integer number
     @item stepsize is an inexact number
 @end itemize



@section Background Colour 
@subsection @code{(set-background-colour red green blue)}
@cindex @code{(set-background-colour red green blue)}
 
where: 
 @itemize 
     @item red is an unknown type
     @item green is an unknown type
     @item blue is an unknown type
 @end itemize


@subsection @code{(background-is-black-p)}
@cindex @code{(background-is-black-p)}
 

@section Ligand Fitting Functions 
@subsection @code{(set-ligand-acceptable-fit-fraction f)}
@cindex @code{(set-ligand-acceptable-fit-fraction f)}
 
where: 
 @itemize 
     @item f is an inexact number
 @end itemize


@subsection @code{(set-ligand-cluster-sigma-level f)}
@cindex @code{(set-ligand-cluster-sigma-level f)}
 
where: 
 @itemize 
     @item f is an inexact number
 @end itemize


@subsection @code{(set-ligand-flexible-ligand-n-samples i)}
@cindex @code{(set-ligand-flexible-ligand-n-samples i)}
 
where: 
 @itemize 
     @item i is an exact integer number
 @end itemize


@subsection @code{(set-ligand-verbose-reporting i)}
@cindex @code{(set-ligand-verbose-reporting i)}
 
where: 
 @itemize 
     @item i is an exact integer number
 @end itemize


@subsection @code{(set-find-ligand-n-top-ligands n)}
@cindex @code{(set-find-ligand-n-top-ligands n)}
 
where: 
 @itemize 
     @item n is an exact integer number
 @end itemize


@subsection @code{(set-find-ligand-mask-waters istate)}
@cindex @code{(set-find-ligand-mask-waters istate)}
 
where: 
 @itemize 
     @item istate is an exact integer number
 @end itemize


@subsection @code{(set-ligand-search-protein-molecule imol)}
@cindex @code{(set-ligand-search-protein-molecule imol)}
 
where: 
 @itemize 
     @item imol is an exact integer number
 @end itemize


@subsection @code{(set-ligand-search-map-molecule imol-map)}
@cindex @code{(set-ligand-search-map-molecule imol-map)}
 
where: 
 @itemize 
     @item imol-map is an exact integer number
 @end itemize


@subsection @code{(add-ligand-search-ligand-molecule imol-ligand)}
@cindex @code{(add-ligand-search-ligand-molecule imol-ligand)}
 
where: 
 @itemize 
     @item imol-ligand is an exact integer number
 @end itemize


@subsection @code{(add-ligand-search-wiggly-ligand-molecule imol-ligand)}
@cindex @code{(add-ligand-search-wiggly-ligand-molecule imol-ligand)}
 
where: 
 @itemize 
     @item imol-ligand is an exact integer number
 @end itemize


@subsection @code{(execute-ligand-search)}
@cindex @code{(execute-ligand-search)}
 
@subsection @code{(ligand-expert)}
@cindex @code{(ligand-expert)}
 
@subsection @code{(do-find-ligands-dialog)}
@cindex @code{(do-find-ligands-dialog)}
 

@section Water Fitting Functions 
@subsection @code{(renumber-waters imol)}
@cindex @code{(renumber-waters imol)}
 
where: 
 @itemize 
     @item imol is an exact integer number
 @end itemize


@subsection @code{(set-value-for-find-waters-sigma-cut-off f)}
@cindex @code{(set-value-for-find-waters-sigma-cut-off f)}
 
where: 
 @itemize 
     @item f is an inexact number
 @end itemize


@subsection @code{(set-ligand-water-spherical-variance-limit f)}
@cindex @code{(set-ligand-water-spherical-variance-limit f)}
 
where: 
 @itemize 
     @item f is an inexact number
 @end itemize


@subsection @code{(set-ligand-water-to-protein-distance-limits f1 f2)}
@cindex @code{(set-ligand-water-to-protein-distance-limits f1 f2)}
 
where: 
 @itemize 
     @item f1 is an inexact number
     @item f2 is an inexact number
 @end itemize


@subsection @code{(set-ligand-water-n-cycles i)}
@cindex @code{(set-ligand-water-n-cycles i)}
 
where: 
 @itemize 
     @item i is an exact integer number
 @end itemize


@subsection @code{(set-write-peaksearched-waters)}
@cindex @code{(set-write-peaksearched-waters)}
 
@subsection @code{(execute-find-blobs imol-model imol-for-map cut-off interactive-flag)}
@cindex @code{(execute-find-blobs imol-model imol-for-map cut-off interactive-flag)}
 
where: 
 @itemize 
     @item imol-model is an exact integer number
     @item imol-for-map is an exact integer number
     @item cut-off is an inexact number
     @item interactive-flag is an exact integer number
 @end itemize



@section Bond Representation 
@subsection @code{(set-default-bond-thickness t)}
@cindex @code{(set-default-bond-thickness t)}
 
where: 
 @itemize 
     @item t is an exact integer number
 @end itemize


@subsection @code{(set-bond-thickness imol t)}
@cindex @code{(set-bond-thickness imol t)}
 
where: 
 @itemize 
     @item imol is an exact integer number
     @item t is an inexact number
 @end itemize


@subsection @code{(set-bond-thickness-intermediate-atoms t)}
@cindex @code{(set-bond-thickness-intermediate-atoms t)}
 
where: 
 @itemize 
     @item t is an inexact number
 @end itemize


@subsection @code{(set-unbonded-atom-star-size f)}
@cindex @code{(set-unbonded-atom-star-size f)}
 
where: 
 @itemize 
     @item f is an inexact number
 @end itemize


@subsection @code{(set-draw-zero-occ-markers status)}
@cindex @code{(set-draw-zero-occ-markers status)}
 
where: 
 @itemize 
     @item status is an exact integer number
 @end itemize


@subsection @code{(set-draw-hydrogens imol istat)}
@cindex @code{(set-draw-hydrogens imol istat)}
 
where: 
 @itemize 
     @item imol is an exact integer number
     @item istat is an exact integer number
 @end itemize


@subsection @code{( imol)}
@cindex @code{( imol)}
 
where: 
 @itemize 
     @item imol is an exact integer number
 @end itemize


@subsection @code{( imol)}
@cindex @code{( imol)}
 
where: 
 @itemize 
     @item imol is an exact integer number
 @end itemize


@subsection @code{(graphics-to-bonds-no-waters-representation imol)}
@cindex @code{(graphics-to-bonds-no-waters-representation imol)}
 
where: 
 @itemize 
     @item imol is an exact integer number
 @end itemize


@subsection @code{(graphics-to-bonds-representation mol)}
@cindex @code{(graphics-to-bonds-representation mol)}
 
where: 
 @itemize 
     @item mol is an exact integer number
 @end itemize


@subsection @code{(graphics-to-ca-plus-ligands-sec-struct-representation imol)}
@cindex @code{(graphics-to-ca-plus-ligands-sec-struct-representation imol)}
 
where: 
 @itemize 
     @item imol is an exact integer number
 @end itemize


@subsection @code{(graphics-to-sec-struct-bonds-representation imol)}
@cindex @code{(graphics-to-sec-struct-bonds-representation imol)}
 
where: 
 @itemize 
     @item imol is an exact integer number
 @end itemize


@subsection @code{(graphics-to-rainbow-representation imol)}
@cindex @code{(graphics-to-rainbow-representation imol)}
 
where: 
 @itemize 
     @item imol is an exact integer number
 @end itemize


@subsection @code{(graphics-to-b-factor-representation imol)}
@cindex @code{(graphics-to-b-factor-representation imol)}
 
where: 
 @itemize 
     @item imol is an exact integer number
 @end itemize


@subsection @code{(graphics-to-occupancy-represenation imol)}
@cindex @code{(graphics-to-occupancy-represenation imol)}
 
where: 
 @itemize 
     @item imol is an exact integer number
 @end itemize


@subsection @code{(graphics-molecule-bond-type imol)}
@cindex @code{(graphics-molecule-bond-type imol)}
 
where: 
 @itemize 
     @item imol is an exact integer number
 @end itemize


@subsection @code{(clear-ball-and-stick imol)}
@cindex @code{(clear-ball-and-stick imol)}
 
where: 
 @itemize 
     @item imol is an exact integer number
 @end itemize


@subsection @code{(clear-dots imol dots-handle)}
@cindex @code{(clear-dots imol dots-handle)}
 
where: 
 @itemize 
     @item imol is an exact integer number
     @item dots-handle is an exact integer number
 @end itemize


@subsection @code{(n-dots-sets imol)}
@cindex @code{(n-dots-sets imol)}
 
where: 
 @itemize 
     @item imol is an exact integer number
 @end itemize



@section Pep-flip 
@subsection @code{(do-pepflip state)}
@cindex @code{(do-pepflip state)}
 
where: 
 @itemize 
     @item state is an exact integer number
 @end itemize


@subsection @code{(pepflip ires chain-id imol)}
@cindex @code{(pepflip ires chain-id imol)}
 
where: 
 @itemize 
     @item ires is an exact integer number
     @item chain-id is a string
     @item imol is an exact integer number
 @end itemize



@section Rigid Body Refinement 
@subsection @code{(do-rigid-body-refine state)}
@cindex @code{(do-rigid-body-refine state)}
 
where: 
 @itemize 
     @item state is an exact integer number
 @end itemize


@subsection @code{(execute-rigid-body-refine auto-range-flag)}
@cindex @code{(execute-rigid-body-refine auto-range-flag)}
 
where: 
 @itemize 
     @item auto-range-flag is an exact integer number
 @end itemize


@subsection @code{(set-rigid-body-fit-acceptable-fit-fraction f)}
@cindex @code{(set-rigid-body-fit-acceptable-fit-fraction f)}
 
where: 
 @itemize 
     @item f is an inexact number
 @end itemize



@section Dynamic Map 
@subsection @code{(toggle-dynamic-map-display-size)}
@cindex @code{(toggle-dynamic-map-display-size)}
 
@subsection @code{(toggle-dynamic-map-sampling)}
@cindex @code{(toggle-dynamic-map-sampling)}
 
@subsection @code{(set-dynamic-map-size-display-on)}
@cindex @code{(set-dynamic-map-size-display-on)}
 
@subsection @code{(set-dynamic-map-size-display-off)}
@cindex @code{(set-dynamic-map-size-display-off)}
 
@subsection @code{(set-dynamic-map-sampling-on)}
@cindex @code{(set-dynamic-map-sampling-on)}
 
@subsection @code{(set-dynamic-map-sampling-off)}
@cindex @code{(set-dynamic-map-sampling-off)}
 
@subsection @code{(set-dynamic-map-zoom-offset i)}
@cindex @code{(set-dynamic-map-zoom-offset i)}
 
where: 
 @itemize 
     @item i is an exact integer number
 @end itemize



@section Add Terminal Residue Functions 
@subsection @code{(do-add-terminal-residue state)}
@cindex @code{(do-add-terminal-residue state)}
 
where: 
 @itemize 
     @item state is an exact integer number
 @end itemize


@subsection @code{(set-add-terminal-residue-n-phi-psi-trials n)}
@cindex @code{(set-add-terminal-residue-n-phi-psi-trials n)}
 
where: 
 @itemize 
     @item n is an exact integer number
 @end itemize


@subsection @code{(set-add-terminal-residue-add-other-residue-flag i)}
@cindex @code{(set-add-terminal-residue-add-other-residue-flag i)}
 
where: 
 @itemize 
     @item i is an exact integer number
 @end itemize


@subsection @code{(set-terminal-residue-do-rigid-body-refine v)}
@cindex @code{(set-terminal-residue-do-rigid-body-refine v)}
 
where: 
 @itemize 
     @item v is an exact integer number
 @end itemize


@subsection @code{(add-terminal-residue-immediate-addition-state)}
@cindex @code{(add-terminal-residue-immediate-addition-state)}
 
@subsection @code{(set-add-terminal-residue-immediate-addition i)}
@cindex @code{(set-add-terminal-residue-immediate-addition i)}
 
where: 
 @itemize 
     @item i is an exact integer number
 @end itemize


@subsection @code{(set-add-terminal-residue-default-residue-type type)}
@cindex @code{(set-add-terminal-residue-default-residue-type type)}
 
where: 
 @itemize 
     @item type is a string
 @end itemize


@subsection @code{(set-add-terminal-residue-do-post-refine istat)}
@cindex @code{(set-add-terminal-residue-do-post-refine istat)}
 
where: 
 @itemize 
     @item istat is an exact integer number
 @end itemize



@section Delete Residues 
@subsection @code{(delete-atom-by-atom-index imol index do-delete-dialog)}
@cindex @code{(delete-atom-by-atom-index imol index do-delete-dialog)}
 
where: 
 @itemize 
     @item imol is an exact integer number
     @item index is an exact integer number
     @item do-delete-dialog is an exact integer number
 @end itemize


@subsection @code{(delete-residue-by-atom-index imol index do-delete-dialog)}
@cindex @code{(delete-residue-by-atom-index imol index do-delete-dialog)}
 
where: 
 @itemize 
     @item imol is an exact integer number
     @item index is an exact integer number
     @item do-delete-dialog is an exact integer number
 @end itemize


@subsection @code{(delete-residue-hydrogens-by-atom-index imol index do-delete-dialog)}
@cindex @code{(delete-residue-hydrogens-by-atom-index imol index do-delete-dialog)}
 
where: 
 @itemize 
     @item imol is an exact integer number
     @item index is an exact integer number
     @item do-delete-dialog is an exact integer number
 @end itemize


@subsection @code{(delete-residue-range imol chain-id resno-start end-resno)}
@cindex @code{(delete-residue-range imol chain-id resno-start end-resno)}
 
where: 
 @itemize 
     @item imol is an exact integer number
     @item chain-id is a string
     @item resno-start is an exact integer number
     @item end-resno is an exact integer number
 @end itemize


@subsection @code{(delete-residue imol chain-id resno inscode)}
@cindex @code{(delete-residue imol chain-id resno inscode)}
 
where: 
 @itemize 
     @item imol is an exact integer number
     @item chain-id is a string
     @item resno is an exact integer number
     @item inscode is a string
 @end itemize


@subsection @code{(delete-residue-with-altconf imol chain-id resno inscode altloc)}
@cindex @code{(delete-residue-with-altconf imol chain-id resno inscode altloc)}
 
where: 
 @itemize 
     @item imol is an exact integer number
     @item chain-id is a string
     @item resno is an exact integer number
     @item inscode is a string
     @item altloc is a string
 @end itemize


@subsection @code{(delete-residue-hydrogens imol chain-id resno inscode altloc)}
@cindex @code{(delete-residue-hydrogens imol chain-id resno inscode altloc)}
 
where: 
 @itemize 
     @item imol is an exact integer number
     @item chain-id is a string
     @item resno is an exact integer number
     @item inscode is a string
     @item altloc is a string
 @end itemize


@subsection @code{(delete-atom imol chain-id resno at-name altloc)}
@cindex @code{(delete-atom imol chain-id resno at-name altloc)}
 
where: 
 @itemize 
     @item imol is an exact integer number
     @item chain-id is a string
     @item resno is an exact integer number
     @item at-name is a string
     @item altloc is a string
 @end itemize


@subsection @code{(delete-residue-sidechain imol chain-id resno ins-code)}
@cindex @code{(delete-residue-sidechain imol chain-id resno ins-code)}
 
where: 
 @itemize 
     @item imol is an exact integer number
     @item chain-id is a string
     @item resno is an exact integer number
     @item ins-code is an unknown type
 @end itemize


@subsection @code{(set-delete-atom-mode)}
@cindex @code{(set-delete-atom-mode)}
 
@subsection @code{(set-delete-residue-mode)}
@cindex @code{(set-delete-residue-mode)}
 
@subsection @code{(set-delete-residue-zone-mode)}
@cindex @code{(set-delete-residue-zone-mode)}
 
@subsection @code{(set-delete-residue-hydrogens-mode)}
@cindex @code{(set-delete-residue-hydrogens-mode)}
 
@subsection @code{(set-delete-water-mode)}
@cindex @code{(set-delete-water-mode)}
 
@subsection @code{(set-delete-sidechain-mode)}
@cindex @code{(set-delete-sidechain-mode)}
 
@subsection @code{(delete-item-mode-is-atom-p)}
@cindex @code{(delete-item-mode-is-atom-p)}
 
@subsection @code{(delete-item-mode-is-residue-p)}
@cindex @code{(delete-item-mode-is-residue-p)}
 
@subsection @code{(delete-item-mode-is-water-p)}
@cindex @code{(delete-item-mode-is-water-p)}
 
@subsection @code{(delete-item-mode-is-sidechain-p)}
@cindex @code{(delete-item-mode-is-sidechain-p)}
 
@subsection @code{(clear-pending-delete-item)}
@cindex @code{(clear-pending-delete-item)}
 
@subsection @code{(set-keep-delete-item-active-state istate)}
@cindex @code{(set-keep-delete-item-active-state istate)}
 
where: 
 @itemize 
     @item istate is an exact integer number
 @end itemize



@section Rotate/Translate Buttons 
@subsection @code{(do-rot-trans-setup state)}
@cindex @code{(do-rot-trans-setup state)}
 
where: 
 @itemize 
     @item state is an exact integer number
 @end itemize


@subsection @code{(rot-trans-reset-previous)}
@cindex @code{(rot-trans-reset-previous)}
 
@subsection @code{(do-cis-trans-conversion-setup istate)}
@cindex @code{(do-cis-trans-conversion-setup istate)}
 
where: 
 @itemize 
     @item istate is an exact integer number
 @end itemize



@section Mainchain Building Functions 
@subsection @code{(do-db-main state)}
@cindex @code{(do-db-main state)}
 
where: 
 @itemize 
     @item state is an exact integer number
 @end itemize



@section Close Molecule FUnctions 
@subsection @code{(close-molecule imol)}
@cindex @code{(close-molecule imol)}
 
where: 
 @itemize 
     @item imol is an exact integer number
 @end itemize



@section Rotatmer Functions 
@subsection @code{(setup-rotamers state)}
@cindex @code{(setup-rotamers state)}
 
where: 
 @itemize 
     @item state is an exact integer number
 @end itemize


@subsection @code{(do-rotamers atom-index imol)}
@cindex @code{(do-rotamers atom-index imol)}
 
where: 
 @itemize 
     @item atom-index is an exact integer number
     @item imol is an exact integer number
 @end itemize


@subsection @code{(set-rotamer-lowest-probability f)}
@cindex @code{(set-rotamer-lowest-probability f)}
 
where: 
 @itemize 
     @item f is an inexact number
 @end itemize


@subsection @code{(set-rotamer-check-clashes i)}
@cindex @code{(set-rotamer-check-clashes i)}
 
where: 
 @itemize 
     @item i is an exact integer number
 @end itemize


@subsection @code{(set-auto-fit-best-rotamer-clash-flag i)}
@cindex @code{(set-auto-fit-best-rotamer-clash-flag i)}
 
where: 
 @itemize 
     @item i is an exact integer number
 @end itemize


@subsection @code{(setup-auto-fit-rotamer state)}
@cindex @code{(setup-auto-fit-rotamer state)}
 
where: 
 @itemize 
     @item state is an exact integer number
 @end itemize


@subsection @code{(fill-partial-residues imol)}
@cindex @code{(fill-partial-residues imol)}
 
where: 
 @itemize 
     @item imol is an exact integer number
 @end itemize


@subsection @code{(setup-180-degree-flip state)}
@cindex @code{(setup-180-degree-flip state)}
 
where: 
 @itemize 
     @item state is an exact integer number
 @end itemize



@section Mutate Functions 
@subsection @code{(setup-mutate state)}
@cindex @code{(setup-mutate state)}
 
where: 
 @itemize 
     @item state is an exact integer number
 @end itemize


@subsection @code{(setup-mutate-auto-fit state)}
@cindex @code{(setup-mutate-auto-fit state)}
 
where: 
 @itemize 
     @item state is an exact integer number
 @end itemize


@subsection @code{(do-mutation type is-stub-flag)}
@cindex @code{(do-mutation type is-stub-flag)}
 
where: 
 @itemize 
     @item type is a string
     @item is-stub-flag is an exact integer number
 @end itemize


@subsection @code{(progressive-residues-in-chain-check chain-id imol)}
@cindex @code{(progressive-residues-in-chain-check chain-id imol)}
 
where: 
 @itemize 
     @item chain-id is a string
     @item imol is an exact integer number
 @end itemize


@subsection @code{(mutate ires chain-id imol target-res-type)}
@cindex @code{(mutate ires chain-id imol target-res-type)}
 
where: 
 @itemize 
     @item ires is an exact integer number
     @item chain-id is a string
     @item imol is an exact integer number
     @item target-res-type is a string
 @end itemize


@subsection @code{(set-mutate-auto-fit-do-post-refine istate)}
@cindex @code{(set-mutate-auto-fit-do-post-refine istate)}
 
where: 
 @itemize 
     @item istate is an exact integer number
 @end itemize


@subsection @code{(mutate-auto-fit-do-post-refine-state)}
@cindex @code{(mutate-auto-fit-do-post-refine-state)}
 
@subsection @code{(do-base-mutation type)}
@cindex @code{(do-base-mutation type)}
 
where: 
 @itemize 
     @item type is a string
 @end itemize


@subsection @code{(set-residue-type-chooser-stub-state istat)}
@cindex @code{(set-residue-type-chooser-stub-state istat)}
 
where: 
 @itemize 
     @item istat is an exact integer number
 @end itemize



@section Alternative Conformation 
@subsection @code{(alt-conf-split-type-number)}
@cindex @code{(alt-conf-split-type-number)}
 
@subsection @code{(set-add-alt-conf-split-type-number i)}
@cindex @code{(set-add-alt-conf-split-type-number i)}
 
where: 
 @itemize 
     @item i is an exact integer number
 @end itemize


@subsection @code{(unset-add-alt-conf-dialog)}
@cindex @code{(unset-add-alt-conf-dialog)}
 
@subsection @code{(unset-add-alt-conf-define)}
@cindex @code{(unset-add-alt-conf-define)}
 
@subsection @code{(altconf)}
@cindex @code{(altconf)}
 
@subsection @code{(set-add-alt-conf-new-atoms-occupancy f)}
@cindex @code{(set-add-alt-conf-new-atoms-occupancy f)}
 
where: 
 @itemize 
     @item f is an inexact number
 @end itemize


@subsection @code{(set-show-alt-conf-intermediate-atoms i)}
@cindex @code{(set-show-alt-conf-intermediate-atoms i)}
 
where: 
 @itemize 
     @item i is an exact integer number
 @end itemize


@subsection @code{(show-alt-conf-intermediate-atoms-state)}
@cindex @code{(show-alt-conf-intermediate-atoms-state)}
 
@subsection @code{(zero-occupancy-residue-range imol chain-id ires1 ires2)}
@cindex @code{(zero-occupancy-residue-range imol chain-id ires1 ires2)}
 
where: 
 @itemize 
     @item imol is an exact integer number
     @item chain-id is a string
     @item ires1 is an exact integer number
     @item ires2 is an exact integer number
 @end itemize


@subsection @code{(fill-occupancy-residue-range imol chain-id ires1 ires2)}
@cindex @code{(fill-occupancy-residue-range imol chain-id ires1 ires2)}
 
where: 
 @itemize 
     @item imol is an exact integer number
     @item chain-id is a string
     @item ires1 is an exact integer number
     @item ires2 is an exact integer number
 @end itemize



@section Pointer Atom Functions 
@subsection @code{(place-atom-at-pointer)}
@cindex @code{(place-atom-at-pointer)}
 
@subsection @code{(place-typed-atom-at-pointer type)}
@cindex @code{(place-typed-atom-at-pointer type)}
 
where: 
 @itemize 
     @item type is a string
 @end itemize


@subsection @code{(set-pointer-atom-is-dummy i)}
@cindex @code{(set-pointer-atom-is-dummy i)}
 
where: 
 @itemize 
     @item i is an exact integer number
 @end itemize


@subsection @code{(display-where-is-pointer)}
@cindex @code{(display-where-is-pointer)}
 

@section Baton Build Functions 
@subsection @code{(set-baton-mode i)}
@cindex @code{(set-baton-mode i)}
 
where: 
 @itemize 
     @item i is an exact integer number
 @end itemize


@subsection @code{(set-draw-baton i)}
@cindex @code{(set-draw-baton i)}
 
where: 
 @itemize 
     @item i is an exact integer number
 @end itemize


@subsection @code{(accept-baton-position)}
@cindex @code{(accept-baton-position)}
 
@subsection @code{(baton-try-another)}
@cindex @code{(baton-try-another)}
 
@subsection @code{(shorten-baton)}
@cindex @code{(shorten-baton)}
 
@subsection @code{(lengthen-baton)}
@cindex @code{(lengthen-baton)}
 
@subsection @code{(baton-build-delete-last-residue)}
@cindex @code{(baton-build-delete-last-residue)}
 
@subsection @code{(set-baton-build-params istart-resno chain-id backwards)}
@cindex @code{(set-baton-build-params istart-resno chain-id backwards)}
 
where: 
 @itemize 
     @item istart-resno is an exact integer number
     @item chain-id is a string
     @item backwards is a string
 @end itemize



@section Post-Baton Functions 
@subsection @code{(reverse-direction-of-fragment imol chain-id resno)}
@cindex @code{(reverse-direction-of-fragment imol chain-id resno)}
 
where: 
 @itemize 
     @item imol is an exact integer number
     @item chain-id is a string
     @item resno is an exact integer number
 @end itemize


@subsection @code{(setup-reverse-direction i)}
@cindex @code{(setup-reverse-direction i)}
 
where: 
 @itemize 
     @item i is an exact integer number
 @end itemize



@section Terminal OXT Atom 
@subsection @code{(add-OXT-to-residue imol reso insertion-code chain-id)}
@cindex @code{(add-OXT-to-residue imol reso insertion-code chain-id)}
 
where: 
 @itemize 
     @item imol is an exact integer number
     @item reso is an exact integer number
     @item insertion-code is a string
     @item chain-id is a string
 @end itemize



@section Crosshairs 
@subsection @code{(set-draw-crosshairs i)}
@cindex @code{(set-draw-crosshairs i)}
 
where: 
 @itemize 
     @item i is an exact integer number
 @end itemize


@subsection @code{(draw-crosshairs-state)}
@cindex @code{(draw-crosshairs-state)}
 

@section Edit Chi Angles 
@subsection @code{(setup-edit-chi-angles state)}
@cindex @code{(setup-edit-chi-angles state)}
 
where: 
 @itemize 
     @item state is an exact integer number
 @end itemize


@subsection @code{(set-find-hydrogen-torsion state)}
@cindex @code{(set-find-hydrogen-torsion state)}
 
where: 
 @itemize 
     @item state is an exact integer number
 @end itemize


@subsection @code{(set-graphics-edit-current-chi ichi)}
@cindex @code{(set-graphics-edit-current-chi ichi)}
 
where: 
 @itemize 
     @item ichi is an exact integer number
 @end itemize


@subsection @code{(unset-moving-atom-move-chis)}
@cindex @code{(unset-moving-atom-move-chis)}
 
@subsection @code{(set-show-chi-angle-bond imode)}
@cindex @code{(set-show-chi-angle-bond imode)}
 
where: 
 @itemize 
     @item imode is an exact integer number
 @end itemize



@section Masks 
@subsection @code{(mask-map-by-molecule map-mol-no coord-mol-no invert-flag)}
@cindex @code{(mask-map-by-molecule map-mol-no coord-mol-no invert-flag)}
 
where: 
 @itemize 
     @item map-mol-no is an exact integer number
     @item coord-mol-no is an exact integer number
     @item invert-flag is an exact integer number
 @end itemize


@subsection @code{(mask-map-by-atom-selection map-mol-no coords-mol-no mmdb-atom-selection invert-flag)}
@cindex @code{(mask-map-by-atom-selection map-mol-no coords-mol-no mmdb-atom-selection invert-flag)}
 
where: 
 @itemize 
     @item map-mol-no is an exact integer number
     @item coords-mol-no is an exact integer number
     @item mmdb-atom-selection is a string
     @item invert-flag is an exact integer number
 @end itemize


@subsection @code{(set-map-mask-atom-radius rad)}
@cindex @code{(set-map-mask-atom-radius rad)}
 
where: 
 @itemize 
     @item rad is an inexact number
 @end itemize



@section Check Waters Interface 
@subsection @code{(set-check-waters-b-factor-limit f)}
@cindex @code{(set-check-waters-b-factor-limit f)}
 
where: 
 @itemize 
     @item f is an inexact number
 @end itemize


@subsection @code{(set-check-waters-map-sigma-limit f)}
@cindex @code{(set-check-waters-map-sigma-limit f)}
 
where: 
 @itemize 
     @item f is an inexact number
 @end itemize


@subsection @code{(set-check-waters-min-dist-limit f)}
@cindex @code{(set-check-waters-min-dist-limit f)}
 
where: 
 @itemize 
     @item f is an inexact number
 @end itemize


@subsection @code{(set-check-waters-max-dist-limit f)}
@cindex @code{(set-check-waters-max-dist-limit f)}
 
where: 
 @itemize 
     @item f is an inexact number
 @end itemize


@subsection @code{(check-waters-by-difference-map-sigma-level-state)}
@cindex @code{(check-waters-by-difference-map-sigma-level-state)}
 
@subsection @code{(set-check-waters-by-difference-map-sigma-level f)}
@cindex @code{(set-check-waters-by-difference-map-sigma-level f)}
 
where: 
 @itemize 
     @item f is an inexact number
 @end itemize



@section Least-Squares matching 
@subsection @code{(clear-lsq-matches)}
@cindex @code{(clear-lsq-matches)}
 
@subsection @code{(apply-lsq-matches imol-reference imol-moving)}
@cindex @code{(apply-lsq-matches imol-reference imol-moving)}
 
where: 
 @itemize 
     @item imol-reference is an exact integer number
     @item imol-moving is an exact integer number
 @end itemize



@section Least-Squares plane interface 
@subsection @code{(setup-lsq-deviation state)}
@cindex @code{(setup-lsq-deviation state)}
 
where: 
 @itemize 
     @item state is an exact integer number
 @end itemize


@subsection @code{(setup-lsq-plane-define state)}
@cindex @code{(setup-lsq-plane-define state)}
 
where: 
 @itemize 
     @item state is an exact integer number
 @end itemize


@subsection @code{(unset-lsq-plane-dialog)}
@cindex @code{(unset-lsq-plane-dialog)}
 
@subsection @code{(remove-last-lsq-plane-atom)}
@cindex @code{(remove-last-lsq-plane-atom)}
 

@section Trim 
@subsection @code{(raster3d rd3-filename)}
@cindex @code{(raster3d rd3-filename)}
 
where: 
 @itemize 
     @item rd3-filename is a string
 @end itemize


@subsection @code{(povray filename)}
@cindex @code{(povray filename)}
 
where: 
 @itemize 
     @item filename is a string
 @end itemize


@subsection @code{(make-image-raster3d filename)}
@cindex @code{(make-image-raster3d filename)}
 
where: 
 @itemize 
     @item filename is a string
 @end itemize


@subsection @code{(make-image-povray filename)}
@cindex @code{(make-image-povray filename)}
 
where: 
 @itemize 
     @item filename is a string
 @end itemize


@subsection @code{(set-raster3d-bond-thickness f)}
@cindex @code{(set-raster3d-bond-thickness f)}
 
where: 
 @itemize 
     @item f is an inexact number
 @end itemize


@subsection @code{(set-raster3d-density-thickness f)}
@cindex @code{(set-raster3d-density-thickness f)}
 
where: 
 @itemize 
     @item f is an inexact number
 @end itemize


@subsection @code{(set-renderer-show-atoms istate)}
@cindex @code{(set-renderer-show-atoms istate)}
 
where: 
 @itemize 
     @item istate is an exact integer number
 @end itemize


@subsection @code{(raster-screen-shot)}
@cindex @code{(raster-screen-shot)}
 
@subsection @code{(citation-notice-off)}
@cindex @code{(citation-notice-off)}
 

@section Superposition (SSM) 
@subsection @code{(superpose imol1 imol2 move-imol2-flag)}
@cindex @code{(superpose imol1 imol2 move-imol2-flag)}
 
where: 
 @itemize 
     @item imol1 is an exact integer number
     @item imol2 is an exact integer number
     @item move-imol2-flag is an exact integer number
 @end itemize



@section NCS 
@subsection @code{(set-draw-ncs-ghosts imol istate)}
@cindex @code{(set-draw-ncs-ghosts imol istate)}
 
where: 
 @itemize 
     @item imol is an exact integer number
     @item istate is an exact integer number
 @end itemize


@subsection @code{(set-ncs-ghost-bond-thickness imol f)}
@cindex @code{(set-ncs-ghost-bond-thickness imol f)}
 
where: 
 @itemize 
     @item imol is an exact integer number
     @item f is an inexact number
 @end itemize


@subsection @code{(ncs-update-ghosts imol)}
@cindex @code{(ncs-update-ghosts imol)}
 
where: 
 @itemize 
     @item imol is an exact integer number
 @end itemize


@subsection @code{(make-dynamically-transformed-ncs-maps imol-model imol-map)}
@cindex @code{(make-dynamically-transformed-ncs-maps imol-model imol-map)}
 
where: 
 @itemize 
     @item imol-model is an exact integer number
     @item imol-map is an exact integer number
 @end itemize


@subsection @code{(make-ncs-ghosts-maybe imol)}
@cindex @code{(make-ncs-ghosts-maybe imol)}
 
where: 
 @itemize 
     @item imol is an exact integer number
 @end itemize


@subsection @code{(show-strict-ncs-state imol)}
@cindex @code{(show-strict-ncs-state imol)}
 
where: 
 @itemize 
     @item imol is an exact integer number
 @end itemize


@subsection @code{(set-show-strict-ncs imol state)}
@cindex @code{(set-show-strict-ncs imol state)}
 
where: 
 @itemize 
     @item imol is an exact integer number
     @item state is an exact integer number
 @end itemize


@subsection @code{(set-ncs-homology-level flev)}
@cindex @code{(set-ncs-homology-level flev)}
 
where: 
 @itemize 
     @item flev is an inexact number
 @end itemize


@subsection @code{(copy-chain imol from-chain to-chain)}
@cindex @code{(copy-chain imol from-chain to-chain)}
 
where: 
 @itemize 
     @item imol is an exact integer number
     @item from-chain is a string
     @item to-chain is a string
 @end itemize


@subsection @code{(copy-from-ncs-master-to-others imol chain-id)}
@cindex @code{(copy-from-ncs-master-to-others imol chain-id)}
 
where: 
 @itemize 
     @item imol is an exact integer number
     @item chain-id is a string
 @end itemize


@subsection @code{(ncs-control-change-ncs-master-to-chain imol ichain)}
@cindex @code{(ncs-control-change-ncs-master-to-chain imol ichain)}
 
where: 
 @itemize 
     @item imol is an exact integer number
     @item ichain is an exact integer number
 @end itemize


@subsection @code{(ncs-control-display-chain imol ichain state)}
@cindex @code{(ncs-control-display-chain imol ichain state)}
 
where: 
 @itemize 
     @item imol is an exact integer number
     @item ichain is an exact integer number
     @item state is an exact integer number
 @end itemize


@subsection @code{(place-helix-here)}
@cindex @code{(place-helix-here)}
 
@subsection @code{(new-molecule-by-residue-type-selection imol residue-type)}
@cindex @code{(new-molecule-by-residue-type-selection imol residue-type)}
 
where: 
 @itemize 
     @item imol is an exact integer number
     @item residue-type is a string
 @end itemize


@subsection @code{(new-molecule-by-atom-selection imol atom-selection)}
@cindex @code{(new-molecule-by-atom-selection imol atom-selection)}
 
where: 
 @itemize 
     @item imol is an exact integer number
     @item atom-selection is an unknown type
 @end itemize



@section Miguel's orientation axes matrix 

@section RNA/DNA 

@section Sequence (Assignment) 
@subsection @code{(print-sequence-chain imol chain-id)}
@cindex @code{(print-sequence-chain imol chain-id)}
 
where: 
 @itemize 
     @item imol is an exact integer number
     @item chain-id is a string
 @end itemize


@subsection @code{(assign-fasta-sequence imol chain-id-in seq)}
@cindex @code{(assign-fasta-sequence imol chain-id-in seq)}
 
where: 
 @itemize 
     @item imol is an exact integer number
     @item chain-id-in is a string
     @item seq is a string
 @end itemize


@subsection @code{(assign-sequence imol-model imol-map chain-id)}
@cindex @code{(assign-sequence imol-model imol-map chain-id)}
 
where: 
 @itemize 
     @item imol-model is an exact integer number
     @item imol-map is an exact integer number
     @item chain-id is a string
 @end itemize



@section Surfaces 
@subsection @code{(do-surface imol istate)}
@cindex @code{(do-surface imol istate)}
 
where: 
 @itemize 
     @item imol is an exact integer number
     @item istate is an exact integer number
 @end itemize


@subsection @code{(fffear-search imol-model imol-map)}
@cindex @code{(fffear-search imol-model imol-map)}
 
where: 
 @itemize 
     @item imol-model is an exact integer number
     @item imol-map is an exact integer number
 @end itemize


@subsection @code{(set-fffear-angular-resolution f)}
@cindex @code{(set-fffear-angular-resolution f)}
 
where: 
 @itemize 
     @item f is an inexact number
 @end itemize


@subsection @code{(fffear-angular-resolution)}
@cindex @code{(fffear-angular-resolution)}
 

@section Remote Control 
@subsection @code{(make-socket-listener-maybe)}
@cindex @code{(make-socket-listener-maybe)}
 
@subsection @code{(set-coot-listener-socket-state-internal sock-state)}
@cindex @code{(set-coot-listener-socket-state-internal sock-state)}
 
where: 
 @itemize 
     @item sock-state is an exact integer number
 @end itemize



@section Display Lists for Maps 
@subsection @code{(set-display-lists-for-maps i)}
@cindex @code{(set-display-lists-for-maps i)}
 
where: 
 @itemize 
     @item i is an exact integer number
 @end itemize



@section Preferences 
@subsection @code{(preferences)}
@cindex @code{(preferences)}
 
@subsection @code{(clear-preferences)}
@cindex @code{(clear-preferences)}
 
@subsection @code{(set-mark-cis-peptides-as-bad istate)}
@cindex @code{(set-mark-cis-peptides-as-bad istate)}
 
where: 
 @itemize 
     @item istate is an exact integer number
 @end itemize


@subsection @code{(show-mark-cis-peptides-as-bad-state)}
@cindex @code{(show-mark-cis-peptides-as-bad-state)}
 
@subsection @code{(browser-url url)}
@cindex @code{(browser-url url)}
 
where: 
 @itemize 
     @item url is a string
 @end itemize


@subsection @code{(set-browser-interface browser)}
@cindex @code{(set-browser-interface browser)}
 
where: 
 @itemize 
     @item browser is a string
 @end itemize


@subsection @code{(handle-online-coot-search-request entry-text)}
@cindex @code{(handle-online-coot-search-request entry-text)}
 
where: 
 @itemize 
     @item entry-text is a string
 @end itemize


@subsection @code{(new-generic-object-number objname)}
@cindex @code{(new-generic-object-number objname)}
 
where: 
 @itemize 
     @item objname is a string
 @end itemize


@subsection @code{(to-generic-object-add-display-list-handle object-number display-list-id)}
@cindex @code{(to-generic-object-add-display-list-handle object-number display-list-id)}
 
where: 
 @itemize 
     @item object-number is an exact integer number
     @item display-list-id is an exact integer number
 @end itemize


@subsection @code{(set-display-generic-object object-number istate)}
@cindex @code{(set-display-generic-object object-number istate)}
 
where: 
 @itemize 
     @item object-number is an exact integer number
     @item istate is an exact integer number
 @end itemize


@subsection @code{(generic-object-is-displayed-p object-number)}
@cindex @code{(generic-object-is-displayed-p object-number)}
 
where: 
 @itemize 
     @item object-number is an exact integer number
 @end itemize


@subsection @code{(generic-object-index name)}
@cindex @code{(generic-object-index name)}
 
where: 
 @itemize 
     @item name is a string
 @end itemize


@subsection @code{(number-of-generic-objects)}
@cindex @code{(number-of-generic-objects)}
 
@subsection @code{(generic-object-info)}
@cindex @code{(generic-object-info)}
 
@subsection @code{(generic-object-has-objects-p obj-no)}
@cindex @code{(generic-object-has-objects-p obj-no)}
 
where: 
 @itemize 
     @item obj-no is an exact integer number
 @end itemize


@subsection @code{(close-generic-object object-number)}
@cindex @code{(close-generic-object object-number)}
 
where: 
 @itemize 
     @item object-number is an exact integer number
 @end itemize


@subsection @code{(is-closed-generic-object-p object-number)}
@cindex @code{(is-closed-generic-object-p object-number)}
 
where: 
 @itemize 
     @item object-number is an exact integer number
 @end itemize


@subsection @code{(generic-objects-gui-wrapper)}
@cindex @code{(generic-objects-gui-wrapper)}
 
@subsection @code{(handle-read-draw-probe-dots dots-file)}
@cindex @code{(handle-read-draw-probe-dots dots-file)}
 
where: 
 @itemize 
     @item dots-file is a string
 @end itemize


@subsection @code{(handle-read-draw-probe-dots-unformatted dots-file imol show-clash-gui-flag)}
@cindex @code{(handle-read-draw-probe-dots-unformatted dots-file imol show-clash-gui-flag)}
 
where: 
 @itemize 
     @item dots-file is a string
     @item imol is an exact integer number
     @item show-clash-gui-flag is an exact integer number
 @end itemize


@subsection @code{(set-do-probe-dots-on-rotamers-and-chis state)}
@cindex @code{(set-do-probe-dots-on-rotamers-and-chis state)}
 
where: 
 @itemize 
     @item state is an exact integer number
 @end itemize


@subsection @code{(do-probe-dots-on-rotamers-and-chis-state)}
@cindex @code{(do-probe-dots-on-rotamers-and-chis-state)}
 
@subsection @code{(set-do-probe-dots-post-refine state)}
@cindex @code{(set-do-probe-dots-post-refine state)}
 
where: 
 @itemize 
     @item state is an exact integer number
 @end itemize


@subsection @code{(do-probe-dots-post-refine-state)}
@cindex @code{(do-probe-dots-post-refine-state)}
 
@subsection @code{(set-interactive-probe-dots-molprobity-radius r)}
@cindex @code{(set-interactive-probe-dots-molprobity-radius r)}
 
where: 
 @itemize 
     @item r is an inexact number
 @end itemize


@subsection @code{(interactive-probe-dots-molprobity-radius)}
@cindex @code{(interactive-probe-dots-molprobity-radius)}
 
@subsection @code{(probe-available-p)}
@cindex @code{(probe-available-p)}
 
@subsection @code{(set-dti-stereo-mode state)}
@cindex @code{(set-dti-stereo-mode state)}
 
where: 
 @itemize 
     @item state is an exact integer number
 @end itemize


@subsection @code{(do-smiles-gui)}
@cindex @code{(do-smiles-gui)}
 

@section Fun 
@subsection @code{(do-tw)}
@cindex @code{(do-tw)}
 
@subsection @code{(place-text text x y z size)}
@cindex @code{(place-text text x y z size)}
 
where: 
 @itemize 
     @item text is an unknown type
     @item x is an inexact number
     @item y is an inexact number
     @item z is an inexact number
     @item size is an exact integer number
 @end itemize


@subsection @code{(remove-text text-handle)}
@cindex @code{(remove-text text-handle)}
 
where: 
 @itemize 
     @item text-handle is an exact integer number
 @end itemize




%
\documentclass{book}
\usepackage{a4}
\usepackage{palatino}
%\usepackage{times}
%\usepackage{utopia}
\usepackage{euler}
\usepackage{fancyhdr}
\usepackage{epsf}

\newcommand {\atilde} {$_{\char '176}$} % tilde(~) character

\title{The Coot Reference Manual}
\author{Paul Emsley \\\textsf{\small emsley@ysbl.york.ac.uk}}
%\makeindex  % Not at the moment.  There are no index markups (yet).

\begin{document}
\maketitle
\tableofcontents

\chapter{Acknowledgments}
Paul Emsley is extremely grateful to use the library code of the
following people, without whom Coot could not have been realised:

\begin{trivlist}
\item Kevin Cowtan
\item Eugene Krissinel
\item Stuart McNicholas
\item Raghavendra Chandrashekara
\item Paul Bourke \& Cory Gene Bloyd
\end{trivlist}

Roland Dunbrack \& co-workers for rotamer library data.

Also (for generally useful software used in Coot):

\begin{trivlist}
\item Matteo Frigo \& Steven G. Johnson
\item Gary Houston \& other Guile developers
\item Python developers
\item Gtk+ and GNOME-Canvas developers
\item GNU Scientific Library developers
\item OpenGL developers
\item Janne L\"of
\end{trivlist}

Also those with whom Paul has corresponded about or provided
features and bug fixes and built the software:

\begin{tabular}{ll}
 William G. Scott & Bernhard Lohkamp \\
 Joel Bard  & Ezra Peisach           \\
 Alex Schuettelkopf & Charlie Bond 
\end{tabular}

Not forgetting the testers\footnote{in no particular order}

%\begin{trivlist}
%\item Eleanor J. Dodson
%\item Jan Dohnalek
%\item Karen McLuskey
%\item Bernhard Lohkamp
%\item Aleks Roszak
%\item Florence Vincent
%\item Roberto Steiner
%\item Alex Schuettelkopf
%\item Charlie Bond
%\item Constantina Fotinou
%\item William G. Scott
%\item Adrian Lapthorn
%\end{trivlist}

\begin{tabular}{ll}
Eleanor J. Dodson & Jan Dohnalek \\
Constantina Fotinou & Alex Roszak  \\
Florence Vincent  & Roberto Steiner \\
Karen McLuskey & Adrian Lapthorn   
\end{tabular}

\vspace{5mm}

Those with experience of Quanta, XFit and O will notice similarities
between Coot and those programs, it's fair to say that they have had
considerable influence in the look of Coot, so Paul respectively
thanks for inspiration: Tom Oldfield, Alwyn Jones and Duncan McRee
(and their co-workers).

\chapter{Design Overview}
\section{Why?}
``Why does Coot exist?'' you might ask.  ``Given that other molecular
graphics\footnote{molecular graphics with protein modeling and
  density fitting functions, that is.} programs exist, why write
another?''

Because I like having the source code to programs I use and think that
others feel the same.  Because the other programs don't quite work how
I wanted them to\footnote{and of course, there was no way to fix
  that.}. Because there was the possibility to integrate molecular
graphics into the CCP4 Suite.  

As to why write Coot when CCP4MG was available: that is not how it
happened. Coot\footnote{it was called ``MapView'' at the time.} was
released over a year before CCP4MG was available.  I followed my own
design, toolkit and aesthetic decisions - for good or bad\footnote{for
  example, I was (and remain) less concerned about porting to various
  shades of Microsoft Windows operating systems than the CCP4MG
  developers.}.

\section{Hacker's Guide}

The are several core libraries that are fundamental to Coot:

\begin{itemize}
\item Clipper: Kevin Cowtan's General crystallographic object library
\item mmdb: CCP4's Coordinate Library
\item GTk+: GNU's GUI toolkit.
\end{itemize}

\subsection{GUI}
The GUI is almost entirely built using glade.  Glade writes out its
code in pure C.  This causes a problem.  \texttt{src/interface.h} and
\texttt{src/support.h} both get regenerated in ``C mode'' every time
glade is run.  So, after every time we change the GUI with glade, we
need to run \texttt{post-glade} to introduce the C/C++ linking type
declaration wrapper into these files.

Not all of the GUI is build with glade - there are dynamic elements,
for example the ``Map and Molecule (Display) Control'' window the
frame of which are generated in glade, but the hboxs are filled using
hand-made code (see \texttt{gtk-manual.c}).

\subsection{GUI/State Variables}
The graphics\_info\_t class contains a host of static state variables,
mostly manipulated by GUI element (\emph{e.g} button)
callbacks\footnote{mostly button clicked signals and menu item
  activative signals}. For historical reasons they are initially set
in \texttt{globjects.cc}.  Because the callbacks are written in C by
glade\footnote{the GUI builder}, these variables need a functional
interface to set the variables, and that interface is used by both the
GUI button\footnote{and other GUI elements} callbacks and is exported
to the scripting level.  These function declarations are in
\texttt{c-interface.h}.  All manipulations of graphics\_info\_t's
state variables go via \texttt{c-interface.h}.

Notice that MMDB functions are not allowed in
this interface\footnote{because SWIG chokes on them}. 

\subsection{Scripting}
So, SWIG uses \texttt{c-interface.h} to generate the python/scheme
scripting interface. The scripting language is chosen at
configure-time using either \texttt{--with-guile} or
\texttt{--with-python}.

\section{Validation}
As I write this, a few of us are cobbling together a XML-based system
for validation.  We think that validation data should be presented as
XML data that can be passed between packages and programs.  Either the
program itself will output the data, or we will write a wrapper to
turn the output into the appropriate XML format.  

These XML data will be then available for use in the molecular
graphics and will provide information for a ``Next Unusual Feature''
button.  The library to provide the XML cabability for this is expat,
the same library used in Perl's XML::Parser, Python's XML parser
Pyexpat and Mozilla's XML parser.

\subsection{Example: Temperature Factor Analysis}
Recall that the kurtosis of a distribution, $k$ is given by:

\begin{equation}
  \label{eq:kurtosis}
  k = \frac{\Sigma(X_i - \mu)^4} {N \sigma^4} - 3 
\end{equation}

We calculate the kurtosis for the isotropic temperature factors for
each residue in the molecule and residues with the most leptokurtic
distributions are written out to a file.  The format of the file is
XML.

This is an example of how we expect validation data to be presented to
molecular graphics programs.



\chapter{Refinement and Regularization}
\input{../../coot/doc/derivatives-part}

\chapter{Exported Functions}
\input{functions}

%\input{reference.ind} % no indexes yet
\end{document}
 % no indexes yet
\end{document}
 % no indexes yet
\end{document}
 % no indexes yet
\end{document}
