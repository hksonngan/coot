
% Have you documented:
%
% Merge molecules dialog    : done
% Mutate sequence dialog    : done
% Add OXT to residue dialog : done
% Bond Parameters dialog
% Renumber Residues dialog
% Find Bad Chiral Atoms     : done
% Validate Waters (simple)
% Validation Graphs 
% Pointer distances
% Torsions

\documentclass{book}
\usepackage{a4}
\usepackage{palatino}
%\usepackage{times}
%\usepackage{utopia}
\usepackage{euler}
\usepackage{fancyhdr}
\usepackage{epsf}

\newcommand {\atilde} {$_{\char '176}$} % tilde(~) character

%\date{1st April 2004}

\title{The Coot User Manual}
\author{Paul Emsley \\\textsf{\small emsley@ysbl.york.ac.uk}}
\makeindex
\hyphenation{tri-angle}

\begin{document}
\thispagestyle{empty}

%% Make a title page: I can't use \maketitlepage because I want a line

\vspace*{30mm}

{\huge The Coot User Manual}

\begin{picture}(300,2)
\linethickness{5pt}
\put(0,0){\line(1,0){347}}
\end{picture}

\begin{flushright}
%  \today
  1st April 2004
\end{flushright}

\vspace*{20mm}


\begin{center}
  \leavevmode
  \epsfxsize 127mm \epsffile{coot-plain-2.eps}
\end{center}

\vspace*{20mm}

\begin{picture}(300,2)
\linethickness{5pt}
\put(0,0){\line(1,0){347}}
\end{picture}

\begin{flushright}

  Paul Emsley\\\textsf{\small emsley@ysbl.york.ac.uk}
\end{flushright}

%\begin{picture}(width,height)(xoffset,yoffset)
%\end{picture}

%\newpage
\tableofcontents
\pagestyle{headings}

\chapter{Introduction}

\section{This document}
This document is the Coot User Manual, giving a brief overview of the
interactive features.  Other documentations includes (or it is planned
to include) the \index{reference manual}Coot Reference Manual and the
Coot \index{tutorial} Tutorial.  These documents should be distributed
with the source code.

\section{What is Coot?}

Coot is a stand-alone portion of CCP4's Molecular Graphics project. Its
focus is crystallographic model-building and manipulation rather than
representation (\emph{i.e.} more like Frodo than
\index{Rasmol}Rasmol).

Coot is Free Software.  You can give it away. If you don't like the
way it behaves, you can fix it yourself.

\section{What Coot is Not}
Coot is not:
\begin{itemize}
\item CCP4's official Molecular Graphics program\footnote{Coot is
    \emph{part of} that project. The official program (which contains
parts of Coot), ccp4mg is under
    the direct control of Liz Potterton and Stuart McNicholas.}
\item a program to do refinement\footnote{although it does have a
    local refinement algorithm it is no substitute for \textsc{refmac}
    (a wrapper for \textsc{refmac} is available).}
\item a database, in any respect
\item a protein crystallographic suite\footnote{that's the job of the
    CCP4 Program Suite.}.
\end{itemize}

\section{Hardware Requirements}
The code is designed to be portable to any Unix-like operating
system\footnote{including Cygwin, but currently coot is ``unstable''
  on a Cygwin system.}.  Coot certainly runs on SGI IRIX64, RedHat
Linux of various sorts, SuSe Linux\footnote{so far only 8.2 verified.}
and MacOS X (10.2).  The sgi Coot binaries shouold also 
work on IRIX.

If you want to port to some other operating system, you are
welcome\footnote{it's Free Software after all and I could give you a
  hand.}.  Note that your task will be eased by using GNU GCC to compile
the programs components.

\subsection{Mouse}
\index{mouse}Coot works best with a 3-button mouse and works better if
it has a scroll-wheel too (see Chapter 2 for more details)\footnote{I
  can get by with a one button Machintosh - but it's not ideal.}.

\section{Environment Variables}
Coot responds to several command line arguments that modify its
behaviour.  

\begin{trivlist}
\item [\texttt{COOT\_STANDARD\_RESIDUES}] The filename of the pdb file
  containing the standard amino acid residues in ``standard
  conformation''\footnote{as it is known in Clipper.}
\item [\texttt{COOT\_SCHEME\_DIR}] The directory containing auxiliary scheme
  files 
\item [\texttt{COOT\_REF\_STRUCTS}] The directory containing a set of
  high resolution pdb files used as \index{reference
    strctures}reference structures to build backbone atoms from
  C$\alpha$ positions
\item [\texttt{COOT\_REFMAC\_LIB\_DIR}] \index{dictionary, cif}Refmac's
  CIF directory containing the monomers and link descriptions.  In the future
  this may simply be the same directory in which refmac looks to find
  the library dictionary.
\item [\texttt{COOT\_RESOURCES\_DIR}] The directory that contains the
  splash screen image and the GTk application resources.
\item [\texttt{COOT\_BACKUP\_DIR}] The directory to which backup are
  written (if it exists as a directory).  If it is not, then backups
  are written to the current directory (the directory in which coot
  was started).
\end{trivlist}
  
And of course extension language environment variables are used too:

\begin{trivlist}
\item [\texttt{PYTHONPATH}] (for python modules)
\item [\texttt{GUILE\_LOAD\_PATH}] (for guile modules)
\end{trivlist}

Normally, these environment variables will be set correctly in the
coot setup script (which can be found in the \texttt{setup} directory
in the binary distribution.  See the web site (Section \ref{webpage})
for setup details.

\section{Command Line Arguments}
\index{command line arguments}
\label{sec:command_line_arguments}
Rather that using the GUI to read in information, you can use the
following command line arguments:
\begin{itemize}
\item \texttt{--script} to run a script on start up
\item \texttt{--pdb}  for pdb/coordinates file
\item \texttt{--data} for mtz, phs or mmCIF data file
\item \texttt{--map}  for a (currently only CCP4) map
\end{itemize}
So, for example, one might use: 
\begin{trivlist}
\item \texttt{coot --pdb pre-refinement.pdb --pdb post-refinement.pdb}
\end{trivlist}

\section{Web Page}
\label{webpage}
Coot has a \index{web page}web page:

\begin{trivlist}
\item \texttt{http://www.ysbl.york.ac.uk/\atilde emsley/coot}
\end{trivlist}

There you can read more about the CCP4 molecular graphics project in
general and other projects which are important for coot\footnote{coot
  has several influences and dependencies, but these will not be
  discussed here in the User Manual.}.

The web page also contains an example ``setup'' file which assigns the
environment variables to change the behaviour of Coot.

\section{Crash}
\label{sec:crash}
\index{crash}
Coot might crash on you - it shouldn't.  

\index{recover session}\index{crash recovery}There are backup files in
the directory \texttt{coot-backup}\footnote{\$COOT\_BACKUP\_DIR is used
  in preference if set}. You can recover the session (until the last
edit) by reading in the pdb file that you started with last time and
then use \textsf{File $\rightarrow$ Recover Session\ldots}.

I would like to know about coot crashing\footnote{The map-reading
  problem (documented in Section \ref{map-reading-bug}) is already
  known.} so that I can fix it as soon as possible. If you want your
problem fixed, this involves some work on your part sadly.

First please make sure that you are using the most recent version of
coot.  I will often need to know as much as possible about what you
did to cause the bug.  If you can reproduce the bug and send me the
files that are needed to cause it, I can almost certainly fix -
it\footnote{now there's a hostage to fortune.} especially if you
\index{debugger}\index{gdb}use the debugger (gdb) and send a backtrace
too\footnote{to do so, please send me the output of the following:
  \texttt{\$ gdb `which coot` \emph{corefile}} and then at the
  \texttt{(gdb)} prompt type: \texttt{where}, where
  \texttt{\emph{corefile}} is the core dump file, \texttt{core} or
  \texttt{core.4536} or some such.}.

% -----------------------------------------------------------
\chapter{Mousing and Keyboarding}
% -----------------------------------------------------------
\index{mouse buttons}
How do we move around and select things?

\vspace{0.5cm}
  \begin{tabular}{ll}
    Left-mouse Drag & Rotate view \\
    Ctrl Left-Mouse Drag &  Translates view \\
    Shift Left-Mouse &  Label Atom\\
    Right-Mouse Drag &  Zoom in and out\index{zoom}\\
    Shift Right-Mouse Drag &  Rotate View around Screen Z axis\\
    Middle-mouse & Centre on atom\\
    Scroll-wheel Forward &  Increase map contour level\\
    Scroll-wheel Backward &  Decrease map contour level
  \end{tabular}
\vspace{3mm}

See also Chapter \ref{chap-hints} for more help.

\section{Next Residue}
\begin{tabular}{ll}
  ``Space'' & Next Residue \\
  ``Shift'' ``Space'' & Previous Residue
\end{tabular}

See also ``Recentring View'' (Section \ref{sec:recentring-view}).

\section{Keyboard Contouring}

Use \texttt{+} or \texttt{-} on the keyboard if you don't have a
scroll-wheel.

\section{Keyboard Rotation}
\index{keyboard rotation}By popular request keyboard equivalents of
rotations have been added\footnote{particularly for those with
  PowerMates (the amount of rotation can be changed to 2$^\circ$ (from
  the default 1$^\circ$) using \texttt{(set-idle-function-rotate-angle
    2.0)}).}: \vspace{3mm}

\begin{tabular}{ll}
  Q & Rotate + X Axis \\
  W & Rotate - X Axis \\
  E & Rotate + Y Axis \\
  R & Rotate - Y Axis \\
  T & Rotate + Z Axis \\
  Y & Rotate - Z Axis \\
  I & Continuous Y Axis Rotation
\end{tabular}
% document rotate-x-sceen nsteps step-size here?

\section{Keyboard Translation}
\index{translation, keyboard}
\label{keyboard_translation}
\begin{tabular}{ll}
  Keypad 3 & Push View (+Z translation)\\
  Keypad . & Pull View (-Z translation)
\end{tabular}


\section{Keyboard Zoom and Clip}

\begin{tabular}{ll}

  N & Zoom out   \\
  M & Zoom in    \\
  D & Slim clip  \\
  F & Fatten clip\\
\end{tabular}

\vspace{3mm}

\section{Scrollwheel}
When there is no map, using the scroll-wheel has no effect.  If there
is exactly one map displayed, \index{contouring, map} the scroll-wheel
will change the contour level of that map.  If there are two or more
maps, the map for which the contour level is changed can be set by
\textsf{HID $\rightarrow$ Scrollwheel $\rightarrow$ Attach scroll-wheel
  to which map?} and selecting a map number.

\section{Selecting Atoms}
Several Coot functions require the selecting of atoms to specify a
residue range (for example: Regularize, Refine (Section
\ref{sec:randr}) or Rigid Body Fit Zone (Section
\ref{sec:RigidBodyRefinement})).  Select atoms with the Left-mouse.
See also Picking (Section \ref{sec:picking}).

Use the scripting function
\index{quanta-buttons}\texttt{(quanta-buttons)} to make the mouse
functions more like other molecular graphics programs to which you may
be more accustomed\footnote{See also \ref{sec:quanta-zooming}}.

\section{Virtual Trackball}
\index{trackball, virtual} You may not completely like the way the
molecule is moved by the mouse movement\footnote{Mouse movement in
  ``Spherical Surface'' mde generates a component of (often
  undesirable) screen z-rotation, particularly noticeable when the
  mouse is at the edge of the screen.}.  To change this, try:
\textsf{HID $\rightarrow$ Virtual Trackball $\rightarrow$ Flat}.  To
do this from the scripting interface: \texttt{(set-vt
  1)}\footnote{\texttt{(set-vt 0)} to turn it back to ``Spherical''
  mode.}.

If you \emph{do} want \index{screen-z rotation}\index{z-rotation}
screen-z rotation, you can either use Shift Right-Mouse Drag or set
the Virtual Trackball to Spherical Surface mode and move the mouse
along the bottom edget of the screen.

\section{More on Zooming}
\label{sec:quanta-zooming}
The function \texttt{(quanta-like-zoom)} adds the ability to zoom the
view using just Shift + Mouse movement\footnote{this is off by default
  because I find it annoying.}.

There is also a Zoom slider\index{zoom, slider} (\textsf{Draw
  $\rightarrow$ Zoom}) for those without a right-mouse button.

% -----------------------------------------------------------
\chapter{General Features}
% -----------------------------------------------------------

The map-fitting and model-building tools can be accessed by using
\textsf{Calculate $\rightarrow$ Model/Fit/Refine\ldots}.  Many
functions have \index{tooltips}tooltips\footnote{Put your mouse over a
  widget for a couple of seconds, if that widget has a tooltip, it
  will pop-up in a yellow box.}\index{tooltips} describing the
particular features and are documented in Chapter
\ref{modelling,building}.

\section{Version number}
\index{version number}
The version number of Coot can be found at the top of the ``About''
window (\textsf{Help $\rightarrow$ About}).

There is also a script function to return the version of coot:

\texttt{(coot-version)}

\section{Antialiasing}
Antialiasing (for what it's worth) can be enabled using:

\texttt{(set-do-anti-aliasing 1)}

The default is \texttt{0} (off).

\section{Molecule Number}
\index{molecule number} 
Coot is based on the concept of molecules.  Maps and coordinates are
different representations of molecules.  The access to the molecule is
\emph{via} the \emph{molecule number}.  It is often important therefore to
know the molecule number of a particular molecule.

Molecule numbers can be found by clicking on an atom in that molecule
(if it has coordinates of course). The first number in brackets in the
resulting text in the console is the molecule number.  The molecule
number can also be found in Display Control window (Section
\ref{sec:display_manager}).  It is also displayed on the left-hand
side of the molecule name in the option menus of the ``Save
Coordinates'' and ``Go To Atom'' windows.

\section{Display Issues}
The ``graphics'' window is drawn using \index{OpenGL}OpenGL.  It is
considerably smoother when using a 3D accelerated X server. 

The view is orthographic (\emph{i.e.} the back is the same size as the
front).  The default clipping is about right for viewing coordinate
data, but is often a little too ``thick'' for viewing electron
density.  It is easily changed (see Section \ref{clipping
  manipulation}).

Depth-cueing\index{depth-cueing} is linear and fixed on. There is no
anti-aliasing\footnote{coot is not the program for snazzy graphics -
  CCP4mg is the program for that sort of thing.}.

The graphics window can be resized, but it has a minimum size of
400x400 pixels.

\subsection{Origin Marker}

A yellow box\index{yellow box} called the ``origin
marker''\index{origin marker} marks the origin.  It can be removed
using:

\texttt{(set-show-origin-marker 0)}

Its state can be queried like this:

\texttt{(show-origin-marker-state)}

which returns an number (an integer).

\subsection{Raster3D output}
\label{Raster3D}Output suitable for use by Raster3D\index{Raster3D}'s
``render''\index{render} can be generated using the scripting function

\texttt{(raster3d \emph{file-name})}

where \texttt{\emph{file-name}} is such as
\texttt{"test.r3d"}\footnote{Also povray will be supported in the
  future.}.

There is a keyboard key to generate this file, run ``render'' and
display the image: Function key F8.

You can also use the function

\texttt{(render-image)}

which will create a file \texttt{coot.r3d}, from which ``render'' produces
\texttt{coot.png}. This png file is displayed using ImageMagick's display
program (by default).  Use something like:

\texttt{(set! coot-png-display-program "gqview")}

to change that to different display program ("gqview" in this case).

To change the widths of the bonds and density ``lines'' use (for example):

\texttt{(set-raster3d-bond-thickness 0.1)}

and 

\texttt{(set-raster3d-density-thickness 0.01)}

To turn off the representations of the atoms (spheres):

\texttt{(set-renderer-show-atoms 0)}



\section{Display Manager}
\label{sec:display_manager}
\index{Display Manager} This is also known as ``Map and molecule
(coordinates) display control''.  Here you can select which maps and
molecules you can see and how they are drawn\footnote{to a limited
  extent.}.  The ``Display'' and ``Active'' are toggle buttons, either
depressed (active) or undepressed (inactive).  The ``Display'' buttons
control whether a molecule (or map) is drawn and the ``Active'' button
controls if the molecule is clickable\footnote{the substantial
  majority of the time you will want your the buttons to be both
  either depressed or undepressed, rarely one but not the other.}
(\emph{i.e.} if the molecule's atoms can be labeled).

By default, the path names of the files are not displayed in the
Display Manager.  To turn them on:

\texttt{(set-show-paths-in-display-manager 1)}

\index{colour by chain}\index{atom colouring}If you pull across the
horizontal scrollbar in a Molecule view, you will see the ``Render
as'' menu.  You can use this to change between normal ``Bonds (Colour
by Atom)'',``Bonds (Colour by Chain)'' and ``C$\alpha$''
representation\index{C$\alpha$ representation}.  There is also
available ``No Waters'' and ``C$\alpha$ + ligands'' representations.

\section{The file selector}
\subsection{File-name Filtering}
\index{file-name filtering} The ``Filter'' button in the fileselection
filters the filenames according to extension.  For coordinates files
the extensions are ``.pdb'' ``.brk'' ``.mmcif'' and others.  For data:
``.mtz'', ``.hkl'', ``.phs'', ``.cif'' and for (CCP4) maps ``.ext'',
``.msk'' and ``.map''.  If you want to add to the extensions, the
following functions are available:

\begin{trivlist}
\item \texttt{(add-coordinates-glob-extension \emph{extension})}
\item \texttt{(add-data-glob-extension \emph{extension})}
\item \texttt{(add-map-glob-extension \emph{extension})}
\item \texttt{(add-dictionary-glob-extension \emph{extension})}
\end{trivlist}
where \texttt{\emph{extension}} is something like: \texttt{".mycif"}.

\subsection{Filename Sorting}
If you like your files initially sorted by date (rather than
lexographically, which is the default use:

\texttt{(set-sticky-sort-by-date)}

\section{Scripting}
\index{scripting} There is an compile-time option of adding a script
interpreter.  Currently the options are python and guile.  Hopefully,
in the near future you will be able to use both in the same
executable, but that's not available today.

Hundreds of commands are made available for use in scripting by using
SWIG.  These are are currently not well documented but can be found in
the Coot Reference Manual or the source code (\texttt{c-interface.h}).

Commands described throughout this manual (such as \texttt{(vt-surface
  1))} can be evaluated\index{executing commands} directly by Coot by
using the ``Scripting Window'' (\textsf{Calculate $\rightarrow$
  Scripting\ldots}). Note that you type the commands in the lower
entry widget and the command gets echoed (in red) and the return vaule
and any output is displayed in the text widget above.  The typed
command should be terminated with a carriage return\footnote{which
  causes the evalution of the command.}.  Files\footnote{such as the
  Coot state file (Section \ref{sec:coot_state}).} can be evaluated
(executed) using \textsf{Calculate $\rightarrow$ Run Script\ldots}.
Note that in scheme (the usual scripting language of Coot), the
parentheses are important.

\subsection{Python}
\index{python} Coot has an (optional) embedded python interpreter.
Thus the full power of python is available to you.  Coot will look for
an initialization script \index{startup settings
  (python)}\index{\texttt{.coot.py}}(\texttt{\$HOME/.coot.py}) and
will execute it if found.  This file should contain python commands
that set your personal preferences.


\subsubsection{Python Commands}
The scripting functions described in this manual are formatted
suitable for use with guile, \emph{i.e.}:

\texttt{(\emph{function} \emph{arg1} \emph{arg2\ldots})}

If you are using Python instead: the format needs to be changed to:

\texttt{\emph{function}(\emph{arg1},\emph{arg2\ldots})}

Note that dashes in guile function names become underscores for
python, so that (for example) \texttt{(raster-screen-shot)} becomes
\texttt{raster\_screen\_shot()}.


\subsection{Scheme}
\index{guile}%
\index{scheme} The scheme interpreter is made available by embedding
guile.  The initialization script used by this interpreter is
\index{startup settings (scheme)} \index{\texttt{.coot}}
\texttt{\$HOME/.coot}.  This file should contain scheme commands that
set your personal preferences.


\subsection{State}
\label{sec:coot_state}
The ``state''\index{state} of coot is saved on Exit and written to a
file called \texttt{0-coot.state.scm} (scheme)
\texttt{0-coot.state.py} (python).   This
state file contains information about the screen centre, the
clipping, colour map rotation size, the symmetry radius, and other
molecule related parameters such as filename, column labels,
coordinate filename \emph{etc.}.

Use \textsf{Calculate $\rightarrow$ Run Script\ldots} to use this file
to re-create the loaded maps and models that you had when you finished
using Coot\footnote{in that particular directory.} last time.
A state file can be saved at any time using \texttt{(save-state)}
which saves to file \texttt{0-coot.state.scm} or
\texttt{(save-state-filename "thing.scm")} which saves to file
\texttt{thing.scm}.

When Coot starts it can optionally run the commands in
\texttt{0-coot.state.scm}.  Use \texttt{(set-run-state-file-status i)}
to change the behaviour: \texttt{i} is \texttt{0} to never run this
state file at \index{startup dialog (state)}startup, \texttt{i} is
\texttt{1} to get a dialog option (this is the default) and \texttt{i}
is \texttt{2} to run the commands without question.

\section{Backups and Undo}
\label{sec:backups_undo}\index{backups}\index{undo} By default, each 
time a modification is made to a model, the old coordinates are
written out\footnote{this might be surprising since this could chew up
  a lot of disk space.  However, disk space is cheap compared to
  losing you molecule.}.  The backups are kept in a backup directory
and are tagged with the date and the history number (lower numbers are
more ancient\footnote{The coordinates are written in pdb format.}).
The ``Undo'' function discards the current molecule and loads itself
from the most recent backup coordinates.  Thus you do not have to
remember to ``Save Changes'' - coot will do it for you\footnote{unless
  you tell it not to, of course - use (\emph{e.g.})
  \texttt{(turn-off-backup 0)} to turn off the backup (for molecule 0
  in this case).}.

If you have made changes to more than one molecule, Coot will pop-up a
dialog box in which you should set the ``Undo Molecule'' \emph{i.e.}
the molecule to which the Undo operations will apply.  Further Undo
operations will continue to apply to this molecule until there are
none left.  If another Undo is requested Coot checks to see if there
are other molecules that can be undone, if there is exactly one, then
that molecule becomes the ``Undo Molecule'', if there are more than
one, then another Undo selection dialog will be displayed.



\subsection{Redo}
\index{redo}The ``undone'' modifications can be re-done using this
button.  This is not available immediately after a
modification\footnote{It works like the ``Forwards'' buttons in a web
  browser - which is not available immediately after viewing a new
  page.}.

\subsection{Restoring from Backup}
\index{restore after crash} There may be certain
circumstances\footnote{for example, if coot crashes.} in which you
wish to restore from a backup but can't get it by the ``Undo''
mechanism described above.  In that case, start coot as normal and
then open the (typically most recent) coordinates file in the
directory \texttt{coot-backup} (or the directory pointed to the
environment varialble \texttt{COOT\_BACKUP\_DIR} if it was set) .
This file should contain your most recent edits.  In such a case, it
is sensible for neatness purposes to immediately save the coordinates
(probably to the current directory) so that you are not modifying a
file in the backup directory.

See also Section \ref{sec:crash}.

\section{View Matrix}
\index{view matrix}It is sometimes useful to use this to orient the
view and export this orientation to other programs.  The orientation
matrix of the view can be displayed (in the console) using:

\texttt{(view-matrix)}

\section{Space Group}
Occassionally you may want to know the space group of a particular
molecule.  Interactively (for maps) you can see it using the Map
Properties button in the Molecule Display Control dialog.

There is a scripting interface function that returns the space group
for a given molecule \footnote{if no space group has been assigned it
  returns \texttt{``No spacegroup for this molecule''}}:

\texttt{(show-spacegroup \emph{imol})}

\section{Recentring View}
\label{sec:recentring-view}
\index{recentring view}
\begin{trivlist}
\item Use Control + left-mouse to drag around the view
\item or
\item middle-mouse over an atom.  In this case, you will often see
  ``slide-recentring'', the graphics smoothly changes between the
  current centre and the newly selected centre.
\item or
\item Use \textsf{Draw$ \rightarrow$ Go To Atom\ldots} to select an atom
  using the keyboard.  Note that you can subsequently use ``Space'' in
  the ``graphics'' window (OpenGL canvas) to recentre on the next
  C$\alpha$.
\end{trivlist}

If you don't want smooth recentring (sliding)\index{sliding}
\textsf{Draw $\rightarrow$ Smooth Recentring $\rightarrow$ Off}.  You
can also use this dialog to speed it up a bit (by decreasing the
number of steps instead of turning it off).

\section{Clipping manipulation}
\label{clipping manipulation}
\index{clipping} The clipping planes (a.k.a. ``slab''\index{slab}) can
be adjusted using \textsf{Edit $\rightarrow$ Clipping} and adjusting
the slider.  There is only one parameter to change and it affects both
the front and the back clipping planes\footnote{I find a clipping
  level of about 3.5 to 4 comfortable for viewing electron density
  maps - it is a little ``thinner'' than the default startup
  thickness.}.
The clipping can also be changed using keyboard ``D'' and ``F''.

One can ``push'' and ``pull'' the view in the screen-Z direction using
keypad 3 and keypad ``.'' (see Section \ref{keyboard_translation}).

\section{Background colour}
\index{background colour}
The background colour can be set either using a GUI dialog
(\textsf{Edit$ \rightarrow$ Background Colour}) or the function
\texttt{(set-background-colour 0.00 0.00 0.00)}, where the arguments
are 3 numbers between 0.0 and 1.0, which respectively represent the
red, green and blue components of the background colour.  The default
is (0.0, 0.0, 0.0) (black).

\section{Unit Cell}
\index{unit cell} If coordinates have symmetry available then unit
cells can be drawn for molecules (\textsf{Draw $\rightarrow$ Cell \&
  Symmetry $\rightarrow$ Show Unit Cell?}).

The unit cell of maps can be drawn without needing to setup CCP4 first.

\section{Rotation Centre Pointer}
\index{rotation centre pointer} There is a pink pointer\index{pink
  pointer} at the centre of the screen that marks the rotation centre.
The size of the pointer can be changed using \textsf{Edit
  $\rightarrow$ Pink Pointer Size\ldots} or using scripting commands:
\texttt{(set-rotation-centre-size 0.3)}.

\subsection{Pointer Distances}
The Rotation Centre Pointer is sometimes called simply ``Pointer''.
One can find distances to the pointer from any active set of atoms
using ``Pointer Distances'' (under Measures).  If you move the Pointer
(\emph{e.g.} by centering on an atom) and want to update the distances
to it, you have to toggle off and on the ``Show Pointer Distances'' on
the Pointer Distances dialog.

\section{Crosshairs}
\index{crosshairs}Crosshairs can be drawn at the centre of the screen,
using either ``c''\footnote{and ``c'' again to toggle them off.} in
graphics window or \textsf{Draw $\rightarrow$ Crosshairs\ldots}.  The
ticks are at 1.54\AA, 2.7\AA\ and 3.8\AA.

\section{Frame Rate}
\index{frame rate}
Sometimes, you might as yourself ``how fast is the
computer?''\footnote{compared to some other one.}.  Using
\texttt{Calculate $\rightarrow$ Frames/Sec} you can see how fast the
molecule is rotating, giving an indication of graphics performance.
It is often better to use a map that is more realistic and stop the
picture whizzing round.  The output is written to the console, you need
to give it a few seconds to ``settle down''.  It is best not to have
other widgets overlaying the GL canvas as you do this.

The contouring elapsed time\footnote{prompted by changing the contour
  level.} gives an indication of CPU performance.

\section{Program Output}
\index{output} Due to its ``in development'' nature (at the moment),
Coot produces a lot of ``console''\footnote{\emph{i.e.} the terminal
  in which you started Coot.} output - much of it debugging or
``informational''.  This will go away in due course.  You are advised
to run Coot so that you can see the console and the graphics window at
the same time, since feedback from atom clicking (for example) is
often written there rather than displayed in the graphics window.

\begin{itemize}
\item Output that starts ``ERROR...'' is a programming problem (and
  ideally, you should never see it).
\item Output that starts ``WARNING...'' means that something propably
unintented happened due to the unexpected nature of your input or
file(s).
\item Output that starts ``DEBUG...'' has (obviously enough) been
  added to aid debugging.  Most of them should have been cleaned up
  before release, but as Coot is constantly being developed, a few may
  slip through.  Just ignore them.
\end{itemize}


% -----------------------------------------------------------
\chapter{Coordinate-Related Features}
% -----------------------------------------------------------


\section{Read coordinates}
The format\index{coordinates format} of coordinates that can be read
by coot is either PDB or mmCIF.  To read coordinates, choose
\textsf{File $\rightarrow$ Read Coordinates} from the menu-bar.
Immediately after the coordinates have been read, the view is (by
default) recentred to the centre of this new molecule and the molecule
is displayed.  To disable the recentring of the view on reading a
coordinates file, use: \texttt{(recentre-on-read-pdb 0)}.

\subsection{Read multiple coordinate files}
\index{reading multiple pdb files}\index{multiple coordinates files}
The reading multiple files using the GUI is not available (at the
moment).  However the following scripting functions are available:

\texttt{(read-pdb-all)}

which reads all the ``*.pdb'' files in the current directory

\texttt{(multi-read-pdb \emph{glob-pattern} \emph{dir})}

which reads all the files matching \texttt{\emph{glob-pattern}} in
directory \texttt{\emph{dir}}.  Typical usage of this might be:

\texttt{(multi-read-pdb "a*.pdb" ".")}

Alternatively you can specify the files to be opened on the command
line when you start coot (see Section
\ref{sec:command_line_arguments}).

\section{Atom Info}
\index{atom info}\index{residue info} Information about about a
particular atom is displayed in the text console when you click using
middle-mouse.  Information for all the atoms in a residue is available
using \textsf{Info $\rightarrow$ Residue Info\ldots}.

\index{edit B-factors}\index{edit occupancy}The temperature factors
and occupancy of the atoms in a residue can be set by using
\textsf{Edit $\rightarrow$ Residue Info\ldots}.

\section{Atom Labeling}
\index{atom labeling}
\label{sec:atom}
Use Shift + left-mouse to label atom.  Do the same to toggle off the
label.  The font size is changeable using \textsf{Edit $\rightarrow
  $Font Size\ldots}.  The newly centred atom is labelled by default.
To turn this off use:

\texttt{(set-label-on-recentre-flag 0)}

\index{atom label, brief}Some people prefer to have atom labels that
are shorter, without the slashes and residue name:

\texttt{(set-brief-atom-labels 1)}

\section{Atom Colouring}
The atom colouring \index{colouring, atoms} \index{atom colouring}
system in coot is unsophisticated. Typically, atoms are coloured by
element: carbons are yellow, oxygens red, nitrogens blue, hydrogens
white and everything else green (see Section \ref{sec:display_manager}
for colour by chain).  However, it is useful to be able to distinguish
different molecules by colour, so by default coot rotates the colour
map of the atoms (\emph{i.e.} changes the H value in the
HSV\footnote{Hue Saturation Value (Intensity).}  colour system).  The
amount of the rotation depends on the molecule number and a
user-settable parameter:
\begin{trivlist}
\item \texttt{(set-colour-map-rotation-on-read-pdb 30)}.
\end{trivlist}

The default value is 31$^\circ$.

Also one is able to select only the Carbon atoms to change colour in
this manner: \texttt{(set-colour-map-rotation-on-read-pdb-c-only-flag
  1)}.

\section{Bond Parameters}
The various bond parameters can be set using the GUI dialog
\textsf{Draw $\rightarrow$ Bond Parameters} or \emph{via} scripting
functions.

\subsection{Bond Thickness}
\index{bond thickness}\index{width, bonds} The thickness (width) of
bonds of inividual molecules can be changed.  This can be done via the
\textsf{Bond Parameters} dialog or the scripting interface:

\texttt{(set-bond-thickness thickness imol)}

where \texttt{imol} is the molecule number. The default thickness is
3.0. The bond thickness also applies to the symmetry atoms of the
molecule.  There is no means to change the bond thickness of a residue
selection within a molecule.

\subsection{Display Hydrogens}
\index{hydrogens}Initially, hydrogens are displayed.  They can be
undisplayed using 

\texttt{(set-draw-hydrogens mol-no 0)}\footnote{they
  can be redisplayed using \texttt{(set-draw-hydrogens mol-no 1)}.}

where \texttt{mol-no} is the molecule number.

\subsection{NCS Ghosts Coordinates}
\index{NCS}It is occasionally useful when analysing
non-crystallographically related molecules to have ``images'' of the
other related molecules appear matched onto the current coordinates.
As you read in coordinates in Coot, they are checked for NCS
relationships and clicking on ``Show NCS Ghosts'' $\rightarrow$
``Yes'' $\rightarrow$ ``Apply'' will create ``ghost'' copies of them
over the reference chain\footnote{the reference chain is the first
  chain of that type in the coordinates file.}.

\subsection{NCS Maps}
Coot can use the relative transformations of the NCS-related molecules
in a coordinates molecule to transform maps. Use \textsf{Calulate}
$\rightarrow$ \textsf{NCS Maps\ldots} to do this (note the NCS maps
only make sense in the region of the reference chain (see above).
\index{NCS averaging}This will also create an NCS averaged
map\footnote{that also only makes sense in the region of the reference
  chain.}.

\section{Download coordinates}
Coot provides the possibility to download coordinates from an
\index{OCA}OCA\footnote{OCA is ``goose'' in Spanish (and Italian).
  \index{goose}} (\emph{e.g.} EBI) server\footnote{the default is the
  Weizmann Institute - which for reasons I won't go into here is
  currently much faster than the EBI server.} (\textsf{File
  $\rightarrow$ Get PDB Using Code\ldots}). A popup entry box is
displayed into which you can type a PDB accession code.  Coot will
then connect to the web server and transfer the file.  Coot blocks as
it does this (which is not ideal) but on a semi-decent internet
connection, it's not too bad.  The downloaded coordinates are saved
into a directory called \texttt{.coot}.

It is also possible to download mmCIF data and generate a map.  This
currently requires a properly formatted database structure factors
mmCIF file\footnote{which (currently) only a fraction are.}.

\section{Save Coordinates}
On selecting from the menus \textsf{File $\rightarrow$ Save
  Coordinates\ldots} you are first presented with a list of molecules
which have coordinates.  As well as the molecule number, there is the
molecule name - very frequently the name of the file that was read in
to generate the coordinates in coot initially.  However, this is only
a \emph{molecule} name and should not be confused with the filename to
which the coordinates are saved.  The coordinates \emph{filename} can
be selected using the \textsf{Select Filename\ldots} button.

If your filename ends in \texttt{.cif}, \texttt{.mmcif} or
\texttt{.mmCIF} then an mmCIF file will be written (not a ``PDB''
file).

\section{Anisotropic Atoms}
\index{anisotropic atoms} By default anisotropic atom information is
not represented\footnote{using thermal ellipsoids}.  To turn them on,
use \textsf{Draw $\rightarrow$ Anisotropic Atoms $\rightarrow$ Show
  Anisotropic Atoms?  $\rightarrow$ Yes}, or the command:
\texttt{(set-show-aniso 1)}.

You cannot currently display thermal ellipsoids\footnote{in the case
  of isotropic atoms, ellipsoids are spherical, of course.} for
isotropic atoms.

\section{Symmetry}
\index{symmetry} Coordinates symmetry is ``dynamic''.  Symmetry atoms
can be labeled\footnote{symmetry labels are in pale blue and also
  provide the symmetry operator number and the translations along the
  $a$, $b$ and $c$ axes.}.  Every time you recentre, the symmetry gets
updated.  The information shown contains the atom information and the
symmetry operation number and translations needed to generate the atom
in that position.

The symmetry can be represented as C$\alpha$s\index{C$\alpha$ symmetry
  representation}.  This along with representation of the molecule as
C$\alpha$s (Section \ref{sec:display_manager}) allow the production of
a packing diagram\index{packing diagram}.

\section{Sequence View}
\index{sequence view} The protein is represented by one letter codes
and coloured according to secondary structure.  These one letter codes
are active - if you click on them, they will change the centre of the
graphics window - in much the same way as clicking on a residue in the
Ramachandran plot.

\section{Environment Distances}
% not this residue, to symmetry if symmetry is on
% coloured bumps (C)
Environment distances are turned on using \textsf{Info $\rightarrow$
  Environment Distances\ldots}.  Contacts to other residues are shown
and to symmetry-related atoms if symmetry is being displayed.  The
contacts are coloured by atom type\footnote{contacts not involving a
  carbon atom are yellow.}.

\section{Distances and Angles}
The distance between atoms can be found using \textsf{Info
  $\rightarrow$ Distance}\footnote{Use \textsf{Angle} for an angle, of
  course.}.  The result is displayed graphically, and written to the
console.

\section{Zero Occupancy Marker}
\index{zero occupancy}Atoms of zero occupancy are marked with a grey
spot. To turn off these markers, use:

\texttt{(set-draw-zero-occ-markers 0)}

Use an argument of 1 to turn them on.

\section{Mean, Median Temperature Factors}
Coot can be used to calculate the \index{mean B-factor}mean (average)
and \index{median B-factor}median temperatures factors:

\texttt{(average-temperature-factor \emph{imol})}

\texttt{(median-temperature-factor \emph{imol})}

$-1$ is returned if there was a problem\footnote{\emph{e.g.} this
  molecule was a map or a closed molecule.}.

\section{Least-Squares Fitting}
There is currently no GUI specified for this, the scripting interface
is as follows:

\texttt{(simple-lsq-match \emph{ref-start-resno ref-end-resno ref-chain-id imol-ref
           mov-start-resno mov-end-resno mov-chain-id imol-mov
           match-type})}

where:
\begin{trivlist}
\item \texttt{\emph{ref-start-resno}} is the starting residue number
  of the reference molecule
\item \texttt{\emph{ref-end-resno}} is the last residue number
  of the reference molecule
\item \texttt{\emph{mov-start-resno}} is the starting residue number
  of the moving molecule
\item \texttt{\emph{mov-end-resno}} is the last residue number
  of the moving molecule
\item \texttt{\emph{match-type}} is one of \texttt{'CA},
  \texttt{'main}, or \texttt{'all}.
\end{trivlist}

\emph{e.g.}: 
\texttt{(simple-lsq-match 940 950 "A" 0 940 950 "A" 1 'main)}

More sophisticated (match molecule number 1 chain ``B'' on to molecule
number 0 chain ``A''):

\vspace{-2mm}
\begin{quote}
\texttt{(define match1 (list 840 850 "A" 440 450 "B" 'all))}\\
\texttt{(define match2 (list 940 950 "A" 540 550 "B" 'main))}\\
\texttt{(clear-lsq-matches)}\\
\texttt{(set-match-element match1)}\\
\texttt{(set-match-element match2)}\\
\texttt{(lsq-match 0 1)} ; match mol number 1 one mol number 0.
\end{quote}

%% \begin{trivlist}
%% \item \texttt{(define match1 (list 840 850 "A" 440 450 "B" 'all))}
%% \item \texttt{(define match2 (list 940 950 "A" 540 550 "B" 'main))}
%% \item \texttt{(clear-lsq-matches)}
%% \item \texttt{(set-match-element match1)}
%% \item \texttt{(set-match-element match2)}
%% \item \texttt{(lsq-match 0 1)}
%% \end{trivlist}

\section{More on Moving Molecules}
There are scripting functions available for this sort of thing:

\texttt{(molecule-centre \emph{imol})} 

will tell you the molecule centre \index{molecule centre} of the
\texttt{\emph{imol}}th molecule.

\texttt{(translate-by \texttt{imol x-shift y-shift z-shift})}

will translate all the atoms in molecule \texttt{\emph{imol}} by the
given amount (in {\AA}ngstr\"{o}ms)\index{translate molecule}.

\texttt{(move-molecule-to-screen-centre \emph{imol})}

will move the \texttt{\emph{imol}}th molecule to the current centre of
the screen (sometimes useful for imported ligands).  Note that this
moves the atoms of the molecule - not just the view of the molecule.


% -----------------------------------------------------------
\chapter{Modelling and Building}
% -----------------------------------------------------------
\label{modelling,building}

The functions described in this chapter manipulate, extend or build
molecules and can be found under \textsf{Calculate $\rightarrow$
  Model/Fit/Refine\ldots}.

\section{Regularization and Real Space Refinement}
\label{sec:randr}
If you have CCP4 installed, coot will read the geometry restraints for
refmac and use them in fragment (zone) idealization - this is called
``Regularization''\index{regularization}.  The geometrical restraints
are, by default, bonds, angles, planes\index{planes} and non-bonded
contacts.  You can additionally use torsion restraints\index{torsion
  restraints} by \textsf{Calculate $\rightarrow$
  Model/Fit/Refine\ldots $\rightarrow$ Refine/Regularize Control
  $\rightarrow$ Use Torsion Restraints}.

% cite Bob Diamond (1971) here somewhere.



``RS (Real Space) Refinement''\index{refinement} (after Diamond,
1971\footnote{Diamond, R. (1971). A Real-Space Refinement Procedure
  for Proteins. \emph{Acta Crystallographica} \textbf{A}27, 436-452.
  }) in Coot is the use of the map in addition to geometry terms to
improve the positions of the atoms.  Select ``Regularize'' from the
``Model/Fit/Refine'' dialog and click on 2 atoms to define the zone
(you can of course click on the same atom twice if you only want to
regularize one residue).  Coot then regularizes the residue range.  At
the end Coot, displays the intermediate atoms in white and also
displays a dialog, in which you can accept or reject this
regularization.  In the console are displayed the $\chi^2$ values of
the various geometrical restraints for the zone before and after the
regularization.  Usually the $\chi^2$ values are considerably
decreased - structure idealization such as this should drive the
$\chi^2$ values toward zero.

The use of ``Refinement'' is similar - with the addition of using a
map.  The map used to refine the structure is set by using the
``Refine/Regularize Control'' dialog.  If you have read/created only
one map into Coot, then that map will be used (there is no need to set
it explicitly).


Use, for example, \index{\texttt{set-matrix}}\texttt{(set-matrix 20.0)}
\footnote{\texttt{set\_matrix(20.0)} (using python).} to change the
weight of the map gradients to geometric gradients.  The higher the
number the more weight that is given to the map terms\footnote{but the
  resulting $\chi^2$ values are higher.}.  The default is 150.0.  This
will be needed for maps generated from data not on (or close to) the
absolute scale or maps that have been scaled (for example so that
the sigma level has been scaled to 1.0).

For both ``Regularize Zone'' and ``Refine Zone'' one is able to use a
single click to \index{single click refine}\index{refine single
  click}refine a residue range.  Pressing ``A'' on the keyboard while
selecting an atom in a residue will automatically create a residue
range with that residue in the middle.  By default the zone is
extended one residue either size of the central residue.  This can be
changed to 2 either side using \texttt{(set-refine-auto-range-step
  2)}.

Intermediate (white) atoms can be moved around with the mouse (click
and drag with left-mouse, by default).  \marginpar{\footnotesize
  \textsf{This is a useful feature}} Refinement will proceed from the
new atom positions when the mouse button is released.  It is possible
to create incorrect atom nomenclature and/or chiral volumes in this
manner - so some care must be taken.  Press the ``A'' key as you
left-mouse click to move atoms more ``locally'' (rather than a linear
shear) and Cntrl key as you left-mouse click to move just one atom.

To prevent the unintentional refinement of a large number of residues,
there is a ``heuristic fencepost'' of 20 residues.  A selection of
than 20 residues will not be regularized or refined.  The limit can be
changed using the scripting function: \emph{e.g.}
\texttt{(set-refine-max-residues 30)}.

\subsection{Dictionary}
\label{cif-dictionary}\index{cif dictionary, mmCIF dictionary}By default, 
the geometry dictionary entries for only the standard
residues are read in at the start \footnote{And a few extras, such as
  phospate}.  It may be that you particular ligand is not amongst
these.  To interactively add a dictionary entry use \textsf{File
  $\rightarrow$ Import CIF Dictionary}.  Alternatively, you can use
the function:

\texttt{(read-cif-dictionary \emph{filename})}

and add this to your \texttt{.coot} file (this may be the prefered
method if you want to read the file on more than one occassion).  

Note: the dictionary also provides the description of the ligand's
torsions.


\section{Rotate/Translate Zone}
\label{sec:rot_trans_zone}\index{rotate/translate, manual}``Rotate/Translate 
Zone'' from the ``Model/Fit/Refine'' menu allows manual movement of a
zone.  After pressing the ``Rotate/Translate Zone'' button, select two
atoms in the graphics canvas to define a residue range\footnote{if you
  want to move only one residue, then click the same atom twice.}, the
second atom that you click will be the local rotation centre for the
zone.  The atoms selected in the moving fragment have the same
alternate conformation code as the first atom you click.  To actuate a
transformation, click and drag horizontally across the relevant button
in the newly-created ``Rotation \& Translation'' dialog. The axis
system of the rotations and translations are the screen coordinates.
Alternatively \footnote{like Refinement and Regularization}, you can
click using left-mouse on an atom in the fragment and drag the
fragment around. Use Control Left-mouse to move just one atom, rather
than the whole fragment.  Click ``OK'' when the transformation is
complete.

\section{Rigid Body Refinement}
\label{sec:RigidBodyRefinement} \index{refinement, rigid body}
\index{rigid body fit}``Rigid Body Fit Zone'' from the
``Model/Fit/Refine'' dialog provides rigid body refinement.  The
selection is zone-based\footnote{like Regularization and Refinement.}.
So to refine just one residue, click on one atom twice.

Sometimes no results are displayed after Rigid Body Fit Zone.  This is
because the final model positions had too many final atom positions in
negative density.  If you want to over-rule the default fraction of
atoms in the zone that have an acceptable fit (0.75), to be (say)
0.25:

\texttt{(set-rigid-body-fit-acceptable-fit-fraction 0.25)}

\section{Baton Build}
\index{baton build} Baton build is most useful if a skeleton is
already calculated and displayed (see Section \ref{skeletonization}).
When three or more atoms have been built in a chain, Coot will use a
prior probability distribution for the next position based on the
position of the previous three.  The analysis is similar to Oldfield
\& Hubbard\footnote{T. J.  Oldfield \& R. E. Hubbard.  ``Analysis of
  C-Alpha Geometry in Protein Structures'' \emph{Proteins-Structure
    Function and Genetics} \textbf{18(4)} 324 -- 337.}, however it is
based on a more recent and considerably larger database.

Little crosses are drawn representing directions in which is is
possible that the chain goes, and a baton is drawn from the current
point to one of these new positions.  If you don't like this
particular direction\footnote{which is quite likely at first since
  coot has no knowledge of where the chain has been and cannot score
  according to geometric criteria.}, use \textsf{Try Another}.  The
list of directions is scored according to the above criterion and
sorted so that the most likely is at the top of the list and displayed
first as the baton direction.

When starting baton building, be sure to be about 3.8\AA\ from the
position of the first-placed C$\alpha$, this is because the next
C$\alpha$ is placed at the end of the baton, the baton root being at
the centre of the screen.  So, when trying to baton-build a chain
starting at residue 1, centre the screen at about the position of
residue 2.

% ``b'' key in GL canvas
\index{baton mode}Occasionally, every point is not where you want to
position the next atom.  In that case you can either shorten or
lengthen the baton, or position it yourself using the mouse.  Use
``b'' on the keyboard to swap to baton mode for the
mouse\footnote{``b'' again toggles the mode off.}.

Baton-built atoms are placed into a molecule called ``Baton Atom'' and
it is often sensible to save the coordinates of this molecule before
quitting coot.

If you try to trace a high resolution map (1.5\AA\  or better) you will
need to increase the skeleton search depth from the default (10), for
example:

\texttt{(set-max-skeleton-search-depth 20)}

Alternatively, you could generate a new map using data
to a more moderate resolution (2\AA), the map may be easier to
interpret at that resolution anyhow\footnote{high-resolution map
  interpretation is planned.}.

The guide positions are updated every time the ``Accept'' button is
clicked.  The molecule name for these atoms is ``Baton Build Guide Points''
and is is not usually necessary to keep them.

\subsection{Building Backwards}
The following senario is not uncommon: you find a nice streatch of
density and start baton building in it.  After a while you come to a
point where you stop (dismissing the baton build dialog).  You want to
go back to where you started and build the other way.  How do you do
that?

\begin{itemize}
\item Use the command: \texttt{(set-baton-build-params start-resno
    chain-id "backwards")}, where \texttt{start-resno} would typically
  be 0\footnote{\emph{i.e.} one less than the starting residue in the
    forward direction (defaults to 1).} and \texttt{chain-id} would be
  \texttt{""} (default).
\item Recentre the graphics window on the first atom of the just-build
  fragment
\item Select ``C$\alpha$ Baton Mode'' and select a baton direction
  that goes in the ``opposite'' direction to what is typically residue
  2.  This is slightly awkward because the initial baton atoms build
  in the ``opposite'' direction are not dependent on the first few
  atoms of the previously build fragment.
\end{itemize}


\subsection{Undo}
There is also an ``Undo'' button for baton-building.  Pressing this
will delete the most recently placed C$\alpha$ and the guide points
will be recalculated for the previous position.  The number of
``Undo''s is unlimited.  Note that you should use the ``Undo'' button
in the Baton Build dialog, not the one in the ``Model/Fit/Refine''
dialog (Section \ref{sec:backups_undo}).

\subsection{Missing Skeleton}
\index{skeleton, missing}Sometimes (especially at loops) you can see
the direction in which the chain should go, but there is no skeleton
(see Section \ref{skeletonization}) is displayed (and consequently no
guide points) in that direction. In that case, ``Undo'' the previous
atom and decrease the skeletonization level (\textsf{Edit
  $\rightarrow$ Skeleton Parameters $\rightarrow$ Skeletonization
  Level}).  Accept the atom (in the same place as last time) and now
when the new guide points are displayed, there should be an option to
build in a new direction.


\section{C$\alpha \rightarrow$ Mainchain}
\index{mainchain} Mainchain can be generated using a set of C$\alpha$s
as guide-points (such as those from Baton-building) along the line of
Esnouf\footnote{R. M. Esnouf ``Polyalanine Reconstruction from
  C$\alpha$ Positions Using the Program \emph{CALPHA} Can Aid Initial
  Phasing of Data by Molecular Replacement Procedures'' \emph{Acta
    Cryst. }, D\textbf{53}, 666-672 (1997).} or Jones and
coworkers\footnote{T.A.  Jones \& S. Thirup ``Using known
  substructures in protein model building and crystallography''
  \emph{EMBO J.} \textbf{5}, 819--822 (1986).}.  Briefly, 6-residue
fragments of are generated from a list of high-quality\footnote{and
  high resolution} structures. The C$\alpha$ atoms of these fragments
are matched against overlapping sets of the guide-point C$\alpha$s.
The resulting matches are merged to provide positions for the
mainchain (and C$\beta$) atoms.  This proceedure works well for
helices and strands, but less well\footnote{\emph{i.e.}  there are
  severely misplaced atoms} for less common structural features.

This function is also available from the scripting interface:

\texttt{(db-mainchain imol chain-id resno-start resno-end direction)}
    
where direction is either \texttt{"backwards"} or \texttt{"forwards"}.

% Withdrawn due to being to difficult to calculate the atom positions 
% given the phi and psi
%
%\section{Edit Phi/Psi}
%\index{edit $\phi/\psi$}This generates a Ramachandran plot with only
%one residue represented.  You can click and drag this residue round
%the plot and the coordinates in the graphics window will change to the
%$\phi/\psi$ values in the Ramachandran plot.

\section{Backbone Torsion Angles}
It is possible to edit the backbone $\phi$ and $\psi$ angles
indirectly using an option in the Model/Fit/Refine's dialog: ``Edit
Backbone Torsions..''. When clicked and an atom of a peptide is
selected, this produces a new dialog that offers ``Rotate Peptide''
which changes this residues $\psi$ and ``Rotate Carbonyl'' which
changes $\phi$.  Click and drag across the button\footnote{as for
  Rotate/Translate Zone (Section \ref{sec:rot_trans_zone}).} to rotate
the moving atoms in the graphics window.  You should know, of course,
that making these modifications alter the $\phi/\psi$ angles of more
than one residue.


\section{Rotamers}
\label{sec:rotamers}
\index{Dunbrack, Roland}\index{rotamers} The rotamers are generated
from the backbone independent sidechain library of Roland Dunbrack and
co-workers\footnote{R. L.  Dunbrack, Jr. \& F. E.  Cohen. "Bayesian
  statistical analysis of protein sidechain rotamer preferences"
  \emph{Protein Science}, \textbf{6}, 1661--1681 (1997). }. According
to this analysis, some sidechains have many rotamer
options\footnote{LYS, for example has 81.}.  By default only rotamers
with a probability (as derived from the structural database) greater
than 1\% are considered. The initial position is the most likely for
that residue type\footnote{Use \emph{e.g.}
  \texttt{(set-rotamer-lowest-probability 0.5)} to change the
  probability lower limit for the rotamer selection (note that this is
  a percentage, therefore 0.5(\%) is quite low and will allow the
  choice of more rotamers than the default.}.

Use keyboard ``.'' and ``,'' to cycle round the rotamers.

\subsection{Auto Fit Rotamer}
\index{auto-fit rotamer}``Auto Fit Rotamer'' will try to fit the
rotamer to the electron density.  Each rotamer is generated, rigid
body refined and scored according to the fit to the map.  Fitting the
second conformation of a dual conformation in this way will often fail
- the algorithm will pick the best fit to the density - ignoring the
position of the other atoms.

The algorithm doesn't know if the other atoms in the structure are in
sensible positions.  If they are, then it is sensible not to put this
residue too close to them, if they are not then there should be no
restriction from the other atoms as to the position of this residue -
the default is ``are sensible'', which means that the algorithm is
prevented from finding solutions that are too close to the atoms of
other residues. \texttt{(set-rotamer-check-clashes 0)} will stop this.

There is a scripting interface to auto-fitting rotamers:

\texttt{(auto-fit-best-rotamer \emph{resno alt-loc ins-code chain-id\\imol-coords
imol-map clash-flag lowest-rotamer-probability})}

where:

\texttt{\emph{resno}} is the residue number

\texttt{\emph{alt-loc}} is the alternate/alternative location symbol
(\emph{e.g.} \texttt{"A"} or \texttt{"B"}, but most often \texttt{""})

\texttt{\emph{ins-code}} is the insertion code (usually \texttt{""})

\texttt{\emph{imol-coords}} is the molecule number of the coordinates molecule

\texttt{\emph{imol-map}} is the molecule number of the map to which
you wish to fit the side chains

\texttt{\emph{clash-flag}} should the positions of other residues be
included in the scoring of the rotamers (\emph{i.e.} clashing with other
other atoms gets marked as bad/unlikely)

\texttt{\emph{lowest-rotamer-probability}}: some rotamers of some side
chains are so unlikely that they shouldn't be considered - typically
0.01 (1\%).

\subsection{De-clashing residues}
Sometimes you don't have a map\footnote{for example, in preparation of
  a model for molecular replacement} but nevertheless there are
clashing residues\index{clashing residues}\footnote{atoms of residues
  that are too close to each other} (for example after mutation of a
residue range) and you need to rotate side-chains to a non-clashing
rotamer.  There is a scripting interface:

\texttt{(de-clash \texttt{imol chain-id start-resno end-resno})}

\texttt{\emph{start-resno}} is the residue number of the first residue
you wish to de-clash.

\texttt{\emph{start-resno}} is the residue number of the last residue
you wish to de-clash

\texttt{\emph{imol}} is the molecule number of the coordinates molecule

This interface will not change residues with insertion codes or
alternate conformation.  The
\texttt{\emph{lowest-rotamer-probability}} is set to 0.01.


\section{Editing $\chi$ Angles}
\index{edit $\chi$ angles}Instead of using Rotamers, one can instead
change the $\chi$ angles \index{torsions}(often called ``torsions'')
``by hand'' (using ``Edit Chi Angles'' from the ``Model/Fit/Refine''
dialog). To edit a residue's $\chi_1$ press ``1'': to edit $\chi_2$,
``2'': $\chi_3$ ``3'' and $\chi_4$ ``4''.  Use left-mouse click and
drag to change the $\chi$ value.  Use keyboard ``0''\footnote{that's
  ``zero''.} to go back to ordinary view mode at any time during the
editing.  Alternatively, one can use the ``View Rotation Mode'' or use
the Ctrl key when moving the mouse in the graphics window.  Use the
Accept/Reject dialog when you have finished editing the $\chi$ angles.

It should be emphasised that for standard residues this is an option
of last resort - use the other rotamer manipulation options first.

\subsection{Ligand Torsion angles}
\index{torsion angles, ligand}\index{ligand torsion angles}For
ligands, you will need to read the mmCIF file that contains a
description of the ligand's geometry (see Section
\ref{cif-dictionary}).  By default, torsions that move hydrogens are
not included.  Only 9 torsion angles are available from the keyboard
torsion angle selection.

\section{Pep-flip}
\index{pepflip}\index{flip peptide} Coot uses the same pepflip scheme
as is used in $O$ (\emph{i.e.} the C, N and O atoms are rotated
180$^o$ round a line joining the C$\alpha$ atoms of the residues
involved in the peptide).  Flip the peptide again to return the atoms
to their previous position.


\section{Add Alternate Conformation}
\label{sec:add_alt_conf}
The allows the addition alternate (\index{dual conformations}dual,
triple \emph{etc.})  conformations to the picked residue.  By default,
this provides a choice of rotamer (Section \ref{sec:rotamers}).  If
there are not the correct main chain atoms a rotamer choice cannot be
provided, and Coot falls back to providing intermediate atoms.

The default occupancy for new atoms is 0.5.  This can be changed by
using use slider on the rotamer selection window or by using the
scripting function:

\texttt{(set-add-alt-conf-new-atoms-occupancy 0.4)}

% The intermediate atoms interface can be forced using:

% \texttt{(set-show-alt-conf-intermediate-atoms 1)}


\section{Mutation}
\index{mutate} Mutations are available on a 1-by-1 basis using the
graphics.  After selecting ``Mutate\ldots'' from the
``Model/Fit/Refine'' dialog, click on an atom in the graphics.  A
``Residue Type'' window will now appear.  Select the new residue type
you wish and the residue in the graphics is updated to the new residue
type\footnote{Note that selecting a residue type that matches the
  residue in the graphics will also result in a mutation}.  The
initial position of the new rotatmer is the \emph{a priori} most
likely rotamer. Note that in interactive mode, such as this, a residue
type match\footnote{\emph{i.e.} the current residue type matches the
  residue type to which you wish to mutate the residue} will not stop
the mutation action occurring.

\subsection{Multiple mutations}
This dialog can be found under \textsf{Calculate $\rightarrow$ Mutate
  Residue Range}.  A residue range can be assigned a sequence and
optionally fitted to the map.  This is useful converting a poly-ALA
model to the correct sequence\footnote{\emph{e.g.} after using Ca
  $\rightarrow$ Mainchain.}.

Multiple mutations\index{multi-mutate} are also supported \emph{via}
the scripting interface.  Unlike the single residue mutation function,
a residue type match \emph{will} prevent a modification of the
residue\footnote{\emph{i.e.} the residue atoms will remain untouched}.
Two functions are provided: To mutate a whole chain, use
\texttt{(mutate-chain \emph{imol} \emph{chain-id sequence})} where:

\texttt{\emph{chain-id}} is the chain identifier of the chain that you wish
to mutate (\emph{e.g.} \texttt{"A"}) and 

\texttt{\emph{imol}} is molecule number.  

\texttt{\emph{sequence}} is a list of single-letter residue codes,
such as \texttt{"GYRESDF"} (this should be a straight string with no
additional spaces or carriage returns).

Note that the number of residues in the sequence chain and those in
the chain of the protein must match exactly (\emph{i.e.} the whole of
  the chain is mutated (except residues that have a matching residue
  type).)

To mutate a residue range, use 

\begin{trivlist}
\item 
\texttt{(mutate-residue-range \emph{chain-id}
  \emph{start-res-no} \emph{stop-res-no \newline sequence})}
\end{trivlist}

where

\texttt{\emph{start-res-no}} is the starting residue for mutation

\texttt{\emph{stop-res-no}} is the last residue for mutation, \emph{i.e.}
using values of 2 and 3 for \texttt{\emph{start-res-no}} and
\texttt{\emph{stop-res-no}} respectively will mutate 2 residues.

Again, the length of the sequence must correspond to the residue range
length.

\subsection{Mutate and Autofit}
The function combines Mutation and Auto Fit Rotamer and is the easiest
way to make a mutation and then fit to the map.

\subsection{Renumbering}
\index{renumbering residues}Renumbering is straightforward using the
renumber dialog available under \textsf{Calculate $\rightarrow$
  Renumber Residue Range\ldots}.  There is also a scripting interface:

\texttt{(renumber-residue-range \emph{imol chain-id start-res-no
    last-resno offset})}

\section{Find Ligands}
\index{ligands} You are offered a selection of maps to search (you can
only choose one at a time) and a selection of molecules that act as a
mask to this map.  Finally you must choose which ligand types you are
going to search for in this map\footnote{you can search for many
  different ligand types.}.  Only molecules with less than 400 atoms
are suggested as potential ligands.  New ligands are placed where the
map density is and protein (mask) atoms are \emph{not}).  The masked
map is searched for clusters using a default cut-off of 1.0$\sigma$.
In weak density this cut-off may be too high and in such a case the
cut-off value can be changed using something such as:

\texttt{(set-ligand-cluster-sigma-level 0.8)}

However, if the map to be searched for ligands is a difference map, a
cluster level of 2.0 or 3.0 would probably be more
appropriate\footnote{less likely to generate spurious sites.}.

Each ligand is fitted with rigid body refinement to each potential
ligand site in the map and the best one for each site selected and
written out as a pdb file.  The clusters are sorted by size, the
biggest one first (with an index of 0).  The output placed ligands
files have a prefix ``best-overall'' and are tagged by the cluster
index and residue type of the best fit ligand in that site.

By default, the top 10 sites are tested for ligands - to increase this
use:

\texttt{(set-ligand-n-top-ligands 20)}

\subsection{Flexible Ligands}
\index{ligands, flexible}
If the ``Flexible?'' checkbutton is activated, coot will generate a
number of variable conformations (default 100) by rotating around the
rotatable bonds (torsions).  Each of these conformations will be fitted
to each of the potential ligand sites in the map and the best one will
be selected (again, if it passes the fitting criteria above).

Before you search for flexible ligands you must have read the mmCIF
dictionary for that particular ligand residue type (\textsf{File
  $\rightarrow$ Import CIF dictionary\index{dictionary, cif}}).

Use:

\texttt{(set-ligand-flexible-ligand-n-samples \emph{n-samples})}

where \texttt{\emph{n-samples}} is the number of samples of flexiblity
made for each ligand.  The more the number of rotatable bonds, the
bigger this number should be.

\subsection{Adding Ligands to Model}
After successful ligand searching, one may well want to add that
displayed ligand to the current model (the coordinates set that
provided the map mask).  To do so, use Merge Molecules (Section
\ref{sec:merge_molecules}).


\section{Find Waters}
\index{waters, finding} As with finding ligands, you are given a chose
of maps, protein (masking) atoms. A final selection has to be made for
the cut-off level, note that this value is the number of standard
deviation of the density of the map \emph{after} the map has been
masked.  Then the map is masked by the masking atoms and a search is
made of features in the map about the electron density cut-off value.
Waters are added if the feature is approximately water-sized and can
make sensible hydrogen bonds to the protein atoms.  The new waters are
optionally created in a new molecule called ``Waters''.

You have control over several parameters used in the water finding:

\texttt{(set-write-peaksearched-waters)} 

which writes \texttt{ligand-waters-peaksearch-results.pdb}, which
contains the water peaks (from the clusters) without any filtering and
\texttt{ligand-waters.pdb} which are a disk copy filtered waters that
have been either added to the molecule or from which a new molecule
has been created.

\texttt{(set-ligand-water-spherical-variance-limit min-d max-d)} sets
the minimum and maximum allowable distances between new waters and the
masking molecule (usually the protein).

\texttt{(set-ligand-water-spherical-variance-limit varlim)} sets the
upper limit for the density variance around water atoms. The default
is 0.12.
% $electrons^2/\AA^6$.

The map that is maked by the protein and is searched to find the
waters is written out in CCP4 format as \texttt{"masked-for-waters.map"}.

\subsection{Blobs}
After a water search, Coot will create a blobs dialog (see Section
\ref{sec:blobs}).

\subsection{Check Waters via Difference Map}
Another check of waters that one can perform is the following:

\texttt{(check-waters-by-difference-map \emph{imol-coords}
  \emph{imol-diff-map})}

where \texttt{\emph{imol-coords}} is the molecule number of the
coordinates that contain the waters to be checked

\texttt{\emph{imol-diff-map}} is the molecule number of the difference
map (it must be a difference map, not an ``ordinary'' map).  This
difference map must have been calculated using the waters. So there is
no point in doing this check immediately after ``Find Waters''.  You
will need to run Refmac first\footnote{and remember to check the
  difference map button in the ``Run Refmac'' dialog}.

This analysis will return a list of water atoms that have
outstandingly high local variance of the difference map (by default a
sphere of 1.5\AA\ centred about the atom position).  This analysis
might find waters that are actually something else, for example: part
of a ligand, a sulfate, an anion or cation, only partially occupied or
should be deleted entirely.  Coot\footnote{as yet} doesn't decide what
should be done about these atoms, it merely brings them to your
attention.  It may be interesting to use an anomalous map to do this
analysis.

There is no GUI for this feature.

\section{Add Terminal Residue}
\index{terminal residue} This creates a new residue at the C or N
terminus by fitting to the map.  $\phi/\psi$ angle pairs are selected
at random based on the Ramachandran plot probability (for a generic
residue).  By default there are 100 trials.  It is possible that a
wrong position will be selected for the terminal residue and if so,
you can reject this fit and try again with Fit Terminal
Residue\footnote{usually if this still fails after two repetitions
  then it never seems to work.}. Each of the trial positions are
scored according to their fit to the map\footnote{The map is selected
  using ``Refine/Regularize Control''} and the best one selected.  It
is probably a good idea to run ``Refine Zone'' on these new residues.

\texttt{(set-terminal-residue-do-rigid-body-refine 0)} will disable
rigid body fitting of the terminal residue fragment for
each trial residue position (the default is 1 (on)) - this may help if
the search does not provide good results.

\texttt{(set-add-terminal-residue-n-phi-psi-trials 50)} will change
the number of trials (default is 100).

\section{Add OXT Atom to Residue}
\index{terminal oxygen}\index{OXT atom}At the
C-terminus\index{C-terminus} of a chain of amino-acid residues, there
is a ``modification'' so that the C-O becomes a carbonyl, \emph{i.e.}
an extra (terminal) oxygen (OXT) needs to be added.  This atom is
added so that it is in the plane of the C$\alpha$, C and O atoms of
the residue.

Scripting usage:

\texttt{(add-OXT-to-residue imol residue-number \newline insertion-code
  chain-id)}\footnote{\emph{e.g.} \texttt{(add-OXT-to-residue 0 428 "" "A")}}, 

where \texttt{insertion-code} is typically \texttt{""}.  

Note, in order to place OXT, the N, CA, C and O atoms must be present
in the residue - if (for example) the existing carbonyl oxygen atom is
called ``OE1'' then this function will not work.

\section{Add Atom at Pointer}
By default, ``Add Atom At Pointer'' will pop-up a dialog from which
you can choose the atom type you wish to insert\footnote{including
  sulfate or phosphate ions (in such a case, it is probably useful to
do a ``Rigid Body Fit Zone'' on that new residue).}.  Using
\texttt{(set-pointer-atom-is-dummy 1)} you can by-pass this dialog and
immediately create a dummy atom at the pointer position.  Use an
argument of \texttt{0} to revert to using the atom type selection
pop-up on a button press.

The atoms are added to a new molecule called ``Pointer Atoms''.  They
should be saved and merged with your coordinates outside of Coot.

\section{Merge Molecules}
\index{merge molecules}\label{sec:merge_molecules}
This dialog can be found under ``Calculate'' in the main menubar.
This is typically used to add molecule fragments or residues that are
in one molecule to the ``working'' coordinates\footnote{For example,
  after a ligand search has been performed.}.


\section{Running Refmac}
\index{refmac}\index{running refmac}
Use the ``Run Refmac...'' button to select the dataset and the
coordinates on which you would like to run Refmac.  Note that only
dataset which had Refmac parameters set as the MTZ file was read are
offered as dataset that can be used with Refmac. By default, Coot
displays the new coordinates and the new map generated from refmac's
output MTZ file.  Optionally, you can also display the difference map.

You can add extra parameters \index{refmac parameters} (data lines) to
refmac's input by storing them in a file called
\texttt{refmac-extra-params} in the directory in which you started
coot.

Coot ``blocks''\footnote{\emph{i.e.} Coot is idle and ignores all
  input.} until Refmac has terminated\footnote{This is not an idea
  feature, of course and will be addressed in future.... Digressive
  Musing: If only computers were fast enough to run Refmac
  interactively\ldots}.

The default refmac executable\index{refmac, default}\index{default
  refmac version} is \texttt{refmac5} it is presumed to be in the
path.  If you don't want this, it can be overridden using a
re-definition either at the scripting interface or in one's
\texttt{~/.coot} file \emph{e.g.}:
\begin{trivlist}
\item \texttt{(define refmac-exec "/e/refmac-new/bin/refmac5.6.3")}
\end{trivlist}

\index{refmac map colour}After running refmac several times, you may
find that you prefer if the new map that refmac creates (after refmac
refinement) is the same colour as the previous one (from before this
refmac refinement).  If so, use:

\texttt{(set-keep-map-colour-after-refmac 1)}

which will swap the colours of then new and old refmac map so that the
post-refmac map has the same colour as the pre-refmac map and the
pre-refmac map is coloured with a different colour.


\section{Clear Pending Picks}
\index{Clear Pending Picks}\index{atom picking}Sometimes one can click
on a button\footnote{such that Coot would subsequently expect an atom
  selection ``pick'' in the graphics window.} unintentionally. This
button is there for such a case.  It clears the expectation of an
atom pick.  This works not only for modelling functions, but also
geometry functions (such as Distance and Angle).

\section{Delete}
\index{delete} Single atoms or residues can be deleted from the
molecule using ``Delete\ldots'' from the ``Model/Fit/Refine''dialog.
Pressing this button results in a new dialog, with the options of
``Residue'' (the default), ``Atom'' and ``Hydrogen Atoms''.  Now click
on an atom in the graphics - the deleted object will be the whole
residue of the atom if ``Residue'' was selected and just that atom if
``Atom'' was selected.

If you want to delete multiple items you can either use check the
``Keep Delete Active'' check-button on this dialog or use the Ctrl key
as you click on an atom.  Either of these will keep the dialog open,
ready for deletion of next item.

% document delete-atom, delete-residue, delete-residue-with-altconf here.


\section{Sequence Assignment}
You can assign a (fasta format) sequence to a molecule using:

\texttt{(assign-fasta-sequence imol chain-id fasta-seq)}

This function has been provided as a precursor to functions that will
(as automatically as possible) mutate your current coordinates to one
that has the desired sequence. It will be used in automatic side-chain
assignment (at some stage in the future).

\section{Building Links and Loops}

Coot can make an attempt to build missing linking regions or
loops\footnote{the current single function doesn't always perform very
  well in tests, which is why it is currenty available only in the
  scripting format.}.  This is an area of Coot that needs to be
improved, currently O does it much better.  We will have several
different loop tools here\footnote{I suspect that there is not one
  tool that fits for all.}.  For now:

\texttt{(fit-gap \emph{imol} \emph{chain-id} \emph{start-resno} \emph{stop-resno})}

and 

\texttt{(fit-gap \emph{imol} \emph{chain-id} \emph{start-resno} \emph{stop-resno} \emph{sequence})}

the second form will also mutate and try to rotamer fit the provided sequence.

Example usage: let's say for molecule number 0 in chain \texttt{"A"}
we have residues up to 56 and then a gap after which we have residues
62 and beyond:

\texttt{(fit-gap 0 "A" 57 61 "TYPWS")}

\section{Setting Occupancies}
As well as the editing ``Residue Info'' to change occupancies of
individual atoms, one can use a scripting function to change
occupancies of a whole residue range:

\begin{trivlist}
\item \texttt{(zero-occupancy-residue-range \emph{imol chain-id \\
resno-start resno-last})}
\end{trivlist}

example usage:

\texttt{(zero-occupancy-residue-range 0 "A" 23 28)}

This is often useful to zero out a questionable loop before submitting
for refinement.  After refinement (with refmac) there should be
relatively unbiased density in the resulting 2Fo-Fc-style and
difference maps.

Similarly there is a function to reverse this operation:

\begin{trivlist}
\item \texttt{(fill-occupancy-residue-range \emph{imol chain-id \\
      resno-start resno-last})}
\end{trivlist}




% -----------------------------------------------------------
\chapter{Map-Related Features}
% -----------------------------------------------------------

\section{Maps in General}
Maps are ``infinite,'' not limited to pre-calculated volume (the
``Everywhere You Click - There Is Electron Density''
(EYC-TIED)\index{EYC-TIED} paradigm) symmetry-related electron
density is generated automatically. Maps are easily re-contoured.
Simply use the scroll wheel on you mouse to alter the contour level
(or -/+ on the keyboard)\index{change contour level}.
 
Maps follow the molecule.  As you recentre or move about the crystal,
the map quickly follows.  If your computer is not up to re-contouring
all the maps for every frame, then use \textsf{Draw $\rightarrow$
  Dragged Map\ldots} to turn off this feature.

Unfortunately, there is a bug in map-reading\label{map-reading-bug}.
If the map is not a bona-fide CCP4 map\footnote{\emph{e.g.} it's a
  directory or a coordinate filename.}, then coot will crash.  Sorry.
A fix is in the works but ``it's complicated''.

\section{Create a map}
From MTZ, mmCIF and .phs (\textsc{phases} format)\index{phases format}
data use \textsf{File $\rightarrow$ Read Dataset\ldots}. From a CCP4
map use \textsf{File $\rightarrow$ Read Map}.  After being
generated/read, the map is immediately contoured and centred on the
current rotation centre.

\subsection{Reading CIF data}
There are several maps that can be generated from CIF files that
contain observed Fs, calculated Fs and calculated phases:

\begin{trivlist}
\item \texttt{(read-cif-data-with-phases-fo-alpha-calc
    \emph{cif-file-name})} Calculate an atom map using F$_{obs}$ and
  $\alpha_{calc}$
\item \texttt{(read-cif-data-with-phases-2fo-fc \emph{cif-file-name})}
 Calculate an atom map using F$_{obs}$, F$_{calc}$ and
  $\alpha_{calc}$
\item \texttt{(read-cif-data-with-phases-fo-fc \emph{cif-file-name})}
 Calculate an difference map using F$_{obs}$, F$_{calc}$ and
  $\alpha_{calc}$.
\end{trivlist}

\section{Map Contouring}
\index{contouring, map}Maps can be re-contoured using the middle-mouse
scroll-wheel (buttons 4 and 5 in X Window System$^{\textrm{\tiny TM}}$
terminology).  Scrolling the mouse wheel will change the map contour
level and the map it redrawn.  If you have several maps displayed then
the map that is has its contour level changed can be set using
\textsf{HID$ \rightarrow$ Scrollwheel $\rightarrow$ Attach scroll-wheel
  to which map?}.  If there is only one map displayed, then that is
the map that has its contour level changed (no matter what the
scroll-wheel is attached to in the menu).  The level of the electron
density is displayed in the top right hand corner of the OpenGL canvas.

Use Keyboard + or - to change the contour level if you don't have a
scroll-wheel\footnote{like I don't on my Mac.}.

If you are creating your map from an MTZ file, you can choose to click
on the ``is difference map''\index{difference map} button on the Column
Label selection widget (after a data set filename has been selected)
then this map will be displayed in 2 colours corresponding to + and -
the map contour level.

If you read in a map it is a difference map then there is
a checkbutton to tell Coot that.

If you want to tell Coot that a map is a difference
map\index{difference map colours} after it has been read, use:

\texttt{(set-map-is-difference-map \emph{imol})}

where \texttt{\emph{imol}} is the molecule number.

By default the map radius\footnote{actually, it's a box.} is 10\AA.
The default increment to the electron density depends on whether or
not this is a difference map (0.05 $e^-$/\AA$^3$ for a ``2Fo-Fc''
style map and 0.005 $e^-$/\AA$^3$ for a difference map).  You can
change these using \textsf{Edit $\rightarrow$ Map Parameters} or by
using the ``Properties'' button of a particular map in the Display
Control (Display Manager) window.

\section{Map contour ``scrolling'' limits}
Usually one doesn't want to look at \index{negative contour
  levels}negative contour levels of a map\footnote{in a coot
  difference map you will get to see the negative level contoured at
  the inverted level of the positive level, what I mean is that you
  don't want to see the ``positive'' level going less than 0.}, so
Coot has by default a limit that stops the contour level going beyond
(less than) 0.  To remove the limit:

\texttt{(set-stop-scroll-iso-map 0)} {for a 2Fo-Fc style map}

\texttt{(set-stop-scroll-diff-map 0)} {for a difference map}

To set the limits to negative (\emph{e.g.} -0.6) levels:

\texttt{(set-stop-scroll-iso-map-level -0.6)}

and similarly: 

\texttt{(set-stop-scroll-diff-map-level -0.6)}

where the level is specified in electrons/\AA$^3$.

\section{Map Line Width}
\index{map line width}\index{density line thickness}\index{thickness
  of density lines}The width of the lines that descibe the density can
be changed like this:

\texttt{(set-map-line-width 2)}

The default line width is 1.

\section{``Dynamic'' Map colouring}
\index{colouring, map} By default, maps get coloured according to
their molecule number.  The starting colour (\emph{i.e.} for molecule
0) is blue.  The colour of a map can be changed by \textsf{Edit
  $\rightarrow$ Map Colour..}. The map colour gets updated as you
change the value in the colour selector\footnote{takes you right back
  to the good old Frodo days, no?}.  Use ``OK'' to fix that colour.

\section{Difference Map Colouring}
For some strange reason, some crystallographers\footnote{Jan Dohnalek,
  for instance.} like to have their difference maps coloured with red
as positive and green as negative, this option is for them:

\texttt{(set-swap-difference-map-colours 1)}


\section{Map Sampling}
By default, the Shannon sampling factor is the conventional 1.5.  Use
larger values (\textsf{Edit $\rightarrow$ Map Parameters $\rightarrow$
  Sampling Rate}) for smoother maps\footnote{a value of 2.5 is often
  sufficient.}.

\section{Dragged Map}
By default, the map is re-contoured at every frame during a drag (Ctrl
Left-mouse).  Sometimes this can be annoyingly slow and jerky so it is
possible to turn it off: \textsf{Draw $\rightarrow$ Dragged Map
  $\rightarrow$ No}.

To change this by scripting:

\texttt{(set-active-map-drag-flag 0)}


\section{Dynamic Map Sampling and Display Size}
If activated (\textsf{Edit $\rightarrow$ Map Parameters $\rightarrow$
  Dynamic Map Sampling}) the map will be re-sampled on a courser grid
when the view is zoomed out.  If ``Display Size'' is also activated,
the box of electron density will be increased in size also.  In this
way, you can see electron density for \index{big maps}big maps (many
unit cells) and the graphics still remain rotatable.

\section{Skeletonization}
\label{skeletonization}
\index{skeletonization} \index{bones} The skeleton (also known as
``Bones''\footnote{If you're living in Sweden... or Captain Kirk, that
  is.}) can be displayed for any map.  A map can be skeletonized using
\textsf{Calculate $\rightarrow$ Map Skeleton\ldots}.  Use the option
menu to choose the map and click ``On'' then ``OK'' to the generate
the map (the skeleton is off by default).

The level of the skeleton can be changed by using \textsf{Edit
  $\rightarrow$ Skeleton Parameters\ldots $\rightarrow$
  Skeletonization Level\ldots} and corresponds to the electron density
level in the map.  By default this value is 1.2 map standard
deviations.  The amount of map can be changed using \textsf{Edit
  $\rightarrow$ Skeleton Parameters\ldots $\rightarrow$ Skeleton Box
  Radius\ldots}\footnote{you may think it strange that a box has a
  radius, this is an idiosyncrasy of coot.}.  The units are in \AA
ngstr\"oms, with 40 as the default value.

The skeleton is often recalculated as the screen centre changes - but
not always since it can be an irritatingly slow calculation.
\index{skeleton regeneration}If you want to force a regeneration of
the displayed skeleton, simply centre on an atom (using the middle
mouse button) or press the ``s'' key.

\section{Masks}
\label{masks}
\index{masks} A map can be masked by a set of coordinates. Use the
scripting function: 

\texttt{(mask-map-by-protein map-number
  coords-number 0)}\footnote{the 0 is a placeholder for an as yet
  unimplemented feature (\texttt{invert?}).}.  

This will create a new
map that has density where there are no (close) coordinates.  So for
example, if you wanted to show the density around your ligand, you
would create a coordinates file that contained all the protein except
for the ligand and use those coordinates to mask the map.

There is no GUI interface to this feature at the moment.

\subsubsection{Example}
If one wanted to show just the density around a ligand:

\begin{enumerate}
\item Make a pdb file the contains just the ligand and read it in to
  Coot - let's say it is molecule 1 and the ligand is residue 3 of
  chain ``L''.
\item Get a map that covers the ligand (\emph{e.g.} from refmac).
  Let's say this map is molecule number 2.
\item Mask the map:

\texttt{(mask-map-by-molecule 2 1 \#f)}

This creates a new map.  Turn the other maps off, leaving only the
masked map.

\end{enumerate}

To get a nice rendered image, press F8 (see Section \ref{Raster3D}).


\section{Trimming}
\index{trimming atoms}
If you want to remove all the atoms\footnote{or set their occupancy to
  zero} that lie ``outside the map'' (\emph{i.e.} in low density) you can use

\texttt{(trim-molecule-by-map \emph{imol-coords imol-map density-level\\ delete/zero-occ?})}

where \texttt{\emph{delete/zero-occ?}} is \texttt{0} to remove the atoms and
\texttt{1} to set their occupancy to zero.

There is no GUI interface to this feature.


% -----------------------------------------------------------
\chapter{Validation}
% -----------------------------------------------------------

The validation functions are in the process of being written.  In
future there will be more functions, particularly those that will
interface to other programs\footnote{such as the Richardsons' reduce
  and probe}.

\section{Ramachandran Plots}
\index{Ramachandran plot} Ramachandran plots are ``dynamic''.  When
you change the molecule (\emph{i.e.} move the coordinates of some of
atoms) the Ramachandran plot gets updated to reflect those changes.
Also the underlying $\phi/\psi$ probability density changes according
to the selected residue type (\emph{i.e.} the residue under the mouse
in the plot).  There are 3 different residue types: GLY, PRO, and
not-GLY-or-PRO\footnote{the not-GLY-or-PRO is the most familiar
  Ramachandran plot.}.

When you mouse over a representation of a residue (a little square or
triangle\footnote{prolines have a grey outline rather than a black
  one, triangles are glycines.}) the residue label pops up.  The
residue is ``active'' \emph{i.e.} it can be clicked.  The ``graphics''
view changes so that the C$\alpha$ of the selected residue is centred.
In the Ramachandran plot window, the current residue is highlighted by
a green square.

% The probability levels for acceptable (yellow) and preferred (red) are
% 0.12\% and 6\% respectively and have been chosen to look like those
% from Procheck\index{Procheck}.

\section{Chiral Volumes}
The dictionary is used to identify the chiral atoms of each of the
model's residues.  A clickable list is created of atoms whose chiral
volume in the model is of a different sign to that in the dictionary.

\section{Blobs: a.k.a. Unmodelled density}
\label{sec:blobs}
This is an interface to the Blobs\index{blobs}\index{unmodelled
  density} dialog.  A map and a set of coordinates that model the
protein are required.

A blob is region of relatively high residual election density that
cannot be explained by a simple water\index{unexplained density}. So,
for example, sulfates, ligands, mis-placed sidechains or unbuilt
terminal residues might appear as blobs.  The blobs are in order, the
biggest \footnote{and therefore most interesting} at the top.

\section{Check Waters by Difference Map}
Sometimes waters can be misplaced - taking the place of sidechains or
ligands or crystallization agents such as phosphate for
example\footnote{or the water should be more properly modelled as
  anistrotropic or a split partial site}.  In such cases the variance
of the difference map can be used to identify them.

This function is also useful to check anomalous maps.  Often waters
are placed in density that is really a cation.  If such an atom
diffracts anomalously this can be identified and corrected.

By default the waters with a map variance greater than 3.5 $\sigma$ are
listed.  One can be more rigorous by using a lower cut-off:

\texttt{(set-check-waters-by-difference-map-sigma-level 3.0)}


\section{Validation Graphs}

Coot provides several graphs that are useful for model validation (on
a residue by residue basis): residue denisty fit, geometry distortion,
temperature factor variance, peptide distortion and rotamer analysis.

\subsection{Residue Density Fit}

The residue density fit is by default scaled to a map that is
calculated on the absolute scale.  Some users use maps that have maps
with density levels considerably different to this, which makes the
residue density fit graph less useful.  To correct for this you can
use the scripting function:

\texttt{(set-residue-density-fit-scale-factor \emph{factor})}

where \texttt{\emph{factor}} would be $1/(4\sigma_{map})$ (as a rule
of thumb).

\texttt{(residue-density-fit-scale-factor)} returns the current scale
factor (default 1.0).

\subsection{Rotamer Analysis}
Residue rotamers are scored according to the prior likelihood.  Note
that when CD1 and CD2 of a PHE residue are exchanged (simply a
nomenclature error) this can lead to large red blocks in the graph
(apparently due to very unlikely rotamers).  There are several other
residues that can have nomenclature errors like this.

\subsection{Temperature Factor Variance}

\subsection{Peptide $\omega$ Distortion}

\subsection{Geometry}


% -----------------------------------------------------------
\chapter{Hints}
% -----------------------------------------------------------
\label{chap-hints}
\section{Getting out of ``Translate'' Mode}
If you get stuck in "translate" mode in the GL canvas
(\emph{i.e.} mouse does not rotate the view as you would expect) simply
press and release the Ctrl key to return to "rotate" mode.

\section{Getting out of ``Label Atom Only'' Mode}
Similarly, if you are stuck in a mode where the ``Model/Fit/Refine''
buttons don't work (the atoms are not selected, only the atom gets
labelled), press and release the Shift key.

\section{Button Labels}
Button labels ending in ``\ldots'' mean that a new dialog will pop-up
when this button is pressed.

\section{Picking}
\label{sec:picking}\index{picking} Note that left-mouse in the 
graphics window is used for both atom picking and rotating the view,
so try not to click over an atom when trying to rotate the view when
in atom selection mode.  

% This was a Matrix (GL_PROJECTION) bug.  Fixed now.
%
%Sometimes, when trying to pick an atom you
%get the message ``Model atom pick failed''\index{model atom pick} even
%though you have clicked accurately over the atom.  The work-around is
%to give the model a little wiggle (using the mouse) and try the pick
%again.

\section{Resizing View}
\index{resizing view}\index{zoom} Click and drag using right-mouse (up
and down or left and right) to zoom in and out.

\section{Map}
If the ``Display'' button for the map in the ``Display Manager''
window stops working, close the ``Display Control'' window and re-open
it.  The button should now respond to clicks.

To change the map to which the scroll-wheel is attached, use
\textsf{HID $\rightarrow$ Scrollwheel $\rightarrow $Attach Scrollwheel
  to which map?}
 
\section{Slow Computer Configuration}
\index{slow computer}Several of the parameters of Coot are chosen
because they are reasonable on my ``middle-ground'' development
machine.  However, these parameters can be tweeked so that slower
computers perform better:

\begin{trivlist}
\item \texttt{(set-smooth-scroll-steps 4) ; default 8 }
\item \texttt{(set-smooth-scroll-limit 30) ; Angstroms}
\item \texttt{(set-residue-selection-flash-frames-number 3);}
\item \texttt{(set-skeleton-box-size 20.0) ; A (default 40).}
\item \texttt{(set-active-map-drag-flag 0) ; turn off recontouring every step}
\item \texttt{(set-idle-function-rotate-angle 1.5) ; turn up to 1.5 degrees}
\end{trivlist}

%\appendix
%\chapter{Some Extras}




% Have you documented:
%
% Merge molecules dialog    : done
% Mutate sequence dialog    : done
% Add OXT to residue dialog : done
% Bond Parameters dialog
% Renumber Residues dialog
% Find Bad Chiral Atoms     : done
% Validate Waters (simple)
% Validation Graphs 
% Pointer distances
% Torsions

\documentclass{book}
\usepackage{a4}
\usepackage{palatino}
%\usepackage{times}
%\usepackage{utopia}
\usepackage{euler}
\usepackage{fancyhdr}
\usepackage{epsf}

\newcommand {\atilde} {$_{\char '176}$} % tilde(~) character

%\date{1st April 2004}

\title{The Coot User Manual}
\author{Paul Emsley \\\textsf{\small emsley@ysbl.york.ac.uk}}
\makeindex
\hyphenation{tri-angle}

\begin{document}
\thispagestyle{empty}

%% Make a title page: I can't use \maketitlepage because I want a line

\vspace*{30mm}

{\huge The Coot User Manual}

\begin{picture}(300,2)
\linethickness{5pt}
\put(0,0){\line(1,0){347}}
\end{picture}

\begin{flushright}
%  \today
  1st April 2004
\end{flushright}

\vspace*{20mm}


\begin{center}
  \leavevmode
  \epsfxsize 127mm \epsffile{coot-plain-2.eps}
\end{center}

\vspace*{20mm}

\begin{picture}(300,2)
\linethickness{5pt}
\put(0,0){\line(1,0){347}}
\end{picture}

\begin{flushright}

  Paul Emsley\\\textsf{\small emsley@ysbl.york.ac.uk}
\end{flushright}

%\begin{picture}(width,height)(xoffset,yoffset)
%\end{picture}

%\newpage
\tableofcontents
\pagestyle{headings}

\chapter{Introduction}

\section{This document}
This document is the Coot User Manual, giving a brief overview of the
interactive features.  Other documentations includes (or it is planned
to include) the \index{reference manual}Coot Reference Manual and the
Coot \index{tutorial} Tutorial.  These documents should be distributed
with the source code.

\section{What is Coot?}

Coot is a stand-alone portion of CCP4's Molecular Graphics project. Its
focus is crystallographic model-building and manipulation rather than
representation (\emph{i.e.} more like Frodo than
\index{Rasmol}Rasmol).

Coot is Free Software.  You can give it away. If you don't like the
way it behaves, you can fix it yourself.

\section{What Coot is Not}
Coot is not:
\begin{itemize}
\item CCP4's official Molecular Graphics program\footnote{Coot is
    \emph{part of} that project. The official program (which contains
parts of Coot), ccp4mg is under
    the direct control of Liz Potterton and Stuart McNicholas.}
\item a program to do refinement\footnote{although it does have a
    local refinement algorithm it is no substitute for \textsc{refmac}
    (a wrapper for \textsc{refmac} is available).}
\item a database, in any respect
\item a protein crystallographic suite\footnote{that's the job of the
    CCP4 Program Suite.}.
\end{itemize}

\section{Hardware Requirements}
The code is designed to be portable to any Unix-like operating
system\footnote{including Cygwin, but currently coot is ``unstable''
  on a Cygwin system.}.  Coot certainly runs on SGI IRIX64, RedHat
Linux of various sorts, SuSe Linux\footnote{so far only 8.2 verified.}
and MacOS X (10.2).  The sgi Coot binaries shouold also 
work on IRIX.

If you want to port to some other operating system, you are
welcome\footnote{it's Free Software after all and I could give you a
  hand.}.  Note that your task will be eased by using GNU GCC to compile
the programs components.

\subsection{Mouse}
\index{mouse}Coot works best with a 3-button mouse and works better if
it has a scroll-wheel too (see Chapter 2 for more details)\footnote{I
  can get by with a one button Machintosh - but it's not ideal.}.

\section{Environment Variables}
Coot responds to several command line arguments that modify its
behaviour.  

\begin{trivlist}
\item [\texttt{COOT\_STANDARD\_RESIDUES}] The filename of the pdb file
  containing the standard amino acid residues in ``standard
  conformation''\footnote{as it is known in Clipper.}
\item [\texttt{COOT\_SCHEME\_DIR}] The directory containing auxiliary scheme
  files 
\item [\texttt{COOT\_REF\_STRUCTS}] The directory containing a set of
  high resolution pdb files used as \index{reference
    strctures}reference structures to build backbone atoms from
  C$\alpha$ positions
\item [\texttt{COOT\_REFMAC\_LIB\_DIR}] \index{dictionary, cif}Refmac's
  CIF directory containing the monomers and link descriptions.  In the future
  this may simply be the same directory in which refmac looks to find
  the library dictionary.
\item [\texttt{COOT\_RESOURCES\_DIR}] The directory that contains the
  splash screen image and the GTk application resources.
\item [\texttt{COOT\_BACKUP\_DIR}] The directory to which backup are
  written (if it exists as a directory).  If it is not, then backups
  are written to the current directory (the directory in which coot
  was started).
\end{trivlist}
  
And of course extension language environment variables are used too:

\begin{trivlist}
\item [\texttt{PYTHONPATH}] (for python modules)
\item [\texttt{GUILE\_LOAD\_PATH}] (for guile modules)
\end{trivlist}

Normally, these environment variables will be set correctly in the
coot setup script (which can be found in the \texttt{setup} directory
in the binary distribution.  See the web site (Section \ref{webpage})
for setup details.

\section{Command Line Arguments}
\index{command line arguments}
\label{sec:command_line_arguments}
Rather that using the GUI to read in information, you can use the
following command line arguments:
\begin{itemize}
\item \texttt{--script} to run a script on start up
\item \texttt{--pdb}  for pdb/coordinates file
\item \texttt{--data} for mtz, phs or mmCIF data file
\item \texttt{--map}  for a (currently only CCP4) map
\end{itemize}
So, for example, one might use: 
\begin{trivlist}
\item \texttt{coot --pdb pre-refinement.pdb --pdb post-refinement.pdb}
\end{trivlist}

\section{Web Page}
\label{webpage}
Coot has a \index{web page}web page:

\begin{trivlist}
\item \texttt{http://www.ysbl.york.ac.uk/\atilde emsley/coot}
\end{trivlist}

There you can read more about the CCP4 molecular graphics project in
general and other projects which are important for coot\footnote{coot
  has several influences and dependencies, but these will not be
  discussed here in the User Manual.}.

The web page also contains an example ``setup'' file which assigns the
environment variables to change the behaviour of Coot.

\section{Crash}
\label{sec:crash}
\index{crash}
Coot might crash on you - it shouldn't.  

\index{recover session}\index{crash recovery}There are backup files in
the directory \texttt{coot-backup}\footnote{\$COOT\_BACKUP\_DIR is used
  in preference if set}. You can recover the session (until the last
edit) by reading in the pdb file that you started with last time and
then use \textsf{File $\rightarrow$ Recover Session\ldots}.

I would like to know about coot crashing\footnote{The map-reading
  problem (documented in Section \ref{map-reading-bug}) is already
  known.} so that I can fix it as soon as possible. If you want your
problem fixed, this involves some work on your part sadly.

First please make sure that you are using the most recent version of
coot.  I will often need to know as much as possible about what you
did to cause the bug.  If you can reproduce the bug and send me the
files that are needed to cause it, I can almost certainly fix -
it\footnote{now there's a hostage to fortune.} especially if you
\index{debugger}\index{gdb}use the debugger (gdb) and send a backtrace
too\footnote{to do so, please send me the output of the following:
  \texttt{\$ gdb `which coot` \emph{corefile}} and then at the
  \texttt{(gdb)} prompt type: \texttt{where}, where
  \texttt{\emph{corefile}} is the core dump file, \texttt{core} or
  \texttt{core.4536} or some such.}.

% -----------------------------------------------------------
\chapter{Mousing and Keyboarding}
% -----------------------------------------------------------
\index{mouse buttons}
How do we move around and select things?

\vspace{0.5cm}
  \begin{tabular}{ll}
    Left-mouse Drag & Rotate view \\
    Ctrl Left-Mouse Drag &  Translates view \\
    Shift Left-Mouse &  Label Atom\\
    Right-Mouse Drag &  Zoom in and out\index{zoom}\\
    Shift Right-Mouse Drag &  Rotate View around Screen Z axis\\
    Middle-mouse & Centre on atom\\
    Scroll-wheel Forward &  Increase map contour level\\
    Scroll-wheel Backward &  Decrease map contour level
  \end{tabular}
\vspace{3mm}

See also Chapter \ref{chap-hints} for more help.

\section{Next Residue}
\begin{tabular}{ll}
  ``Space'' & Next Residue \\
  ``Shift'' ``Space'' & Previous Residue
\end{tabular}

See also ``Recentring View'' (Section \ref{sec:recentring-view}).

\section{Keyboard Contouring}

Use \texttt{+} or \texttt{-} on the keyboard if you don't have a
scroll-wheel.

\section{Keyboard Rotation}
\index{keyboard rotation}By popular request keyboard equivalents of
rotations have been added\footnote{particularly for those with
  PowerMates (the amount of rotation can be changed to 2$^\circ$ (from
  the default 1$^\circ$) using \texttt{(set-idle-function-rotate-angle
    2.0)}).}: \vspace{3mm}

\begin{tabular}{ll}
  Q & Rotate + X Axis \\
  W & Rotate - X Axis \\
  E & Rotate + Y Axis \\
  R & Rotate - Y Axis \\
  T & Rotate + Z Axis \\
  Y & Rotate - Z Axis \\
  I & Continuous Y Axis Rotation
\end{tabular}
% document rotate-x-sceen nsteps step-size here?

\section{Keyboard Translation}
\index{translation, keyboard}
\label{keyboard_translation}
\begin{tabular}{ll}
  Keypad 3 & Push View (+Z translation)\\
  Keypad . & Pull View (-Z translation)
\end{tabular}


\section{Keyboard Zoom and Clip}

\begin{tabular}{ll}

  N & Zoom out   \\
  M & Zoom in    \\
  D & Slim clip  \\
  F & Fatten clip\\
\end{tabular}

\vspace{3mm}

\section{Scrollwheel}
When there is no map, using the scroll-wheel has no effect.  If there
is exactly one map displayed, \index{contouring, map} the scroll-wheel
will change the contour level of that map.  If there are two or more
maps, the map for which the contour level is changed can be set by
\textsf{HID $\rightarrow$ Scrollwheel $\rightarrow$ Attach scroll-wheel
  to which map?} and selecting a map number.

\section{Selecting Atoms}
Several Coot functions require the selecting of atoms to specify a
residue range (for example: Regularize, Refine (Section
\ref{sec:randr}) or Rigid Body Fit Zone (Section
\ref{sec:RigidBodyRefinement})).  Select atoms with the Left-mouse.
See also Picking (Section \ref{sec:picking}).

Use the scripting function
\index{quanta-buttons}\texttt{(quanta-buttons)} to make the mouse
functions more like other molecular graphics programs to which you may
be more accustomed\footnote{See also \ref{sec:quanta-zooming}}.

\section{Virtual Trackball}
\index{trackball, virtual} You may not completely like the way the
molecule is moved by the mouse movement\footnote{Mouse movement in
  ``Spherical Surface'' mde generates a component of (often
  undesirable) screen z-rotation, particularly noticeable when the
  mouse is at the edge of the screen.}.  To change this, try:
\textsf{HID $\rightarrow$ Virtual Trackball $\rightarrow$ Flat}.  To
do this from the scripting interface: \texttt{(set-vt
  1)}\footnote{\texttt{(set-vt 0)} to turn it back to ``Spherical''
  mode.}.

If you \emph{do} want \index{screen-z rotation}\index{z-rotation}
screen-z rotation, you can either use Shift Right-Mouse Drag or set
the Virtual Trackball to Spherical Surface mode and move the mouse
along the bottom edget of the screen.

\section{More on Zooming}
\label{sec:quanta-zooming}
The function \texttt{(quanta-like-zoom)} adds the ability to zoom the
view using just Shift + Mouse movement\footnote{this is off by default
  because I find it annoying.}.

There is also a Zoom slider\index{zoom, slider} (\textsf{Draw
  $\rightarrow$ Zoom}) for those without a right-mouse button.

% -----------------------------------------------------------
\chapter{General Features}
% -----------------------------------------------------------

The map-fitting and model-building tools can be accessed by using
\textsf{Calculate $\rightarrow$ Model/Fit/Refine\ldots}.  Many
functions have \index{tooltips}tooltips\footnote{Put your mouse over a
  widget for a couple of seconds, if that widget has a tooltip, it
  will pop-up in a yellow box.}\index{tooltips} describing the
particular features and are documented in Chapter
\ref{modelling,building}.

\section{Version number}
\index{version number}
The version number of Coot can be found at the top of the ``About''
window (\textsf{Help $\rightarrow$ About}).

There is also a script function to return the version of coot:

\texttt{(coot-version)}

\section{Antialiasing}
Antialiasing (for what it's worth) can be enabled using:

\texttt{(set-do-anti-aliasing 1)}

The default is \texttt{0} (off).

\section{Molecule Number}
\index{molecule number} 
Coot is based on the concept of molecules.  Maps and coordinates are
different representations of molecules.  The access to the molecule is
\emph{via} the \emph{molecule number}.  It is often important therefore to
know the molecule number of a particular molecule.

Molecule numbers can be found by clicking on an atom in that molecule
(if it has coordinates of course). The first number in brackets in the
resulting text in the console is the molecule number.  The molecule
number can also be found in Display Control window (Section
\ref{sec:display_manager}).  It is also displayed on the left-hand
side of the molecule name in the option menus of the ``Save
Coordinates'' and ``Go To Atom'' windows.

\section{Display Issues}
The ``graphics'' window is drawn using \index{OpenGL}OpenGL.  It is
considerably smoother when using a 3D accelerated X server. 

The view is orthographic (\emph{i.e.} the back is the same size as the
front).  The default clipping is about right for viewing coordinate
data, but is often a little too ``thick'' for viewing electron
density.  It is easily changed (see Section \ref{clipping
  manipulation}).

Depth-cueing\index{depth-cueing} is linear and fixed on. There is no
anti-aliasing\footnote{coot is not the program for snazzy graphics -
  CCP4mg is the program for that sort of thing.}.

The graphics window can be resized, but it has a minimum size of
400x400 pixels.

\subsection{Origin Marker}

A yellow box\index{yellow box} called the ``origin
marker''\index{origin marker} marks the origin.  It can be removed
using:

\texttt{(set-show-origin-marker 0)}

Its state can be queried like this:

\texttt{(show-origin-marker-state)}

which returns an number (an integer).

\subsection{Raster3D output}
\label{Raster3D}Output suitable for use by Raster3D\index{Raster3D}'s
``render''\index{render} can be generated using the scripting function

\texttt{(raster3d \emph{file-name})}

where \texttt{\emph{file-name}} is such as
\texttt{"test.r3d"}\footnote{Also povray will be supported in the
  future.}.

There is a keyboard key to generate this file, run ``render'' and
display the image: Function key F8.

You can also use the function

\texttt{(render-image)}

which will create a file \texttt{coot.r3d}, from which ``render'' produces
\texttt{coot.png}. This png file is displayed using ImageMagick's display
program (by default).  Use something like:

\texttt{(set! coot-png-display-program "gqview")}

to change that to different display program ("gqview" in this case).

To change the widths of the bonds and density ``lines'' use (for example):

\texttt{(set-raster3d-bond-thickness 0.1)}

and 

\texttt{(set-raster3d-density-thickness 0.01)}

To turn off the representations of the atoms (spheres):

\texttt{(set-renderer-show-atoms 0)}



\section{Display Manager}
\label{sec:display_manager}
\index{Display Manager} This is also known as ``Map and molecule
(coordinates) display control''.  Here you can select which maps and
molecules you can see and how they are drawn\footnote{to a limited
  extent.}.  The ``Display'' and ``Active'' are toggle buttons, either
depressed (active) or undepressed (inactive).  The ``Display'' buttons
control whether a molecule (or map) is drawn and the ``Active'' button
controls if the molecule is clickable\footnote{the substantial
  majority of the time you will want your the buttons to be both
  either depressed or undepressed, rarely one but not the other.}
(\emph{i.e.} if the molecule's atoms can be labeled).

By default, the path names of the files are not displayed in the
Display Manager.  To turn them on:

\texttt{(set-show-paths-in-display-manager 1)}

\index{colour by chain}\index{atom colouring}If you pull across the
horizontal scrollbar in a Molecule view, you will see the ``Render
as'' menu.  You can use this to change between normal ``Bonds (Colour
by Atom)'',``Bonds (Colour by Chain)'' and ``C$\alpha$''
representation\index{C$\alpha$ representation}.  There is also
available ``No Waters'' and ``C$\alpha$ + ligands'' representations.

\section{The file selector}
\subsection{File-name Filtering}
\index{file-name filtering} The ``Filter'' button in the fileselection
filters the filenames according to extension.  For coordinates files
the extensions are ``.pdb'' ``.brk'' ``.mmcif'' and others.  For data:
``.mtz'', ``.hkl'', ``.phs'', ``.cif'' and for (CCP4) maps ``.ext'',
``.msk'' and ``.map''.  If you want to add to the extensions, the
following functions are available:

\begin{trivlist}
\item \texttt{(add-coordinates-glob-extension \emph{extension})}
\item \texttt{(add-data-glob-extension \emph{extension})}
\item \texttt{(add-map-glob-extension \emph{extension})}
\item \texttt{(add-dictionary-glob-extension \emph{extension})}
\end{trivlist}
where \texttt{\emph{extension}} is something like: \texttt{".mycif"}.

\subsection{Filename Sorting}
If you like your files initially sorted by date (rather than
lexographically, which is the default use:

\texttt{(set-sticky-sort-by-date)}

\section{Scripting}
\index{scripting} There is an compile-time option of adding a script
interpreter.  Currently the options are python and guile.  Hopefully,
in the near future you will be able to use both in the same
executable, but that's not available today.

Hundreds of commands are made available for use in scripting by using
SWIG.  These are are currently not well documented but can be found in
the Coot Reference Manual or the source code (\texttt{c-interface.h}).

Commands described throughout this manual (such as \texttt{(vt-surface
  1))} can be evaluated\index{executing commands} directly by Coot by
using the ``Scripting Window'' (\textsf{Calculate $\rightarrow$
  Scripting\ldots}). Note that you type the commands in the lower
entry widget and the command gets echoed (in red) and the return vaule
and any output is displayed in the text widget above.  The typed
command should be terminated with a carriage return\footnote{which
  causes the evalution of the command.}.  Files\footnote{such as the
  Coot state file (Section \ref{sec:coot_state}).} can be evaluated
(executed) using \textsf{Calculate $\rightarrow$ Run Script\ldots}.
Note that in scheme (the usual scripting language of Coot), the
parentheses are important.

\subsection{Python}
\index{python} Coot has an (optional) embedded python interpreter.
Thus the full power of python is available to you.  Coot will look for
an initialization script \index{startup settings
  (python)}\index{\texttt{.coot.py}}(\texttt{\$HOME/.coot.py}) and
will execute it if found.  This file should contain python commands
that set your personal preferences.


\subsubsection{Python Commands}
The scripting functions described in this manual are formatted
suitable for use with guile, \emph{i.e.}:

\texttt{(\emph{function} \emph{arg1} \emph{arg2\ldots})}

If you are using Python instead: the format needs to be changed to:

\texttt{\emph{function}(\emph{arg1},\emph{arg2\ldots})}

Note that dashes in guile function names become underscores for
python, so that (for example) \texttt{(raster-screen-shot)} becomes
\texttt{raster\_screen\_shot()}.


\subsection{Scheme}
\index{guile}%
\index{scheme} The scheme interpreter is made available by embedding
guile.  The initialization script used by this interpreter is
\index{startup settings (scheme)} \index{\texttt{.coot}}
\texttt{\$HOME/.coot}.  This file should contain scheme commands that
set your personal preferences.


\subsection{State}
\label{sec:coot_state}
The ``state''\index{state} of coot is saved on Exit and written to a
file called \texttt{0-coot.state.scm} (scheme)
\texttt{0-coot.state.py} (python).   This
state file contains information about the screen centre, the
clipping, colour map rotation size, the symmetry radius, and other
molecule related parameters such as filename, column labels,
coordinate filename \emph{etc.}.

Use \textsf{Calculate $\rightarrow$ Run Script\ldots} to use this file
to re-create the loaded maps and models that you had when you finished
using Coot\footnote{in that particular directory.} last time.
A state file can be saved at any time using \texttt{(save-state)}
which saves to file \texttt{0-coot.state.scm} or
\texttt{(save-state-filename "thing.scm")} which saves to file
\texttt{thing.scm}.

When Coot starts it can optionally run the commands in
\texttt{0-coot.state.scm}.  Use \texttt{(set-run-state-file-status i)}
to change the behaviour: \texttt{i} is \texttt{0} to never run this
state file at \index{startup dialog (state)}startup, \texttt{i} is
\texttt{1} to get a dialog option (this is the default) and \texttt{i}
is \texttt{2} to run the commands without question.

\section{Backups and Undo}
\label{sec:backups_undo}\index{backups}\index{undo} By default, each 
time a modification is made to a model, the old coordinates are
written out\footnote{this might be surprising since this could chew up
  a lot of disk space.  However, disk space is cheap compared to
  losing you molecule.}.  The backups are kept in a backup directory
and are tagged with the date and the history number (lower numbers are
more ancient\footnote{The coordinates are written in pdb format.}).
The ``Undo'' function discards the current molecule and loads itself
from the most recent backup coordinates.  Thus you do not have to
remember to ``Save Changes'' - coot will do it for you\footnote{unless
  you tell it not to, of course - use (\emph{e.g.})
  \texttt{(turn-off-backup 0)} to turn off the backup (for molecule 0
  in this case).}.

If you have made changes to more than one molecule, Coot will pop-up a
dialog box in which you should set the ``Undo Molecule'' \emph{i.e.}
the molecule to which the Undo operations will apply.  Further Undo
operations will continue to apply to this molecule until there are
none left.  If another Undo is requested Coot checks to see if there
are other molecules that can be undone, if there is exactly one, then
that molecule becomes the ``Undo Molecule'', if there are more than
one, then another Undo selection dialog will be displayed.



\subsection{Redo}
\index{redo}The ``undone'' modifications can be re-done using this
button.  This is not available immediately after a
modification\footnote{It works like the ``Forwards'' buttons in a web
  browser - which is not available immediately after viewing a new
  page.}.

\subsection{Restoring from Backup}
\index{restore after crash} There may be certain
circumstances\footnote{for example, if coot crashes.} in which you
wish to restore from a backup but can't get it by the ``Undo''
mechanism described above.  In that case, start coot as normal and
then open the (typically most recent) coordinates file in the
directory \texttt{coot-backup} (or the directory pointed to the
environment varialble \texttt{COOT\_BACKUP\_DIR} if it was set) .
This file should contain your most recent edits.  In such a case, it
is sensible for neatness purposes to immediately save the coordinates
(probably to the current directory) so that you are not modifying a
file in the backup directory.

See also Section \ref{sec:crash}.

\section{View Matrix}
\index{view matrix}It is sometimes useful to use this to orient the
view and export this orientation to other programs.  The orientation
matrix of the view can be displayed (in the console) using:

\texttt{(view-matrix)}

\section{Space Group}
Occassionally you may want to know the space group of a particular
molecule.  Interactively (for maps) you can see it using the Map
Properties button in the Molecule Display Control dialog.

There is a scripting interface function that returns the space group
for a given molecule \footnote{if no space group has been assigned it
  returns \texttt{``No spacegroup for this molecule''}}:

\texttt{(show-spacegroup \emph{imol})}

\section{Recentring View}
\label{sec:recentring-view}
\index{recentring view}
\begin{trivlist}
\item Use Control + left-mouse to drag around the view
\item or
\item middle-mouse over an atom.  In this case, you will often see
  ``slide-recentring'', the graphics smoothly changes between the
  current centre and the newly selected centre.
\item or
\item Use \textsf{Draw$ \rightarrow$ Go To Atom\ldots} to select an atom
  using the keyboard.  Note that you can subsequently use ``Space'' in
  the ``graphics'' window (OpenGL canvas) to recentre on the next
  C$\alpha$.
\end{trivlist}

If you don't want smooth recentring (sliding)\index{sliding}
\textsf{Draw $\rightarrow$ Smooth Recentring $\rightarrow$ Off}.  You
can also use this dialog to speed it up a bit (by decreasing the
number of steps instead of turning it off).

\section{Clipping manipulation}
\label{clipping manipulation}
\index{clipping} The clipping planes (a.k.a. ``slab''\index{slab}) can
be adjusted using \textsf{Edit $\rightarrow$ Clipping} and adjusting
the slider.  There is only one parameter to change and it affects both
the front and the back clipping planes\footnote{I find a clipping
  level of about 3.5 to 4 comfortable for viewing electron density
  maps - it is a little ``thinner'' than the default startup
  thickness.}.
The clipping can also be changed using keyboard ``D'' and ``F''.

One can ``push'' and ``pull'' the view in the screen-Z direction using
keypad 3 and keypad ``.'' (see Section \ref{keyboard_translation}).

\section{Background colour}
\index{background colour}
The background colour can be set either using a GUI dialog
(\textsf{Edit$ \rightarrow$ Background Colour}) or the function
\texttt{(set-background-colour 0.00 0.00 0.00)}, where the arguments
are 3 numbers between 0.0 and 1.0, which respectively represent the
red, green and blue components of the background colour.  The default
is (0.0, 0.0, 0.0) (black).

\section{Unit Cell}
\index{unit cell} If coordinates have symmetry available then unit
cells can be drawn for molecules (\textsf{Draw $\rightarrow$ Cell \&
  Symmetry $\rightarrow$ Show Unit Cell?}).

The unit cell of maps can be drawn without needing to setup CCP4 first.

\section{Rotation Centre Pointer}
\index{rotation centre pointer} There is a pink pointer\index{pink
  pointer} at the centre of the screen that marks the rotation centre.
The size of the pointer can be changed using \textsf{Edit
  $\rightarrow$ Pink Pointer Size\ldots} or using scripting commands:
\texttt{(set-rotation-centre-size 0.3)}.

\subsection{Pointer Distances}
The Rotation Centre Pointer is sometimes called simply ``Pointer''.
One can find distances to the pointer from any active set of atoms
using ``Pointer Distances'' (under Measures).  If you move the Pointer
(\emph{e.g.} by centering on an atom) and want to update the distances
to it, you have to toggle off and on the ``Show Pointer Distances'' on
the Pointer Distances dialog.

\section{Crosshairs}
\index{crosshairs}Crosshairs can be drawn at the centre of the screen,
using either ``c''\footnote{and ``c'' again to toggle them off.} in
graphics window or \textsf{Draw $\rightarrow$ Crosshairs\ldots}.  The
ticks are at 1.54\AA, 2.7\AA\ and 3.8\AA.

\section{Frame Rate}
\index{frame rate}
Sometimes, you might as yourself ``how fast is the
computer?''\footnote{compared to some other one.}.  Using
\texttt{Calculate $\rightarrow$ Frames/Sec} you can see how fast the
molecule is rotating, giving an indication of graphics performance.
It is often better to use a map that is more realistic and stop the
picture whizzing round.  The output is written to the console, you need
to give it a few seconds to ``settle down''.  It is best not to have
other widgets overlaying the GL canvas as you do this.

The contouring elapsed time\footnote{prompted by changing the contour
  level.} gives an indication of CPU performance.

\section{Program Output}
\index{output} Due to its ``in development'' nature (at the moment),
Coot produces a lot of ``console''\footnote{\emph{i.e.} the terminal
  in which you started Coot.} output - much of it debugging or
``informational''.  This will go away in due course.  You are advised
to run Coot so that you can see the console and the graphics window at
the same time, since feedback from atom clicking (for example) is
often written there rather than displayed in the graphics window.

\begin{itemize}
\item Output that starts ``ERROR...'' is a programming problem (and
  ideally, you should never see it).
\item Output that starts ``WARNING...'' means that something propably
unintented happened due to the unexpected nature of your input or
file(s).
\item Output that starts ``DEBUG...'' has (obviously enough) been
  added to aid debugging.  Most of them should have been cleaned up
  before release, but as Coot is constantly being developed, a few may
  slip through.  Just ignore them.
\end{itemize}


% -----------------------------------------------------------
\chapter{Coordinate-Related Features}
% -----------------------------------------------------------


\section{Read coordinates}
The format\index{coordinates format} of coordinates that can be read
by coot is either PDB or mmCIF.  To read coordinates, choose
\textsf{File $\rightarrow$ Read Coordinates} from the menu-bar.
Immediately after the coordinates have been read, the view is (by
default) recentred to the centre of this new molecule and the molecule
is displayed.  To disable the recentring of the view on reading a
coordinates file, use: \texttt{(recentre-on-read-pdb 0)}.

\subsection{Read multiple coordinate files}
\index{reading multiple pdb files}\index{multiple coordinates files}
The reading multiple files using the GUI is not available (at the
moment).  However the following scripting functions are available:

\texttt{(read-pdb-all)}

which reads all the ``*.pdb'' files in the current directory

\texttt{(multi-read-pdb \emph{glob-pattern} \emph{dir})}

which reads all the files matching \texttt{\emph{glob-pattern}} in
directory \texttt{\emph{dir}}.  Typical usage of this might be:

\texttt{(multi-read-pdb "a*.pdb" ".")}

Alternatively you can specify the files to be opened on the command
line when you start coot (see Section
\ref{sec:command_line_arguments}).

\section{Atom Info}
\index{atom info}\index{residue info} Information about about a
particular atom is displayed in the text console when you click using
middle-mouse.  Information for all the atoms in a residue is available
using \textsf{Info $\rightarrow$ Residue Info\ldots}.

\index{edit B-factors}\index{edit occupancy}The temperature factors
and occupancy of the atoms in a residue can be set by using
\textsf{Edit $\rightarrow$ Residue Info\ldots}.

\section{Atom Labeling}
\index{atom labeling}
\label{sec:atom}
Use Shift + left-mouse to label atom.  Do the same to toggle off the
label.  The font size is changeable using \textsf{Edit $\rightarrow
  $Font Size\ldots}.  The newly centred atom is labelled by default.
To turn this off use:

\texttt{(set-label-on-recentre-flag 0)}

\index{atom label, brief}Some people prefer to have atom labels that
are shorter, without the slashes and residue name:

\texttt{(set-brief-atom-labels 1)}

\section{Atom Colouring}
The atom colouring \index{colouring, atoms} \index{atom colouring}
system in coot is unsophisticated. Typically, atoms are coloured by
element: carbons are yellow, oxygens red, nitrogens blue, hydrogens
white and everything else green (see Section \ref{sec:display_manager}
for colour by chain).  However, it is useful to be able to distinguish
different molecules by colour, so by default coot rotates the colour
map of the atoms (\emph{i.e.} changes the H value in the
HSV\footnote{Hue Saturation Value (Intensity).}  colour system).  The
amount of the rotation depends on the molecule number and a
user-settable parameter:
\begin{trivlist}
\item \texttt{(set-colour-map-rotation-on-read-pdb 30)}.
\end{trivlist}

The default value is 31$^\circ$.

Also one is able to select only the Carbon atoms to change colour in
this manner: \texttt{(set-colour-map-rotation-on-read-pdb-c-only-flag
  1)}.

\section{Bond Parameters}
The various bond parameters can be set using the GUI dialog
\textsf{Draw $\rightarrow$ Bond Parameters} or \emph{via} scripting
functions.

\subsection{Bond Thickness}
\index{bond thickness}\index{width, bonds} The thickness (width) of
bonds of inividual molecules can be changed.  This can be done via the
\textsf{Bond Parameters} dialog or the scripting interface:

\texttt{(set-bond-thickness thickness imol)}

where \texttt{imol} is the molecule number. The default thickness is
3.0. The bond thickness also applies to the symmetry atoms of the
molecule.  There is no means to change the bond thickness of a residue
selection within a molecule.

\subsection{Display Hydrogens}
\index{hydrogens}Initially, hydrogens are displayed.  They can be
undisplayed using 

\texttt{(set-draw-hydrogens mol-no 0)}\footnote{they
  can be redisplayed using \texttt{(set-draw-hydrogens mol-no 1)}.}

where \texttt{mol-no} is the molecule number.

\subsection{NCS Ghosts Coordinates}
\index{NCS}It is occasionally useful when analysing
non-crystallographically related molecules to have ``images'' of the
other related molecules appear matched onto the current coordinates.
As you read in coordinates in Coot, they are checked for NCS
relationships and clicking on ``Show NCS Ghosts'' $\rightarrow$
``Yes'' $\rightarrow$ ``Apply'' will create ``ghost'' copies of them
over the reference chain\footnote{the reference chain is the first
  chain of that type in the coordinates file.}.

\subsection{NCS Maps}
Coot can use the relative transformations of the NCS-related molecules
in a coordinates molecule to transform maps. Use \textsf{Calulate}
$\rightarrow$ \textsf{NCS Maps\ldots} to do this (note the NCS maps
only make sense in the region of the reference chain (see above).
\index{NCS averaging}This will also create an NCS averaged
map\footnote{that also only makes sense in the region of the reference
  chain.}.

\section{Download coordinates}
Coot provides the possibility to download coordinates from an
\index{OCA}OCA\footnote{OCA is ``goose'' in Spanish (and Italian).
  \index{goose}} (\emph{e.g.} EBI) server\footnote{the default is the
  Weizmann Institute - which for reasons I won't go into here is
  currently much faster than the EBI server.} (\textsf{File
  $\rightarrow$ Get PDB Using Code\ldots}). A popup entry box is
displayed into which you can type a PDB accession code.  Coot will
then connect to the web server and transfer the file.  Coot blocks as
it does this (which is not ideal) but on a semi-decent internet
connection, it's not too bad.  The downloaded coordinates are saved
into a directory called \texttt{.coot}.

It is also possible to download mmCIF data and generate a map.  This
currently requires a properly formatted database structure factors
mmCIF file\footnote{which (currently) only a fraction are.}.

\section{Save Coordinates}
On selecting from the menus \textsf{File $\rightarrow$ Save
  Coordinates\ldots} you are first presented with a list of molecules
which have coordinates.  As well as the molecule number, there is the
molecule name - very frequently the name of the file that was read in
to generate the coordinates in coot initially.  However, this is only
a \emph{molecule} name and should not be confused with the filename to
which the coordinates are saved.  The coordinates \emph{filename} can
be selected using the \textsf{Select Filename\ldots} button.

If your filename ends in \texttt{.cif}, \texttt{.mmcif} or
\texttt{.mmCIF} then an mmCIF file will be written (not a ``PDB''
file).

\section{Anisotropic Atoms}
\index{anisotropic atoms} By default anisotropic atom information is
not represented\footnote{using thermal ellipsoids}.  To turn them on,
use \textsf{Draw $\rightarrow$ Anisotropic Atoms $\rightarrow$ Show
  Anisotropic Atoms?  $\rightarrow$ Yes}, or the command:
\texttt{(set-show-aniso 1)}.

You cannot currently display thermal ellipsoids\footnote{in the case
  of isotropic atoms, ellipsoids are spherical, of course.} for
isotropic atoms.

\section{Symmetry}
\index{symmetry} Coordinates symmetry is ``dynamic''.  Symmetry atoms
can be labeled\footnote{symmetry labels are in pale blue and also
  provide the symmetry operator number and the translations along the
  $a$, $b$ and $c$ axes.}.  Every time you recentre, the symmetry gets
updated.  The information shown contains the atom information and the
symmetry operation number and translations needed to generate the atom
in that position.

The symmetry can be represented as C$\alpha$s\index{C$\alpha$ symmetry
  representation}.  This along with representation of the molecule as
C$\alpha$s (Section \ref{sec:display_manager}) allow the production of
a packing diagram\index{packing diagram}.

\section{Sequence View}
\index{sequence view} The protein is represented by one letter codes
and coloured according to secondary structure.  These one letter codes
are active - if you click on them, they will change the centre of the
graphics window - in much the same way as clicking on a residue in the
Ramachandran plot.

\section{Environment Distances}
% not this residue, to symmetry if symmetry is on
% coloured bumps (C)
Environment distances are turned on using \textsf{Info $\rightarrow$
  Environment Distances\ldots}.  Contacts to other residues are shown
and to symmetry-related atoms if symmetry is being displayed.  The
contacts are coloured by atom type\footnote{contacts not involving a
  carbon atom are yellow.}.

\section{Distances and Angles}
The distance between atoms can be found using \textsf{Info
  $\rightarrow$ Distance}\footnote{Use \textsf{Angle} for an angle, of
  course.}.  The result is displayed graphically, and written to the
console.

\section{Zero Occupancy Marker}
\index{zero occupancy}Atoms of zero occupancy are marked with a grey
spot. To turn off these markers, use:

\texttt{(set-draw-zero-occ-markers 0)}

Use an argument of 1 to turn them on.

\section{Mean, Median Temperature Factors}
Coot can be used to calculate the \index{mean B-factor}mean (average)
and \index{median B-factor}median temperatures factors:

\texttt{(average-temperature-factor \emph{imol})}

\texttt{(median-temperature-factor \emph{imol})}

$-1$ is returned if there was a problem\footnote{\emph{e.g.} this
  molecule was a map or a closed molecule.}.

\section{Least-Squares Fitting}
There is currently no GUI specified for this, the scripting interface
is as follows:

\texttt{(simple-lsq-match \emph{ref-start-resno ref-end-resno ref-chain-id imol-ref
           mov-start-resno mov-end-resno mov-chain-id imol-mov
           match-type})}

where:
\begin{trivlist}
\item \texttt{\emph{ref-start-resno}} is the starting residue number
  of the reference molecule
\item \texttt{\emph{ref-end-resno}} is the last residue number
  of the reference molecule
\item \texttt{\emph{mov-start-resno}} is the starting residue number
  of the moving molecule
\item \texttt{\emph{mov-end-resno}} is the last residue number
  of the moving molecule
\item \texttt{\emph{match-type}} is one of \texttt{'CA},
  \texttt{'main}, or \texttt{'all}.
\end{trivlist}

\emph{e.g.}: 
\texttt{(simple-lsq-match 940 950 "A" 0 940 950 "A" 1 'main)}

More sophisticated (match molecule number 1 chain ``B'' on to molecule
number 0 chain ``A''):

\vspace{-2mm}
\begin{quote}
\texttt{(define match1 (list 840 850 "A" 440 450 "B" 'all))}\\
\texttt{(define match2 (list 940 950 "A" 540 550 "B" 'main))}\\
\texttt{(clear-lsq-matches)}\\
\texttt{(set-match-element match1)}\\
\texttt{(set-match-element match2)}\\
\texttt{(lsq-match 0 1)} ; match mol number 1 one mol number 0.
\end{quote}

%% \begin{trivlist}
%% \item \texttt{(define match1 (list 840 850 "A" 440 450 "B" 'all))}
%% \item \texttt{(define match2 (list 940 950 "A" 540 550 "B" 'main))}
%% \item \texttt{(clear-lsq-matches)}
%% \item \texttt{(set-match-element match1)}
%% \item \texttt{(set-match-element match2)}
%% \item \texttt{(lsq-match 0 1)}
%% \end{trivlist}

\section{More on Moving Molecules}
There are scripting functions available for this sort of thing:

\texttt{(molecule-centre \emph{imol})} 

will tell you the molecule centre \index{molecule centre} of the
\texttt{\emph{imol}}th molecule.

\texttt{(translate-by \texttt{imol x-shift y-shift z-shift})}

will translate all the atoms in molecule \texttt{\emph{imol}} by the
given amount (in {\AA}ngstr\"{o}ms)\index{translate molecule}.

\texttt{(move-molecule-to-screen-centre \emph{imol})}

will move the \texttt{\emph{imol}}th molecule to the current centre of
the screen (sometimes useful for imported ligands).  Note that this
moves the atoms of the molecule - not just the view of the molecule.


% -----------------------------------------------------------
\chapter{Modelling and Building}
% -----------------------------------------------------------
\label{modelling,building}

The functions described in this chapter manipulate, extend or build
molecules and can be found under \textsf{Calculate $\rightarrow$
  Model/Fit/Refine\ldots}.

\section{Regularization and Real Space Refinement}
\label{sec:randr}
If you have CCP4 installed, coot will read the geometry restraints for
refmac and use them in fragment (zone) idealization - this is called
``Regularization''\index{regularization}.  The geometrical restraints
are, by default, bonds, angles, planes\index{planes} and non-bonded
contacts.  You can additionally use torsion restraints\index{torsion
  restraints} by \textsf{Calculate $\rightarrow$
  Model/Fit/Refine\ldots $\rightarrow$ Refine/Regularize Control
  $\rightarrow$ Use Torsion Restraints}.

% cite Bob Diamond (1971) here somewhere.



``RS (Real Space) Refinement''\index{refinement} (after Diamond,
1971\footnote{Diamond, R. (1971). A Real-Space Refinement Procedure
  for Proteins. \emph{Acta Crystallographica} \textbf{A}27, 436-452.
  }) in Coot is the use of the map in addition to geometry terms to
improve the positions of the atoms.  Select ``Regularize'' from the
``Model/Fit/Refine'' dialog and click on 2 atoms to define the zone
(you can of course click on the same atom twice if you only want to
regularize one residue).  Coot then regularizes the residue range.  At
the end Coot, displays the intermediate atoms in white and also
displays a dialog, in which you can accept or reject this
regularization.  In the console are displayed the $\chi^2$ values of
the various geometrical restraints for the zone before and after the
regularization.  Usually the $\chi^2$ values are considerably
decreased - structure idealization such as this should drive the
$\chi^2$ values toward zero.

The use of ``Refinement'' is similar - with the addition of using a
map.  The map used to refine the structure is set by using the
``Refine/Regularize Control'' dialog.  If you have read/created only
one map into Coot, then that map will be used (there is no need to set
it explicitly).


Use, for example, \index{\texttt{set-matrix}}\texttt{(set-matrix 20.0)}
\footnote{\texttt{set\_matrix(20.0)} (using python).} to change the
weight of the map gradients to geometric gradients.  The higher the
number the more weight that is given to the map terms\footnote{but the
  resulting $\chi^2$ values are higher.}.  The default is 150.0.  This
will be needed for maps generated from data not on (or close to) the
absolute scale or maps that have been scaled (for example so that
the sigma level has been scaled to 1.0).

For both ``Regularize Zone'' and ``Refine Zone'' one is able to use a
single click to \index{single click refine}\index{refine single
  click}refine a residue range.  Pressing ``A'' on the keyboard while
selecting an atom in a residue will automatically create a residue
range with that residue in the middle.  By default the zone is
extended one residue either size of the central residue.  This can be
changed to 2 either side using \texttt{(set-refine-auto-range-step
  2)}.

Intermediate (white) atoms can be moved around with the mouse (click
and drag with left-mouse, by default).  \marginpar{\footnotesize
  \textsf{This is a useful feature}} Refinement will proceed from the
new atom positions when the mouse button is released.  It is possible
to create incorrect atom nomenclature and/or chiral volumes in this
manner - so some care must be taken.  Press the ``A'' key as you
left-mouse click to move atoms more ``locally'' (rather than a linear
shear) and Cntrl key as you left-mouse click to move just one atom.

To prevent the unintentional refinement of a large number of residues,
there is a ``heuristic fencepost'' of 20 residues.  A selection of
than 20 residues will not be regularized or refined.  The limit can be
changed using the scripting function: \emph{e.g.}
\texttt{(set-refine-max-residues 30)}.

\subsection{Dictionary}
\label{cif-dictionary}\index{cif dictionary, mmCIF dictionary}By default, 
the geometry dictionary entries for only the standard
residues are read in at the start \footnote{And a few extras, such as
  phospate}.  It may be that you particular ligand is not amongst
these.  To interactively add a dictionary entry use \textsf{File
  $\rightarrow$ Import CIF Dictionary}.  Alternatively, you can use
the function:

\texttt{(read-cif-dictionary \emph{filename})}

and add this to your \texttt{.coot} file (this may be the prefered
method if you want to read the file on more than one occassion).  

Note: the dictionary also provides the description of the ligand's
torsions.


\section{Rotate/Translate Zone}
\label{sec:rot_trans_zone}\index{rotate/translate, manual}``Rotate/Translate 
Zone'' from the ``Model/Fit/Refine'' menu allows manual movement of a
zone.  After pressing the ``Rotate/Translate Zone'' button, select two
atoms in the graphics canvas to define a residue range\footnote{if you
  want to move only one residue, then click the same atom twice.}, the
second atom that you click will be the local rotation centre for the
zone.  The atoms selected in the moving fragment have the same
alternate conformation code as the first atom you click.  To actuate a
transformation, click and drag horizontally across the relevant button
in the newly-created ``Rotation \& Translation'' dialog. The axis
system of the rotations and translations are the screen coordinates.
Alternatively \footnote{like Refinement and Regularization}, you can
click using left-mouse on an atom in the fragment and drag the
fragment around. Use Control Left-mouse to move just one atom, rather
than the whole fragment.  Click ``OK'' when the transformation is
complete.

\section{Rigid Body Refinement}
\label{sec:RigidBodyRefinement} \index{refinement, rigid body}
\index{rigid body fit}``Rigid Body Fit Zone'' from the
``Model/Fit/Refine'' dialog provides rigid body refinement.  The
selection is zone-based\footnote{like Regularization and Refinement.}.
So to refine just one residue, click on one atom twice.

Sometimes no results are displayed after Rigid Body Fit Zone.  This is
because the final model positions had too many final atom positions in
negative density.  If you want to over-rule the default fraction of
atoms in the zone that have an acceptable fit (0.75), to be (say)
0.25:

\texttt{(set-rigid-body-fit-acceptable-fit-fraction 0.25)}

\section{Baton Build}
\index{baton build} Baton build is most useful if a skeleton is
already calculated and displayed (see Section \ref{skeletonization}).
When three or more atoms have been built in a chain, Coot will use a
prior probability distribution for the next position based on the
position of the previous three.  The analysis is similar to Oldfield
\& Hubbard\footnote{T. J.  Oldfield \& R. E. Hubbard.  ``Analysis of
  C-Alpha Geometry in Protein Structures'' \emph{Proteins-Structure
    Function and Genetics} \textbf{18(4)} 324 -- 337.}, however it is
based on a more recent and considerably larger database.

Little crosses are drawn representing directions in which is is
possible that the chain goes, and a baton is drawn from the current
point to one of these new positions.  If you don't like this
particular direction\footnote{which is quite likely at first since
  coot has no knowledge of where the chain has been and cannot score
  according to geometric criteria.}, use \textsf{Try Another}.  The
list of directions is scored according to the above criterion and
sorted so that the most likely is at the top of the list and displayed
first as the baton direction.

When starting baton building, be sure to be about 3.8\AA\ from the
position of the first-placed C$\alpha$, this is because the next
C$\alpha$ is placed at the end of the baton, the baton root being at
the centre of the screen.  So, when trying to baton-build a chain
starting at residue 1, centre the screen at about the position of
residue 2.

% ``b'' key in GL canvas
\index{baton mode}Occasionally, every point is not where you want to
position the next atom.  In that case you can either shorten or
lengthen the baton, or position it yourself using the mouse.  Use
``b'' on the keyboard to swap to baton mode for the
mouse\footnote{``b'' again toggles the mode off.}.

Baton-built atoms are placed into a molecule called ``Baton Atom'' and
it is often sensible to save the coordinates of this molecule before
quitting coot.

If you try to trace a high resolution map (1.5\AA\  or better) you will
need to increase the skeleton search depth from the default (10), for
example:

\texttt{(set-max-skeleton-search-depth 20)}

Alternatively, you could generate a new map using data
to a more moderate resolution (2\AA), the map may be easier to
interpret at that resolution anyhow\footnote{high-resolution map
  interpretation is planned.}.

The guide positions are updated every time the ``Accept'' button is
clicked.  The molecule name for these atoms is ``Baton Build Guide Points''
and is is not usually necessary to keep them.

\subsection{Building Backwards}
The following senario is not uncommon: you find a nice streatch of
density and start baton building in it.  After a while you come to a
point where you stop (dismissing the baton build dialog).  You want to
go back to where you started and build the other way.  How do you do
that?

\begin{itemize}
\item Use the command: \texttt{(set-baton-build-params start-resno
    chain-id "backwards")}, where \texttt{start-resno} would typically
  be 0\footnote{\emph{i.e.} one less than the starting residue in the
    forward direction (defaults to 1).} and \texttt{chain-id} would be
  \texttt{""} (default).
\item Recentre the graphics window on the first atom of the just-build
  fragment
\item Select ``C$\alpha$ Baton Mode'' and select a baton direction
  that goes in the ``opposite'' direction to what is typically residue
  2.  This is slightly awkward because the initial baton atoms build
  in the ``opposite'' direction are not dependent on the first few
  atoms of the previously build fragment.
\end{itemize}


\subsection{Undo}
There is also an ``Undo'' button for baton-building.  Pressing this
will delete the most recently placed C$\alpha$ and the guide points
will be recalculated for the previous position.  The number of
``Undo''s is unlimited.  Note that you should use the ``Undo'' button
in the Baton Build dialog, not the one in the ``Model/Fit/Refine''
dialog (Section \ref{sec:backups_undo}).

\subsection{Missing Skeleton}
\index{skeleton, missing}Sometimes (especially at loops) you can see
the direction in which the chain should go, but there is no skeleton
(see Section \ref{skeletonization}) is displayed (and consequently no
guide points) in that direction. In that case, ``Undo'' the previous
atom and decrease the skeletonization level (\textsf{Edit
  $\rightarrow$ Skeleton Parameters $\rightarrow$ Skeletonization
  Level}).  Accept the atom (in the same place as last time) and now
when the new guide points are displayed, there should be an option to
build in a new direction.


\section{C$\alpha \rightarrow$ Mainchain}
\index{mainchain} Mainchain can be generated using a set of C$\alpha$s
as guide-points (such as those from Baton-building) along the line of
Esnouf\footnote{R. M. Esnouf ``Polyalanine Reconstruction from
  C$\alpha$ Positions Using the Program \emph{CALPHA} Can Aid Initial
  Phasing of Data by Molecular Replacement Procedures'' \emph{Acta
    Cryst. }, D\textbf{53}, 666-672 (1997).} or Jones and
coworkers\footnote{T.A.  Jones \& S. Thirup ``Using known
  substructures in protein model building and crystallography''
  \emph{EMBO J.} \textbf{5}, 819--822 (1986).}.  Briefly, 6-residue
fragments of are generated from a list of high-quality\footnote{and
  high resolution} structures. The C$\alpha$ atoms of these fragments
are matched against overlapping sets of the guide-point C$\alpha$s.
The resulting matches are merged to provide positions for the
mainchain (and C$\beta$) atoms.  This proceedure works well for
helices and strands, but less well\footnote{\emph{i.e.}  there are
  severely misplaced atoms} for less common structural features.

This function is also available from the scripting interface:

\texttt{(db-mainchain imol chain-id resno-start resno-end direction)}
    
where direction is either \texttt{"backwards"} or \texttt{"forwards"}.

% Withdrawn due to being to difficult to calculate the atom positions 
% given the phi and psi
%
%\section{Edit Phi/Psi}
%\index{edit $\phi/\psi$}This generates a Ramachandran plot with only
%one residue represented.  You can click and drag this residue round
%the plot and the coordinates in the graphics window will change to the
%$\phi/\psi$ values in the Ramachandran plot.

\section{Backbone Torsion Angles}
It is possible to edit the backbone $\phi$ and $\psi$ angles
indirectly using an option in the Model/Fit/Refine's dialog: ``Edit
Backbone Torsions..''. When clicked and an atom of a peptide is
selected, this produces a new dialog that offers ``Rotate Peptide''
which changes this residues $\psi$ and ``Rotate Carbonyl'' which
changes $\phi$.  Click and drag across the button\footnote{as for
  Rotate/Translate Zone (Section \ref{sec:rot_trans_zone}).} to rotate
the moving atoms in the graphics window.  You should know, of course,
that making these modifications alter the $\phi/\psi$ angles of more
than one residue.


\section{Rotamers}
\label{sec:rotamers}
\index{Dunbrack, Roland}\index{rotamers} The rotamers are generated
from the backbone independent sidechain library of Roland Dunbrack and
co-workers\footnote{R. L.  Dunbrack, Jr. \& F. E.  Cohen. "Bayesian
  statistical analysis of protein sidechain rotamer preferences"
  \emph{Protein Science}, \textbf{6}, 1661--1681 (1997). }. According
to this analysis, some sidechains have many rotamer
options\footnote{LYS, for example has 81.}.  By default only rotamers
with a probability (as derived from the structural database) greater
than 1\% are considered. The initial position is the most likely for
that residue type\footnote{Use \emph{e.g.}
  \texttt{(set-rotamer-lowest-probability 0.5)} to change the
  probability lower limit for the rotamer selection (note that this is
  a percentage, therefore 0.5(\%) is quite low and will allow the
  choice of more rotamers than the default.}.

Use keyboard ``.'' and ``,'' to cycle round the rotamers.

\subsection{Auto Fit Rotamer}
\index{auto-fit rotamer}``Auto Fit Rotamer'' will try to fit the
rotamer to the electron density.  Each rotamer is generated, rigid
body refined and scored according to the fit to the map.  Fitting the
second conformation of a dual conformation in this way will often fail
- the algorithm will pick the best fit to the density - ignoring the
position of the other atoms.

The algorithm doesn't know if the other atoms in the structure are in
sensible positions.  If they are, then it is sensible not to put this
residue too close to them, if they are not then there should be no
restriction from the other atoms as to the position of this residue -
the default is ``are sensible'', which means that the algorithm is
prevented from finding solutions that are too close to the atoms of
other residues. \texttt{(set-rotamer-check-clashes 0)} will stop this.

There is a scripting interface to auto-fitting rotamers:

\texttt{(auto-fit-best-rotamer \emph{resno alt-loc ins-code chain-id\\imol-coords
imol-map clash-flag lowest-rotamer-probability})}

where:

\texttt{\emph{resno}} is the residue number

\texttt{\emph{alt-loc}} is the alternate/alternative location symbol
(\emph{e.g.} \texttt{"A"} or \texttt{"B"}, but most often \texttt{""})

\texttt{\emph{ins-code}} is the insertion code (usually \texttt{""})

\texttt{\emph{imol-coords}} is the molecule number of the coordinates molecule

\texttt{\emph{imol-map}} is the molecule number of the map to which
you wish to fit the side chains

\texttt{\emph{clash-flag}} should the positions of other residues be
included in the scoring of the rotamers (\emph{i.e.} clashing with other
other atoms gets marked as bad/unlikely)

\texttt{\emph{lowest-rotamer-probability}}: some rotamers of some side
chains are so unlikely that they shouldn't be considered - typically
0.01 (1\%).

\subsection{De-clashing residues}
Sometimes you don't have a map\footnote{for example, in preparation of
  a model for molecular replacement} but nevertheless there are
clashing residues\index{clashing residues}\footnote{atoms of residues
  that are too close to each other} (for example after mutation of a
residue range) and you need to rotate side-chains to a non-clashing
rotamer.  There is a scripting interface:

\texttt{(de-clash \texttt{imol chain-id start-resno end-resno})}

\texttt{\emph{start-resno}} is the residue number of the first residue
you wish to de-clash.

\texttt{\emph{start-resno}} is the residue number of the last residue
you wish to de-clash

\texttt{\emph{imol}} is the molecule number of the coordinates molecule

This interface will not change residues with insertion codes or
alternate conformation.  The
\texttt{\emph{lowest-rotamer-probability}} is set to 0.01.


\section{Editing $\chi$ Angles}
\index{edit $\chi$ angles}Instead of using Rotamers, one can instead
change the $\chi$ angles \index{torsions}(often called ``torsions'')
``by hand'' (using ``Edit Chi Angles'' from the ``Model/Fit/Refine''
dialog). To edit a residue's $\chi_1$ press ``1'': to edit $\chi_2$,
``2'': $\chi_3$ ``3'' and $\chi_4$ ``4''.  Use left-mouse click and
drag to change the $\chi$ value.  Use keyboard ``0''\footnote{that's
  ``zero''.} to go back to ordinary view mode at any time during the
editing.  Alternatively, one can use the ``View Rotation Mode'' or use
the Ctrl key when moving the mouse in the graphics window.  Use the
Accept/Reject dialog when you have finished editing the $\chi$ angles.

It should be emphasised that for standard residues this is an option
of last resort - use the other rotamer manipulation options first.

\subsection{Ligand Torsion angles}
\index{torsion angles, ligand}\index{ligand torsion angles}For
ligands, you will need to read the mmCIF file that contains a
description of the ligand's geometry (see Section
\ref{cif-dictionary}).  By default, torsions that move hydrogens are
not included.  Only 9 torsion angles are available from the keyboard
torsion angle selection.

\section{Pep-flip}
\index{pepflip}\index{flip peptide} Coot uses the same pepflip scheme
as is used in $O$ (\emph{i.e.} the C, N and O atoms are rotated
180$^o$ round a line joining the C$\alpha$ atoms of the residues
involved in the peptide).  Flip the peptide again to return the atoms
to their previous position.


\section{Add Alternate Conformation}
\label{sec:add_alt_conf}
The allows the addition alternate (\index{dual conformations}dual,
triple \emph{etc.})  conformations to the picked residue.  By default,
this provides a choice of rotamer (Section \ref{sec:rotamers}).  If
there are not the correct main chain atoms a rotamer choice cannot be
provided, and Coot falls back to providing intermediate atoms.

The default occupancy for new atoms is 0.5.  This can be changed by
using use slider on the rotamer selection window or by using the
scripting function:

\texttt{(set-add-alt-conf-new-atoms-occupancy 0.4)}

% The intermediate atoms interface can be forced using:

% \texttt{(set-show-alt-conf-intermediate-atoms 1)}


\section{Mutation}
\index{mutate} Mutations are available on a 1-by-1 basis using the
graphics.  After selecting ``Mutate\ldots'' from the
``Model/Fit/Refine'' dialog, click on an atom in the graphics.  A
``Residue Type'' window will now appear.  Select the new residue type
you wish and the residue in the graphics is updated to the new residue
type\footnote{Note that selecting a residue type that matches the
  residue in the graphics will also result in a mutation}.  The
initial position of the new rotatmer is the \emph{a priori} most
likely rotamer. Note that in interactive mode, such as this, a residue
type match\footnote{\emph{i.e.} the current residue type matches the
  residue type to which you wish to mutate the residue} will not stop
the mutation action occurring.

\subsection{Multiple mutations}
This dialog can be found under \textsf{Calculate $\rightarrow$ Mutate
  Residue Range}.  A residue range can be assigned a sequence and
optionally fitted to the map.  This is useful converting a poly-ALA
model to the correct sequence\footnote{\emph{e.g.} after using Ca
  $\rightarrow$ Mainchain.}.

Multiple mutations\index{multi-mutate} are also supported \emph{via}
the scripting interface.  Unlike the single residue mutation function,
a residue type match \emph{will} prevent a modification of the
residue\footnote{\emph{i.e.} the residue atoms will remain untouched}.
Two functions are provided: To mutate a whole chain, use
\texttt{(mutate-chain \emph{imol} \emph{chain-id sequence})} where:

\texttt{\emph{chain-id}} is the chain identifier of the chain that you wish
to mutate (\emph{e.g.} \texttt{"A"}) and 

\texttt{\emph{imol}} is molecule number.  

\texttt{\emph{sequence}} is a list of single-letter residue codes,
such as \texttt{"GYRESDF"} (this should be a straight string with no
additional spaces or carriage returns).

Note that the number of residues in the sequence chain and those in
the chain of the protein must match exactly (\emph{i.e.} the whole of
  the chain is mutated (except residues that have a matching residue
  type).)

To mutate a residue range, use 

\begin{trivlist}
\item 
\texttt{(mutate-residue-range \emph{chain-id}
  \emph{start-res-no} \emph{stop-res-no \newline sequence})}
\end{trivlist}

where

\texttt{\emph{start-res-no}} is the starting residue for mutation

\texttt{\emph{stop-res-no}} is the last residue for mutation, \emph{i.e.}
using values of 2 and 3 for \texttt{\emph{start-res-no}} and
\texttt{\emph{stop-res-no}} respectively will mutate 2 residues.

Again, the length of the sequence must correspond to the residue range
length.

\subsection{Mutate and Autofit}
The function combines Mutation and Auto Fit Rotamer and is the easiest
way to make a mutation and then fit to the map.

\subsection{Renumbering}
\index{renumbering residues}Renumbering is straightforward using the
renumber dialog available under \textsf{Calculate $\rightarrow$
  Renumber Residue Range\ldots}.  There is also a scripting interface:

\texttt{(renumber-residue-range \emph{imol chain-id start-res-no
    last-resno offset})}

\section{Find Ligands}
\index{ligands} You are offered a selection of maps to search (you can
only choose one at a time) and a selection of molecules that act as a
mask to this map.  Finally you must choose which ligand types you are
going to search for in this map\footnote{you can search for many
  different ligand types.}.  Only molecules with less than 400 atoms
are suggested as potential ligands.  New ligands are placed where the
map density is and protein (mask) atoms are \emph{not}).  The masked
map is searched for clusters using a default cut-off of 1.0$\sigma$.
In weak density this cut-off may be too high and in such a case the
cut-off value can be changed using something such as:

\texttt{(set-ligand-cluster-sigma-level 0.8)}

However, if the map to be searched for ligands is a difference map, a
cluster level of 2.0 or 3.0 would probably be more
appropriate\footnote{less likely to generate spurious sites.}.

Each ligand is fitted with rigid body refinement to each potential
ligand site in the map and the best one for each site selected and
written out as a pdb file.  The clusters are sorted by size, the
biggest one first (with an index of 0).  The output placed ligands
files have a prefix ``best-overall'' and are tagged by the cluster
index and residue type of the best fit ligand in that site.

By default, the top 10 sites are tested for ligands - to increase this
use:

\texttt{(set-ligand-n-top-ligands 20)}

\subsection{Flexible Ligands}
\index{ligands, flexible}
If the ``Flexible?'' checkbutton is activated, coot will generate a
number of variable conformations (default 100) by rotating around the
rotatable bonds (torsions).  Each of these conformations will be fitted
to each of the potential ligand sites in the map and the best one will
be selected (again, if it passes the fitting criteria above).

Before you search for flexible ligands you must have read the mmCIF
dictionary for that particular ligand residue type (\textsf{File
  $\rightarrow$ Import CIF dictionary\index{dictionary, cif}}).

Use:

\texttt{(set-ligand-flexible-ligand-n-samples \emph{n-samples})}

where \texttt{\emph{n-samples}} is the number of samples of flexiblity
made for each ligand.  The more the number of rotatable bonds, the
bigger this number should be.

\subsection{Adding Ligands to Model}
After successful ligand searching, one may well want to add that
displayed ligand to the current model (the coordinates set that
provided the map mask).  To do so, use Merge Molecules (Section
\ref{sec:merge_molecules}).


\section{Find Waters}
\index{waters, finding} As with finding ligands, you are given a chose
of maps, protein (masking) atoms. A final selection has to be made for
the cut-off level, note that this value is the number of standard
deviation of the density of the map \emph{after} the map has been
masked.  Then the map is masked by the masking atoms and a search is
made of features in the map about the electron density cut-off value.
Waters are added if the feature is approximately water-sized and can
make sensible hydrogen bonds to the protein atoms.  The new waters are
optionally created in a new molecule called ``Waters''.

You have control over several parameters used in the water finding:

\texttt{(set-write-peaksearched-waters)} 

which writes \texttt{ligand-waters-peaksearch-results.pdb}, which
contains the water peaks (from the clusters) without any filtering and
\texttt{ligand-waters.pdb} which are a disk copy filtered waters that
have been either added to the molecule or from which a new molecule
has been created.

\texttt{(set-ligand-water-spherical-variance-limit min-d max-d)} sets
the minimum and maximum allowable distances between new waters and the
masking molecule (usually the protein).

\texttt{(set-ligand-water-spherical-variance-limit varlim)} sets the
upper limit for the density variance around water atoms. The default
is 0.12.
% $electrons^2/\AA^6$.

The map that is maked by the protein and is searched to find the
waters is written out in CCP4 format as \texttt{"masked-for-waters.map"}.

\subsection{Blobs}
After a water search, Coot will create a blobs dialog (see Section
\ref{sec:blobs}).

\subsection{Check Waters via Difference Map}
Another check of waters that one can perform is the following:

\texttt{(check-waters-by-difference-map \emph{imol-coords}
  \emph{imol-diff-map})}

where \texttt{\emph{imol-coords}} is the molecule number of the
coordinates that contain the waters to be checked

\texttt{\emph{imol-diff-map}} is the molecule number of the difference
map (it must be a difference map, not an ``ordinary'' map).  This
difference map must have been calculated using the waters. So there is
no point in doing this check immediately after ``Find Waters''.  You
will need to run Refmac first\footnote{and remember to check the
  difference map button in the ``Run Refmac'' dialog}.

This analysis will return a list of water atoms that have
outstandingly high local variance of the difference map (by default a
sphere of 1.5\AA\ centred about the atom position).  This analysis
might find waters that are actually something else, for example: part
of a ligand, a sulfate, an anion or cation, only partially occupied or
should be deleted entirely.  Coot\footnote{as yet} doesn't decide what
should be done about these atoms, it merely brings them to your
attention.  It may be interesting to use an anomalous map to do this
analysis.

There is no GUI for this feature.

\section{Add Terminal Residue}
\index{terminal residue} This creates a new residue at the C or N
terminus by fitting to the map.  $\phi/\psi$ angle pairs are selected
at random based on the Ramachandran plot probability (for a generic
residue).  By default there are 100 trials.  It is possible that a
wrong position will be selected for the terminal residue and if so,
you can reject this fit and try again with Fit Terminal
Residue\footnote{usually if this still fails after two repetitions
  then it never seems to work.}. Each of the trial positions are
scored according to their fit to the map\footnote{The map is selected
  using ``Refine/Regularize Control''} and the best one selected.  It
is probably a good idea to run ``Refine Zone'' on these new residues.

\texttt{(set-terminal-residue-do-rigid-body-refine 0)} will disable
rigid body fitting of the terminal residue fragment for
each trial residue position (the default is 1 (on)) - this may help if
the search does not provide good results.

\texttt{(set-add-terminal-residue-n-phi-psi-trials 50)} will change
the number of trials (default is 100).

\section{Add OXT Atom to Residue}
\index{terminal oxygen}\index{OXT atom}At the
C-terminus\index{C-terminus} of a chain of amino-acid residues, there
is a ``modification'' so that the C-O becomes a carbonyl, \emph{i.e.}
an extra (terminal) oxygen (OXT) needs to be added.  This atom is
added so that it is in the plane of the C$\alpha$, C and O atoms of
the residue.

Scripting usage:

\texttt{(add-OXT-to-residue imol residue-number \newline insertion-code
  chain-id)}\footnote{\emph{e.g.} \texttt{(add-OXT-to-residue 0 428 "" "A")}}, 

where \texttt{insertion-code} is typically \texttt{""}.  

Note, in order to place OXT, the N, CA, C and O atoms must be present
in the residue - if (for example) the existing carbonyl oxygen atom is
called ``OE1'' then this function will not work.

\section{Add Atom at Pointer}
By default, ``Add Atom At Pointer'' will pop-up a dialog from which
you can choose the atom type you wish to insert\footnote{including
  sulfate or phosphate ions (in such a case, it is probably useful to
do a ``Rigid Body Fit Zone'' on that new residue).}.  Using
\texttt{(set-pointer-atom-is-dummy 1)} you can by-pass this dialog and
immediately create a dummy atom at the pointer position.  Use an
argument of \texttt{0} to revert to using the atom type selection
pop-up on a button press.

The atoms are added to a new molecule called ``Pointer Atoms''.  They
should be saved and merged with your coordinates outside of Coot.

\section{Merge Molecules}
\index{merge molecules}\label{sec:merge_molecules}
This dialog can be found under ``Calculate'' in the main menubar.
This is typically used to add molecule fragments or residues that are
in one molecule to the ``working'' coordinates\footnote{For example,
  after a ligand search has been performed.}.


\section{Running Refmac}
\index{refmac}\index{running refmac}
Use the ``Run Refmac...'' button to select the dataset and the
coordinates on which you would like to run Refmac.  Note that only
dataset which had Refmac parameters set as the MTZ file was read are
offered as dataset that can be used with Refmac. By default, Coot
displays the new coordinates and the new map generated from refmac's
output MTZ file.  Optionally, you can also display the difference map.

You can add extra parameters \index{refmac parameters} (data lines) to
refmac's input by storing them in a file called
\texttt{refmac-extra-params} in the directory in which you started
coot.

Coot ``blocks''\footnote{\emph{i.e.} Coot is idle and ignores all
  input.} until Refmac has terminated\footnote{This is not an idea
  feature, of course and will be addressed in future.... Digressive
  Musing: If only computers were fast enough to run Refmac
  interactively\ldots}.

The default refmac executable\index{refmac, default}\index{default
  refmac version} is \texttt{refmac5} it is presumed to be in the
path.  If you don't want this, it can be overridden using a
re-definition either at the scripting interface or in one's
\texttt{~/.coot} file \emph{e.g.}:
\begin{trivlist}
\item \texttt{(define refmac-exec "/e/refmac-new/bin/refmac5.6.3")}
\end{trivlist}

\index{refmac map colour}After running refmac several times, you may
find that you prefer if the new map that refmac creates (after refmac
refinement) is the same colour as the previous one (from before this
refmac refinement).  If so, use:

\texttt{(set-keep-map-colour-after-refmac 1)}

which will swap the colours of then new and old refmac map so that the
post-refmac map has the same colour as the pre-refmac map and the
pre-refmac map is coloured with a different colour.


\section{Clear Pending Picks}
\index{Clear Pending Picks}\index{atom picking}Sometimes one can click
on a button\footnote{such that Coot would subsequently expect an atom
  selection ``pick'' in the graphics window.} unintentionally. This
button is there for such a case.  It clears the expectation of an
atom pick.  This works not only for modelling functions, but also
geometry functions (such as Distance and Angle).

\section{Delete}
\index{delete} Single atoms or residues can be deleted from the
molecule using ``Delete\ldots'' from the ``Model/Fit/Refine''dialog.
Pressing this button results in a new dialog, with the options of
``Residue'' (the default), ``Atom'' and ``Hydrogen Atoms''.  Now click
on an atom in the graphics - the deleted object will be the whole
residue of the atom if ``Residue'' was selected and just that atom if
``Atom'' was selected.

If you want to delete multiple items you can either use check the
``Keep Delete Active'' check-button on this dialog or use the Ctrl key
as you click on an atom.  Either of these will keep the dialog open,
ready for deletion of next item.

% document delete-atom, delete-residue, delete-residue-with-altconf here.


\section{Sequence Assignment}
You can assign a (fasta format) sequence to a molecule using:

\texttt{(assign-fasta-sequence imol chain-id fasta-seq)}

This function has been provided as a precursor to functions that will
(as automatically as possible) mutate your current coordinates to one
that has the desired sequence. It will be used in automatic side-chain
assignment (at some stage in the future).

\section{Building Links and Loops}

Coot can make an attempt to build missing linking regions or
loops\footnote{the current single function doesn't always perform very
  well in tests, which is why it is currenty available only in the
  scripting format.}.  This is an area of Coot that needs to be
improved, currently O does it much better.  We will have several
different loop tools here\footnote{I suspect that there is not one
  tool that fits for all.}.  For now:

\texttt{(fit-gap \emph{imol} \emph{chain-id} \emph{start-resno} \emph{stop-resno})}

and 

\texttt{(fit-gap \emph{imol} \emph{chain-id} \emph{start-resno} \emph{stop-resno} \emph{sequence})}

the second form will also mutate and try to rotamer fit the provided sequence.

Example usage: let's say for molecule number 0 in chain \texttt{"A"}
we have residues up to 56 and then a gap after which we have residues
62 and beyond:

\texttt{(fit-gap 0 "A" 57 61 "TYPWS")}

\section{Setting Occupancies}
As well as the editing ``Residue Info'' to change occupancies of
individual atoms, one can use a scripting function to change
occupancies of a whole residue range:

\begin{trivlist}
\item \texttt{(zero-occupancy-residue-range \emph{imol chain-id \\
resno-start resno-last})}
\end{trivlist}

example usage:

\texttt{(zero-occupancy-residue-range 0 "A" 23 28)}

This is often useful to zero out a questionable loop before submitting
for refinement.  After refinement (with refmac) there should be
relatively unbiased density in the resulting 2Fo-Fc-style and
difference maps.

Similarly there is a function to reverse this operation:

\begin{trivlist}
\item \texttt{(fill-occupancy-residue-range \emph{imol chain-id \\
      resno-start resno-last})}
\end{trivlist}




% -----------------------------------------------------------
\chapter{Map-Related Features}
% -----------------------------------------------------------

\section{Maps in General}
Maps are ``infinite,'' not limited to pre-calculated volume (the
``Everywhere You Click - There Is Electron Density''
(EYC-TIED)\index{EYC-TIED} paradigm) symmetry-related electron
density is generated automatically. Maps are easily re-contoured.
Simply use the scroll wheel on you mouse to alter the contour level
(or -/+ on the keyboard)\index{change contour level}.
 
Maps follow the molecule.  As you recentre or move about the crystal,
the map quickly follows.  If your computer is not up to re-contouring
all the maps for every frame, then use \textsf{Draw $\rightarrow$
  Dragged Map\ldots} to turn off this feature.

Unfortunately, there is a bug in map-reading\label{map-reading-bug}.
If the map is not a bona-fide CCP4 map\footnote{\emph{e.g.} it's a
  directory or a coordinate filename.}, then coot will crash.  Sorry.
A fix is in the works but ``it's complicated''.

\section{Create a map}
From MTZ, mmCIF and .phs (\textsc{phases} format)\index{phases format}
data use \textsf{File $\rightarrow$ Read Dataset\ldots}. From a CCP4
map use \textsf{File $\rightarrow$ Read Map}.  After being
generated/read, the map is immediately contoured and centred on the
current rotation centre.

\subsection{Reading CIF data}
There are several maps that can be generated from CIF files that
contain observed Fs, calculated Fs and calculated phases:

\begin{trivlist}
\item \texttt{(read-cif-data-with-phases-fo-alpha-calc
    \emph{cif-file-name})} Calculate an atom map using F$_{obs}$ and
  $\alpha_{calc}$
\item \texttt{(read-cif-data-with-phases-2fo-fc \emph{cif-file-name})}
 Calculate an atom map using F$_{obs}$, F$_{calc}$ and
  $\alpha_{calc}$
\item \texttt{(read-cif-data-with-phases-fo-fc \emph{cif-file-name})}
 Calculate an difference map using F$_{obs}$, F$_{calc}$ and
  $\alpha_{calc}$.
\end{trivlist}

\section{Map Contouring}
\index{contouring, map}Maps can be re-contoured using the middle-mouse
scroll-wheel (buttons 4 and 5 in X Window System$^{\textrm{\tiny TM}}$
terminology).  Scrolling the mouse wheel will change the map contour
level and the map it redrawn.  If you have several maps displayed then
the map that is has its contour level changed can be set using
\textsf{HID$ \rightarrow$ Scrollwheel $\rightarrow$ Attach scroll-wheel
  to which map?}.  If there is only one map displayed, then that is
the map that has its contour level changed (no matter what the
scroll-wheel is attached to in the menu).  The level of the electron
density is displayed in the top right hand corner of the OpenGL canvas.

Use Keyboard + or - to change the contour level if you don't have a
scroll-wheel\footnote{like I don't on my Mac.}.

If you are creating your map from an MTZ file, you can choose to click
on the ``is difference map''\index{difference map} button on the Column
Label selection widget (after a data set filename has been selected)
then this map will be displayed in 2 colours corresponding to + and -
the map contour level.

If you read in a map it is a difference map then there is
a checkbutton to tell Coot that.

If you want to tell Coot that a map is a difference
map\index{difference map colours} after it has been read, use:

\texttt{(set-map-is-difference-map \emph{imol})}

where \texttt{\emph{imol}} is the molecule number.

By default the map radius\footnote{actually, it's a box.} is 10\AA.
The default increment to the electron density depends on whether or
not this is a difference map (0.05 $e^-$/\AA$^3$ for a ``2Fo-Fc''
style map and 0.005 $e^-$/\AA$^3$ for a difference map).  You can
change these using \textsf{Edit $\rightarrow$ Map Parameters} or by
using the ``Properties'' button of a particular map in the Display
Control (Display Manager) window.

\section{Map contour ``scrolling'' limits}
Usually one doesn't want to look at \index{negative contour
  levels}negative contour levels of a map\footnote{in a coot
  difference map you will get to see the negative level contoured at
  the inverted level of the positive level, what I mean is that you
  don't want to see the ``positive'' level going less than 0.}, so
Coot has by default a limit that stops the contour level going beyond
(less than) 0.  To remove the limit:

\texttt{(set-stop-scroll-iso-map 0)} {for a 2Fo-Fc style map}

\texttt{(set-stop-scroll-diff-map 0)} {for a difference map}

To set the limits to negative (\emph{e.g.} -0.6) levels:

\texttt{(set-stop-scroll-iso-map-level -0.6)}

and similarly: 

\texttt{(set-stop-scroll-diff-map-level -0.6)}

where the level is specified in electrons/\AA$^3$.

\section{Map Line Width}
\index{map line width}\index{density line thickness}\index{thickness
  of density lines}The width of the lines that descibe the density can
be changed like this:

\texttt{(set-map-line-width 2)}

The default line width is 1.

\section{``Dynamic'' Map colouring}
\index{colouring, map} By default, maps get coloured according to
their molecule number.  The starting colour (\emph{i.e.} for molecule
0) is blue.  The colour of a map can be changed by \textsf{Edit
  $\rightarrow$ Map Colour..}. The map colour gets updated as you
change the value in the colour selector\footnote{takes you right back
  to the good old Frodo days, no?}.  Use ``OK'' to fix that colour.

\section{Difference Map Colouring}
For some strange reason, some crystallographers\footnote{Jan Dohnalek,
  for instance.} like to have their difference maps coloured with red
as positive and green as negative, this option is for them:

\texttt{(set-swap-difference-map-colours 1)}


\section{Map Sampling}
By default, the Shannon sampling factor is the conventional 1.5.  Use
larger values (\textsf{Edit $\rightarrow$ Map Parameters $\rightarrow$
  Sampling Rate}) for smoother maps\footnote{a value of 2.5 is often
  sufficient.}.

\section{Dragged Map}
By default, the map is re-contoured at every frame during a drag (Ctrl
Left-mouse).  Sometimes this can be annoyingly slow and jerky so it is
possible to turn it off: \textsf{Draw $\rightarrow$ Dragged Map
  $\rightarrow$ No}.

To change this by scripting:

\texttt{(set-active-map-drag-flag 0)}


\section{Dynamic Map Sampling and Display Size}
If activated (\textsf{Edit $\rightarrow$ Map Parameters $\rightarrow$
  Dynamic Map Sampling}) the map will be re-sampled on a courser grid
when the view is zoomed out.  If ``Display Size'' is also activated,
the box of electron density will be increased in size also.  In this
way, you can see electron density for \index{big maps}big maps (many
unit cells) and the graphics still remain rotatable.

\section{Skeletonization}
\label{skeletonization}
\index{skeletonization} \index{bones} The skeleton (also known as
``Bones''\footnote{If you're living in Sweden... or Captain Kirk, that
  is.}) can be displayed for any map.  A map can be skeletonized using
\textsf{Calculate $\rightarrow$ Map Skeleton\ldots}.  Use the option
menu to choose the map and click ``On'' then ``OK'' to the generate
the map (the skeleton is off by default).

The level of the skeleton can be changed by using \textsf{Edit
  $\rightarrow$ Skeleton Parameters\ldots $\rightarrow$
  Skeletonization Level\ldots} and corresponds to the electron density
level in the map.  By default this value is 1.2 map standard
deviations.  The amount of map can be changed using \textsf{Edit
  $\rightarrow$ Skeleton Parameters\ldots $\rightarrow$ Skeleton Box
  Radius\ldots}\footnote{you may think it strange that a box has a
  radius, this is an idiosyncrasy of coot.}.  The units are in \AA
ngstr\"oms, with 40 as the default value.

The skeleton is often recalculated as the screen centre changes - but
not always since it can be an irritatingly slow calculation.
\index{skeleton regeneration}If you want to force a regeneration of
the displayed skeleton, simply centre on an atom (using the middle
mouse button) or press the ``s'' key.

\section{Masks}
\label{masks}
\index{masks} A map can be masked by a set of coordinates. Use the
scripting function: 

\texttt{(mask-map-by-protein map-number
  coords-number 0)}\footnote{the 0 is a placeholder for an as yet
  unimplemented feature (\texttt{invert?}).}.  

This will create a new
map that has density where there are no (close) coordinates.  So for
example, if you wanted to show the density around your ligand, you
would create a coordinates file that contained all the protein except
for the ligand and use those coordinates to mask the map.

There is no GUI interface to this feature at the moment.

\subsubsection{Example}
If one wanted to show just the density around a ligand:

\begin{enumerate}
\item Make a pdb file the contains just the ligand and read it in to
  Coot - let's say it is molecule 1 and the ligand is residue 3 of
  chain ``L''.
\item Get a map that covers the ligand (\emph{e.g.} from refmac).
  Let's say this map is molecule number 2.
\item Mask the map:

\texttt{(mask-map-by-molecule 2 1 \#f)}

This creates a new map.  Turn the other maps off, leaving only the
masked map.

\end{enumerate}

To get a nice rendered image, press F8 (see Section \ref{Raster3D}).


\section{Trimming}
\index{trimming atoms}
If you want to remove all the atoms\footnote{or set their occupancy to
  zero} that lie ``outside the map'' (\emph{i.e.} in low density) you can use

\texttt{(trim-molecule-by-map \emph{imol-coords imol-map density-level\\ delete/zero-occ?})}

where \texttt{\emph{delete/zero-occ?}} is \texttt{0} to remove the atoms and
\texttt{1} to set their occupancy to zero.

There is no GUI interface to this feature.


% -----------------------------------------------------------
\chapter{Validation}
% -----------------------------------------------------------

The validation functions are in the process of being written.  In
future there will be more functions, particularly those that will
interface to other programs\footnote{such as the Richardsons' reduce
  and probe}.

\section{Ramachandran Plots}
\index{Ramachandran plot} Ramachandran plots are ``dynamic''.  When
you change the molecule (\emph{i.e.} move the coordinates of some of
atoms) the Ramachandran plot gets updated to reflect those changes.
Also the underlying $\phi/\psi$ probability density changes according
to the selected residue type (\emph{i.e.} the residue under the mouse
in the plot).  There are 3 different residue types: GLY, PRO, and
not-GLY-or-PRO\footnote{the not-GLY-or-PRO is the most familiar
  Ramachandran plot.}.

When you mouse over a representation of a residue (a little square or
triangle\footnote{prolines have a grey outline rather than a black
  one, triangles are glycines.}) the residue label pops up.  The
residue is ``active'' \emph{i.e.} it can be clicked.  The ``graphics''
view changes so that the C$\alpha$ of the selected residue is centred.
In the Ramachandran plot window, the current residue is highlighted by
a green square.

% The probability levels for acceptable (yellow) and preferred (red) are
% 0.12\% and 6\% respectively and have been chosen to look like those
% from Procheck\index{Procheck}.

\section{Chiral Volumes}
The dictionary is used to identify the chiral atoms of each of the
model's residues.  A clickable list is created of atoms whose chiral
volume in the model is of a different sign to that in the dictionary.

\section{Blobs: a.k.a. Unmodelled density}
\label{sec:blobs}
This is an interface to the Blobs\index{blobs}\index{unmodelled
  density} dialog.  A map and a set of coordinates that model the
protein are required.

A blob is region of relatively high residual election density that
cannot be explained by a simple water\index{unexplained density}. So,
for example, sulfates, ligands, mis-placed sidechains or unbuilt
terminal residues might appear as blobs.  The blobs are in order, the
biggest \footnote{and therefore most interesting} at the top.

\section{Check Waters by Difference Map}
Sometimes waters can be misplaced - taking the place of sidechains or
ligands or crystallization agents such as phosphate for
example\footnote{or the water should be more properly modelled as
  anistrotropic or a split partial site}.  In such cases the variance
of the difference map can be used to identify them.

This function is also useful to check anomalous maps.  Often waters
are placed in density that is really a cation.  If such an atom
diffracts anomalously this can be identified and corrected.

By default the waters with a map variance greater than 3.5 $\sigma$ are
listed.  One can be more rigorous by using a lower cut-off:

\texttt{(set-check-waters-by-difference-map-sigma-level 3.0)}


\section{Validation Graphs}

Coot provides several graphs that are useful for model validation (on
a residue by residue basis): residue denisty fit, geometry distortion,
temperature factor variance, peptide distortion and rotamer analysis.

\subsection{Residue Density Fit}

The residue density fit is by default scaled to a map that is
calculated on the absolute scale.  Some users use maps that have maps
with density levels considerably different to this, which makes the
residue density fit graph less useful.  To correct for this you can
use the scripting function:

\texttt{(set-residue-density-fit-scale-factor \emph{factor})}

where \texttt{\emph{factor}} would be $1/(4\sigma_{map})$ (as a rule
of thumb).

\texttt{(residue-density-fit-scale-factor)} returns the current scale
factor (default 1.0).

\subsection{Rotamer Analysis}
Residue rotamers are scored according to the prior likelihood.  Note
that when CD1 and CD2 of a PHE residue are exchanged (simply a
nomenclature error) this can lead to large red blocks in the graph
(apparently due to very unlikely rotamers).  There are several other
residues that can have nomenclature errors like this.

\subsection{Temperature Factor Variance}

\subsection{Peptide $\omega$ Distortion}

\subsection{Geometry}


% -----------------------------------------------------------
\chapter{Hints}
% -----------------------------------------------------------
\label{chap-hints}
\section{Getting out of ``Translate'' Mode}
If you get stuck in "translate" mode in the GL canvas
(\emph{i.e.} mouse does not rotate the view as you would expect) simply
press and release the Ctrl key to return to "rotate" mode.

\section{Getting out of ``Label Atom Only'' Mode}
Similarly, if you are stuck in a mode where the ``Model/Fit/Refine''
buttons don't work (the atoms are not selected, only the atom gets
labelled), press and release the Shift key.

\section{Button Labels}
Button labels ending in ``\ldots'' mean that a new dialog will pop-up
when this button is pressed.

\section{Picking}
\label{sec:picking}\index{picking} Note that left-mouse in the 
graphics window is used for both atom picking and rotating the view,
so try not to click over an atom when trying to rotate the view when
in atom selection mode.  

% This was a Matrix (GL_PROJECTION) bug.  Fixed now.
%
%Sometimes, when trying to pick an atom you
%get the message ``Model atom pick failed''\index{model atom pick} even
%though you have clicked accurately over the atom.  The work-around is
%to give the model a little wiggle (using the mouse) and try the pick
%again.

\section{Resizing View}
\index{resizing view}\index{zoom} Click and drag using right-mouse (up
and down or left and right) to zoom in and out.

\section{Map}
If the ``Display'' button for the map in the ``Display Manager''
window stops working, close the ``Display Control'' window and re-open
it.  The button should now respond to clicks.

To change the map to which the scroll-wheel is attached, use
\textsf{HID $\rightarrow$ Scrollwheel $\rightarrow $Attach Scrollwheel
  to which map?}
 
\section{Slow Computer Configuration}
\index{slow computer}Several of the parameters of Coot are chosen
because they are reasonable on my ``middle-ground'' development
machine.  However, these parameters can be tweeked so that slower
computers perform better:

\begin{trivlist}
\item \texttt{(set-smooth-scroll-steps 4) ; default 8 }
\item \texttt{(set-smooth-scroll-limit 30) ; Angstroms}
\item \texttt{(set-residue-selection-flash-frames-number 3);}
\item \texttt{(set-skeleton-box-size 20.0) ; A (default 40).}
\item \texttt{(set-active-map-drag-flag 0) ; turn off recontouring every step}
\item \texttt{(set-idle-function-rotate-angle 1.5) ; turn up to 1.5 degrees}
\end{trivlist}

%\appendix
%\chapter{Some Extras}




% Have you documented:
%
% Merge molecules dialog    : done
% Mutate sequence dialog    : done
% Add OXT to residue dialog : done
% Bond Parameters dialog
% Renumber Residues dialog
% Find Bad Chiral Atoms     : done
% Validate Waters (simple)
% Validation Graphs 
% Pointer distances
% Torsions

\documentclass{book}
\usepackage{a4}
\usepackage{palatino}
%\usepackage{times}
%\usepackage{utopia}
\usepackage{euler}
\usepackage{fancyhdr}
\usepackage{epsf}

\newcommand {\atilde} {$_{\char '176}$} % tilde(~) character

%\date{1st April 2004}

\title{The Coot User Manual}
\author{Paul Emsley \\\textsf{\small emsley@ysbl.york.ac.uk}}
\makeindex
\hyphenation{tri-angle}

\begin{document}
\thispagestyle{empty}

%% Make a title page: I can't use \maketitlepage because I want a line

\vspace*{30mm}

{\huge The Coot User Manual}

\begin{picture}(300,2)
\linethickness{5pt}
\put(0,0){\line(1,0){347}}
\end{picture}

\begin{flushright}
%  \today
  1st April 2004
\end{flushright}

\vspace*{20mm}


\begin{center}
  \leavevmode
  \epsfxsize 127mm \epsffile{coot-plain-2.eps}
\end{center}

\vspace*{20mm}

\begin{picture}(300,2)
\linethickness{5pt}
\put(0,0){\line(1,0){347}}
\end{picture}

\begin{flushright}

  Paul Emsley\\\textsf{\small emsley@ysbl.york.ac.uk}
\end{flushright}

%\begin{picture}(width,height)(xoffset,yoffset)
%\end{picture}

%\newpage
\tableofcontents
\pagestyle{headings}

\chapter{Introduction}

\section{This document}
This document is the Coot User Manual, giving a brief overview of the
interactive features.  Other documentations includes (or it is planned
to include) the \index{reference manual}Coot Reference Manual and the
Coot \index{tutorial} Tutorial.  These documents should be distributed
with the source code.

\section{What is Coot?}

Coot is a stand-alone portion of CCP4's Molecular Graphics project. Its
focus is crystallographic model-building and manipulation rather than
representation (\emph{i.e.} more like Frodo than
\index{Rasmol}Rasmol).

Coot is Free Software.  You can give it away. If you don't like the
way it behaves, you can fix it yourself.

\section{What Coot is Not}
Coot is not:
\begin{itemize}
\item CCP4's official Molecular Graphics program\footnote{Coot is
    \emph{part of} that project. The official program (which contains
parts of Coot), ccp4mg is under
    the direct control of Liz Potterton and Stuart McNicholas.}
\item a program to do refinement\footnote{although it does have a
    local refinement algorithm it is no substitute for \textsc{refmac}
    (a wrapper for \textsc{refmac} is available).}
\item a database, in any respect
\item a protein crystallographic suite\footnote{that's the job of the
    CCP4 Program Suite.}.
\end{itemize}

\section{Hardware Requirements}
The code is designed to be portable to any Unix-like operating
system\footnote{including Cygwin, but currently coot is ``unstable''
  on a Cygwin system.}.  Coot certainly runs on SGI IRIX64, RedHat
Linux of various sorts, SuSe Linux\footnote{so far only 8.2 verified.}
and MacOS X (10.2).  The sgi Coot binaries shouold also 
work on IRIX.

If you want to port to some other operating system, you are
welcome\footnote{it's Free Software after all and I could give you a
  hand.}.  Note that your task will be eased by using GNU GCC to compile
the programs components.

\subsection{Mouse}
\index{mouse}Coot works best with a 3-button mouse and works better if
it has a scroll-wheel too (see Chapter 2 for more details)\footnote{I
  can get by with a one button Machintosh - but it's not ideal.}.

\section{Environment Variables}
Coot responds to several command line arguments that modify its
behaviour.  

\begin{trivlist}
\item [\texttt{COOT\_STANDARD\_RESIDUES}] The filename of the pdb file
  containing the standard amino acid residues in ``standard
  conformation''\footnote{as it is known in Clipper.}
\item [\texttt{COOT\_SCHEME\_DIR}] The directory containing auxiliary scheme
  files 
\item [\texttt{COOT\_REF\_STRUCTS}] The directory containing a set of
  high resolution pdb files used as \index{reference
    strctures}reference structures to build backbone atoms from
  C$\alpha$ positions
\item [\texttt{COOT\_REFMAC\_LIB\_DIR}] \index{dictionary, cif}Refmac's
  CIF directory containing the monomers and link descriptions.  In the future
  this may simply be the same directory in which refmac looks to find
  the library dictionary.
\item [\texttt{COOT\_RESOURCES\_DIR}] The directory that contains the
  splash screen image and the GTk application resources.
\item [\texttt{COOT\_BACKUP\_DIR}] The directory to which backup are
  written (if it exists as a directory).  If it is not, then backups
  are written to the current directory (the directory in which coot
  was started).
\end{trivlist}
  
And of course extension language environment variables are used too:

\begin{trivlist}
\item [\texttt{PYTHONPATH}] (for python modules)
\item [\texttt{GUILE\_LOAD\_PATH}] (for guile modules)
\end{trivlist}

Normally, these environment variables will be set correctly in the
coot setup script (which can be found in the \texttt{setup} directory
in the binary distribution.  See the web site (Section \ref{webpage})
for setup details.

\section{Command Line Arguments}
\index{command line arguments}
\label{sec:command_line_arguments}
Rather that using the GUI to read in information, you can use the
following command line arguments:
\begin{itemize}
\item \texttt{--script} to run a script on start up
\item \texttt{--pdb}  for pdb/coordinates file
\item \texttt{--data} for mtz, phs or mmCIF data file
\item \texttt{--map}  for a (currently only CCP4) map
\end{itemize}
So, for example, one might use: 
\begin{trivlist}
\item \texttt{coot --pdb pre-refinement.pdb --pdb post-refinement.pdb}
\end{trivlist}

\section{Web Page}
\label{webpage}
Coot has a \index{web page}web page:

\begin{trivlist}
\item \texttt{http://www.ysbl.york.ac.uk/\atilde emsley/coot}
\end{trivlist}

There you can read more about the CCP4 molecular graphics project in
general and other projects which are important for coot\footnote{coot
  has several influences and dependencies, but these will not be
  discussed here in the User Manual.}.

The web page also contains an example ``setup'' file which assigns the
environment variables to change the behaviour of Coot.

\section{Crash}
\label{sec:crash}
\index{crash}
Coot might crash on you - it shouldn't.  

\index{recover session}\index{crash recovery}There are backup files in
the directory \texttt{coot-backup}\footnote{\$COOT\_BACKUP\_DIR is used
  in preference if set}. You can recover the session (until the last
edit) by reading in the pdb file that you started with last time and
then use \textsf{File $\rightarrow$ Recover Session\ldots}.

I would like to know about coot crashing\footnote{The map-reading
  problem (documented in Section \ref{map-reading-bug}) is already
  known.} so that I can fix it as soon as possible. If you want your
problem fixed, this involves some work on your part sadly.

First please make sure that you are using the most recent version of
coot.  I will often need to know as much as possible about what you
did to cause the bug.  If you can reproduce the bug and send me the
files that are needed to cause it, I can almost certainly fix -
it\footnote{now there's a hostage to fortune.} especially if you
\index{debugger}\index{gdb}use the debugger (gdb) and send a backtrace
too\footnote{to do so, please send me the output of the following:
  \texttt{\$ gdb `which coot` \emph{corefile}} and then at the
  \texttt{(gdb)} prompt type: \texttt{where}, where
  \texttt{\emph{corefile}} is the core dump file, \texttt{core} or
  \texttt{core.4536} or some such.}.

% -----------------------------------------------------------
\chapter{Mousing and Keyboarding}
% -----------------------------------------------------------
\index{mouse buttons}
How do we move around and select things?

\vspace{0.5cm}
  \begin{tabular}{ll}
    Left-mouse Drag & Rotate view \\
    Ctrl Left-Mouse Drag &  Translates view \\
    Shift Left-Mouse &  Label Atom\\
    Right-Mouse Drag &  Zoom in and out\index{zoom}\\
    Shift Right-Mouse Drag &  Rotate View around Screen Z axis\\
    Middle-mouse & Centre on atom\\
    Scroll-wheel Forward &  Increase map contour level\\
    Scroll-wheel Backward &  Decrease map contour level
  \end{tabular}
\vspace{3mm}

See also Chapter \ref{chap-hints} for more help.

\section{Next Residue}
\begin{tabular}{ll}
  ``Space'' & Next Residue \\
  ``Shift'' ``Space'' & Previous Residue
\end{tabular}

See also ``Recentring View'' (Section \ref{sec:recentring-view}).

\section{Keyboard Contouring}

Use \texttt{+} or \texttt{-} on the keyboard if you don't have a
scroll-wheel.

\section{Keyboard Rotation}
\index{keyboard rotation}By popular request keyboard equivalents of
rotations have been added\footnote{particularly for those with
  PowerMates (the amount of rotation can be changed to 2$^\circ$ (from
  the default 1$^\circ$) using \texttt{(set-idle-function-rotate-angle
    2.0)}).}: \vspace{3mm}

\begin{tabular}{ll}
  Q & Rotate + X Axis \\
  W & Rotate - X Axis \\
  E & Rotate + Y Axis \\
  R & Rotate - Y Axis \\
  T & Rotate + Z Axis \\
  Y & Rotate - Z Axis \\
  I & Continuous Y Axis Rotation
\end{tabular}
% document rotate-x-sceen nsteps step-size here?

\section{Keyboard Translation}
\index{translation, keyboard}
\label{keyboard_translation}
\begin{tabular}{ll}
  Keypad 3 & Push View (+Z translation)\\
  Keypad . & Pull View (-Z translation)
\end{tabular}


\section{Keyboard Zoom and Clip}

\begin{tabular}{ll}

  N & Zoom out   \\
  M & Zoom in    \\
  D & Slim clip  \\
  F & Fatten clip\\
\end{tabular}

\vspace{3mm}

\section{Scrollwheel}
When there is no map, using the scroll-wheel has no effect.  If there
is exactly one map displayed, \index{contouring, map} the scroll-wheel
will change the contour level of that map.  If there are two or more
maps, the map for which the contour level is changed can be set by
\textsf{HID $\rightarrow$ Scrollwheel $\rightarrow$ Attach scroll-wheel
  to which map?} and selecting a map number.

\section{Selecting Atoms}
Several Coot functions require the selecting of atoms to specify a
residue range (for example: Regularize, Refine (Section
\ref{sec:randr}) or Rigid Body Fit Zone (Section
\ref{sec:RigidBodyRefinement})).  Select atoms with the Left-mouse.
See also Picking (Section \ref{sec:picking}).

Use the scripting function
\index{quanta-buttons}\texttt{(quanta-buttons)} to make the mouse
functions more like other molecular graphics programs to which you may
be more accustomed\footnote{See also \ref{sec:quanta-zooming}}.

\section{Virtual Trackball}
\index{trackball, virtual} You may not completely like the way the
molecule is moved by the mouse movement\footnote{Mouse movement in
  ``Spherical Surface'' mde generates a component of (often
  undesirable) screen z-rotation, particularly noticeable when the
  mouse is at the edge of the screen.}.  To change this, try:
\textsf{HID $\rightarrow$ Virtual Trackball $\rightarrow$ Flat}.  To
do this from the scripting interface: \texttt{(set-vt
  1)}\footnote{\texttt{(set-vt 0)} to turn it back to ``Spherical''
  mode.}.

If you \emph{do} want \index{screen-z rotation}\index{z-rotation}
screen-z rotation, you can either use Shift Right-Mouse Drag or set
the Virtual Trackball to Spherical Surface mode and move the mouse
along the bottom edget of the screen.

\section{More on Zooming}
\label{sec:quanta-zooming}
The function \texttt{(quanta-like-zoom)} adds the ability to zoom the
view using just Shift + Mouse movement\footnote{this is off by default
  because I find it annoying.}.

There is also a Zoom slider\index{zoom, slider} (\textsf{Draw
  $\rightarrow$ Zoom}) for those without a right-mouse button.

% -----------------------------------------------------------
\chapter{General Features}
% -----------------------------------------------------------

The map-fitting and model-building tools can be accessed by using
\textsf{Calculate $\rightarrow$ Model/Fit/Refine\ldots}.  Many
functions have \index{tooltips}tooltips\footnote{Put your mouse over a
  widget for a couple of seconds, if that widget has a tooltip, it
  will pop-up in a yellow box.}\index{tooltips} describing the
particular features and are documented in Chapter
\ref{modelling,building}.

\section{Version number}
\index{version number}
The version number of Coot can be found at the top of the ``About''
window (\textsf{Help $\rightarrow$ About}).

There is also a script function to return the version of coot:

\texttt{(coot-version)}

\section{Antialiasing}
Antialiasing (for what it's worth) can be enabled using:

\texttt{(set-do-anti-aliasing 1)}

The default is \texttt{0} (off).

\section{Molecule Number}
\index{molecule number} 
Coot is based on the concept of molecules.  Maps and coordinates are
different representations of molecules.  The access to the molecule is
\emph{via} the \emph{molecule number}.  It is often important therefore to
know the molecule number of a particular molecule.

Molecule numbers can be found by clicking on an atom in that molecule
(if it has coordinates of course). The first number in brackets in the
resulting text in the console is the molecule number.  The molecule
number can also be found in Display Control window (Section
\ref{sec:display_manager}).  It is also displayed on the left-hand
side of the molecule name in the option menus of the ``Save
Coordinates'' and ``Go To Atom'' windows.

\section{Display Issues}
The ``graphics'' window is drawn using \index{OpenGL}OpenGL.  It is
considerably smoother when using a 3D accelerated X server. 

The view is orthographic (\emph{i.e.} the back is the same size as the
front).  The default clipping is about right for viewing coordinate
data, but is often a little too ``thick'' for viewing electron
density.  It is easily changed (see Section \ref{clipping
  manipulation}).

Depth-cueing\index{depth-cueing} is linear and fixed on. There is no
anti-aliasing\footnote{coot is not the program for snazzy graphics -
  CCP4mg is the program for that sort of thing.}.

The graphics window can be resized, but it has a minimum size of
400x400 pixels.

\subsection{Origin Marker}

A yellow box\index{yellow box} called the ``origin
marker''\index{origin marker} marks the origin.  It can be removed
using:

\texttt{(set-show-origin-marker 0)}

Its state can be queried like this:

\texttt{(show-origin-marker-state)}

which returns an number (an integer).

\subsection{Raster3D output}
\label{Raster3D}Output suitable for use by Raster3D\index{Raster3D}'s
``render''\index{render} can be generated using the scripting function

\texttt{(raster3d \emph{file-name})}

where \texttt{\emph{file-name}} is such as
\texttt{"test.r3d"}\footnote{Also povray will be supported in the
  future.}.

There is a keyboard key to generate this file, run ``render'' and
display the image: Function key F8.

You can also use the function

\texttt{(render-image)}

which will create a file \texttt{coot.r3d}, from which ``render'' produces
\texttt{coot.png}. This png file is displayed using ImageMagick's display
program (by default).  Use something like:

\texttt{(set! coot-png-display-program "gqview")}

to change that to different display program ("gqview" in this case).

To change the widths of the bonds and density ``lines'' use (for example):

\texttt{(set-raster3d-bond-thickness 0.1)}

and 

\texttt{(set-raster3d-density-thickness 0.01)}

To turn off the representations of the atoms (spheres):

\texttt{(set-renderer-show-atoms 0)}



\section{Display Manager}
\label{sec:display_manager}
\index{Display Manager} This is also known as ``Map and molecule
(coordinates) display control''.  Here you can select which maps and
molecules you can see and how they are drawn\footnote{to a limited
  extent.}.  The ``Display'' and ``Active'' are toggle buttons, either
depressed (active) or undepressed (inactive).  The ``Display'' buttons
control whether a molecule (or map) is drawn and the ``Active'' button
controls if the molecule is clickable\footnote{the substantial
  majority of the time you will want your the buttons to be both
  either depressed or undepressed, rarely one but not the other.}
(\emph{i.e.} if the molecule's atoms can be labeled).

By default, the path names of the files are not displayed in the
Display Manager.  To turn them on:

\texttt{(set-show-paths-in-display-manager 1)}

\index{colour by chain}\index{atom colouring}If you pull across the
horizontal scrollbar in a Molecule view, you will see the ``Render
as'' menu.  You can use this to change between normal ``Bonds (Colour
by Atom)'',``Bonds (Colour by Chain)'' and ``C$\alpha$''
representation\index{C$\alpha$ representation}.  There is also
available ``No Waters'' and ``C$\alpha$ + ligands'' representations.

\section{The file selector}
\subsection{File-name Filtering}
\index{file-name filtering} The ``Filter'' button in the fileselection
filters the filenames according to extension.  For coordinates files
the extensions are ``.pdb'' ``.brk'' ``.mmcif'' and others.  For data:
``.mtz'', ``.hkl'', ``.phs'', ``.cif'' and for (CCP4) maps ``.ext'',
``.msk'' and ``.map''.  If you want to add to the extensions, the
following functions are available:

\begin{trivlist}
\item \texttt{(add-coordinates-glob-extension \emph{extension})}
\item \texttt{(add-data-glob-extension \emph{extension})}
\item \texttt{(add-map-glob-extension \emph{extension})}
\item \texttt{(add-dictionary-glob-extension \emph{extension})}
\end{trivlist}
where \texttt{\emph{extension}} is something like: \texttt{".mycif"}.

\subsection{Filename Sorting}
If you like your files initially sorted by date (rather than
lexographically, which is the default use:

\texttt{(set-sticky-sort-by-date)}

\section{Scripting}
\index{scripting} There is an compile-time option of adding a script
interpreter.  Currently the options are python and guile.  Hopefully,
in the near future you will be able to use both in the same
executable, but that's not available today.

Hundreds of commands are made available for use in scripting by using
SWIG.  These are are currently not well documented but can be found in
the Coot Reference Manual or the source code (\texttt{c-interface.h}).

Commands described throughout this manual (such as \texttt{(vt-surface
  1))} can be evaluated\index{executing commands} directly by Coot by
using the ``Scripting Window'' (\textsf{Calculate $\rightarrow$
  Scripting\ldots}). Note that you type the commands in the lower
entry widget and the command gets echoed (in red) and the return vaule
and any output is displayed in the text widget above.  The typed
command should be terminated with a carriage return\footnote{which
  causes the evalution of the command.}.  Files\footnote{such as the
  Coot state file (Section \ref{sec:coot_state}).} can be evaluated
(executed) using \textsf{Calculate $\rightarrow$ Run Script\ldots}.
Note that in scheme (the usual scripting language of Coot), the
parentheses are important.

\subsection{Python}
\index{python} Coot has an (optional) embedded python interpreter.
Thus the full power of python is available to you.  Coot will look for
an initialization script \index{startup settings
  (python)}\index{\texttt{.coot.py}}(\texttt{\$HOME/.coot.py}) and
will execute it if found.  This file should contain python commands
that set your personal preferences.


\subsubsection{Python Commands}
The scripting functions described in this manual are formatted
suitable for use with guile, \emph{i.e.}:

\texttt{(\emph{function} \emph{arg1} \emph{arg2\ldots})}

If you are using Python instead: the format needs to be changed to:

\texttt{\emph{function}(\emph{arg1},\emph{arg2\ldots})}

Note that dashes in guile function names become underscores for
python, so that (for example) \texttt{(raster-screen-shot)} becomes
\texttt{raster\_screen\_shot()}.


\subsection{Scheme}
\index{guile}%
\index{scheme} The scheme interpreter is made available by embedding
guile.  The initialization script used by this interpreter is
\index{startup settings (scheme)} \index{\texttt{.coot}}
\texttt{\$HOME/.coot}.  This file should contain scheme commands that
set your personal preferences.


\subsection{State}
\label{sec:coot_state}
The ``state''\index{state} of coot is saved on Exit and written to a
file called \texttt{0-coot.state.scm} (scheme)
\texttt{0-coot.state.py} (python).   This
state file contains information about the screen centre, the
clipping, colour map rotation size, the symmetry radius, and other
molecule related parameters such as filename, column labels,
coordinate filename \emph{etc.}.

Use \textsf{Calculate $\rightarrow$ Run Script\ldots} to use this file
to re-create the loaded maps and models that you had when you finished
using Coot\footnote{in that particular directory.} last time.
A state file can be saved at any time using \texttt{(save-state)}
which saves to file \texttt{0-coot.state.scm} or
\texttt{(save-state-filename "thing.scm")} which saves to file
\texttt{thing.scm}.

When Coot starts it can optionally run the commands in
\texttt{0-coot.state.scm}.  Use \texttt{(set-run-state-file-status i)}
to change the behaviour: \texttt{i} is \texttt{0} to never run this
state file at \index{startup dialog (state)}startup, \texttt{i} is
\texttt{1} to get a dialog option (this is the default) and \texttt{i}
is \texttt{2} to run the commands without question.

\section{Backups and Undo}
\label{sec:backups_undo}\index{backups}\index{undo} By default, each 
time a modification is made to a model, the old coordinates are
written out\footnote{this might be surprising since this could chew up
  a lot of disk space.  However, disk space is cheap compared to
  losing you molecule.}.  The backups are kept in a backup directory
and are tagged with the date and the history number (lower numbers are
more ancient\footnote{The coordinates are written in pdb format.}).
The ``Undo'' function discards the current molecule and loads itself
from the most recent backup coordinates.  Thus you do not have to
remember to ``Save Changes'' - coot will do it for you\footnote{unless
  you tell it not to, of course - use (\emph{e.g.})
  \texttt{(turn-off-backup 0)} to turn off the backup (for molecule 0
  in this case).}.

If you have made changes to more than one molecule, Coot will pop-up a
dialog box in which you should set the ``Undo Molecule'' \emph{i.e.}
the molecule to which the Undo operations will apply.  Further Undo
operations will continue to apply to this molecule until there are
none left.  If another Undo is requested Coot checks to see if there
are other molecules that can be undone, if there is exactly one, then
that molecule becomes the ``Undo Molecule'', if there are more than
one, then another Undo selection dialog will be displayed.



\subsection{Redo}
\index{redo}The ``undone'' modifications can be re-done using this
button.  This is not available immediately after a
modification\footnote{It works like the ``Forwards'' buttons in a web
  browser - which is not available immediately after viewing a new
  page.}.

\subsection{Restoring from Backup}
\index{restore after crash} There may be certain
circumstances\footnote{for example, if coot crashes.} in which you
wish to restore from a backup but can't get it by the ``Undo''
mechanism described above.  In that case, start coot as normal and
then open the (typically most recent) coordinates file in the
directory \texttt{coot-backup} (or the directory pointed to the
environment varialble \texttt{COOT\_BACKUP\_DIR} if it was set) .
This file should contain your most recent edits.  In such a case, it
is sensible for neatness purposes to immediately save the coordinates
(probably to the current directory) so that you are not modifying a
file in the backup directory.

See also Section \ref{sec:crash}.

\section{View Matrix}
\index{view matrix}It is sometimes useful to use this to orient the
view and export this orientation to other programs.  The orientation
matrix of the view can be displayed (in the console) using:

\texttt{(view-matrix)}

\section{Space Group}
Occassionally you may want to know the space group of a particular
molecule.  Interactively (for maps) you can see it using the Map
Properties button in the Molecule Display Control dialog.

There is a scripting interface function that returns the space group
for a given molecule \footnote{if no space group has been assigned it
  returns \texttt{``No spacegroup for this molecule''}}:

\texttt{(show-spacegroup \emph{imol})}

\section{Recentring View}
\label{sec:recentring-view}
\index{recentring view}
\begin{trivlist}
\item Use Control + left-mouse to drag around the view
\item or
\item middle-mouse over an atom.  In this case, you will often see
  ``slide-recentring'', the graphics smoothly changes between the
  current centre and the newly selected centre.
\item or
\item Use \textsf{Draw$ \rightarrow$ Go To Atom\ldots} to select an atom
  using the keyboard.  Note that you can subsequently use ``Space'' in
  the ``graphics'' window (OpenGL canvas) to recentre on the next
  C$\alpha$.
\end{trivlist}

If you don't want smooth recentring (sliding)\index{sliding}
\textsf{Draw $\rightarrow$ Smooth Recentring $\rightarrow$ Off}.  You
can also use this dialog to speed it up a bit (by decreasing the
number of steps instead of turning it off).

\section{Clipping manipulation}
\label{clipping manipulation}
\index{clipping} The clipping planes (a.k.a. ``slab''\index{slab}) can
be adjusted using \textsf{Edit $\rightarrow$ Clipping} and adjusting
the slider.  There is only one parameter to change and it affects both
the front and the back clipping planes\footnote{I find a clipping
  level of about 3.5 to 4 comfortable for viewing electron density
  maps - it is a little ``thinner'' than the default startup
  thickness.}.
The clipping can also be changed using keyboard ``D'' and ``F''.

One can ``push'' and ``pull'' the view in the screen-Z direction using
keypad 3 and keypad ``.'' (see Section \ref{keyboard_translation}).

\section{Background colour}
\index{background colour}
The background colour can be set either using a GUI dialog
(\textsf{Edit$ \rightarrow$ Background Colour}) or the function
\texttt{(set-background-colour 0.00 0.00 0.00)}, where the arguments
are 3 numbers between 0.0 and 1.0, which respectively represent the
red, green and blue components of the background colour.  The default
is (0.0, 0.0, 0.0) (black).

\section{Unit Cell}
\index{unit cell} If coordinates have symmetry available then unit
cells can be drawn for molecules (\textsf{Draw $\rightarrow$ Cell \&
  Symmetry $\rightarrow$ Show Unit Cell?}).

The unit cell of maps can be drawn without needing to setup CCP4 first.

\section{Rotation Centre Pointer}
\index{rotation centre pointer} There is a pink pointer\index{pink
  pointer} at the centre of the screen that marks the rotation centre.
The size of the pointer can be changed using \textsf{Edit
  $\rightarrow$ Pink Pointer Size\ldots} or using scripting commands:
\texttt{(set-rotation-centre-size 0.3)}.

\subsection{Pointer Distances}
The Rotation Centre Pointer is sometimes called simply ``Pointer''.
One can find distances to the pointer from any active set of atoms
using ``Pointer Distances'' (under Measures).  If you move the Pointer
(\emph{e.g.} by centering on an atom) and want to update the distances
to it, you have to toggle off and on the ``Show Pointer Distances'' on
the Pointer Distances dialog.

\section{Crosshairs}
\index{crosshairs}Crosshairs can be drawn at the centre of the screen,
using either ``c''\footnote{and ``c'' again to toggle them off.} in
graphics window or \textsf{Draw $\rightarrow$ Crosshairs\ldots}.  The
ticks are at 1.54\AA, 2.7\AA\ and 3.8\AA.

\section{Frame Rate}
\index{frame rate}
Sometimes, you might as yourself ``how fast is the
computer?''\footnote{compared to some other one.}.  Using
\texttt{Calculate $\rightarrow$ Frames/Sec} you can see how fast the
molecule is rotating, giving an indication of graphics performance.
It is often better to use a map that is more realistic and stop the
picture whizzing round.  The output is written to the console, you need
to give it a few seconds to ``settle down''.  It is best not to have
other widgets overlaying the GL canvas as you do this.

The contouring elapsed time\footnote{prompted by changing the contour
  level.} gives an indication of CPU performance.

\section{Program Output}
\index{output} Due to its ``in development'' nature (at the moment),
Coot produces a lot of ``console''\footnote{\emph{i.e.} the terminal
  in which you started Coot.} output - much of it debugging or
``informational''.  This will go away in due course.  You are advised
to run Coot so that you can see the console and the graphics window at
the same time, since feedback from atom clicking (for example) is
often written there rather than displayed in the graphics window.

\begin{itemize}
\item Output that starts ``ERROR...'' is a programming problem (and
  ideally, you should never see it).
\item Output that starts ``WARNING...'' means that something propably
unintented happened due to the unexpected nature of your input or
file(s).
\item Output that starts ``DEBUG...'' has (obviously enough) been
  added to aid debugging.  Most of them should have been cleaned up
  before release, but as Coot is constantly being developed, a few may
  slip through.  Just ignore them.
\end{itemize}


% -----------------------------------------------------------
\chapter{Coordinate-Related Features}
% -----------------------------------------------------------


\section{Read coordinates}
The format\index{coordinates format} of coordinates that can be read
by coot is either PDB or mmCIF.  To read coordinates, choose
\textsf{File $\rightarrow$ Read Coordinates} from the menu-bar.
Immediately after the coordinates have been read, the view is (by
default) recentred to the centre of this new molecule and the molecule
is displayed.  To disable the recentring of the view on reading a
coordinates file, use: \texttt{(recentre-on-read-pdb 0)}.

\subsection{Read multiple coordinate files}
\index{reading multiple pdb files}\index{multiple coordinates files}
The reading multiple files using the GUI is not available (at the
moment).  However the following scripting functions are available:

\texttt{(read-pdb-all)}

which reads all the ``*.pdb'' files in the current directory

\texttt{(multi-read-pdb \emph{glob-pattern} \emph{dir})}

which reads all the files matching \texttt{\emph{glob-pattern}} in
directory \texttt{\emph{dir}}.  Typical usage of this might be:

\texttt{(multi-read-pdb "a*.pdb" ".")}

Alternatively you can specify the files to be opened on the command
line when you start coot (see Section
\ref{sec:command_line_arguments}).

\section{Atom Info}
\index{atom info}\index{residue info} Information about about a
particular atom is displayed in the text console when you click using
middle-mouse.  Information for all the atoms in a residue is available
using \textsf{Info $\rightarrow$ Residue Info\ldots}.

\index{edit B-factors}\index{edit occupancy}The temperature factors
and occupancy of the atoms in a residue can be set by using
\textsf{Edit $\rightarrow$ Residue Info\ldots}.

\section{Atom Labeling}
\index{atom labeling}
\label{sec:atom}
Use Shift + left-mouse to label atom.  Do the same to toggle off the
label.  The font size is changeable using \textsf{Edit $\rightarrow
  $Font Size\ldots}.  The newly centred atom is labelled by default.
To turn this off use:

\texttt{(set-label-on-recentre-flag 0)}

\index{atom label, brief}Some people prefer to have atom labels that
are shorter, without the slashes and residue name:

\texttt{(set-brief-atom-labels 1)}

\section{Atom Colouring}
The atom colouring \index{colouring, atoms} \index{atom colouring}
system in coot is unsophisticated. Typically, atoms are coloured by
element: carbons are yellow, oxygens red, nitrogens blue, hydrogens
white and everything else green (see Section \ref{sec:display_manager}
for colour by chain).  However, it is useful to be able to distinguish
different molecules by colour, so by default coot rotates the colour
map of the atoms (\emph{i.e.} changes the H value in the
HSV\footnote{Hue Saturation Value (Intensity).}  colour system).  The
amount of the rotation depends on the molecule number and a
user-settable parameter:
\begin{trivlist}
\item \texttt{(set-colour-map-rotation-on-read-pdb 30)}.
\end{trivlist}

The default value is 31$^\circ$.

Also one is able to select only the Carbon atoms to change colour in
this manner: \texttt{(set-colour-map-rotation-on-read-pdb-c-only-flag
  1)}.

\section{Bond Parameters}
The various bond parameters can be set using the GUI dialog
\textsf{Draw $\rightarrow$ Bond Parameters} or \emph{via} scripting
functions.

\subsection{Bond Thickness}
\index{bond thickness}\index{width, bonds} The thickness (width) of
bonds of inividual molecules can be changed.  This can be done via the
\textsf{Bond Parameters} dialog or the scripting interface:

\texttt{(set-bond-thickness thickness imol)}

where \texttt{imol} is the molecule number. The default thickness is
3.0. The bond thickness also applies to the symmetry atoms of the
molecule.  There is no means to change the bond thickness of a residue
selection within a molecule.

\subsection{Display Hydrogens}
\index{hydrogens}Initially, hydrogens are displayed.  They can be
undisplayed using 

\texttt{(set-draw-hydrogens mol-no 0)}\footnote{they
  can be redisplayed using \texttt{(set-draw-hydrogens mol-no 1)}.}

where \texttt{mol-no} is the molecule number.

\subsection{NCS Ghosts Coordinates}
\index{NCS}It is occasionally useful when analysing
non-crystallographically related molecules to have ``images'' of the
other related molecules appear matched onto the current coordinates.
As you read in coordinates in Coot, they are checked for NCS
relationships and clicking on ``Show NCS Ghosts'' $\rightarrow$
``Yes'' $\rightarrow$ ``Apply'' will create ``ghost'' copies of them
over the reference chain\footnote{the reference chain is the first
  chain of that type in the coordinates file.}.

\subsection{NCS Maps}
Coot can use the relative transformations of the NCS-related molecules
in a coordinates molecule to transform maps. Use \textsf{Calulate}
$\rightarrow$ \textsf{NCS Maps\ldots} to do this (note the NCS maps
only make sense in the region of the reference chain (see above).
\index{NCS averaging}This will also create an NCS averaged
map\footnote{that also only makes sense in the region of the reference
  chain.}.

\section{Download coordinates}
Coot provides the possibility to download coordinates from an
\index{OCA}OCA\footnote{OCA is ``goose'' in Spanish (and Italian).
  \index{goose}} (\emph{e.g.} EBI) server\footnote{the default is the
  Weizmann Institute - which for reasons I won't go into here is
  currently much faster than the EBI server.} (\textsf{File
  $\rightarrow$ Get PDB Using Code\ldots}). A popup entry box is
displayed into which you can type a PDB accession code.  Coot will
then connect to the web server and transfer the file.  Coot blocks as
it does this (which is not ideal) but on a semi-decent internet
connection, it's not too bad.  The downloaded coordinates are saved
into a directory called \texttt{.coot}.

It is also possible to download mmCIF data and generate a map.  This
currently requires a properly formatted database structure factors
mmCIF file\footnote{which (currently) only a fraction are.}.

\section{Save Coordinates}
On selecting from the menus \textsf{File $\rightarrow$ Save
  Coordinates\ldots} you are first presented with a list of molecules
which have coordinates.  As well as the molecule number, there is the
molecule name - very frequently the name of the file that was read in
to generate the coordinates in coot initially.  However, this is only
a \emph{molecule} name and should not be confused with the filename to
which the coordinates are saved.  The coordinates \emph{filename} can
be selected using the \textsf{Select Filename\ldots} button.

If your filename ends in \texttt{.cif}, \texttt{.mmcif} or
\texttt{.mmCIF} then an mmCIF file will be written (not a ``PDB''
file).

\section{Anisotropic Atoms}
\index{anisotropic atoms} By default anisotropic atom information is
not represented\footnote{using thermal ellipsoids}.  To turn them on,
use \textsf{Draw $\rightarrow$ Anisotropic Atoms $\rightarrow$ Show
  Anisotropic Atoms?  $\rightarrow$ Yes}, or the command:
\texttt{(set-show-aniso 1)}.

You cannot currently display thermal ellipsoids\footnote{in the case
  of isotropic atoms, ellipsoids are spherical, of course.} for
isotropic atoms.

\section{Symmetry}
\index{symmetry} Coordinates symmetry is ``dynamic''.  Symmetry atoms
can be labeled\footnote{symmetry labels are in pale blue and also
  provide the symmetry operator number and the translations along the
  $a$, $b$ and $c$ axes.}.  Every time you recentre, the symmetry gets
updated.  The information shown contains the atom information and the
symmetry operation number and translations needed to generate the atom
in that position.

The symmetry can be represented as C$\alpha$s\index{C$\alpha$ symmetry
  representation}.  This along with representation of the molecule as
C$\alpha$s (Section \ref{sec:display_manager}) allow the production of
a packing diagram\index{packing diagram}.

\section{Sequence View}
\index{sequence view} The protein is represented by one letter codes
and coloured according to secondary structure.  These one letter codes
are active - if you click on them, they will change the centre of the
graphics window - in much the same way as clicking on a residue in the
Ramachandran plot.

\section{Environment Distances}
% not this residue, to symmetry if symmetry is on
% coloured bumps (C)
Environment distances are turned on using \textsf{Info $\rightarrow$
  Environment Distances\ldots}.  Contacts to other residues are shown
and to symmetry-related atoms if symmetry is being displayed.  The
contacts are coloured by atom type\footnote{contacts not involving a
  carbon atom are yellow.}.

\section{Distances and Angles}
The distance between atoms can be found using \textsf{Info
  $\rightarrow$ Distance}\footnote{Use \textsf{Angle} for an angle, of
  course.}.  The result is displayed graphically, and written to the
console.

\section{Zero Occupancy Marker}
\index{zero occupancy}Atoms of zero occupancy are marked with a grey
spot. To turn off these markers, use:

\texttt{(set-draw-zero-occ-markers 0)}

Use an argument of 1 to turn them on.

\section{Mean, Median Temperature Factors}
Coot can be used to calculate the \index{mean B-factor}mean (average)
and \index{median B-factor}median temperatures factors:

\texttt{(average-temperature-factor \emph{imol})}

\texttt{(median-temperature-factor \emph{imol})}

$-1$ is returned if there was a problem\footnote{\emph{e.g.} this
  molecule was a map or a closed molecule.}.

\section{Least-Squares Fitting}
There is currently no GUI specified for this, the scripting interface
is as follows:

\texttt{(simple-lsq-match \emph{ref-start-resno ref-end-resno ref-chain-id imol-ref
           mov-start-resno mov-end-resno mov-chain-id imol-mov
           match-type})}

where:
\begin{trivlist}
\item \texttt{\emph{ref-start-resno}} is the starting residue number
  of the reference molecule
\item \texttt{\emph{ref-end-resno}} is the last residue number
  of the reference molecule
\item \texttt{\emph{mov-start-resno}} is the starting residue number
  of the moving molecule
\item \texttt{\emph{mov-end-resno}} is the last residue number
  of the moving molecule
\item \texttt{\emph{match-type}} is one of \texttt{'CA},
  \texttt{'main}, or \texttt{'all}.
\end{trivlist}

\emph{e.g.}: 
\texttt{(simple-lsq-match 940 950 "A" 0 940 950 "A" 1 'main)}

More sophisticated (match molecule number 1 chain ``B'' on to molecule
number 0 chain ``A''):

\vspace{-2mm}
\begin{quote}
\texttt{(define match1 (list 840 850 "A" 440 450 "B" 'all))}\\
\texttt{(define match2 (list 940 950 "A" 540 550 "B" 'main))}\\
\texttt{(clear-lsq-matches)}\\
\texttt{(set-match-element match1)}\\
\texttt{(set-match-element match2)}\\
\texttt{(lsq-match 0 1)} ; match mol number 1 one mol number 0.
\end{quote}

%% \begin{trivlist}
%% \item \texttt{(define match1 (list 840 850 "A" 440 450 "B" 'all))}
%% \item \texttt{(define match2 (list 940 950 "A" 540 550 "B" 'main))}
%% \item \texttt{(clear-lsq-matches)}
%% \item \texttt{(set-match-element match1)}
%% \item \texttt{(set-match-element match2)}
%% \item \texttt{(lsq-match 0 1)}
%% \end{trivlist}

\section{More on Moving Molecules}
There are scripting functions available for this sort of thing:

\texttt{(molecule-centre \emph{imol})} 

will tell you the molecule centre \index{molecule centre} of the
\texttt{\emph{imol}}th molecule.

\texttt{(translate-by \texttt{imol x-shift y-shift z-shift})}

will translate all the atoms in molecule \texttt{\emph{imol}} by the
given amount (in {\AA}ngstr\"{o}ms)\index{translate molecule}.

\texttt{(move-molecule-to-screen-centre \emph{imol})}

will move the \texttt{\emph{imol}}th molecule to the current centre of
the screen (sometimes useful for imported ligands).  Note that this
moves the atoms of the molecule - not just the view of the molecule.


% -----------------------------------------------------------
\chapter{Modelling and Building}
% -----------------------------------------------------------
\label{modelling,building}

The functions described in this chapter manipulate, extend or build
molecules and can be found under \textsf{Calculate $\rightarrow$
  Model/Fit/Refine\ldots}.

\section{Regularization and Real Space Refinement}
\label{sec:randr}
If you have CCP4 installed, coot will read the geometry restraints for
refmac and use them in fragment (zone) idealization - this is called
``Regularization''\index{regularization}.  The geometrical restraints
are, by default, bonds, angles, planes\index{planes} and non-bonded
contacts.  You can additionally use torsion restraints\index{torsion
  restraints} by \textsf{Calculate $\rightarrow$
  Model/Fit/Refine\ldots $\rightarrow$ Refine/Regularize Control
  $\rightarrow$ Use Torsion Restraints}.

% cite Bob Diamond (1971) here somewhere.



``RS (Real Space) Refinement''\index{refinement} (after Diamond,
1971\footnote{Diamond, R. (1971). A Real-Space Refinement Procedure
  for Proteins. \emph{Acta Crystallographica} \textbf{A}27, 436-452.
  }) in Coot is the use of the map in addition to geometry terms to
improve the positions of the atoms.  Select ``Regularize'' from the
``Model/Fit/Refine'' dialog and click on 2 atoms to define the zone
(you can of course click on the same atom twice if you only want to
regularize one residue).  Coot then regularizes the residue range.  At
the end Coot, displays the intermediate atoms in white and also
displays a dialog, in which you can accept or reject this
regularization.  In the console are displayed the $\chi^2$ values of
the various geometrical restraints for the zone before and after the
regularization.  Usually the $\chi^2$ values are considerably
decreased - structure idealization such as this should drive the
$\chi^2$ values toward zero.

The use of ``Refinement'' is similar - with the addition of using a
map.  The map used to refine the structure is set by using the
``Refine/Regularize Control'' dialog.  If you have read/created only
one map into Coot, then that map will be used (there is no need to set
it explicitly).


Use, for example, \index{\texttt{set-matrix}}\texttt{(set-matrix 20.0)}
\footnote{\texttt{set\_matrix(20.0)} (using python).} to change the
weight of the map gradients to geometric gradients.  The higher the
number the more weight that is given to the map terms\footnote{but the
  resulting $\chi^2$ values are higher.}.  The default is 150.0.  This
will be needed for maps generated from data not on (or close to) the
absolute scale or maps that have been scaled (for example so that
the sigma level has been scaled to 1.0).

For both ``Regularize Zone'' and ``Refine Zone'' one is able to use a
single click to \index{single click refine}\index{refine single
  click}refine a residue range.  Pressing ``A'' on the keyboard while
selecting an atom in a residue will automatically create a residue
range with that residue in the middle.  By default the zone is
extended one residue either size of the central residue.  This can be
changed to 2 either side using \texttt{(set-refine-auto-range-step
  2)}.

Intermediate (white) atoms can be moved around with the mouse (click
and drag with left-mouse, by default).  \marginpar{\footnotesize
  \textsf{This is a useful feature}} Refinement will proceed from the
new atom positions when the mouse button is released.  It is possible
to create incorrect atom nomenclature and/or chiral volumes in this
manner - so some care must be taken.  Press the ``A'' key as you
left-mouse click to move atoms more ``locally'' (rather than a linear
shear) and Cntrl key as you left-mouse click to move just one atom.

To prevent the unintentional refinement of a large number of residues,
there is a ``heuristic fencepost'' of 20 residues.  A selection of
than 20 residues will not be regularized or refined.  The limit can be
changed using the scripting function: \emph{e.g.}
\texttt{(set-refine-max-residues 30)}.

\subsection{Dictionary}
\label{cif-dictionary}\index{cif dictionary, mmCIF dictionary}By default, 
the geometry dictionary entries for only the standard
residues are read in at the start \footnote{And a few extras, such as
  phospate}.  It may be that you particular ligand is not amongst
these.  To interactively add a dictionary entry use \textsf{File
  $\rightarrow$ Import CIF Dictionary}.  Alternatively, you can use
the function:

\texttt{(read-cif-dictionary \emph{filename})}

and add this to your \texttt{.coot} file (this may be the prefered
method if you want to read the file on more than one occassion).  

Note: the dictionary also provides the description of the ligand's
torsions.


\section{Rotate/Translate Zone}
\label{sec:rot_trans_zone}\index{rotate/translate, manual}``Rotate/Translate 
Zone'' from the ``Model/Fit/Refine'' menu allows manual movement of a
zone.  After pressing the ``Rotate/Translate Zone'' button, select two
atoms in the graphics canvas to define a residue range\footnote{if you
  want to move only one residue, then click the same atom twice.}, the
second atom that you click will be the local rotation centre for the
zone.  The atoms selected in the moving fragment have the same
alternate conformation code as the first atom you click.  To actuate a
transformation, click and drag horizontally across the relevant button
in the newly-created ``Rotation \& Translation'' dialog. The axis
system of the rotations and translations are the screen coordinates.
Alternatively \footnote{like Refinement and Regularization}, you can
click using left-mouse on an atom in the fragment and drag the
fragment around. Use Control Left-mouse to move just one atom, rather
than the whole fragment.  Click ``OK'' when the transformation is
complete.

\section{Rigid Body Refinement}
\label{sec:RigidBodyRefinement} \index{refinement, rigid body}
\index{rigid body fit}``Rigid Body Fit Zone'' from the
``Model/Fit/Refine'' dialog provides rigid body refinement.  The
selection is zone-based\footnote{like Regularization and Refinement.}.
So to refine just one residue, click on one atom twice.

Sometimes no results are displayed after Rigid Body Fit Zone.  This is
because the final model positions had too many final atom positions in
negative density.  If you want to over-rule the default fraction of
atoms in the zone that have an acceptable fit (0.75), to be (say)
0.25:

\texttt{(set-rigid-body-fit-acceptable-fit-fraction 0.25)}

\section{Baton Build}
\index{baton build} Baton build is most useful if a skeleton is
already calculated and displayed (see Section \ref{skeletonization}).
When three or more atoms have been built in a chain, Coot will use a
prior probability distribution for the next position based on the
position of the previous three.  The analysis is similar to Oldfield
\& Hubbard\footnote{T. J.  Oldfield \& R. E. Hubbard.  ``Analysis of
  C-Alpha Geometry in Protein Structures'' \emph{Proteins-Structure
    Function and Genetics} \textbf{18(4)} 324 -- 337.}, however it is
based on a more recent and considerably larger database.

Little crosses are drawn representing directions in which is is
possible that the chain goes, and a baton is drawn from the current
point to one of these new positions.  If you don't like this
particular direction\footnote{which is quite likely at first since
  coot has no knowledge of where the chain has been and cannot score
  according to geometric criteria.}, use \textsf{Try Another}.  The
list of directions is scored according to the above criterion and
sorted so that the most likely is at the top of the list and displayed
first as the baton direction.

When starting baton building, be sure to be about 3.8\AA\ from the
position of the first-placed C$\alpha$, this is because the next
C$\alpha$ is placed at the end of the baton, the baton root being at
the centre of the screen.  So, when trying to baton-build a chain
starting at residue 1, centre the screen at about the position of
residue 2.

% ``b'' key in GL canvas
\index{baton mode}Occasionally, every point is not where you want to
position the next atom.  In that case you can either shorten or
lengthen the baton, or position it yourself using the mouse.  Use
``b'' on the keyboard to swap to baton mode for the
mouse\footnote{``b'' again toggles the mode off.}.

Baton-built atoms are placed into a molecule called ``Baton Atom'' and
it is often sensible to save the coordinates of this molecule before
quitting coot.

If you try to trace a high resolution map (1.5\AA\  or better) you will
need to increase the skeleton search depth from the default (10), for
example:

\texttt{(set-max-skeleton-search-depth 20)}

Alternatively, you could generate a new map using data
to a more moderate resolution (2\AA), the map may be easier to
interpret at that resolution anyhow\footnote{high-resolution map
  interpretation is planned.}.

The guide positions are updated every time the ``Accept'' button is
clicked.  The molecule name for these atoms is ``Baton Build Guide Points''
and is is not usually necessary to keep them.

\subsection{Building Backwards}
The following senario is not uncommon: you find a nice streatch of
density and start baton building in it.  After a while you come to a
point where you stop (dismissing the baton build dialog).  You want to
go back to where you started and build the other way.  How do you do
that?

\begin{itemize}
\item Use the command: \texttt{(set-baton-build-params start-resno
    chain-id "backwards")}, where \texttt{start-resno} would typically
  be 0\footnote{\emph{i.e.} one less than the starting residue in the
    forward direction (defaults to 1).} and \texttt{chain-id} would be
  \texttt{""} (default).
\item Recentre the graphics window on the first atom of the just-build
  fragment
\item Select ``C$\alpha$ Baton Mode'' and select a baton direction
  that goes in the ``opposite'' direction to what is typically residue
  2.  This is slightly awkward because the initial baton atoms build
  in the ``opposite'' direction are not dependent on the first few
  atoms of the previously build fragment.
\end{itemize}


\subsection{Undo}
There is also an ``Undo'' button for baton-building.  Pressing this
will delete the most recently placed C$\alpha$ and the guide points
will be recalculated for the previous position.  The number of
``Undo''s is unlimited.  Note that you should use the ``Undo'' button
in the Baton Build dialog, not the one in the ``Model/Fit/Refine''
dialog (Section \ref{sec:backups_undo}).

\subsection{Missing Skeleton}
\index{skeleton, missing}Sometimes (especially at loops) you can see
the direction in which the chain should go, but there is no skeleton
(see Section \ref{skeletonization}) is displayed (and consequently no
guide points) in that direction. In that case, ``Undo'' the previous
atom and decrease the skeletonization level (\textsf{Edit
  $\rightarrow$ Skeleton Parameters $\rightarrow$ Skeletonization
  Level}).  Accept the atom (in the same place as last time) and now
when the new guide points are displayed, there should be an option to
build in a new direction.


\section{C$\alpha \rightarrow$ Mainchain}
\index{mainchain} Mainchain can be generated using a set of C$\alpha$s
as guide-points (such as those from Baton-building) along the line of
Esnouf\footnote{R. M. Esnouf ``Polyalanine Reconstruction from
  C$\alpha$ Positions Using the Program \emph{CALPHA} Can Aid Initial
  Phasing of Data by Molecular Replacement Procedures'' \emph{Acta
    Cryst. }, D\textbf{53}, 666-672 (1997).} or Jones and
coworkers\footnote{T.A.  Jones \& S. Thirup ``Using known
  substructures in protein model building and crystallography''
  \emph{EMBO J.} \textbf{5}, 819--822 (1986).}.  Briefly, 6-residue
fragments of are generated from a list of high-quality\footnote{and
  high resolution} structures. The C$\alpha$ atoms of these fragments
are matched against overlapping sets of the guide-point C$\alpha$s.
The resulting matches are merged to provide positions for the
mainchain (and C$\beta$) atoms.  This proceedure works well for
helices and strands, but less well\footnote{\emph{i.e.}  there are
  severely misplaced atoms} for less common structural features.

This function is also available from the scripting interface:

\texttt{(db-mainchain imol chain-id resno-start resno-end direction)}
    
where direction is either \texttt{"backwards"} or \texttt{"forwards"}.

% Withdrawn due to being to difficult to calculate the atom positions 
% given the phi and psi
%
%\section{Edit Phi/Psi}
%\index{edit $\phi/\psi$}This generates a Ramachandran plot with only
%one residue represented.  You can click and drag this residue round
%the plot and the coordinates in the graphics window will change to the
%$\phi/\psi$ values in the Ramachandran plot.

\section{Backbone Torsion Angles}
It is possible to edit the backbone $\phi$ and $\psi$ angles
indirectly using an option in the Model/Fit/Refine's dialog: ``Edit
Backbone Torsions..''. When clicked and an atom of a peptide is
selected, this produces a new dialog that offers ``Rotate Peptide''
which changes this residues $\psi$ and ``Rotate Carbonyl'' which
changes $\phi$.  Click and drag across the button\footnote{as for
  Rotate/Translate Zone (Section \ref{sec:rot_trans_zone}).} to rotate
the moving atoms in the graphics window.  You should know, of course,
that making these modifications alter the $\phi/\psi$ angles of more
than one residue.


\section{Rotamers}
\label{sec:rotamers}
\index{Dunbrack, Roland}\index{rotamers} The rotamers are generated
from the backbone independent sidechain library of Roland Dunbrack and
co-workers\footnote{R. L.  Dunbrack, Jr. \& F. E.  Cohen. "Bayesian
  statistical analysis of protein sidechain rotamer preferences"
  \emph{Protein Science}, \textbf{6}, 1661--1681 (1997). }. According
to this analysis, some sidechains have many rotamer
options\footnote{LYS, for example has 81.}.  By default only rotamers
with a probability (as derived from the structural database) greater
than 1\% are considered. The initial position is the most likely for
that residue type\footnote{Use \emph{e.g.}
  \texttt{(set-rotamer-lowest-probability 0.5)} to change the
  probability lower limit for the rotamer selection (note that this is
  a percentage, therefore 0.5(\%) is quite low and will allow the
  choice of more rotamers than the default.}.

Use keyboard ``.'' and ``,'' to cycle round the rotamers.

\subsection{Auto Fit Rotamer}
\index{auto-fit rotamer}``Auto Fit Rotamer'' will try to fit the
rotamer to the electron density.  Each rotamer is generated, rigid
body refined and scored according to the fit to the map.  Fitting the
second conformation of a dual conformation in this way will often fail
- the algorithm will pick the best fit to the density - ignoring the
position of the other atoms.

The algorithm doesn't know if the other atoms in the structure are in
sensible positions.  If they are, then it is sensible not to put this
residue too close to them, if they are not then there should be no
restriction from the other atoms as to the position of this residue -
the default is ``are sensible'', which means that the algorithm is
prevented from finding solutions that are too close to the atoms of
other residues. \texttt{(set-rotamer-check-clashes 0)} will stop this.

There is a scripting interface to auto-fitting rotamers:

\texttt{(auto-fit-best-rotamer \emph{resno alt-loc ins-code chain-id\\imol-coords
imol-map clash-flag lowest-rotamer-probability})}

where:

\texttt{\emph{resno}} is the residue number

\texttt{\emph{alt-loc}} is the alternate/alternative location symbol
(\emph{e.g.} \texttt{"A"} or \texttt{"B"}, but most often \texttt{""})

\texttt{\emph{ins-code}} is the insertion code (usually \texttt{""})

\texttt{\emph{imol-coords}} is the molecule number of the coordinates molecule

\texttt{\emph{imol-map}} is the molecule number of the map to which
you wish to fit the side chains

\texttt{\emph{clash-flag}} should the positions of other residues be
included in the scoring of the rotamers (\emph{i.e.} clashing with other
other atoms gets marked as bad/unlikely)

\texttt{\emph{lowest-rotamer-probability}}: some rotamers of some side
chains are so unlikely that they shouldn't be considered - typically
0.01 (1\%).

\subsection{De-clashing residues}
Sometimes you don't have a map\footnote{for example, in preparation of
  a model for molecular replacement} but nevertheless there are
clashing residues\index{clashing residues}\footnote{atoms of residues
  that are too close to each other} (for example after mutation of a
residue range) and you need to rotate side-chains to a non-clashing
rotamer.  There is a scripting interface:

\texttt{(de-clash \texttt{imol chain-id start-resno end-resno})}

\texttt{\emph{start-resno}} is the residue number of the first residue
you wish to de-clash.

\texttt{\emph{start-resno}} is the residue number of the last residue
you wish to de-clash

\texttt{\emph{imol}} is the molecule number of the coordinates molecule

This interface will not change residues with insertion codes or
alternate conformation.  The
\texttt{\emph{lowest-rotamer-probability}} is set to 0.01.


\section{Editing $\chi$ Angles}
\index{edit $\chi$ angles}Instead of using Rotamers, one can instead
change the $\chi$ angles \index{torsions}(often called ``torsions'')
``by hand'' (using ``Edit Chi Angles'' from the ``Model/Fit/Refine''
dialog). To edit a residue's $\chi_1$ press ``1'': to edit $\chi_2$,
``2'': $\chi_3$ ``3'' and $\chi_4$ ``4''.  Use left-mouse click and
drag to change the $\chi$ value.  Use keyboard ``0''\footnote{that's
  ``zero''.} to go back to ordinary view mode at any time during the
editing.  Alternatively, one can use the ``View Rotation Mode'' or use
the Ctrl key when moving the mouse in the graphics window.  Use the
Accept/Reject dialog when you have finished editing the $\chi$ angles.

It should be emphasised that for standard residues this is an option
of last resort - use the other rotamer manipulation options first.

\subsection{Ligand Torsion angles}
\index{torsion angles, ligand}\index{ligand torsion angles}For
ligands, you will need to read the mmCIF file that contains a
description of the ligand's geometry (see Section
\ref{cif-dictionary}).  By default, torsions that move hydrogens are
not included.  Only 9 torsion angles are available from the keyboard
torsion angle selection.

\section{Pep-flip}
\index{pepflip}\index{flip peptide} Coot uses the same pepflip scheme
as is used in $O$ (\emph{i.e.} the C, N and O atoms are rotated
180$^o$ round a line joining the C$\alpha$ atoms of the residues
involved in the peptide).  Flip the peptide again to return the atoms
to their previous position.


\section{Add Alternate Conformation}
\label{sec:add_alt_conf}
The allows the addition alternate (\index{dual conformations}dual,
triple \emph{etc.})  conformations to the picked residue.  By default,
this provides a choice of rotamer (Section \ref{sec:rotamers}).  If
there are not the correct main chain atoms a rotamer choice cannot be
provided, and Coot falls back to providing intermediate atoms.

The default occupancy for new atoms is 0.5.  This can be changed by
using use slider on the rotamer selection window or by using the
scripting function:

\texttt{(set-add-alt-conf-new-atoms-occupancy 0.4)}

% The intermediate atoms interface can be forced using:

% \texttt{(set-show-alt-conf-intermediate-atoms 1)}


\section{Mutation}
\index{mutate} Mutations are available on a 1-by-1 basis using the
graphics.  After selecting ``Mutate\ldots'' from the
``Model/Fit/Refine'' dialog, click on an atom in the graphics.  A
``Residue Type'' window will now appear.  Select the new residue type
you wish and the residue in the graphics is updated to the new residue
type\footnote{Note that selecting a residue type that matches the
  residue in the graphics will also result in a mutation}.  The
initial position of the new rotatmer is the \emph{a priori} most
likely rotamer. Note that in interactive mode, such as this, a residue
type match\footnote{\emph{i.e.} the current residue type matches the
  residue type to which you wish to mutate the residue} will not stop
the mutation action occurring.

\subsection{Multiple mutations}
This dialog can be found under \textsf{Calculate $\rightarrow$ Mutate
  Residue Range}.  A residue range can be assigned a sequence and
optionally fitted to the map.  This is useful converting a poly-ALA
model to the correct sequence\footnote{\emph{e.g.} after using Ca
  $\rightarrow$ Mainchain.}.

Multiple mutations\index{multi-mutate} are also supported \emph{via}
the scripting interface.  Unlike the single residue mutation function,
a residue type match \emph{will} prevent a modification of the
residue\footnote{\emph{i.e.} the residue atoms will remain untouched}.
Two functions are provided: To mutate a whole chain, use
\texttt{(mutate-chain \emph{imol} \emph{chain-id sequence})} where:

\texttt{\emph{chain-id}} is the chain identifier of the chain that you wish
to mutate (\emph{e.g.} \texttt{"A"}) and 

\texttt{\emph{imol}} is molecule number.  

\texttt{\emph{sequence}} is a list of single-letter residue codes,
such as \texttt{"GYRESDF"} (this should be a straight string with no
additional spaces or carriage returns).

Note that the number of residues in the sequence chain and those in
the chain of the protein must match exactly (\emph{i.e.} the whole of
  the chain is mutated (except residues that have a matching residue
  type).)

To mutate a residue range, use 

\begin{trivlist}
\item 
\texttt{(mutate-residue-range \emph{chain-id}
  \emph{start-res-no} \emph{stop-res-no \newline sequence})}
\end{trivlist}

where

\texttt{\emph{start-res-no}} is the starting residue for mutation

\texttt{\emph{stop-res-no}} is the last residue for mutation, \emph{i.e.}
using values of 2 and 3 for \texttt{\emph{start-res-no}} and
\texttt{\emph{stop-res-no}} respectively will mutate 2 residues.

Again, the length of the sequence must correspond to the residue range
length.

\subsection{Mutate and Autofit}
The function combines Mutation and Auto Fit Rotamer and is the easiest
way to make a mutation and then fit to the map.

\subsection{Renumbering}
\index{renumbering residues}Renumbering is straightforward using the
renumber dialog available under \textsf{Calculate $\rightarrow$
  Renumber Residue Range\ldots}.  There is also a scripting interface:

\texttt{(renumber-residue-range \emph{imol chain-id start-res-no
    last-resno offset})}

\section{Find Ligands}
\index{ligands} You are offered a selection of maps to search (you can
only choose one at a time) and a selection of molecules that act as a
mask to this map.  Finally you must choose which ligand types you are
going to search for in this map\footnote{you can search for many
  different ligand types.}.  Only molecules with less than 400 atoms
are suggested as potential ligands.  New ligands are placed where the
map density is and protein (mask) atoms are \emph{not}).  The masked
map is searched for clusters using a default cut-off of 1.0$\sigma$.
In weak density this cut-off may be too high and in such a case the
cut-off value can be changed using something such as:

\texttt{(set-ligand-cluster-sigma-level 0.8)}

However, if the map to be searched for ligands is a difference map, a
cluster level of 2.0 or 3.0 would probably be more
appropriate\footnote{less likely to generate spurious sites.}.

Each ligand is fitted with rigid body refinement to each potential
ligand site in the map and the best one for each site selected and
written out as a pdb file.  The clusters are sorted by size, the
biggest one first (with an index of 0).  The output placed ligands
files have a prefix ``best-overall'' and are tagged by the cluster
index and residue type of the best fit ligand in that site.

By default, the top 10 sites are tested for ligands - to increase this
use:

\texttt{(set-ligand-n-top-ligands 20)}

\subsection{Flexible Ligands}
\index{ligands, flexible}
If the ``Flexible?'' checkbutton is activated, coot will generate a
number of variable conformations (default 100) by rotating around the
rotatable bonds (torsions).  Each of these conformations will be fitted
to each of the potential ligand sites in the map and the best one will
be selected (again, if it passes the fitting criteria above).

Before you search for flexible ligands you must have read the mmCIF
dictionary for that particular ligand residue type (\textsf{File
  $\rightarrow$ Import CIF dictionary\index{dictionary, cif}}).

Use:

\texttt{(set-ligand-flexible-ligand-n-samples \emph{n-samples})}

where \texttt{\emph{n-samples}} is the number of samples of flexiblity
made for each ligand.  The more the number of rotatable bonds, the
bigger this number should be.

\subsection{Adding Ligands to Model}
After successful ligand searching, one may well want to add that
displayed ligand to the current model (the coordinates set that
provided the map mask).  To do so, use Merge Molecules (Section
\ref{sec:merge_molecules}).


\section{Find Waters}
\index{waters, finding} As with finding ligands, you are given a chose
of maps, protein (masking) atoms. A final selection has to be made for
the cut-off level, note that this value is the number of standard
deviation of the density of the map \emph{after} the map has been
masked.  Then the map is masked by the masking atoms and a search is
made of features in the map about the electron density cut-off value.
Waters are added if the feature is approximately water-sized and can
make sensible hydrogen bonds to the protein atoms.  The new waters are
optionally created in a new molecule called ``Waters''.

You have control over several parameters used in the water finding:

\texttt{(set-write-peaksearched-waters)} 

which writes \texttt{ligand-waters-peaksearch-results.pdb}, which
contains the water peaks (from the clusters) without any filtering and
\texttt{ligand-waters.pdb} which are a disk copy filtered waters that
have been either added to the molecule or from which a new molecule
has been created.

\texttt{(set-ligand-water-spherical-variance-limit min-d max-d)} sets
the minimum and maximum allowable distances between new waters and the
masking molecule (usually the protein).

\texttt{(set-ligand-water-spherical-variance-limit varlim)} sets the
upper limit for the density variance around water atoms. The default
is 0.12.
% $electrons^2/\AA^6$.

The map that is maked by the protein and is searched to find the
waters is written out in CCP4 format as \texttt{"masked-for-waters.map"}.

\subsection{Blobs}
After a water search, Coot will create a blobs dialog (see Section
\ref{sec:blobs}).

\subsection{Check Waters via Difference Map}
Another check of waters that one can perform is the following:

\texttt{(check-waters-by-difference-map \emph{imol-coords}
  \emph{imol-diff-map})}

where \texttt{\emph{imol-coords}} is the molecule number of the
coordinates that contain the waters to be checked

\texttt{\emph{imol-diff-map}} is the molecule number of the difference
map (it must be a difference map, not an ``ordinary'' map).  This
difference map must have been calculated using the waters. So there is
no point in doing this check immediately after ``Find Waters''.  You
will need to run Refmac first\footnote{and remember to check the
  difference map button in the ``Run Refmac'' dialog}.

This analysis will return a list of water atoms that have
outstandingly high local variance of the difference map (by default a
sphere of 1.5\AA\ centred about the atom position).  This analysis
might find waters that are actually something else, for example: part
of a ligand, a sulfate, an anion or cation, only partially occupied or
should be deleted entirely.  Coot\footnote{as yet} doesn't decide what
should be done about these atoms, it merely brings them to your
attention.  It may be interesting to use an anomalous map to do this
analysis.

There is no GUI for this feature.

\section{Add Terminal Residue}
\index{terminal residue} This creates a new residue at the C or N
terminus by fitting to the map.  $\phi/\psi$ angle pairs are selected
at random based on the Ramachandran plot probability (for a generic
residue).  By default there are 100 trials.  It is possible that a
wrong position will be selected for the terminal residue and if so,
you can reject this fit and try again with Fit Terminal
Residue\footnote{usually if this still fails after two repetitions
  then it never seems to work.}. Each of the trial positions are
scored according to their fit to the map\footnote{The map is selected
  using ``Refine/Regularize Control''} and the best one selected.  It
is probably a good idea to run ``Refine Zone'' on these new residues.

\texttt{(set-terminal-residue-do-rigid-body-refine 0)} will disable
rigid body fitting of the terminal residue fragment for
each trial residue position (the default is 1 (on)) - this may help if
the search does not provide good results.

\texttt{(set-add-terminal-residue-n-phi-psi-trials 50)} will change
the number of trials (default is 100).

\section{Add OXT Atom to Residue}
\index{terminal oxygen}\index{OXT atom}At the
C-terminus\index{C-terminus} of a chain of amino-acid residues, there
is a ``modification'' so that the C-O becomes a carbonyl, \emph{i.e.}
an extra (terminal) oxygen (OXT) needs to be added.  This atom is
added so that it is in the plane of the C$\alpha$, C and O atoms of
the residue.

Scripting usage:

\texttt{(add-OXT-to-residue imol residue-number \newline insertion-code
  chain-id)}\footnote{\emph{e.g.} \texttt{(add-OXT-to-residue 0 428 "" "A")}}, 

where \texttt{insertion-code} is typically \texttt{""}.  

Note, in order to place OXT, the N, CA, C and O atoms must be present
in the residue - if (for example) the existing carbonyl oxygen atom is
called ``OE1'' then this function will not work.

\section{Add Atom at Pointer}
By default, ``Add Atom At Pointer'' will pop-up a dialog from which
you can choose the atom type you wish to insert\footnote{including
  sulfate or phosphate ions (in such a case, it is probably useful to
do a ``Rigid Body Fit Zone'' on that new residue).}.  Using
\texttt{(set-pointer-atom-is-dummy 1)} you can by-pass this dialog and
immediately create a dummy atom at the pointer position.  Use an
argument of \texttt{0} to revert to using the atom type selection
pop-up on a button press.

The atoms are added to a new molecule called ``Pointer Atoms''.  They
should be saved and merged with your coordinates outside of Coot.

\section{Merge Molecules}
\index{merge molecules}\label{sec:merge_molecules}
This dialog can be found under ``Calculate'' in the main menubar.
This is typically used to add molecule fragments or residues that are
in one molecule to the ``working'' coordinates\footnote{For example,
  after a ligand search has been performed.}.


\section{Running Refmac}
\index{refmac}\index{running refmac}
Use the ``Run Refmac...'' button to select the dataset and the
coordinates on which you would like to run Refmac.  Note that only
dataset which had Refmac parameters set as the MTZ file was read are
offered as dataset that can be used with Refmac. By default, Coot
displays the new coordinates and the new map generated from refmac's
output MTZ file.  Optionally, you can also display the difference map.

You can add extra parameters \index{refmac parameters} (data lines) to
refmac's input by storing them in a file called
\texttt{refmac-extra-params} in the directory in which you started
coot.

Coot ``blocks''\footnote{\emph{i.e.} Coot is idle and ignores all
  input.} until Refmac has terminated\footnote{This is not an idea
  feature, of course and will be addressed in future.... Digressive
  Musing: If only computers were fast enough to run Refmac
  interactively\ldots}.

The default refmac executable\index{refmac, default}\index{default
  refmac version} is \texttt{refmac5} it is presumed to be in the
path.  If you don't want this, it can be overridden using a
re-definition either at the scripting interface or in one's
\texttt{~/.coot} file \emph{e.g.}:
\begin{trivlist}
\item \texttt{(define refmac-exec "/e/refmac-new/bin/refmac5.6.3")}
\end{trivlist}

\index{refmac map colour}After running refmac several times, you may
find that you prefer if the new map that refmac creates (after refmac
refinement) is the same colour as the previous one (from before this
refmac refinement).  If so, use:

\texttt{(set-keep-map-colour-after-refmac 1)}

which will swap the colours of then new and old refmac map so that the
post-refmac map has the same colour as the pre-refmac map and the
pre-refmac map is coloured with a different colour.


\section{Clear Pending Picks}
\index{Clear Pending Picks}\index{atom picking}Sometimes one can click
on a button\footnote{such that Coot would subsequently expect an atom
  selection ``pick'' in the graphics window.} unintentionally. This
button is there for such a case.  It clears the expectation of an
atom pick.  This works not only for modelling functions, but also
geometry functions (such as Distance and Angle).

\section{Delete}
\index{delete} Single atoms or residues can be deleted from the
molecule using ``Delete\ldots'' from the ``Model/Fit/Refine''dialog.
Pressing this button results in a new dialog, with the options of
``Residue'' (the default), ``Atom'' and ``Hydrogen Atoms''.  Now click
on an atom in the graphics - the deleted object will be the whole
residue of the atom if ``Residue'' was selected and just that atom if
``Atom'' was selected.

If you want to delete multiple items you can either use check the
``Keep Delete Active'' check-button on this dialog or use the Ctrl key
as you click on an atom.  Either of these will keep the dialog open,
ready for deletion of next item.

% document delete-atom, delete-residue, delete-residue-with-altconf here.


\section{Sequence Assignment}
You can assign a (fasta format) sequence to a molecule using:

\texttt{(assign-fasta-sequence imol chain-id fasta-seq)}

This function has been provided as a precursor to functions that will
(as automatically as possible) mutate your current coordinates to one
that has the desired sequence. It will be used in automatic side-chain
assignment (at some stage in the future).

\section{Building Links and Loops}

Coot can make an attempt to build missing linking regions or
loops\footnote{the current single function doesn't always perform very
  well in tests, which is why it is currenty available only in the
  scripting format.}.  This is an area of Coot that needs to be
improved, currently O does it much better.  We will have several
different loop tools here\footnote{I suspect that there is not one
  tool that fits for all.}.  For now:

\texttt{(fit-gap \emph{imol} \emph{chain-id} \emph{start-resno} \emph{stop-resno})}

and 

\texttt{(fit-gap \emph{imol} \emph{chain-id} \emph{start-resno} \emph{stop-resno} \emph{sequence})}

the second form will also mutate and try to rotamer fit the provided sequence.

Example usage: let's say for molecule number 0 in chain \texttt{"A"}
we have residues up to 56 and then a gap after which we have residues
62 and beyond:

\texttt{(fit-gap 0 "A" 57 61 "TYPWS")}

\section{Setting Occupancies}
As well as the editing ``Residue Info'' to change occupancies of
individual atoms, one can use a scripting function to change
occupancies of a whole residue range:

\begin{trivlist}
\item \texttt{(zero-occupancy-residue-range \emph{imol chain-id \\
resno-start resno-last})}
\end{trivlist}

example usage:

\texttt{(zero-occupancy-residue-range 0 "A" 23 28)}

This is often useful to zero out a questionable loop before submitting
for refinement.  After refinement (with refmac) there should be
relatively unbiased density in the resulting 2Fo-Fc-style and
difference maps.

Similarly there is a function to reverse this operation:

\begin{trivlist}
\item \texttt{(fill-occupancy-residue-range \emph{imol chain-id \\
      resno-start resno-last})}
\end{trivlist}




% -----------------------------------------------------------
\chapter{Map-Related Features}
% -----------------------------------------------------------

\section{Maps in General}
Maps are ``infinite,'' not limited to pre-calculated volume (the
``Everywhere You Click - There Is Electron Density''
(EYC-TIED)\index{EYC-TIED} paradigm) symmetry-related electron
density is generated automatically. Maps are easily re-contoured.
Simply use the scroll wheel on you mouse to alter the contour level
(or -/+ on the keyboard)\index{change contour level}.
 
Maps follow the molecule.  As you recentre or move about the crystal,
the map quickly follows.  If your computer is not up to re-contouring
all the maps for every frame, then use \textsf{Draw $\rightarrow$
  Dragged Map\ldots} to turn off this feature.

Unfortunately, there is a bug in map-reading\label{map-reading-bug}.
If the map is not a bona-fide CCP4 map\footnote{\emph{e.g.} it's a
  directory or a coordinate filename.}, then coot will crash.  Sorry.
A fix is in the works but ``it's complicated''.

\section{Create a map}
From MTZ, mmCIF and .phs (\textsc{phases} format)\index{phases format}
data use \textsf{File $\rightarrow$ Read Dataset\ldots}. From a CCP4
map use \textsf{File $\rightarrow$ Read Map}.  After being
generated/read, the map is immediately contoured and centred on the
current rotation centre.

\subsection{Reading CIF data}
There are several maps that can be generated from CIF files that
contain observed Fs, calculated Fs and calculated phases:

\begin{trivlist}
\item \texttt{(read-cif-data-with-phases-fo-alpha-calc
    \emph{cif-file-name})} Calculate an atom map using F$_{obs}$ and
  $\alpha_{calc}$
\item \texttt{(read-cif-data-with-phases-2fo-fc \emph{cif-file-name})}
 Calculate an atom map using F$_{obs}$, F$_{calc}$ and
  $\alpha_{calc}$
\item \texttt{(read-cif-data-with-phases-fo-fc \emph{cif-file-name})}
 Calculate an difference map using F$_{obs}$, F$_{calc}$ and
  $\alpha_{calc}$.
\end{trivlist}

\section{Map Contouring}
\index{contouring, map}Maps can be re-contoured using the middle-mouse
scroll-wheel (buttons 4 and 5 in X Window System$^{\textrm{\tiny TM}}$
terminology).  Scrolling the mouse wheel will change the map contour
level and the map it redrawn.  If you have several maps displayed then
the map that is has its contour level changed can be set using
\textsf{HID$ \rightarrow$ Scrollwheel $\rightarrow$ Attach scroll-wheel
  to which map?}.  If there is only one map displayed, then that is
the map that has its contour level changed (no matter what the
scroll-wheel is attached to in the menu).  The level of the electron
density is displayed in the top right hand corner of the OpenGL canvas.

Use Keyboard + or - to change the contour level if you don't have a
scroll-wheel\footnote{like I don't on my Mac.}.

If you are creating your map from an MTZ file, you can choose to click
on the ``is difference map''\index{difference map} button on the Column
Label selection widget (after a data set filename has been selected)
then this map will be displayed in 2 colours corresponding to + and -
the map contour level.

If you read in a map it is a difference map then there is
a checkbutton to tell Coot that.

If you want to tell Coot that a map is a difference
map\index{difference map colours} after it has been read, use:

\texttt{(set-map-is-difference-map \emph{imol})}

where \texttt{\emph{imol}} is the molecule number.

By default the map radius\footnote{actually, it's a box.} is 10\AA.
The default increment to the electron density depends on whether or
not this is a difference map (0.05 $e^-$/\AA$^3$ for a ``2Fo-Fc''
style map and 0.005 $e^-$/\AA$^3$ for a difference map).  You can
change these using \textsf{Edit $\rightarrow$ Map Parameters} or by
using the ``Properties'' button of a particular map in the Display
Control (Display Manager) window.

\section{Map contour ``scrolling'' limits}
Usually one doesn't want to look at \index{negative contour
  levels}negative contour levels of a map\footnote{in a coot
  difference map you will get to see the negative level contoured at
  the inverted level of the positive level, what I mean is that you
  don't want to see the ``positive'' level going less than 0.}, so
Coot has by default a limit that stops the contour level going beyond
(less than) 0.  To remove the limit:

\texttt{(set-stop-scroll-iso-map 0)} {for a 2Fo-Fc style map}

\texttt{(set-stop-scroll-diff-map 0)} {for a difference map}

To set the limits to negative (\emph{e.g.} -0.6) levels:

\texttt{(set-stop-scroll-iso-map-level -0.6)}

and similarly: 

\texttt{(set-stop-scroll-diff-map-level -0.6)}

where the level is specified in electrons/\AA$^3$.

\section{Map Line Width}
\index{map line width}\index{density line thickness}\index{thickness
  of density lines}The width of the lines that descibe the density can
be changed like this:

\texttt{(set-map-line-width 2)}

The default line width is 1.

\section{``Dynamic'' Map colouring}
\index{colouring, map} By default, maps get coloured according to
their molecule number.  The starting colour (\emph{i.e.} for molecule
0) is blue.  The colour of a map can be changed by \textsf{Edit
  $\rightarrow$ Map Colour..}. The map colour gets updated as you
change the value in the colour selector\footnote{takes you right back
  to the good old Frodo days, no?}.  Use ``OK'' to fix that colour.

\section{Difference Map Colouring}
For some strange reason, some crystallographers\footnote{Jan Dohnalek,
  for instance.} like to have their difference maps coloured with red
as positive and green as negative, this option is for them:

\texttt{(set-swap-difference-map-colours 1)}


\section{Map Sampling}
By default, the Shannon sampling factor is the conventional 1.5.  Use
larger values (\textsf{Edit $\rightarrow$ Map Parameters $\rightarrow$
  Sampling Rate}) for smoother maps\footnote{a value of 2.5 is often
  sufficient.}.

\section{Dragged Map}
By default, the map is re-contoured at every frame during a drag (Ctrl
Left-mouse).  Sometimes this can be annoyingly slow and jerky so it is
possible to turn it off: \textsf{Draw $\rightarrow$ Dragged Map
  $\rightarrow$ No}.

To change this by scripting:

\texttt{(set-active-map-drag-flag 0)}


\section{Dynamic Map Sampling and Display Size}
If activated (\textsf{Edit $\rightarrow$ Map Parameters $\rightarrow$
  Dynamic Map Sampling}) the map will be re-sampled on a courser grid
when the view is zoomed out.  If ``Display Size'' is also activated,
the box of electron density will be increased in size also.  In this
way, you can see electron density for \index{big maps}big maps (many
unit cells) and the graphics still remain rotatable.

\section{Skeletonization}
\label{skeletonization}
\index{skeletonization} \index{bones} The skeleton (also known as
``Bones''\footnote{If you're living in Sweden... or Captain Kirk, that
  is.}) can be displayed for any map.  A map can be skeletonized using
\textsf{Calculate $\rightarrow$ Map Skeleton\ldots}.  Use the option
menu to choose the map and click ``On'' then ``OK'' to the generate
the map (the skeleton is off by default).

The level of the skeleton can be changed by using \textsf{Edit
  $\rightarrow$ Skeleton Parameters\ldots $\rightarrow$
  Skeletonization Level\ldots} and corresponds to the electron density
level in the map.  By default this value is 1.2 map standard
deviations.  The amount of map can be changed using \textsf{Edit
  $\rightarrow$ Skeleton Parameters\ldots $\rightarrow$ Skeleton Box
  Radius\ldots}\footnote{you may think it strange that a box has a
  radius, this is an idiosyncrasy of coot.}.  The units are in \AA
ngstr\"oms, with 40 as the default value.

The skeleton is often recalculated as the screen centre changes - but
not always since it can be an irritatingly slow calculation.
\index{skeleton regeneration}If you want to force a regeneration of
the displayed skeleton, simply centre on an atom (using the middle
mouse button) or press the ``s'' key.

\section{Masks}
\label{masks}
\index{masks} A map can be masked by a set of coordinates. Use the
scripting function: 

\texttt{(mask-map-by-protein map-number
  coords-number 0)}\footnote{the 0 is a placeholder for an as yet
  unimplemented feature (\texttt{invert?}).}.  

This will create a new
map that has density where there are no (close) coordinates.  So for
example, if you wanted to show the density around your ligand, you
would create a coordinates file that contained all the protein except
for the ligand and use those coordinates to mask the map.

There is no GUI interface to this feature at the moment.

\subsubsection{Example}
If one wanted to show just the density around a ligand:

\begin{enumerate}
\item Make a pdb file the contains just the ligand and read it in to
  Coot - let's say it is molecule 1 and the ligand is residue 3 of
  chain ``L''.
\item Get a map that covers the ligand (\emph{e.g.} from refmac).
  Let's say this map is molecule number 2.
\item Mask the map:

\texttt{(mask-map-by-molecule 2 1 \#f)}

This creates a new map.  Turn the other maps off, leaving only the
masked map.

\end{enumerate}

To get a nice rendered image, press F8 (see Section \ref{Raster3D}).


\section{Trimming}
\index{trimming atoms}
If you want to remove all the atoms\footnote{or set their occupancy to
  zero} that lie ``outside the map'' (\emph{i.e.} in low density) you can use

\texttt{(trim-molecule-by-map \emph{imol-coords imol-map density-level\\ delete/zero-occ?})}

where \texttt{\emph{delete/zero-occ?}} is \texttt{0} to remove the atoms and
\texttt{1} to set their occupancy to zero.

There is no GUI interface to this feature.


% -----------------------------------------------------------
\chapter{Validation}
% -----------------------------------------------------------

The validation functions are in the process of being written.  In
future there will be more functions, particularly those that will
interface to other programs\footnote{such as the Richardsons' reduce
  and probe}.

\section{Ramachandran Plots}
\index{Ramachandran plot} Ramachandran plots are ``dynamic''.  When
you change the molecule (\emph{i.e.} move the coordinates of some of
atoms) the Ramachandran plot gets updated to reflect those changes.
Also the underlying $\phi/\psi$ probability density changes according
to the selected residue type (\emph{i.e.} the residue under the mouse
in the plot).  There are 3 different residue types: GLY, PRO, and
not-GLY-or-PRO\footnote{the not-GLY-or-PRO is the most familiar
  Ramachandran plot.}.

When you mouse over a representation of a residue (a little square or
triangle\footnote{prolines have a grey outline rather than a black
  one, triangles are glycines.}) the residue label pops up.  The
residue is ``active'' \emph{i.e.} it can be clicked.  The ``graphics''
view changes so that the C$\alpha$ of the selected residue is centred.
In the Ramachandran plot window, the current residue is highlighted by
a green square.

% The probability levels for acceptable (yellow) and preferred (red) are
% 0.12\% and 6\% respectively and have been chosen to look like those
% from Procheck\index{Procheck}.

\section{Chiral Volumes}
The dictionary is used to identify the chiral atoms of each of the
model's residues.  A clickable list is created of atoms whose chiral
volume in the model is of a different sign to that in the dictionary.

\section{Blobs: a.k.a. Unmodelled density}
\label{sec:blobs}
This is an interface to the Blobs\index{blobs}\index{unmodelled
  density} dialog.  A map and a set of coordinates that model the
protein are required.

A blob is region of relatively high residual election density that
cannot be explained by a simple water\index{unexplained density}. So,
for example, sulfates, ligands, mis-placed sidechains or unbuilt
terminal residues might appear as blobs.  The blobs are in order, the
biggest \footnote{and therefore most interesting} at the top.

\section{Check Waters by Difference Map}
Sometimes waters can be misplaced - taking the place of sidechains or
ligands or crystallization agents such as phosphate for
example\footnote{or the water should be more properly modelled as
  anistrotropic or a split partial site}.  In such cases the variance
of the difference map can be used to identify them.

This function is also useful to check anomalous maps.  Often waters
are placed in density that is really a cation.  If such an atom
diffracts anomalously this can be identified and corrected.

By default the waters with a map variance greater than 3.5 $\sigma$ are
listed.  One can be more rigorous by using a lower cut-off:

\texttt{(set-check-waters-by-difference-map-sigma-level 3.0)}


\section{Validation Graphs}

Coot provides several graphs that are useful for model validation (on
a residue by residue basis): residue denisty fit, geometry distortion,
temperature factor variance, peptide distortion and rotamer analysis.

\subsection{Residue Density Fit}

The residue density fit is by default scaled to a map that is
calculated on the absolute scale.  Some users use maps that have maps
with density levels considerably different to this, which makes the
residue density fit graph less useful.  To correct for this you can
use the scripting function:

\texttt{(set-residue-density-fit-scale-factor \emph{factor})}

where \texttt{\emph{factor}} would be $1/(4\sigma_{map})$ (as a rule
of thumb).

\texttt{(residue-density-fit-scale-factor)} returns the current scale
factor (default 1.0).

\subsection{Rotamer Analysis}
Residue rotamers are scored according to the prior likelihood.  Note
that when CD1 and CD2 of a PHE residue are exchanged (simply a
nomenclature error) this can lead to large red blocks in the graph
(apparently due to very unlikely rotamers).  There are several other
residues that can have nomenclature errors like this.

\subsection{Temperature Factor Variance}

\subsection{Peptide $\omega$ Distortion}

\subsection{Geometry}


% -----------------------------------------------------------
\chapter{Hints}
% -----------------------------------------------------------
\label{chap-hints}
\section{Getting out of ``Translate'' Mode}
If you get stuck in "translate" mode in the GL canvas
(\emph{i.e.} mouse does not rotate the view as you would expect) simply
press and release the Ctrl key to return to "rotate" mode.

\section{Getting out of ``Label Atom Only'' Mode}
Similarly, if you are stuck in a mode where the ``Model/Fit/Refine''
buttons don't work (the atoms are not selected, only the atom gets
labelled), press and release the Shift key.

\section{Button Labels}
Button labels ending in ``\ldots'' mean that a new dialog will pop-up
when this button is pressed.

\section{Picking}
\label{sec:picking}\index{picking} Note that left-mouse in the 
graphics window is used for both atom picking and rotating the view,
so try not to click over an atom when trying to rotate the view when
in atom selection mode.  

% This was a Matrix (GL_PROJECTION) bug.  Fixed now.
%
%Sometimes, when trying to pick an atom you
%get the message ``Model atom pick failed''\index{model atom pick} even
%though you have clicked accurately over the atom.  The work-around is
%to give the model a little wiggle (using the mouse) and try the pick
%again.

\section{Resizing View}
\index{resizing view}\index{zoom} Click and drag using right-mouse (up
and down or left and right) to zoom in and out.

\section{Map}
If the ``Display'' button for the map in the ``Display Manager''
window stops working, close the ``Display Control'' window and re-open
it.  The button should now respond to clicks.

To change the map to which the scroll-wheel is attached, use
\textsf{HID $\rightarrow$ Scrollwheel $\rightarrow $Attach Scrollwheel
  to which map?}
 
\section{Slow Computer Configuration}
\index{slow computer}Several of the parameters of Coot are chosen
because they are reasonable on my ``middle-ground'' development
machine.  However, these parameters can be tweeked so that slower
computers perform better:

\begin{trivlist}
\item \texttt{(set-smooth-scroll-steps 4) ; default 8 }
\item \texttt{(set-smooth-scroll-limit 30) ; Angstroms}
\item \texttt{(set-residue-selection-flash-frames-number 3);}
\item \texttt{(set-skeleton-box-size 20.0) ; A (default 40).}
\item \texttt{(set-active-map-drag-flag 0) ; turn off recontouring every step}
\item \texttt{(set-idle-function-rotate-angle 1.5) ; turn up to 1.5 degrees}
\end{trivlist}

%\appendix
%\chapter{Some Extras}




% Have you documented:
%
% Merge molecules dialog    : done
% Mutate sequence dialog    : done
% Add OXT to residue dialog : done
% Bond Parameters dialog
% Renumber Residues dialog
% Find Bad Chiral Atoms     : done
% Validate Waters (simple)
% Validation Graphs 
% Pointer distances
% Torsions

\documentclass{book}
\usepackage{a4}
\usepackage{palatino}
%\usepackage{times}
%\usepackage{utopia}
\usepackage{euler}
\usepackage{fancyhdr}
\usepackage{epsf}

\newcommand {\atilde} {$_{\char '176}$} % tilde(~) character

%\date{1st April 2004}

\title{The Coot User Manual}
\author{Paul Emsley \\\textsf{\small emsley@ysbl.york.ac.uk}}
\makeindex
\hyphenation{tri-angle}

\begin{document}
\thispagestyle{empty}

%% Make a title page: I can't use \maketitlepage because I want a line

\vspace*{30mm}

{\huge The Coot User Manual}

\begin{picture}(300,2)
\linethickness{5pt}
\put(0,0){\line(1,0){347}}
\end{picture}

\begin{flushright}
%  \today
  1st April 2004
\end{flushright}

\vspace*{20mm}


\begin{center}
  \leavevmode
  \epsfxsize 127mm \epsffile{coot-plain-2.eps}
\end{center}

\vspace*{20mm}

\begin{picture}(300,2)
\linethickness{5pt}
\put(0,0){\line(1,0){347}}
\end{picture}

\begin{flushright}

  Paul Emsley\\\textsf{\small emsley@ysbl.york.ac.uk}
\end{flushright}

%\begin{picture}(width,height)(xoffset,yoffset)
%\end{picture}

%\newpage
\tableofcontents
\pagestyle{headings}

\chapter{Introduction}

\section{This document}
This document is the Coot User Manual, giving a brief overview of the
interactive features.  Other documentations includes (or it is planned
to include) the \index{reference manual}Coot Reference Manual and the
Coot \index{tutorial} Tutorial.  These documents should be distributed
with the source code.

\section{What is Coot?}

Coot is a stand-alone portion of CCP4's Molecular Graphics project. Its
focus is crystallographic model-building and manipulation rather than
representation (\emph{i.e.} more like Frodo than
\index{Rasmol}Rasmol).

Coot is Free Software.  You can give it away. If you don't like the
way it behaves, you can fix it yourself.

\section{What Coot is Not}
Coot is not:
\begin{itemize}
\item CCP4's official Molecular Graphics program\footnote{Coot is
    \emph{part of} that project. The official program (which contains
parts of Coot), ccp4mg is under
    the direct control of Liz Potterton and Stuart McNicholas.}
\item a program to do refinement\footnote{although it does have a
    local refinement algorithm it is no substitute for \textsc{refmac}
    (a wrapper for \textsc{refmac} is available).}
\item a database, in any respect
\item a protein crystallographic suite\footnote{that's the job of the
    CCP4 Program Suite.}.
\end{itemize}

\section{Hardware Requirements}
The code is designed to be portable to any Unix-like operating
system\footnote{including Cygwin, but currently coot is ``unstable''
  on a Cygwin system.}.  Coot certainly runs on SGI IRIX64, RedHat
Linux of various sorts, SuSe Linux\footnote{so far only 8.2 verified.}
and MacOS X (10.2).  The sgi Coot binaries shouold also 
work on IRIX.

If you want to port to some other operating system, you are
welcome\footnote{it's Free Software after all and I could give you a
  hand.}.  Note that your task will be eased by using GNU GCC to compile
the programs components.

\subsection{Mouse}
\index{mouse}Coot works best with a 3-button mouse and works better if
it has a scroll-wheel too (see Chapter 2 for more details)\footnote{I
  can get by with a one button Machintosh - but it's not ideal.}.

\section{Environment Variables}
Coot responds to several command line arguments that modify its
behaviour.  

\begin{trivlist}
\item [\texttt{COOT\_STANDARD\_RESIDUES}] The filename of the pdb file
  containing the standard amino acid residues in ``standard
  conformation''\footnote{as it is known in Clipper.}
\item [\texttt{COOT\_SCHEME\_DIR}] The directory containing auxiliary scheme
  files 
\item [\texttt{COOT\_REF\_STRUCTS}] The directory containing a set of
  high resolution pdb files used as \index{reference
    strctures}reference structures to build backbone atoms from
  C$\alpha$ positions
\item [\texttt{COOT\_REFMAC\_LIB\_DIR}] \index{dictionary, cif}Refmac's
  CIF directory containing the monomers and link descriptions.  In the future
  this may simply be the same directory in which refmac looks to find
  the library dictionary.
\item [\texttt{COOT\_RESOURCES\_DIR}] The directory that contains the
  splash screen image and the GTk application resources.
\item [\texttt{COOT\_BACKUP\_DIR}] The directory to which backup are
  written (if it exists as a directory).  If it is not, then backups
  are written to the current directory (the directory in which coot
  was started).
\end{trivlist}
  
And of course extension language environment variables are used too:

\begin{trivlist}
\item [\texttt{PYTHONPATH}] (for python modules)
\item [\texttt{GUILE\_LOAD\_PATH}] (for guile modules)
\end{trivlist}

Normally, these environment variables will be set correctly in the
coot setup script (which can be found in the \texttt{setup} directory
in the binary distribution.  See the web site (Section \ref{webpage})
for setup details.

\section{Command Line Arguments}
\index{command line arguments}
\label{sec:command_line_arguments}
Rather that using the GUI to read in information, you can use the
following command line arguments:
\begin{itemize}
\item \texttt{--script} to run a script on start up
\item \texttt{--pdb}  for pdb/coordinates file
\item \texttt{--data} for mtz, phs or mmCIF data file
\item \texttt{--map}  for a (currently only CCP4) map
\end{itemize}
So, for example, one might use: 
\begin{trivlist}
\item \texttt{coot --pdb pre-refinement.pdb --pdb post-refinement.pdb}
\end{trivlist}

\section{Web Page}
\label{webpage}
Coot has a \index{web page}web page:

\begin{trivlist}
\item \texttt{http://www.ysbl.york.ac.uk/\atilde emsley/coot}
\end{trivlist}

There you can read more about the CCP4 molecular graphics project in
general and other projects which are important for coot\footnote{coot
  has several influences and dependencies, but these will not be
  discussed here in the User Manual.}.

The web page also contains an example ``setup'' file which assigns the
environment variables to change the behaviour of Coot.

\section{Crash}
\label{sec:crash}
\index{crash}
Coot might crash on you - it shouldn't.  

\index{recover session}\index{crash recovery}There are backup files in
the directory \texttt{coot-backup}\footnote{\$COOT\_BACKUP\_DIR is used
  in preference if set}. You can recover the session (until the last
edit) by reading in the pdb file that you started with last time and
then use \textsf{File $\rightarrow$ Recover Session\ldots}.

I would like to know about coot crashing\footnote{The map-reading
  problem (documented in Section \ref{map-reading-bug}) is already
  known.} so that I can fix it as soon as possible. If you want your
problem fixed, this involves some work on your part sadly.

First please make sure that you are using the most recent version of
coot.  I will often need to know as much as possible about what you
did to cause the bug.  If you can reproduce the bug and send me the
files that are needed to cause it, I can almost certainly fix -
it\footnote{now there's a hostage to fortune.} especially if you
\index{debugger}\index{gdb}use the debugger (gdb) and send a backtrace
too\footnote{to do so, please send me the output of the following:
  \texttt{\$ gdb `which coot` \emph{corefile}} and then at the
  \texttt{(gdb)} prompt type: \texttt{where}, where
  \texttt{\emph{corefile}} is the core dump file, \texttt{core} or
  \texttt{core.4536} or some such.}.

% -----------------------------------------------------------
\chapter{Mousing and Keyboarding}
% -----------------------------------------------------------
\index{mouse buttons}
How do we move around and select things?

\vspace{0.5cm}
  \begin{tabular}{ll}
    Left-mouse Drag & Rotate view \\
    Ctrl Left-Mouse Drag &  Translates view \\
    Shift Left-Mouse &  Label Atom\\
    Right-Mouse Drag &  Zoom in and out\index{zoom}\\
    Shift Right-Mouse Drag &  Rotate View around Screen Z axis\\
    Middle-mouse & Centre on atom\\
    Scroll-wheel Forward &  Increase map contour level\\
    Scroll-wheel Backward &  Decrease map contour level
  \end{tabular}
\vspace{3mm}

See also Chapter \ref{chap-hints} for more help.

\section{Next Residue}
\begin{tabular}{ll}
  ``Space'' & Next Residue \\
  ``Shift'' ``Space'' & Previous Residue
\end{tabular}

See also ``Recentring View'' (Section \ref{sec:recentring-view}).

\section{Keyboard Contouring}

Use \texttt{+} or \texttt{-} on the keyboard if you don't have a
scroll-wheel.

\section{Keyboard Rotation}
\index{keyboard rotation}By popular request keyboard equivalents of
rotations have been added\footnote{particularly for those with
  PowerMates (the amount of rotation can be changed to 2$^\circ$ (from
  the default 1$^\circ$) using \texttt{(set-idle-function-rotate-angle
    2.0)}).}: \vspace{3mm}

\begin{tabular}{ll}
  Q & Rotate + X Axis \\
  W & Rotate - X Axis \\
  E & Rotate + Y Axis \\
  R & Rotate - Y Axis \\
  T & Rotate + Z Axis \\
  Y & Rotate - Z Axis \\
  I & Continuous Y Axis Rotation
\end{tabular}
% document rotate-x-sceen nsteps step-size here?

\section{Keyboard Translation}
\index{translation, keyboard}
\label{keyboard_translation}
\begin{tabular}{ll}
  Keypad 3 & Push View (+Z translation)\\
  Keypad . & Pull View (-Z translation)
\end{tabular}


\section{Keyboard Zoom and Clip}

\begin{tabular}{ll}

  N & Zoom out   \\
  M & Zoom in    \\
  D & Slim clip  \\
  F & Fatten clip\\
\end{tabular}

\vspace{3mm}

\section{Scrollwheel}
When there is no map, using the scroll-wheel has no effect.  If there
is exactly one map displayed, \index{contouring, map} the scroll-wheel
will change the contour level of that map.  If there are two or more
maps, the map for which the contour level is changed can be set by
\textsf{HID $\rightarrow$ Scrollwheel $\rightarrow$ Attach scroll-wheel
  to which map?} and selecting a map number.

\section{Selecting Atoms}
Several Coot functions require the selecting of atoms to specify a
residue range (for example: Regularize, Refine (Section
\ref{sec:randr}) or Rigid Body Fit Zone (Section
\ref{sec:RigidBodyRefinement})).  Select atoms with the Left-mouse.
See also Picking (Section \ref{sec:picking}).

Use the scripting function
\index{quanta-buttons}\texttt{(quanta-buttons)} to make the mouse
functions more like other molecular graphics programs to which you may
be more accustomed\footnote{See also \ref{sec:quanta-zooming}}.

\section{Virtual Trackball}
\index{trackball, virtual} You may not completely like the way the
molecule is moved by the mouse movement\footnote{Mouse movement in
  ``Spherical Surface'' mde generates a component of (often
  undesirable) screen z-rotation, particularly noticeable when the
  mouse is at the edge of the screen.}.  To change this, try:
\textsf{HID $\rightarrow$ Virtual Trackball $\rightarrow$ Flat}.  To
do this from the scripting interface: \texttt{(set-vt
  1)}\footnote{\texttt{(set-vt 0)} to turn it back to ``Spherical''
  mode.}.

If you \emph{do} want \index{screen-z rotation}\index{z-rotation}
screen-z rotation, you can either use Shift Right-Mouse Drag or set
the Virtual Trackball to Spherical Surface mode and move the mouse
along the bottom edget of the screen.

\section{More on Zooming}
\label{sec:quanta-zooming}
The function \texttt{(quanta-like-zoom)} adds the ability to zoom the
view using just Shift + Mouse movement\footnote{this is off by default
  because I find it annoying.}.

There is also a Zoom slider\index{zoom, slider} (\textsf{Draw
  $\rightarrow$ Zoom}) for those without a right-mouse button.

% -----------------------------------------------------------
\chapter{General Features}
% -----------------------------------------------------------

The map-fitting and model-building tools can be accessed by using
\textsf{Calculate $\rightarrow$ Model/Fit/Refine\ldots}.  Many
functions have \index{tooltips}tooltips\footnote{Put your mouse over a
  widget for a couple of seconds, if that widget has a tooltip, it
  will pop-up in a yellow box.}\index{tooltips} describing the
particular features and are documented in Chapter
\ref{modelling,building}.

\section{Version number}
\index{version number}
The version number of Coot can be found at the top of the ``About''
window (\textsf{Help $\rightarrow$ About}).

There is also a script function to return the version of coot:

\texttt{(coot-version)}

\section{Antialiasing}
Antialiasing (for what it's worth) can be enabled using:

\texttt{(set-do-anti-aliasing 1)}

The default is \texttt{0} (off).

\section{Molecule Number}
\index{molecule number} 
Coot is based on the concept of molecules.  Maps and coordinates are
different representations of molecules.  The access to the molecule is
\emph{via} the \emph{molecule number}.  It is often important therefore to
know the molecule number of a particular molecule.

Molecule numbers can be found by clicking on an atom in that molecule
(if it has coordinates of course). The first number in brackets in the
resulting text in the console is the molecule number.  The molecule
number can also be found in Display Control window (Section
\ref{sec:display_manager}).  It is also displayed on the left-hand
side of the molecule name in the option menus of the ``Save
Coordinates'' and ``Go To Atom'' windows.

\section{Display Issues}
The ``graphics'' window is drawn using \index{OpenGL}OpenGL.  It is
considerably smoother when using a 3D accelerated X server. 

The view is orthographic (\emph{i.e.} the back is the same size as the
front).  The default clipping is about right for viewing coordinate
data, but is often a little too ``thick'' for viewing electron
density.  It is easily changed (see Section \ref{clipping
  manipulation}).

Depth-cueing\index{depth-cueing} is linear and fixed on. There is no
anti-aliasing\footnote{coot is not the program for snazzy graphics -
  CCP4mg is the program for that sort of thing.}.

The graphics window can be resized, but it has a minimum size of
400x400 pixels.

\subsection{Origin Marker}

A yellow box\index{yellow box} called the ``origin
marker''\index{origin marker} marks the origin.  It can be removed
using:

\texttt{(set-show-origin-marker 0)}

Its state can be queried like this:

\texttt{(show-origin-marker-state)}

which returns an number (an integer).

\subsection{Raster3D output}
\label{Raster3D}Output suitable for use by Raster3D\index{Raster3D}'s
``render''\index{render} can be generated using the scripting function

\texttt{(raster3d \emph{file-name})}

where \texttt{\emph{file-name}} is such as
\texttt{"test.r3d"}\footnote{Also povray will be supported in the
  future.}.

There is a keyboard key to generate this file, run ``render'' and
display the image: Function key F8.

You can also use the function

\texttt{(render-image)}

which will create a file \texttt{coot.r3d}, from which ``render'' produces
\texttt{coot.png}. This png file is displayed using ImageMagick's display
program (by default).  Use something like:

\texttt{(set! coot-png-display-program "gqview")}

to change that to different display program ("gqview" in this case).

To change the widths of the bonds and density ``lines'' use (for example):

\texttt{(set-raster3d-bond-thickness 0.1)}

and 

\texttt{(set-raster3d-density-thickness 0.01)}

To turn off the representations of the atoms (spheres):

\texttt{(set-renderer-show-atoms 0)}



\section{Display Manager}
\label{sec:display_manager}
\index{Display Manager} This is also known as ``Map and molecule
(coordinates) display control''.  Here you can select which maps and
molecules you can see and how they are drawn\footnote{to a limited
  extent.}.  The ``Display'' and ``Active'' are toggle buttons, either
depressed (active) or undepressed (inactive).  The ``Display'' buttons
control whether a molecule (or map) is drawn and the ``Active'' button
controls if the molecule is clickable\footnote{the substantial
  majority of the time you will want your the buttons to be both
  either depressed or undepressed, rarely one but not the other.}
(\emph{i.e.} if the molecule's atoms can be labeled).

By default, the path names of the files are not displayed in the
Display Manager.  To turn them on:

\texttt{(set-show-paths-in-display-manager 1)}

\index{colour by chain}\index{atom colouring}If you pull across the
horizontal scrollbar in a Molecule view, you will see the ``Render
as'' menu.  You can use this to change between normal ``Bonds (Colour
by Atom)'',``Bonds (Colour by Chain)'' and ``C$\alpha$''
representation\index{C$\alpha$ representation}.  There is also
available ``No Waters'' and ``C$\alpha$ + ligands'' representations.

\section{The file selector}
\subsection{File-name Filtering}
\index{file-name filtering} The ``Filter'' button in the fileselection
filters the filenames according to extension.  For coordinates files
the extensions are ``.pdb'' ``.brk'' ``.mmcif'' and others.  For data:
``.mtz'', ``.hkl'', ``.phs'', ``.cif'' and for (CCP4) maps ``.ext'',
``.msk'' and ``.map''.  If you want to add to the extensions, the
following functions are available:

\begin{trivlist}
\item \texttt{(add-coordinates-glob-extension \emph{extension})}
\item \texttt{(add-data-glob-extension \emph{extension})}
\item \texttt{(add-map-glob-extension \emph{extension})}
\item \texttt{(add-dictionary-glob-extension \emph{extension})}
\end{trivlist}
where \texttt{\emph{extension}} is something like: \texttt{".mycif"}.

\subsection{Filename Sorting}
If you like your files initially sorted by date (rather than
lexographically, which is the default use:

\texttt{(set-sticky-sort-by-date)}

\section{Scripting}
\index{scripting} There is an compile-time option of adding a script
interpreter.  Currently the options are python and guile.  Hopefully,
in the near future you will be able to use both in the same
executable, but that's not available today.

Hundreds of commands are made available for use in scripting by using
SWIG.  These are are currently not well documented but can be found in
the Coot Reference Manual or the source code (\texttt{c-interface.h}).

Commands described throughout this manual (such as \texttt{(vt-surface
  1))} can be evaluated\index{executing commands} directly by Coot by
using the ``Scripting Window'' (\textsf{Calculate $\rightarrow$
  Scripting\ldots}). Note that you type the commands in the lower
entry widget and the command gets echoed (in red) and the return vaule
and any output is displayed in the text widget above.  The typed
command should be terminated with a carriage return\footnote{which
  causes the evalution of the command.}.  Files\footnote{such as the
  Coot state file (Section \ref{sec:coot_state}).} can be evaluated
(executed) using \textsf{Calculate $\rightarrow$ Run Script\ldots}.
Note that in scheme (the usual scripting language of Coot), the
parentheses are important.

\subsection{Python}
\index{python} Coot has an (optional) embedded python interpreter.
Thus the full power of python is available to you.  Coot will look for
an initialization script \index{startup settings
  (python)}\index{\texttt{.coot.py}}(\texttt{\$HOME/.coot.py}) and
will execute it if found.  This file should contain python commands
that set your personal preferences.


\subsubsection{Python Commands}
The scripting functions described in this manual are formatted
suitable for use with guile, \emph{i.e.}:

\texttt{(\emph{function} \emph{arg1} \emph{arg2\ldots})}

If you are using Python instead: the format needs to be changed to:

\texttt{\emph{function}(\emph{arg1},\emph{arg2\ldots})}

Note that dashes in guile function names become underscores for
python, so that (for example) \texttt{(raster-screen-shot)} becomes
\texttt{raster\_screen\_shot()}.


\subsection{Scheme}
\index{guile}%
\index{scheme} The scheme interpreter is made available by embedding
guile.  The initialization script used by this interpreter is
\index{startup settings (scheme)} \index{\texttt{.coot}}
\texttt{\$HOME/.coot}.  This file should contain scheme commands that
set your personal preferences.


\subsection{State}
\label{sec:coot_state}
The ``state''\index{state} of coot is saved on Exit and written to a
file called \texttt{0-coot.state.scm} (scheme)
\texttt{0-coot.state.py} (python).   This
state file contains information about the screen centre, the
clipping, colour map rotation size, the symmetry radius, and other
molecule related parameters such as filename, column labels,
coordinate filename \emph{etc.}.

Use \textsf{Calculate $\rightarrow$ Run Script\ldots} to use this file
to re-create the loaded maps and models that you had when you finished
using Coot\footnote{in that particular directory.} last time.
A state file can be saved at any time using \texttt{(save-state)}
which saves to file \texttt{0-coot.state.scm} or
\texttt{(save-state-filename "thing.scm")} which saves to file
\texttt{thing.scm}.

When Coot starts it can optionally run the commands in
\texttt{0-coot.state.scm}.  Use \texttt{(set-run-state-file-status i)}
to change the behaviour: \texttt{i} is \texttt{0} to never run this
state file at \index{startup dialog (state)}startup, \texttt{i} is
\texttt{1} to get a dialog option (this is the default) and \texttt{i}
is \texttt{2} to run the commands without question.

\section{Backups and Undo}
\label{sec:backups_undo}\index{backups}\index{undo} By default, each 
time a modification is made to a model, the old coordinates are
written out\footnote{this might be surprising since this could chew up
  a lot of disk space.  However, disk space is cheap compared to
  losing you molecule.}.  The backups are kept in a backup directory
and are tagged with the date and the history number (lower numbers are
more ancient\footnote{The coordinates are written in pdb format.}).
The ``Undo'' function discards the current molecule and loads itself
from the most recent backup coordinates.  Thus you do not have to
remember to ``Save Changes'' - coot will do it for you\footnote{unless
  you tell it not to, of course - use (\emph{e.g.})
  \texttt{(turn-off-backup 0)} to turn off the backup (for molecule 0
  in this case).}.

If you have made changes to more than one molecule, Coot will pop-up a
dialog box in which you should set the ``Undo Molecule'' \emph{i.e.}
the molecule to which the Undo operations will apply.  Further Undo
operations will continue to apply to this molecule until there are
none left.  If another Undo is requested Coot checks to see if there
are other molecules that can be undone, if there is exactly one, then
that molecule becomes the ``Undo Molecule'', if there are more than
one, then another Undo selection dialog will be displayed.



\subsection{Redo}
\index{redo}The ``undone'' modifications can be re-done using this
button.  This is not available immediately after a
modification\footnote{It works like the ``Forwards'' buttons in a web
  browser - which is not available immediately after viewing a new
  page.}.

\subsection{Restoring from Backup}
\index{restore after crash} There may be certain
circumstances\footnote{for example, if coot crashes.} in which you
wish to restore from a backup but can't get it by the ``Undo''
mechanism described above.  In that case, start coot as normal and
then open the (typically most recent) coordinates file in the
directory \texttt{coot-backup} (or the directory pointed to the
environment varialble \texttt{COOT\_BACKUP\_DIR} if it was set) .
This file should contain your most recent edits.  In such a case, it
is sensible for neatness purposes to immediately save the coordinates
(probably to the current directory) so that you are not modifying a
file in the backup directory.

See also Section \ref{sec:crash}.

\section{View Matrix}
\index{view matrix}It is sometimes useful to use this to orient the
view and export this orientation to other programs.  The orientation
matrix of the view can be displayed (in the console) using:

\texttt{(view-matrix)}

\section{Space Group}
Occassionally you may want to know the space group of a particular
molecule.  Interactively (for maps) you can see it using the Map
Properties button in the Molecule Display Control dialog.

There is a scripting interface function that returns the space group
for a given molecule \footnote{if no space group has been assigned it
  returns \texttt{``No spacegroup for this molecule''}}:

\texttt{(show-spacegroup \emph{imol})}

\section{Recentring View}
\label{sec:recentring-view}
\index{recentring view}
\begin{trivlist}
\item Use Control + left-mouse to drag around the view
\item or
\item middle-mouse over an atom.  In this case, you will often see
  ``slide-recentring'', the graphics smoothly changes between the
  current centre and the newly selected centre.
\item or
\item Use \textsf{Draw$ \rightarrow$ Go To Atom\ldots} to select an atom
  using the keyboard.  Note that you can subsequently use ``Space'' in
  the ``graphics'' window (OpenGL canvas) to recentre on the next
  C$\alpha$.
\end{trivlist}

If you don't want smooth recentring (sliding)\index{sliding}
\textsf{Draw $\rightarrow$ Smooth Recentring $\rightarrow$ Off}.  You
can also use this dialog to speed it up a bit (by decreasing the
number of steps instead of turning it off).

\section{Clipping manipulation}
\label{clipping manipulation}
\index{clipping} The clipping planes (a.k.a. ``slab''\index{slab}) can
be adjusted using \textsf{Edit $\rightarrow$ Clipping} and adjusting
the slider.  There is only one parameter to change and it affects both
the front and the back clipping planes\footnote{I find a clipping
  level of about 3.5 to 4 comfortable for viewing electron density
  maps - it is a little ``thinner'' than the default startup
  thickness.}.
The clipping can also be changed using keyboard ``D'' and ``F''.

One can ``push'' and ``pull'' the view in the screen-Z direction using
keypad 3 and keypad ``.'' (see Section \ref{keyboard_translation}).

\section{Background colour}
\index{background colour}
The background colour can be set either using a GUI dialog
(\textsf{Edit$ \rightarrow$ Background Colour}) or the function
\texttt{(set-background-colour 0.00 0.00 0.00)}, where the arguments
are 3 numbers between 0.0 and 1.0, which respectively represent the
red, green and blue components of the background colour.  The default
is (0.0, 0.0, 0.0) (black).

\section{Unit Cell}
\index{unit cell} If coordinates have symmetry available then unit
cells can be drawn for molecules (\textsf{Draw $\rightarrow$ Cell \&
  Symmetry $\rightarrow$ Show Unit Cell?}).

The unit cell of maps can be drawn without needing to setup CCP4 first.

\section{Rotation Centre Pointer}
\index{rotation centre pointer} There is a pink pointer\index{pink
  pointer} at the centre of the screen that marks the rotation centre.
The size of the pointer can be changed using \textsf{Edit
  $\rightarrow$ Pink Pointer Size\ldots} or using scripting commands:
\texttt{(set-rotation-centre-size 0.3)}.

\subsection{Pointer Distances}
The Rotation Centre Pointer is sometimes called simply ``Pointer''.
One can find distances to the pointer from any active set of atoms
using ``Pointer Distances'' (under Measures).  If you move the Pointer
(\emph{e.g.} by centering on an atom) and want to update the distances
to it, you have to toggle off and on the ``Show Pointer Distances'' on
the Pointer Distances dialog.

\section{Crosshairs}
\index{crosshairs}Crosshairs can be drawn at the centre of the screen,
using either ``c''\footnote{and ``c'' again to toggle them off.} in
graphics window or \textsf{Draw $\rightarrow$ Crosshairs\ldots}.  The
ticks are at 1.54\AA, 2.7\AA\ and 3.8\AA.

\section{Frame Rate}
\index{frame rate}
Sometimes, you might as yourself ``how fast is the
computer?''\footnote{compared to some other one.}.  Using
\texttt{Calculate $\rightarrow$ Frames/Sec} you can see how fast the
molecule is rotating, giving an indication of graphics performance.
It is often better to use a map that is more realistic and stop the
picture whizzing round.  The output is written to the console, you need
to give it a few seconds to ``settle down''.  It is best not to have
other widgets overlaying the GL canvas as you do this.

The contouring elapsed time\footnote{prompted by changing the contour
  level.} gives an indication of CPU performance.

\section{Program Output}
\index{output} Due to its ``in development'' nature (at the moment),
Coot produces a lot of ``console''\footnote{\emph{i.e.} the terminal
  in which you started Coot.} output - much of it debugging or
``informational''.  This will go away in due course.  You are advised
to run Coot so that you can see the console and the graphics window at
the same time, since feedback from atom clicking (for example) is
often written there rather than displayed in the graphics window.

\begin{itemize}
\item Output that starts ``ERROR...'' is a programming problem (and
  ideally, you should never see it).
\item Output that starts ``WARNING...'' means that something propably
unintented happened due to the unexpected nature of your input or
file(s).
\item Output that starts ``DEBUG...'' has (obviously enough) been
  added to aid debugging.  Most of them should have been cleaned up
  before release, but as Coot is constantly being developed, a few may
  slip through.  Just ignore them.
\end{itemize}


% -----------------------------------------------------------
\chapter{Coordinate-Related Features}
% -----------------------------------------------------------


\section{Read coordinates}
The format\index{coordinates format} of coordinates that can be read
by coot is either PDB or mmCIF.  To read coordinates, choose
\textsf{File $\rightarrow$ Read Coordinates} from the menu-bar.
Immediately after the coordinates have been read, the view is (by
default) recentred to the centre of this new molecule and the molecule
is displayed.  To disable the recentring of the view on reading a
coordinates file, use: \texttt{(recentre-on-read-pdb 0)}.

\subsection{Read multiple coordinate files}
\index{reading multiple pdb files}\index{multiple coordinates files}
The reading multiple files using the GUI is not available (at the
moment).  However the following scripting functions are available:

\texttt{(read-pdb-all)}

which reads all the ``*.pdb'' files in the current directory

\texttt{(multi-read-pdb \emph{glob-pattern} \emph{dir})}

which reads all the files matching \texttt{\emph{glob-pattern}} in
directory \texttt{\emph{dir}}.  Typical usage of this might be:

\texttt{(multi-read-pdb "a*.pdb" ".")}

Alternatively you can specify the files to be opened on the command
line when you start coot (see Section
\ref{sec:command_line_arguments}).

\section{Atom Info}
\index{atom info}\index{residue info} Information about about a
particular atom is displayed in the text console when you click using
middle-mouse.  Information for all the atoms in a residue is available
using \textsf{Info $\rightarrow$ Residue Info\ldots}.

\index{edit B-factors}\index{edit occupancy}The temperature factors
and occupancy of the atoms in a residue can be set by using
\textsf{Edit $\rightarrow$ Residue Info\ldots}.

\section{Atom Labeling}
\index{atom labeling}
\label{sec:atom}
Use Shift + left-mouse to label atom.  Do the same to toggle off the
label.  The font size is changeable using \textsf{Edit $\rightarrow
  $Font Size\ldots}.  The newly centred atom is labelled by default.
To turn this off use:

\texttt{(set-label-on-recentre-flag 0)}

\index{atom label, brief}Some people prefer to have atom labels that
are shorter, without the slashes and residue name:

\texttt{(set-brief-atom-labels 1)}

\section{Atom Colouring}
The atom colouring \index{colouring, atoms} \index{atom colouring}
system in coot is unsophisticated. Typically, atoms are coloured by
element: carbons are yellow, oxygens red, nitrogens blue, hydrogens
white and everything else green (see Section \ref{sec:display_manager}
for colour by chain).  However, it is useful to be able to distinguish
different molecules by colour, so by default coot rotates the colour
map of the atoms (\emph{i.e.} changes the H value in the
HSV\footnote{Hue Saturation Value (Intensity).}  colour system).  The
amount of the rotation depends on the molecule number and a
user-settable parameter:
\begin{trivlist}
\item \texttt{(set-colour-map-rotation-on-read-pdb 30)}.
\end{trivlist}

The default value is 31$^\circ$.

Also one is able to select only the Carbon atoms to change colour in
this manner: \texttt{(set-colour-map-rotation-on-read-pdb-c-only-flag
  1)}.

\section{Bond Parameters}
The various bond parameters can be set using the GUI dialog
\textsf{Draw $\rightarrow$ Bond Parameters} or \emph{via} scripting
functions.

\subsection{Bond Thickness}
\index{bond thickness}\index{width, bonds} The thickness (width) of
bonds of inividual molecules can be changed.  This can be done via the
\textsf{Bond Parameters} dialog or the scripting interface:

\texttt{(set-bond-thickness thickness imol)}

where \texttt{imol} is the molecule number. The default thickness is
3.0. The bond thickness also applies to the symmetry atoms of the
molecule.  There is no means to change the bond thickness of a residue
selection within a molecule.

\subsection{Display Hydrogens}
\index{hydrogens}Initially, hydrogens are displayed.  They can be
undisplayed using 

\texttt{(set-draw-hydrogens mol-no 0)}\footnote{they
  can be redisplayed using \texttt{(set-draw-hydrogens mol-no 1)}.}

where \texttt{mol-no} is the molecule number.

\subsection{NCS Ghosts Coordinates}
\index{NCS}It is occasionally useful when analysing
non-crystallographically related molecules to have ``images'' of the
other related molecules appear matched onto the current coordinates.
As you read in coordinates in Coot, they are checked for NCS
relationships and clicking on ``Show NCS Ghosts'' $\rightarrow$
``Yes'' $\rightarrow$ ``Apply'' will create ``ghost'' copies of them
over the reference chain\footnote{the reference chain is the first
  chain of that type in the coordinates file.}.

\subsection{NCS Maps}
Coot can use the relative transformations of the NCS-related molecules
in a coordinates molecule to transform maps. Use \textsf{Calulate}
$\rightarrow$ \textsf{NCS Maps\ldots} to do this (note the NCS maps
only make sense in the region of the reference chain (see above).
\index{NCS averaging}This will also create an NCS averaged
map\footnote{that also only makes sense in the region of the reference
  chain.}.

\section{Download coordinates}
Coot provides the possibility to download coordinates from an
\index{OCA}OCA\footnote{OCA is ``goose'' in Spanish (and Italian).
  \index{goose}} (\emph{e.g.} EBI) server\footnote{the default is the
  Weizmann Institute - which for reasons I won't go into here is
  currently much faster than the EBI server.} (\textsf{File
  $\rightarrow$ Get PDB Using Code\ldots}). A popup entry box is
displayed into which you can type a PDB accession code.  Coot will
then connect to the web server and transfer the file.  Coot blocks as
it does this (which is not ideal) but on a semi-decent internet
connection, it's not too bad.  The downloaded coordinates are saved
into a directory called \texttt{.coot}.

It is also possible to download mmCIF data and generate a map.  This
currently requires a properly formatted database structure factors
mmCIF file\footnote{which (currently) only a fraction are.}.

\section{Save Coordinates}
On selecting from the menus \textsf{File $\rightarrow$ Save
  Coordinates\ldots} you are first presented with a list of molecules
which have coordinates.  As well as the molecule number, there is the
molecule name - very frequently the name of the file that was read in
to generate the coordinates in coot initially.  However, this is only
a \emph{molecule} name and should not be confused with the filename to
which the coordinates are saved.  The coordinates \emph{filename} can
be selected using the \textsf{Select Filename\ldots} button.

If your filename ends in \texttt{.cif}, \texttt{.mmcif} or
\texttt{.mmCIF} then an mmCIF file will be written (not a ``PDB''
file).

\section{Anisotropic Atoms}
\index{anisotropic atoms} By default anisotropic atom information is
not represented\footnote{using thermal ellipsoids}.  To turn them on,
use \textsf{Draw $\rightarrow$ Anisotropic Atoms $\rightarrow$ Show
  Anisotropic Atoms?  $\rightarrow$ Yes}, or the command:
\texttt{(set-show-aniso 1)}.

You cannot currently display thermal ellipsoids\footnote{in the case
  of isotropic atoms, ellipsoids are spherical, of course.} for
isotropic atoms.

\section{Symmetry}
\index{symmetry} Coordinates symmetry is ``dynamic''.  Symmetry atoms
can be labeled\footnote{symmetry labels are in pale blue and also
  provide the symmetry operator number and the translations along the
  $a$, $b$ and $c$ axes.}.  Every time you recentre, the symmetry gets
updated.  The information shown contains the atom information and the
symmetry operation number and translations needed to generate the atom
in that position.

The symmetry can be represented as C$\alpha$s\index{C$\alpha$ symmetry
  representation}.  This along with representation of the molecule as
C$\alpha$s (Section \ref{sec:display_manager}) allow the production of
a packing diagram\index{packing diagram}.

\section{Sequence View}
\index{sequence view} The protein is represented by one letter codes
and coloured according to secondary structure.  These one letter codes
are active - if you click on them, they will change the centre of the
graphics window - in much the same way as clicking on a residue in the
Ramachandran plot.

\section{Environment Distances}
% not this residue, to symmetry if symmetry is on
% coloured bumps (C)
Environment distances are turned on using \textsf{Info $\rightarrow$
  Environment Distances\ldots}.  Contacts to other residues are shown
and to symmetry-related atoms if symmetry is being displayed.  The
contacts are coloured by atom type\footnote{contacts not involving a
  carbon atom are yellow.}.

\section{Distances and Angles}
The distance between atoms can be found using \textsf{Info
  $\rightarrow$ Distance}\footnote{Use \textsf{Angle} for an angle, of
  course.}.  The result is displayed graphically, and written to the
console.

\section{Zero Occupancy Marker}
\index{zero occupancy}Atoms of zero occupancy are marked with a grey
spot. To turn off these markers, use:

\texttt{(set-draw-zero-occ-markers 0)}

Use an argument of 1 to turn them on.

\section{Mean, Median Temperature Factors}
Coot can be used to calculate the \index{mean B-factor}mean (average)
and \index{median B-factor}median temperatures factors:

\texttt{(average-temperature-factor \emph{imol})}

\texttt{(median-temperature-factor \emph{imol})}

$-1$ is returned if there was a problem\footnote{\emph{e.g.} this
  molecule was a map or a closed molecule.}.

\section{Least-Squares Fitting}
There is currently no GUI specified for this, the scripting interface
is as follows:

\texttt{(simple-lsq-match \emph{ref-start-resno ref-end-resno ref-chain-id imol-ref
           mov-start-resno mov-end-resno mov-chain-id imol-mov
           match-type})}

where:
\begin{trivlist}
\item \texttt{\emph{ref-start-resno}} is the starting residue number
  of the reference molecule
\item \texttt{\emph{ref-end-resno}} is the last residue number
  of the reference molecule
\item \texttt{\emph{mov-start-resno}} is the starting residue number
  of the moving molecule
\item \texttt{\emph{mov-end-resno}} is the last residue number
  of the moving molecule
\item \texttt{\emph{match-type}} is one of \texttt{'CA},
  \texttt{'main}, or \texttt{'all}.
\end{trivlist}

\emph{e.g.}: 
\texttt{(simple-lsq-match 940 950 "A" 0 940 950 "A" 1 'main)}

More sophisticated (match molecule number 1 chain ``B'' on to molecule
number 0 chain ``A''):

\vspace{-2mm}
\begin{quote}
\texttt{(define match1 (list 840 850 "A" 440 450 "B" 'all))}\\
\texttt{(define match2 (list 940 950 "A" 540 550 "B" 'main))}\\
\texttt{(clear-lsq-matches)}\\
\texttt{(set-match-element match1)}\\
\texttt{(set-match-element match2)}\\
\texttt{(lsq-match 0 1)} ; match mol number 1 one mol number 0.
\end{quote}

%% \begin{trivlist}
%% \item \texttt{(define match1 (list 840 850 "A" 440 450 "B" 'all))}
%% \item \texttt{(define match2 (list 940 950 "A" 540 550 "B" 'main))}
%% \item \texttt{(clear-lsq-matches)}
%% \item \texttt{(set-match-element match1)}
%% \item \texttt{(set-match-element match2)}
%% \item \texttt{(lsq-match 0 1)}
%% \end{trivlist}

\section{More on Moving Molecules}
There are scripting functions available for this sort of thing:

\texttt{(molecule-centre \emph{imol})} 

will tell you the molecule centre \index{molecule centre} of the
\texttt{\emph{imol}}th molecule.

\texttt{(translate-by \texttt{imol x-shift y-shift z-shift})}

will translate all the atoms in molecule \texttt{\emph{imol}} by the
given amount (in {\AA}ngstr\"{o}ms)\index{translate molecule}.

\texttt{(move-molecule-to-screen-centre \emph{imol})}

will move the \texttt{\emph{imol}}th molecule to the current centre of
the screen (sometimes useful for imported ligands).  Note that this
moves the atoms of the molecule - not just the view of the molecule.


% -----------------------------------------------------------
\chapter{Modelling and Building}
% -----------------------------------------------------------
\label{modelling,building}

The functions described in this chapter manipulate, extend or build
molecules and can be found under \textsf{Calculate $\rightarrow$
  Model/Fit/Refine\ldots}.

\section{Regularization and Real Space Refinement}
\label{sec:randr}
If you have CCP4 installed, coot will read the geometry restraints for
refmac and use them in fragment (zone) idealization - this is called
``Regularization''\index{regularization}.  The geometrical restraints
are, by default, bonds, angles, planes\index{planes} and non-bonded
contacts.  You can additionally use torsion restraints\index{torsion
  restraints} by \textsf{Calculate $\rightarrow$
  Model/Fit/Refine\ldots $\rightarrow$ Refine/Regularize Control
  $\rightarrow$ Use Torsion Restraints}.

% cite Bob Diamond (1971) here somewhere.



``RS (Real Space) Refinement''\index{refinement} (after Diamond,
1971\footnote{Diamond, R. (1971). A Real-Space Refinement Procedure
  for Proteins. \emph{Acta Crystallographica} \textbf{A}27, 436-452.
  }) in Coot is the use of the map in addition to geometry terms to
improve the positions of the atoms.  Select ``Regularize'' from the
``Model/Fit/Refine'' dialog and click on 2 atoms to define the zone
(you can of course click on the same atom twice if you only want to
regularize one residue).  Coot then regularizes the residue range.  At
the end Coot, displays the intermediate atoms in white and also
displays a dialog, in which you can accept or reject this
regularization.  In the console are displayed the $\chi^2$ values of
the various geometrical restraints for the zone before and after the
regularization.  Usually the $\chi^2$ values are considerably
decreased - structure idealization such as this should drive the
$\chi^2$ values toward zero.

The use of ``Refinement'' is similar - with the addition of using a
map.  The map used to refine the structure is set by using the
``Refine/Regularize Control'' dialog.  If you have read/created only
one map into Coot, then that map will be used (there is no need to set
it explicitly).


Use, for example, \index{\texttt{set-matrix}}\texttt{(set-matrix 20.0)}
\footnote{\texttt{set\_matrix(20.0)} (using python).} to change the
weight of the map gradients to geometric gradients.  The higher the
number the more weight that is given to the map terms\footnote{but the
  resulting $\chi^2$ values are higher.}.  The default is 150.0.  This
will be needed for maps generated from data not on (or close to) the
absolute scale or maps that have been scaled (for example so that
the sigma level has been scaled to 1.0).

For both ``Regularize Zone'' and ``Refine Zone'' one is able to use a
single click to \index{single click refine}\index{refine single
  click}refine a residue range.  Pressing ``A'' on the keyboard while
selecting an atom in a residue will automatically create a residue
range with that residue in the middle.  By default the zone is
extended one residue either size of the central residue.  This can be
changed to 2 either side using \texttt{(set-refine-auto-range-step
  2)}.

Intermediate (white) atoms can be moved around with the mouse (click
and drag with left-mouse, by default).  \marginpar{\footnotesize
  \textsf{This is a useful feature}} Refinement will proceed from the
new atom positions when the mouse button is released.  It is possible
to create incorrect atom nomenclature and/or chiral volumes in this
manner - so some care must be taken.  Press the ``A'' key as you
left-mouse click to move atoms more ``locally'' (rather than a linear
shear) and Cntrl key as you left-mouse click to move just one atom.

To prevent the unintentional refinement of a large number of residues,
there is a ``heuristic fencepost'' of 20 residues.  A selection of
than 20 residues will not be regularized or refined.  The limit can be
changed using the scripting function: \emph{e.g.}
\texttt{(set-refine-max-residues 30)}.

\subsection{Dictionary}
\label{cif-dictionary}\index{cif dictionary, mmCIF dictionary}By default, 
the geometry dictionary entries for only the standard
residues are read in at the start \footnote{And a few extras, such as
  phospate}.  It may be that you particular ligand is not amongst
these.  To interactively add a dictionary entry use \textsf{File
  $\rightarrow$ Import CIF Dictionary}.  Alternatively, you can use
the function:

\texttt{(read-cif-dictionary \emph{filename})}

and add this to your \texttt{.coot} file (this may be the prefered
method if you want to read the file on more than one occassion).  

Note: the dictionary also provides the description of the ligand's
torsions.


\section{Rotate/Translate Zone}
\label{sec:rot_trans_zone}\index{rotate/translate, manual}``Rotate/Translate 
Zone'' from the ``Model/Fit/Refine'' menu allows manual movement of a
zone.  After pressing the ``Rotate/Translate Zone'' button, select two
atoms in the graphics canvas to define a residue range\footnote{if you
  want to move only one residue, then click the same atom twice.}, the
second atom that you click will be the local rotation centre for the
zone.  The atoms selected in the moving fragment have the same
alternate conformation code as the first atom you click.  To actuate a
transformation, click and drag horizontally across the relevant button
in the newly-created ``Rotation \& Translation'' dialog. The axis
system of the rotations and translations are the screen coordinates.
Alternatively \footnote{like Refinement and Regularization}, you can
click using left-mouse on an atom in the fragment and drag the
fragment around. Use Control Left-mouse to move just one atom, rather
than the whole fragment.  Click ``OK'' when the transformation is
complete.

\section{Rigid Body Refinement}
\label{sec:RigidBodyRefinement} \index{refinement, rigid body}
\index{rigid body fit}``Rigid Body Fit Zone'' from the
``Model/Fit/Refine'' dialog provides rigid body refinement.  The
selection is zone-based\footnote{like Regularization and Refinement.}.
So to refine just one residue, click on one atom twice.

Sometimes no results are displayed after Rigid Body Fit Zone.  This is
because the final model positions had too many final atom positions in
negative density.  If you want to over-rule the default fraction of
atoms in the zone that have an acceptable fit (0.75), to be (say)
0.25:

\texttt{(set-rigid-body-fit-acceptable-fit-fraction 0.25)}

\section{Baton Build}
\index{baton build} Baton build is most useful if a skeleton is
already calculated and displayed (see Section \ref{skeletonization}).
When three or more atoms have been built in a chain, Coot will use a
prior probability distribution for the next position based on the
position of the previous three.  The analysis is similar to Oldfield
\& Hubbard\footnote{T. J.  Oldfield \& R. E. Hubbard.  ``Analysis of
  C-Alpha Geometry in Protein Structures'' \emph{Proteins-Structure
    Function and Genetics} \textbf{18(4)} 324 -- 337.}, however it is
based on a more recent and considerably larger database.

Little crosses are drawn representing directions in which is is
possible that the chain goes, and a baton is drawn from the current
point to one of these new positions.  If you don't like this
particular direction\footnote{which is quite likely at first since
  coot has no knowledge of where the chain has been and cannot score
  according to geometric criteria.}, use \textsf{Try Another}.  The
list of directions is scored according to the above criterion and
sorted so that the most likely is at the top of the list and displayed
first as the baton direction.

When starting baton building, be sure to be about 3.8\AA\ from the
position of the first-placed C$\alpha$, this is because the next
C$\alpha$ is placed at the end of the baton, the baton root being at
the centre of the screen.  So, when trying to baton-build a chain
starting at residue 1, centre the screen at about the position of
residue 2.

% ``b'' key in GL canvas
\index{baton mode}Occasionally, every point is not where you want to
position the next atom.  In that case you can either shorten or
lengthen the baton, or position it yourself using the mouse.  Use
``b'' on the keyboard to swap to baton mode for the
mouse\footnote{``b'' again toggles the mode off.}.

Baton-built atoms are placed into a molecule called ``Baton Atom'' and
it is often sensible to save the coordinates of this molecule before
quitting coot.

If you try to trace a high resolution map (1.5\AA\  or better) you will
need to increase the skeleton search depth from the default (10), for
example:

\texttt{(set-max-skeleton-search-depth 20)}

Alternatively, you could generate a new map using data
to a more moderate resolution (2\AA), the map may be easier to
interpret at that resolution anyhow\footnote{high-resolution map
  interpretation is planned.}.

The guide positions are updated every time the ``Accept'' button is
clicked.  The molecule name for these atoms is ``Baton Build Guide Points''
and is is not usually necessary to keep them.

\subsection{Building Backwards}
The following senario is not uncommon: you find a nice streatch of
density and start baton building in it.  After a while you come to a
point where you stop (dismissing the baton build dialog).  You want to
go back to where you started and build the other way.  How do you do
that?

\begin{itemize}
\item Use the command: \texttt{(set-baton-build-params start-resno
    chain-id "backwards")}, where \texttt{start-resno} would typically
  be 0\footnote{\emph{i.e.} one less than the starting residue in the
    forward direction (defaults to 1).} and \texttt{chain-id} would be
  \texttt{""} (default).
\item Recentre the graphics window on the first atom of the just-build
  fragment
\item Select ``C$\alpha$ Baton Mode'' and select a baton direction
  that goes in the ``opposite'' direction to what is typically residue
  2.  This is slightly awkward because the initial baton atoms build
  in the ``opposite'' direction are not dependent on the first few
  atoms of the previously build fragment.
\end{itemize}


\subsection{Undo}
There is also an ``Undo'' button for baton-building.  Pressing this
will delete the most recently placed C$\alpha$ and the guide points
will be recalculated for the previous position.  The number of
``Undo''s is unlimited.  Note that you should use the ``Undo'' button
in the Baton Build dialog, not the one in the ``Model/Fit/Refine''
dialog (Section \ref{sec:backups_undo}).

\subsection{Missing Skeleton}
\index{skeleton, missing}Sometimes (especially at loops) you can see
the direction in which the chain should go, but there is no skeleton
(see Section \ref{skeletonization}) is displayed (and consequently no
guide points) in that direction. In that case, ``Undo'' the previous
atom and decrease the skeletonization level (\textsf{Edit
  $\rightarrow$ Skeleton Parameters $\rightarrow$ Skeletonization
  Level}).  Accept the atom (in the same place as last time) and now
when the new guide points are displayed, there should be an option to
build in a new direction.


\section{C$\alpha \rightarrow$ Mainchain}
\index{mainchain} Mainchain can be generated using a set of C$\alpha$s
as guide-points (such as those from Baton-building) along the line of
Esnouf\footnote{R. M. Esnouf ``Polyalanine Reconstruction from
  C$\alpha$ Positions Using the Program \emph{CALPHA} Can Aid Initial
  Phasing of Data by Molecular Replacement Procedures'' \emph{Acta
    Cryst. }, D\textbf{53}, 666-672 (1997).} or Jones and
coworkers\footnote{T.A.  Jones \& S. Thirup ``Using known
  substructures in protein model building and crystallography''
  \emph{EMBO J.} \textbf{5}, 819--822 (1986).}.  Briefly, 6-residue
fragments of are generated from a list of high-quality\footnote{and
  high resolution} structures. The C$\alpha$ atoms of these fragments
are matched against overlapping sets of the guide-point C$\alpha$s.
The resulting matches are merged to provide positions for the
mainchain (and C$\beta$) atoms.  This proceedure works well for
helices and strands, but less well\footnote{\emph{i.e.}  there are
  severely misplaced atoms} for less common structural features.

This function is also available from the scripting interface:

\texttt{(db-mainchain imol chain-id resno-start resno-end direction)}
    
where direction is either \texttt{"backwards"} or \texttt{"forwards"}.

% Withdrawn due to being to difficult to calculate the atom positions 
% given the phi and psi
%
%\section{Edit Phi/Psi}
%\index{edit $\phi/\psi$}This generates a Ramachandran plot with only
%one residue represented.  You can click and drag this residue round
%the plot and the coordinates in the graphics window will change to the
%$\phi/\psi$ values in the Ramachandran plot.

\section{Backbone Torsion Angles}
It is possible to edit the backbone $\phi$ and $\psi$ angles
indirectly using an option in the Model/Fit/Refine's dialog: ``Edit
Backbone Torsions..''. When clicked and an atom of a peptide is
selected, this produces a new dialog that offers ``Rotate Peptide''
which changes this residues $\psi$ and ``Rotate Carbonyl'' which
changes $\phi$.  Click and drag across the button\footnote{as for
  Rotate/Translate Zone (Section \ref{sec:rot_trans_zone}).} to rotate
the moving atoms in the graphics window.  You should know, of course,
that making these modifications alter the $\phi/\psi$ angles of more
than one residue.


\section{Rotamers}
\label{sec:rotamers}
\index{Dunbrack, Roland}\index{rotamers} The rotamers are generated
from the backbone independent sidechain library of Roland Dunbrack and
co-workers\footnote{R. L.  Dunbrack, Jr. \& F. E.  Cohen. "Bayesian
  statistical analysis of protein sidechain rotamer preferences"
  \emph{Protein Science}, \textbf{6}, 1661--1681 (1997). }. According
to this analysis, some sidechains have many rotamer
options\footnote{LYS, for example has 81.}.  By default only rotamers
with a probability (as derived from the structural database) greater
than 1\% are considered. The initial position is the most likely for
that residue type\footnote{Use \emph{e.g.}
  \texttt{(set-rotamer-lowest-probability 0.5)} to change the
  probability lower limit for the rotamer selection (note that this is
  a percentage, therefore 0.5(\%) is quite low and will allow the
  choice of more rotamers than the default.}.

Use keyboard ``.'' and ``,'' to cycle round the rotamers.

\subsection{Auto Fit Rotamer}
\index{auto-fit rotamer}``Auto Fit Rotamer'' will try to fit the
rotamer to the electron density.  Each rotamer is generated, rigid
body refined and scored according to the fit to the map.  Fitting the
second conformation of a dual conformation in this way will often fail
- the algorithm will pick the best fit to the density - ignoring the
position of the other atoms.

The algorithm doesn't know if the other atoms in the structure are in
sensible positions.  If they are, then it is sensible not to put this
residue too close to them, if they are not then there should be no
restriction from the other atoms as to the position of this residue -
the default is ``are sensible'', which means that the algorithm is
prevented from finding solutions that are too close to the atoms of
other residues. \texttt{(set-rotamer-check-clashes 0)} will stop this.

There is a scripting interface to auto-fitting rotamers:

\texttt{(auto-fit-best-rotamer \emph{resno alt-loc ins-code chain-id\\imol-coords
imol-map clash-flag lowest-rotamer-probability})}

where:

\texttt{\emph{resno}} is the residue number

\texttt{\emph{alt-loc}} is the alternate/alternative location symbol
(\emph{e.g.} \texttt{"A"} or \texttt{"B"}, but most often \texttt{""})

\texttt{\emph{ins-code}} is the insertion code (usually \texttt{""})

\texttt{\emph{imol-coords}} is the molecule number of the coordinates molecule

\texttt{\emph{imol-map}} is the molecule number of the map to which
you wish to fit the side chains

\texttt{\emph{clash-flag}} should the positions of other residues be
included in the scoring of the rotamers (\emph{i.e.} clashing with other
other atoms gets marked as bad/unlikely)

\texttt{\emph{lowest-rotamer-probability}}: some rotamers of some side
chains are so unlikely that they shouldn't be considered - typically
0.01 (1\%).

\subsection{De-clashing residues}
Sometimes you don't have a map\footnote{for example, in preparation of
  a model for molecular replacement} but nevertheless there are
clashing residues\index{clashing residues}\footnote{atoms of residues
  that are too close to each other} (for example after mutation of a
residue range) and you need to rotate side-chains to a non-clashing
rotamer.  There is a scripting interface:

\texttt{(de-clash \texttt{imol chain-id start-resno end-resno})}

\texttt{\emph{start-resno}} is the residue number of the first residue
you wish to de-clash.

\texttt{\emph{start-resno}} is the residue number of the last residue
you wish to de-clash

\texttt{\emph{imol}} is the molecule number of the coordinates molecule

This interface will not change residues with insertion codes or
alternate conformation.  The
\texttt{\emph{lowest-rotamer-probability}} is set to 0.01.


\section{Editing $\chi$ Angles}
\index{edit $\chi$ angles}Instead of using Rotamers, one can instead
change the $\chi$ angles \index{torsions}(often called ``torsions'')
``by hand'' (using ``Edit Chi Angles'' from the ``Model/Fit/Refine''
dialog). To edit a residue's $\chi_1$ press ``1'': to edit $\chi_2$,
``2'': $\chi_3$ ``3'' and $\chi_4$ ``4''.  Use left-mouse click and
drag to change the $\chi$ value.  Use keyboard ``0''\footnote{that's
  ``zero''.} to go back to ordinary view mode at any time during the
editing.  Alternatively, one can use the ``View Rotation Mode'' or use
the Ctrl key when moving the mouse in the graphics window.  Use the
Accept/Reject dialog when you have finished editing the $\chi$ angles.

It should be emphasised that for standard residues this is an option
of last resort - use the other rotamer manipulation options first.

\subsection{Ligand Torsion angles}
\index{torsion angles, ligand}\index{ligand torsion angles}For
ligands, you will need to read the mmCIF file that contains a
description of the ligand's geometry (see Section
\ref{cif-dictionary}).  By default, torsions that move hydrogens are
not included.  Only 9 torsion angles are available from the keyboard
torsion angle selection.

\section{Pep-flip}
\index{pepflip}\index{flip peptide} Coot uses the same pepflip scheme
as is used in $O$ (\emph{i.e.} the C, N and O atoms are rotated
180$^o$ round a line joining the C$\alpha$ atoms of the residues
involved in the peptide).  Flip the peptide again to return the atoms
to their previous position.


\section{Add Alternate Conformation}
\label{sec:add_alt_conf}
The allows the addition alternate (\index{dual conformations}dual,
triple \emph{etc.})  conformations to the picked residue.  By default,
this provides a choice of rotamer (Section \ref{sec:rotamers}).  If
there are not the correct main chain atoms a rotamer choice cannot be
provided, and Coot falls back to providing intermediate atoms.

The default occupancy for new atoms is 0.5.  This can be changed by
using use slider on the rotamer selection window or by using the
scripting function:

\texttt{(set-add-alt-conf-new-atoms-occupancy 0.4)}

% The intermediate atoms interface can be forced using:

% \texttt{(set-show-alt-conf-intermediate-atoms 1)}


\section{Mutation}
\index{mutate} Mutations are available on a 1-by-1 basis using the
graphics.  After selecting ``Mutate\ldots'' from the
``Model/Fit/Refine'' dialog, click on an atom in the graphics.  A
``Residue Type'' window will now appear.  Select the new residue type
you wish and the residue in the graphics is updated to the new residue
type\footnote{Note that selecting a residue type that matches the
  residue in the graphics will also result in a mutation}.  The
initial position of the new rotatmer is the \emph{a priori} most
likely rotamer. Note that in interactive mode, such as this, a residue
type match\footnote{\emph{i.e.} the current residue type matches the
  residue type to which you wish to mutate the residue} will not stop
the mutation action occurring.

\subsection{Multiple mutations}
This dialog can be found under \textsf{Calculate $\rightarrow$ Mutate
  Residue Range}.  A residue range can be assigned a sequence and
optionally fitted to the map.  This is useful converting a poly-ALA
model to the correct sequence\footnote{\emph{e.g.} after using Ca
  $\rightarrow$ Mainchain.}.

Multiple mutations\index{multi-mutate} are also supported \emph{via}
the scripting interface.  Unlike the single residue mutation function,
a residue type match \emph{will} prevent a modification of the
residue\footnote{\emph{i.e.} the residue atoms will remain untouched}.
Two functions are provided: To mutate a whole chain, use
\texttt{(mutate-chain \emph{imol} \emph{chain-id sequence})} where:

\texttt{\emph{chain-id}} is the chain identifier of the chain that you wish
to mutate (\emph{e.g.} \texttt{"A"}) and 

\texttt{\emph{imol}} is molecule number.  

\texttt{\emph{sequence}} is a list of single-letter residue codes,
such as \texttt{"GYRESDF"} (this should be a straight string with no
additional spaces or carriage returns).

Note that the number of residues in the sequence chain and those in
the chain of the protein must match exactly (\emph{i.e.} the whole of
  the chain is mutated (except residues that have a matching residue
  type).)

To mutate a residue range, use 

\begin{trivlist}
\item 
\texttt{(mutate-residue-range \emph{chain-id}
  \emph{start-res-no} \emph{stop-res-no \newline sequence})}
\end{trivlist}

where

\texttt{\emph{start-res-no}} is the starting residue for mutation

\texttt{\emph{stop-res-no}} is the last residue for mutation, \emph{i.e.}
using values of 2 and 3 for \texttt{\emph{start-res-no}} and
\texttt{\emph{stop-res-no}} respectively will mutate 2 residues.

Again, the length of the sequence must correspond to the residue range
length.

\subsection{Mutate and Autofit}
The function combines Mutation and Auto Fit Rotamer and is the easiest
way to make a mutation and then fit to the map.

\subsection{Renumbering}
\index{renumbering residues}Renumbering is straightforward using the
renumber dialog available under \textsf{Calculate $\rightarrow$
  Renumber Residue Range\ldots}.  There is also a scripting interface:

\texttt{(renumber-residue-range \emph{imol chain-id start-res-no
    last-resno offset})}

\section{Find Ligands}
\index{ligands} You are offered a selection of maps to search (you can
only choose one at a time) and a selection of molecules that act as a
mask to this map.  Finally you must choose which ligand types you are
going to search for in this map\footnote{you can search for many
  different ligand types.}.  Only molecules with less than 400 atoms
are suggested as potential ligands.  New ligands are placed where the
map density is and protein (mask) atoms are \emph{not}).  The masked
map is searched for clusters using a default cut-off of 1.0$\sigma$.
In weak density this cut-off may be too high and in such a case the
cut-off value can be changed using something such as:

\texttt{(set-ligand-cluster-sigma-level 0.8)}

However, if the map to be searched for ligands is a difference map, a
cluster level of 2.0 or 3.0 would probably be more
appropriate\footnote{less likely to generate spurious sites.}.

Each ligand is fitted with rigid body refinement to each potential
ligand site in the map and the best one for each site selected and
written out as a pdb file.  The clusters are sorted by size, the
biggest one first (with an index of 0).  The output placed ligands
files have a prefix ``best-overall'' and are tagged by the cluster
index and residue type of the best fit ligand in that site.

By default, the top 10 sites are tested for ligands - to increase this
use:

\texttt{(set-ligand-n-top-ligands 20)}

\subsection{Flexible Ligands}
\index{ligands, flexible}
If the ``Flexible?'' checkbutton is activated, coot will generate a
number of variable conformations (default 100) by rotating around the
rotatable bonds (torsions).  Each of these conformations will be fitted
to each of the potential ligand sites in the map and the best one will
be selected (again, if it passes the fitting criteria above).

Before you search for flexible ligands you must have read the mmCIF
dictionary for that particular ligand residue type (\textsf{File
  $\rightarrow$ Import CIF dictionary\index{dictionary, cif}}).

Use:

\texttt{(set-ligand-flexible-ligand-n-samples \emph{n-samples})}

where \texttt{\emph{n-samples}} is the number of samples of flexiblity
made for each ligand.  The more the number of rotatable bonds, the
bigger this number should be.

\subsection{Adding Ligands to Model}
After successful ligand searching, one may well want to add that
displayed ligand to the current model (the coordinates set that
provided the map mask).  To do so, use Merge Molecules (Section
\ref{sec:merge_molecules}).


\section{Find Waters}
\index{waters, finding} As with finding ligands, you are given a chose
of maps, protein (masking) atoms. A final selection has to be made for
the cut-off level, note that this value is the number of standard
deviation of the density of the map \emph{after} the map has been
masked.  Then the map is masked by the masking atoms and a search is
made of features in the map about the electron density cut-off value.
Waters are added if the feature is approximately water-sized and can
make sensible hydrogen bonds to the protein atoms.  The new waters are
optionally created in a new molecule called ``Waters''.

You have control over several parameters used in the water finding:

\texttt{(set-write-peaksearched-waters)} 

which writes \texttt{ligand-waters-peaksearch-results.pdb}, which
contains the water peaks (from the clusters) without any filtering and
\texttt{ligand-waters.pdb} which are a disk copy filtered waters that
have been either added to the molecule or from which a new molecule
has been created.

\texttt{(set-ligand-water-spherical-variance-limit min-d max-d)} sets
the minimum and maximum allowable distances between new waters and the
masking molecule (usually the protein).

\texttt{(set-ligand-water-spherical-variance-limit varlim)} sets the
upper limit for the density variance around water atoms. The default
is 0.12.
% $electrons^2/\AA^6$.

The map that is maked by the protein and is searched to find the
waters is written out in CCP4 format as \texttt{"masked-for-waters.map"}.

\subsection{Blobs}
After a water search, Coot will create a blobs dialog (see Section
\ref{sec:blobs}).

\subsection{Check Waters via Difference Map}
Another check of waters that one can perform is the following:

\texttt{(check-waters-by-difference-map \emph{imol-coords}
  \emph{imol-diff-map})}

where \texttt{\emph{imol-coords}} is the molecule number of the
coordinates that contain the waters to be checked

\texttt{\emph{imol-diff-map}} is the molecule number of the difference
map (it must be a difference map, not an ``ordinary'' map).  This
difference map must have been calculated using the waters. So there is
no point in doing this check immediately after ``Find Waters''.  You
will need to run Refmac first\footnote{and remember to check the
  difference map button in the ``Run Refmac'' dialog}.

This analysis will return a list of water atoms that have
outstandingly high local variance of the difference map (by default a
sphere of 1.5\AA\ centred about the atom position).  This analysis
might find waters that are actually something else, for example: part
of a ligand, a sulfate, an anion or cation, only partially occupied or
should be deleted entirely.  Coot\footnote{as yet} doesn't decide what
should be done about these atoms, it merely brings them to your
attention.  It may be interesting to use an anomalous map to do this
analysis.

There is no GUI for this feature.

\section{Add Terminal Residue}
\index{terminal residue} This creates a new residue at the C or N
terminus by fitting to the map.  $\phi/\psi$ angle pairs are selected
at random based on the Ramachandran plot probability (for a generic
residue).  By default there are 100 trials.  It is possible that a
wrong position will be selected for the terminal residue and if so,
you can reject this fit and try again with Fit Terminal
Residue\footnote{usually if this still fails after two repetitions
  then it never seems to work.}. Each of the trial positions are
scored according to their fit to the map\footnote{The map is selected
  using ``Refine/Regularize Control''} and the best one selected.  It
is probably a good idea to run ``Refine Zone'' on these new residues.

\texttt{(set-terminal-residue-do-rigid-body-refine 0)} will disable
rigid body fitting of the terminal residue fragment for
each trial residue position (the default is 1 (on)) - this may help if
the search does not provide good results.

\texttt{(set-add-terminal-residue-n-phi-psi-trials 50)} will change
the number of trials (default is 100).

\section{Add OXT Atom to Residue}
\index{terminal oxygen}\index{OXT atom}At the
C-terminus\index{C-terminus} of a chain of amino-acid residues, there
is a ``modification'' so that the C-O becomes a carbonyl, \emph{i.e.}
an extra (terminal) oxygen (OXT) needs to be added.  This atom is
added so that it is in the plane of the C$\alpha$, C and O atoms of
the residue.

Scripting usage:

\texttt{(add-OXT-to-residue imol residue-number \newline insertion-code
  chain-id)}\footnote{\emph{e.g.} \texttt{(add-OXT-to-residue 0 428 "" "A")}}, 

where \texttt{insertion-code} is typically \texttt{""}.  

Note, in order to place OXT, the N, CA, C and O atoms must be present
in the residue - if (for example) the existing carbonyl oxygen atom is
called ``OE1'' then this function will not work.

\section{Add Atom at Pointer}
By default, ``Add Atom At Pointer'' will pop-up a dialog from which
you can choose the atom type you wish to insert\footnote{including
  sulfate or phosphate ions (in such a case, it is probably useful to
do a ``Rigid Body Fit Zone'' on that new residue).}.  Using
\texttt{(set-pointer-atom-is-dummy 1)} you can by-pass this dialog and
immediately create a dummy atom at the pointer position.  Use an
argument of \texttt{0} to revert to using the atom type selection
pop-up on a button press.

The atoms are added to a new molecule called ``Pointer Atoms''.  They
should be saved and merged with your coordinates outside of Coot.

\section{Merge Molecules}
\index{merge molecules}\label{sec:merge_molecules}
This dialog can be found under ``Calculate'' in the main menubar.
This is typically used to add molecule fragments or residues that are
in one molecule to the ``working'' coordinates\footnote{For example,
  after a ligand search has been performed.}.


\section{Running Refmac}
\index{refmac}\index{running refmac}
Use the ``Run Refmac...'' button to select the dataset and the
coordinates on which you would like to run Refmac.  Note that only
dataset which had Refmac parameters set as the MTZ file was read are
offered as dataset that can be used with Refmac. By default, Coot
displays the new coordinates and the new map generated from refmac's
output MTZ file.  Optionally, you can also display the difference map.

You can add extra parameters \index{refmac parameters} (data lines) to
refmac's input by storing them in a file called
\texttt{refmac-extra-params} in the directory in which you started
coot.

Coot ``blocks''\footnote{\emph{i.e.} Coot is idle and ignores all
  input.} until Refmac has terminated\footnote{This is not an idea
  feature, of course and will be addressed in future.... Digressive
  Musing: If only computers were fast enough to run Refmac
  interactively\ldots}.

The default refmac executable\index{refmac, default}\index{default
  refmac version} is \texttt{refmac5} it is presumed to be in the
path.  If you don't want this, it can be overridden using a
re-definition either at the scripting interface or in one's
\texttt{~/.coot} file \emph{e.g.}:
\begin{trivlist}
\item \texttt{(define refmac-exec "/e/refmac-new/bin/refmac5.6.3")}
\end{trivlist}

\index{refmac map colour}After running refmac several times, you may
find that you prefer if the new map that refmac creates (after refmac
refinement) is the same colour as the previous one (from before this
refmac refinement).  If so, use:

\texttt{(set-keep-map-colour-after-refmac 1)}

which will swap the colours of then new and old refmac map so that the
post-refmac map has the same colour as the pre-refmac map and the
pre-refmac map is coloured with a different colour.


\section{Clear Pending Picks}
\index{Clear Pending Picks}\index{atom picking}Sometimes one can click
on a button\footnote{such that Coot would subsequently expect an atom
  selection ``pick'' in the graphics window.} unintentionally. This
button is there for such a case.  It clears the expectation of an
atom pick.  This works not only for modelling functions, but also
geometry functions (such as Distance and Angle).

\section{Delete}
\index{delete} Single atoms or residues can be deleted from the
molecule using ``Delete\ldots'' from the ``Model/Fit/Refine''dialog.
Pressing this button results in a new dialog, with the options of
``Residue'' (the default), ``Atom'' and ``Hydrogen Atoms''.  Now click
on an atom in the graphics - the deleted object will be the whole
residue of the atom if ``Residue'' was selected and just that atom if
``Atom'' was selected.

If you want to delete multiple items you can either use check the
``Keep Delete Active'' check-button on this dialog or use the Ctrl key
as you click on an atom.  Either of these will keep the dialog open,
ready for deletion of next item.

% document delete-atom, delete-residue, delete-residue-with-altconf here.


\section{Sequence Assignment}
You can assign a (fasta format) sequence to a molecule using:

\texttt{(assign-fasta-sequence imol chain-id fasta-seq)}

This function has been provided as a precursor to functions that will
(as automatically as possible) mutate your current coordinates to one
that has the desired sequence. It will be used in automatic side-chain
assignment (at some stage in the future).

\section{Building Links and Loops}

Coot can make an attempt to build missing linking regions or
loops\footnote{the current single function doesn't always perform very
  well in tests, which is why it is currenty available only in the
  scripting format.}.  This is an area of Coot that needs to be
improved, currently O does it much better.  We will have several
different loop tools here\footnote{I suspect that there is not one
  tool that fits for all.}.  For now:

\texttt{(fit-gap \emph{imol} \emph{chain-id} \emph{start-resno} \emph{stop-resno})}

and 

\texttt{(fit-gap \emph{imol} \emph{chain-id} \emph{start-resno} \emph{stop-resno} \emph{sequence})}

the second form will also mutate and try to rotamer fit the provided sequence.

Example usage: let's say for molecule number 0 in chain \texttt{"A"}
we have residues up to 56 and then a gap after which we have residues
62 and beyond:

\texttt{(fit-gap 0 "A" 57 61 "TYPWS")}

\section{Setting Occupancies}
As well as the editing ``Residue Info'' to change occupancies of
individual atoms, one can use a scripting function to change
occupancies of a whole residue range:

\begin{trivlist}
\item \texttt{(zero-occupancy-residue-range \emph{imol chain-id \\
resno-start resno-last})}
\end{trivlist}

example usage:

\texttt{(zero-occupancy-residue-range 0 "A" 23 28)}

This is often useful to zero out a questionable loop before submitting
for refinement.  After refinement (with refmac) there should be
relatively unbiased density in the resulting 2Fo-Fc-style and
difference maps.

Similarly there is a function to reverse this operation:

\begin{trivlist}
\item \texttt{(fill-occupancy-residue-range \emph{imol chain-id \\
      resno-start resno-last})}
\end{trivlist}




% -----------------------------------------------------------
\chapter{Map-Related Features}
% -----------------------------------------------------------

\section{Maps in General}
Maps are ``infinite,'' not limited to pre-calculated volume (the
``Everywhere You Click - There Is Electron Density''
(EYC-TIED)\index{EYC-TIED} paradigm) symmetry-related electron
density is generated automatically. Maps are easily re-contoured.
Simply use the scroll wheel on you mouse to alter the contour level
(or -/+ on the keyboard)\index{change contour level}.
 
Maps follow the molecule.  As you recentre or move about the crystal,
the map quickly follows.  If your computer is not up to re-contouring
all the maps for every frame, then use \textsf{Draw $\rightarrow$
  Dragged Map\ldots} to turn off this feature.

Unfortunately, there is a bug in map-reading\label{map-reading-bug}.
If the map is not a bona-fide CCP4 map\footnote{\emph{e.g.} it's a
  directory or a coordinate filename.}, then coot will crash.  Sorry.
A fix is in the works but ``it's complicated''.

\section{Create a map}
From MTZ, mmCIF and .phs (\textsc{phases} format)\index{phases format}
data use \textsf{File $\rightarrow$ Read Dataset\ldots}. From a CCP4
map use \textsf{File $\rightarrow$ Read Map}.  After being
generated/read, the map is immediately contoured and centred on the
current rotation centre.

\subsection{Reading CIF data}
There are several maps that can be generated from CIF files that
contain observed Fs, calculated Fs and calculated phases:

\begin{trivlist}
\item \texttt{(read-cif-data-with-phases-fo-alpha-calc
    \emph{cif-file-name})} Calculate an atom map using F$_{obs}$ and
  $\alpha_{calc}$
\item \texttt{(read-cif-data-with-phases-2fo-fc \emph{cif-file-name})}
 Calculate an atom map using F$_{obs}$, F$_{calc}$ and
  $\alpha_{calc}$
\item \texttt{(read-cif-data-with-phases-fo-fc \emph{cif-file-name})}
 Calculate an difference map using F$_{obs}$, F$_{calc}$ and
  $\alpha_{calc}$.
\end{trivlist}

\section{Map Contouring}
\index{contouring, map}Maps can be re-contoured using the middle-mouse
scroll-wheel (buttons 4 and 5 in X Window System$^{\textrm{\tiny TM}}$
terminology).  Scrolling the mouse wheel will change the map contour
level and the map it redrawn.  If you have several maps displayed then
the map that is has its contour level changed can be set using
\textsf{HID$ \rightarrow$ Scrollwheel $\rightarrow$ Attach scroll-wheel
  to which map?}.  If there is only one map displayed, then that is
the map that has its contour level changed (no matter what the
scroll-wheel is attached to in the menu).  The level of the electron
density is displayed in the top right hand corner of the OpenGL canvas.

Use Keyboard + or - to change the contour level if you don't have a
scroll-wheel\footnote{like I don't on my Mac.}.

If you are creating your map from an MTZ file, you can choose to click
on the ``is difference map''\index{difference map} button on the Column
Label selection widget (after a data set filename has been selected)
then this map will be displayed in 2 colours corresponding to + and -
the map contour level.

If you read in a map it is a difference map then there is
a checkbutton to tell Coot that.

If you want to tell Coot that a map is a difference
map\index{difference map colours} after it has been read, use:

\texttt{(set-map-is-difference-map \emph{imol})}

where \texttt{\emph{imol}} is the molecule number.

By default the map radius\footnote{actually, it's a box.} is 10\AA.
The default increment to the electron density depends on whether or
not this is a difference map (0.05 $e^-$/\AA$^3$ for a ``2Fo-Fc''
style map and 0.005 $e^-$/\AA$^3$ for a difference map).  You can
change these using \textsf{Edit $\rightarrow$ Map Parameters} or by
using the ``Properties'' button of a particular map in the Display
Control (Display Manager) window.

\section{Map contour ``scrolling'' limits}
Usually one doesn't want to look at \index{negative contour
  levels}negative contour levels of a map\footnote{in a coot
  difference map you will get to see the negative level contoured at
  the inverted level of the positive level, what I mean is that you
  don't want to see the ``positive'' level going less than 0.}, so
Coot has by default a limit that stops the contour level going beyond
(less than) 0.  To remove the limit:

\texttt{(set-stop-scroll-iso-map 0)} {for a 2Fo-Fc style map}

\texttt{(set-stop-scroll-diff-map 0)} {for a difference map}

To set the limits to negative (\emph{e.g.} -0.6) levels:

\texttt{(set-stop-scroll-iso-map-level -0.6)}

and similarly: 

\texttt{(set-stop-scroll-diff-map-level -0.6)}

where the level is specified in electrons/\AA$^3$.

\section{Map Line Width}
\index{map line width}\index{density line thickness}\index{thickness
  of density lines}The width of the lines that descibe the density can
be changed like this:

\texttt{(set-map-line-width 2)}

The default line width is 1.

\section{``Dynamic'' Map colouring}
\index{colouring, map} By default, maps get coloured according to
their molecule number.  The starting colour (\emph{i.e.} for molecule
0) is blue.  The colour of a map can be changed by \textsf{Edit
  $\rightarrow$ Map Colour..}. The map colour gets updated as you
change the value in the colour selector\footnote{takes you right back
  to the good old Frodo days, no?}.  Use ``OK'' to fix that colour.

\section{Difference Map Colouring}
For some strange reason, some crystallographers\footnote{Jan Dohnalek,
  for instance.} like to have their difference maps coloured with red
as positive and green as negative, this option is for them:

\texttt{(set-swap-difference-map-colours 1)}


\section{Map Sampling}
By default, the Shannon sampling factor is the conventional 1.5.  Use
larger values (\textsf{Edit $\rightarrow$ Map Parameters $\rightarrow$
  Sampling Rate}) for smoother maps\footnote{a value of 2.5 is often
  sufficient.}.

\section{Dragged Map}
By default, the map is re-contoured at every frame during a drag (Ctrl
Left-mouse).  Sometimes this can be annoyingly slow and jerky so it is
possible to turn it off: \textsf{Draw $\rightarrow$ Dragged Map
  $\rightarrow$ No}.

To change this by scripting:

\texttt{(set-active-map-drag-flag 0)}


\section{Dynamic Map Sampling and Display Size}
If activated (\textsf{Edit $\rightarrow$ Map Parameters $\rightarrow$
  Dynamic Map Sampling}) the map will be re-sampled on a courser grid
when the view is zoomed out.  If ``Display Size'' is also activated,
the box of electron density will be increased in size also.  In this
way, you can see electron density for \index{big maps}big maps (many
unit cells) and the graphics still remain rotatable.

\section{Skeletonization}
\label{skeletonization}
\index{skeletonization} \index{bones} The skeleton (also known as
``Bones''\footnote{If you're living in Sweden... or Captain Kirk, that
  is.}) can be displayed for any map.  A map can be skeletonized using
\textsf{Calculate $\rightarrow$ Map Skeleton\ldots}.  Use the option
menu to choose the map and click ``On'' then ``OK'' to the generate
the map (the skeleton is off by default).

The level of the skeleton can be changed by using \textsf{Edit
  $\rightarrow$ Skeleton Parameters\ldots $\rightarrow$
  Skeletonization Level\ldots} and corresponds to the electron density
level in the map.  By default this value is 1.2 map standard
deviations.  The amount of map can be changed using \textsf{Edit
  $\rightarrow$ Skeleton Parameters\ldots $\rightarrow$ Skeleton Box
  Radius\ldots}\footnote{you may think it strange that a box has a
  radius, this is an idiosyncrasy of coot.}.  The units are in \AA
ngstr\"oms, with 40 as the default value.

The skeleton is often recalculated as the screen centre changes - but
not always since it can be an irritatingly slow calculation.
\index{skeleton regeneration}If you want to force a regeneration of
the displayed skeleton, simply centre on an atom (using the middle
mouse button) or press the ``s'' key.

\section{Masks}
\label{masks}
\index{masks} A map can be masked by a set of coordinates. Use the
scripting function: 

\texttt{(mask-map-by-protein map-number
  coords-number 0)}\footnote{the 0 is a placeholder for an as yet
  unimplemented feature (\texttt{invert?}).}.  

This will create a new
map that has density where there are no (close) coordinates.  So for
example, if you wanted to show the density around your ligand, you
would create a coordinates file that contained all the protein except
for the ligand and use those coordinates to mask the map.

There is no GUI interface to this feature at the moment.

\subsubsection{Example}
If one wanted to show just the density around a ligand:

\begin{enumerate}
\item Make a pdb file the contains just the ligand and read it in to
  Coot - let's say it is molecule 1 and the ligand is residue 3 of
  chain ``L''.
\item Get a map that covers the ligand (\emph{e.g.} from refmac).
  Let's say this map is molecule number 2.
\item Mask the map:

\texttt{(mask-map-by-molecule 2 1 \#f)}

This creates a new map.  Turn the other maps off, leaving only the
masked map.

\end{enumerate}

To get a nice rendered image, press F8 (see Section \ref{Raster3D}).


\section{Trimming}
\index{trimming atoms}
If you want to remove all the atoms\footnote{or set their occupancy to
  zero} that lie ``outside the map'' (\emph{i.e.} in low density) you can use

\texttt{(trim-molecule-by-map \emph{imol-coords imol-map density-level\\ delete/zero-occ?})}

where \texttt{\emph{delete/zero-occ?}} is \texttt{0} to remove the atoms and
\texttt{1} to set their occupancy to zero.

There is no GUI interface to this feature.


% -----------------------------------------------------------
\chapter{Validation}
% -----------------------------------------------------------

The validation functions are in the process of being written.  In
future there will be more functions, particularly those that will
interface to other programs\footnote{such as the Richardsons' reduce
  and probe}.

\section{Ramachandran Plots}
\index{Ramachandran plot} Ramachandran plots are ``dynamic''.  When
you change the molecule (\emph{i.e.} move the coordinates of some of
atoms) the Ramachandran plot gets updated to reflect those changes.
Also the underlying $\phi/\psi$ probability density changes according
to the selected residue type (\emph{i.e.} the residue under the mouse
in the plot).  There are 3 different residue types: GLY, PRO, and
not-GLY-or-PRO\footnote{the not-GLY-or-PRO is the most familiar
  Ramachandran plot.}.

When you mouse over a representation of a residue (a little square or
triangle\footnote{prolines have a grey outline rather than a black
  one, triangles are glycines.}) the residue label pops up.  The
residue is ``active'' \emph{i.e.} it can be clicked.  The ``graphics''
view changes so that the C$\alpha$ of the selected residue is centred.
In the Ramachandran plot window, the current residue is highlighted by
a green square.

% The probability levels for acceptable (yellow) and preferred (red) are
% 0.12\% and 6\% respectively and have been chosen to look like those
% from Procheck\index{Procheck}.

\section{Chiral Volumes}
The dictionary is used to identify the chiral atoms of each of the
model's residues.  A clickable list is created of atoms whose chiral
volume in the model is of a different sign to that in the dictionary.

\section{Blobs: a.k.a. Unmodelled density}
\label{sec:blobs}
This is an interface to the Blobs\index{blobs}\index{unmodelled
  density} dialog.  A map and a set of coordinates that model the
protein are required.

A blob is region of relatively high residual election density that
cannot be explained by a simple water\index{unexplained density}. So,
for example, sulfates, ligands, mis-placed sidechains or unbuilt
terminal residues might appear as blobs.  The blobs are in order, the
biggest \footnote{and therefore most interesting} at the top.

\section{Check Waters by Difference Map}
Sometimes waters can be misplaced - taking the place of sidechains or
ligands or crystallization agents such as phosphate for
example\footnote{or the water should be more properly modelled as
  anistrotropic or a split partial site}.  In such cases the variance
of the difference map can be used to identify them.

This function is also useful to check anomalous maps.  Often waters
are placed in density that is really a cation.  If such an atom
diffracts anomalously this can be identified and corrected.

By default the waters with a map variance greater than 3.5 $\sigma$ are
listed.  One can be more rigorous by using a lower cut-off:

\texttt{(set-check-waters-by-difference-map-sigma-level 3.0)}


\section{Validation Graphs}

Coot provides several graphs that are useful for model validation (on
a residue by residue basis): residue denisty fit, geometry distortion,
temperature factor variance, peptide distortion and rotamer analysis.

\subsection{Residue Density Fit}

The residue density fit is by default scaled to a map that is
calculated on the absolute scale.  Some users use maps that have maps
with density levels considerably different to this, which makes the
residue density fit graph less useful.  To correct for this you can
use the scripting function:

\texttt{(set-residue-density-fit-scale-factor \emph{factor})}

where \texttt{\emph{factor}} would be $1/(4\sigma_{map})$ (as a rule
of thumb).

\texttt{(residue-density-fit-scale-factor)} returns the current scale
factor (default 1.0).

\subsection{Rotamer Analysis}
Residue rotamers are scored according to the prior likelihood.  Note
that when CD1 and CD2 of a PHE residue are exchanged (simply a
nomenclature error) this can lead to large red blocks in the graph
(apparently due to very unlikely rotamers).  There are several other
residues that can have nomenclature errors like this.

\subsection{Temperature Factor Variance}

\subsection{Peptide $\omega$ Distortion}

\subsection{Geometry}


% -----------------------------------------------------------
\chapter{Hints}
% -----------------------------------------------------------
\label{chap-hints}
\section{Getting out of ``Translate'' Mode}
If you get stuck in "translate" mode in the GL canvas
(\emph{i.e.} mouse does not rotate the view as you would expect) simply
press and release the Ctrl key to return to "rotate" mode.

\section{Getting out of ``Label Atom Only'' Mode}
Similarly, if you are stuck in a mode where the ``Model/Fit/Refine''
buttons don't work (the atoms are not selected, only the atom gets
labelled), press and release the Shift key.

\section{Button Labels}
Button labels ending in ``\ldots'' mean that a new dialog will pop-up
when this button is pressed.

\section{Picking}
\label{sec:picking}\index{picking} Note that left-mouse in the 
graphics window is used for both atom picking and rotating the view,
so try not to click over an atom when trying to rotate the view when
in atom selection mode.  

% This was a Matrix (GL_PROJECTION) bug.  Fixed now.
%
%Sometimes, when trying to pick an atom you
%get the message ``Model atom pick failed''\index{model atom pick} even
%though you have clicked accurately over the atom.  The work-around is
%to give the model a little wiggle (using the mouse) and try the pick
%again.

\section{Resizing View}
\index{resizing view}\index{zoom} Click and drag using right-mouse (up
and down or left and right) to zoom in and out.

\section{Map}
If the ``Display'' button for the map in the ``Display Manager''
window stops working, close the ``Display Control'' window and re-open
it.  The button should now respond to clicks.

To change the map to which the scroll-wheel is attached, use
\textsf{HID $\rightarrow$ Scrollwheel $\rightarrow $Attach Scrollwheel
  to which map?}
 
\section{Slow Computer Configuration}
\index{slow computer}Several of the parameters of Coot are chosen
because they are reasonable on my ``middle-ground'' development
machine.  However, these parameters can be tweeked so that slower
computers perform better:

\begin{trivlist}
\item \texttt{(set-smooth-scroll-steps 4) ; default 8 }
\item \texttt{(set-smooth-scroll-limit 30) ; Angstroms}
\item \texttt{(set-residue-selection-flash-frames-number 3);}
\item \texttt{(set-skeleton-box-size 20.0) ; A (default 40).}
\item \texttt{(set-active-map-drag-flag 0) ; turn off recontouring every step}
\item \texttt{(set-idle-function-rotate-angle 1.5) ; turn up to 1.5 degrees}
\end{trivlist}

%\appendix
%\chapter{Some Extras}



\input{coot.ind}

\end{document}

% There may be some confusion about the relation of CCP4mg and coot.
% Coot is only the protein model-building portion of ccp4mg and is (for 
% historical (and other) reasons) available as a standalone program.
%
% It is planned to merge many of the features of coot into ccp4mg at some
% stage in the future. Coot is a testbed for the development and 
% refinement of new ideas and the user interface.
%
% Coot is fast-changing and released often.  (I like to listen to (potential)
% users of coot - often comments will be of the form ``you need to add 
% Feature X''. I think how to implement this suggestion, often 
% experimenting to find the best form, and then release.  This takes 
% between a couple hours and a couple of weeks (depending on the feature
% of course)).   A surprising amount of effort has gone into designing Coot's
% GUI and experimentation on this is facilitated by the use of 
% the Gtk gui-builder, glade.
%
% Due to author's attitudes ccp4mg and coot have different release policy.  
% Broadly speaking the design principle of ccp4mg is to be stable, 
% careful, thought-out, written substantially in scripting languages,
% and released infrequently. By comparison, coot is written almost 
% entirely in c++, event-driven (rather than using separate thread to 
% process events), less thoroughly tested at each release, since the code, 
% including new features is released as soon as possible - typically 
% every few days (not that it is intended for coot to be a crash-a-minute 
% program, just that the author is less focussed on the issue).
%
% If the worse comes to the worse, we can live with a bit more duplication.
% It not unknown in the CCP4 Program Suite.  Anyway, it'll resolve itself
%  eventually.
%
% fhscale, scaleit, 
% refmac, sfall
% extend, mapmask
% many others.


\end{document}

% There may be some confusion about the relation of CCP4mg and coot.
% Coot is only the protein model-building portion of ccp4mg and is (for 
% historical (and other) reasons) available as a standalone program.
%
% It is planned to merge many of the features of coot into ccp4mg at some
% stage in the future. Coot is a testbed for the development and 
% refinement of new ideas and the user interface.
%
% Coot is fast-changing and released often.  (I like to listen to (potential)
% users of coot - often comments will be of the form ``you need to add 
% Feature X''. I think how to implement this suggestion, often 
% experimenting to find the best form, and then release.  This takes 
% between a couple hours and a couple of weeks (depending on the feature
% of course)).   A surprising amount of effort has gone into designing Coot's
% GUI and experimentation on this is facilitated by the use of 
% the Gtk gui-builder, glade.
%
% Due to author's attitudes ccp4mg and coot have different release policy.  
% Broadly speaking the design principle of ccp4mg is to be stable, 
% careful, thought-out, written substantially in scripting languages,
% and released infrequently. By comparison, coot is written almost 
% entirely in c++, event-driven (rather than using separate thread to 
% process events), less thoroughly tested at each release, since the code, 
% including new features is released as soon as possible - typically 
% every few days (not that it is intended for coot to be a crash-a-minute 
% program, just that the author is less focussed on the issue).
%
% If the worse comes to the worse, we can live with a bit more duplication.
% It not unknown in the CCP4 Program Suite.  Anyway, it'll resolve itself
%  eventually.
%
% fhscale, scaleit, 
% refmac, sfall
% extend, mapmask
% many others.


\end{document}

% There may be some confusion about the relation of CCP4mg and coot.
% Coot is only the protein model-building portion of ccp4mg and is (for 
% historical (and other) reasons) available as a standalone program.
%
% It is planned to merge many of the features of coot into ccp4mg at some
% stage in the future. Coot is a testbed for the development and 
% refinement of new ideas and the user interface.
%
% Coot is fast-changing and released often.  (I like to listen to (potential)
% users of coot - often comments will be of the form ``you need to add 
% Feature X''. I think how to implement this suggestion, often 
% experimenting to find the best form, and then release.  This takes 
% between a couple hours and a couple of weeks (depending on the feature
% of course)).   A surprising amount of effort has gone into designing Coot's
% GUI and experimentation on this is facilitated by the use of 
% the Gtk gui-builder, glade.
%
% Due to author's attitudes ccp4mg and coot have different release policy.  
% Broadly speaking the design principle of ccp4mg is to be stable, 
% careful, thought-out, written substantially in scripting languages,
% and released infrequently. By comparison, coot is written almost 
% entirely in c++, event-driven (rather than using separate thread to 
% process events), less thoroughly tested at each release, since the code, 
% including new features is released as soon as possible - typically 
% every few days (not that it is intended for coot to be a crash-a-minute 
% program, just that the author is less focussed on the issue).
%
% If the worse comes to the worse, we can live with a bit more duplication.
% It not unknown in the CCP4 Program Suite.  Anyway, it'll resolve itself
%  eventually.
%
% fhscale, scaleit, 
% refmac, sfall
% extend, mapmask
% many others.


\end{document}

% There may be some confusion about the relation of CCP4mg and coot.
% Coot is only the protein model-building portion of ccp4mg and is (for 
% historical (and other) reasons) available as a standalone program.
%
% It is planned to merge many of the features of coot into ccp4mg at some
% stage in the future. Coot is a testbed for the development and 
% refinement of new ideas and the user interface.
%
% Coot is fast-changing and released often.  (I like to listen to (potential)
% users of coot - often comments will be of the form ``you need to add 
% Feature X''. I think how to implement this suggestion, often 
% experimenting to find the best form, and then release.  This takes 
% between a couple hours and a couple of weeks (depending on the feature
% of course)).   A surprising amount of effort has gone into designing Coot's
% GUI and experimentation on this is facilitated by the use of 
% the Gtk gui-builder, glade.
%
% Due to author's attitudes ccp4mg and coot have different release policy.  
% Broadly speaking the design principle of ccp4mg is to be stable, 
% careful, thought-out, written substantially in scripting languages,
% and released infrequently. By comparison, coot is written almost 
% entirely in c++, event-driven (rather than using separate thread to 
% process events), less thoroughly tested at each release, since the code, 
% including new features is released as soon as possible - typically 
% every few days (not that it is intended for coot to be a crash-a-minute 
% program, just that the author is less focussed on the issue).
%
% If the worse comes to the worse, we can live with a bit more duplication.
% It not unknown in the CCP4 Program Suite.  Anyway, it'll resolve itself
%  eventually.
%
% fhscale, scaleit, 
% refmac, sfall
% extend, mapmask
% many others.
