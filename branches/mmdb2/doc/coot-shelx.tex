\documentclass{article}

\usepackage{epsf}
\usepackage{a4}

%\usepackage{palatino}
%\usepackage{euler}

\newcommand {\atilde} {$_{\char '176}$} % tilde(~) character
% I thought tilde was $\sim$...

\begin{document}

\title{``Coot for SHELXL'' Tutorial}
\author{Paul Emsley \& Judit \'{E} Debreczeni}

\maketitle
\tableofcontents

\pagebreak
\section{Introduction}

This is an extended description of the presentations of Coot at the ACA
meeting in Salt Lake City, 2007 and 2014 IUCr meeting in Montreal.

It should also serve as a tutorial for Coot specific for use with
SHELXL. As such, it is intended for those already familiar with the
basics of SHELXL refinement\footnote{If this is not the case you might
  want to read the
  following:\\http://shelx.uni-ac.gwdg.de/SHELX/shelx.pdf and
  http://shelx.uni-ac.gwdg.de/SHELX/pn1a.htm}, and would like to use
Coot as a model building and validation tool between SHELXL refinement
cycles. This is, however, no substitute for the standard Coot
tutorial\footnote{http://www2.mrc-lmb.cam.ac.uk/Personal/pemsley/coot/tutorials/tutorial.pdf}.


\section{A SHELXL Project}
Refinement in SHELXL requires two files:
\begin{itemize}
\item {\bf \texttt{ins}} -- instruction file containing atomic
  coordinates, restraints, constraints and other refinement
  instructions. Atomic coordinates are fractional coordinates and
  protein molecules are represented as one single chain.
\item {\bf \texttt{hkl}} -- SHELX format reflection file (typically HKLF4).
\end{itemize}
and will produce the following:
\begin{itemize}
\item {\bf \texttt{res}} -- similar to \texttt{ins} with updated
  values of refined parameters,
\item {\bf \texttt{fcf}} -- a small molecule style cif file containing
  phases,
\item {\bf \texttt{lst}} -- listing file containing diagnostic
  information, useful for spotting errors in the model or refinement
  parameters,
\item {\bf \texttt{pdb}} -- pdb style coordinate output if WPDB card was given.
\end{itemize}
All extensions are appended to a common root, e.g.
\texttt{\emph{my-project}.ins,\\\emph{my-project}.fcf} etc.

\subsection{Reading and writing SHELX coordinate files}
Coordinate files in SHELX format (\texttt{ins} and \texttt{res}) are
valid coordinate formats in Coot, i.e. they can be read and written.
However, not all SHELX commands are interpreted.

\begin{itemize}

\item {\bf CELL, LATT and SYMM cards} are used to determine the cell
  and space group when reading \texttt{ins/res}.

\item {\bf Atomic coordinates} are reformated upon as they are read:
  the single protein chain is split up based on gaps in residue
  number. It is assumed that a gap larger than 21 indicates the
  beginning of a new chain, where a new chain identifier is assigned.
  Residues are not renumbered, \emph{i.e.} residue numbers correspond
  to those in the SHELX coordinate
  file. % Is 21 the limit really? Would be an odd number to
        % choose. Actually, it would be better to check for gap AND a
        % 1000 step in residue numbers. A 1000 shift is the classical
        % way of treating chains, at least this is what SHELXPRO
        % suggests.

  By default, hydrogens are displayed for protein molecules if they
  are present in \texttt{ins} or \texttt{res}. They can be set to be
  invisible in \textsf{Edit$\to$Bond Parameters\ldots}

\item {\bf PART cards} are converted to alternate conformation
  identifiers (up to four alternate conformations). Negative PART
  numbers translate as lower case identifiers.

\item {\bf Anisotropic displacement parameters} are read and converted
  to B-factors. Coot displays thermal ellipsoids,
  too.% Is it fixed in rev-364?

\item {\bf Free variables} are mostly used in macromolecular
  refinement for\linebreak[4]co-restraining occupancies of atoms in
  alternate conformations. Coot keeps free variables read from
  \texttt{res} files and writes an updated FVAR list to \texttt{ins}.
  It assigns a new free variable on splitting a residue and
  dissociates an already existing fvar from the residue when its
  alternate conformation is deleted, but will not renumber or
  otherwise change fvars. SHELX style occupancies are not unencoded,
  they are used to reconstruct the SHELX format when coordinates are
  written to \texttt{ins}. (They should also serve as a potential
  mechanism of changing free variables within Coot in the future.)

\item {\bf AFIX cards} are handled as attributes of atoms.

\item {\bf Most other} pre-coordinate SHELXL instructions are not
  interpreted\footnote{This includes SHELX style restraints. As a
    consequence, unconventional residues, e.g. ligand molecules,
    require mmcif dictionaries (such as the Refmac dictionary) for
    refinement/regularization in Coot.} and are written to
  \texttt{ins} unchanged. (In other words, there are certain
  limitations of Coot's ability to create a fully valid \texttt{ins}
  file, and will definitely fail if we try to create \texttt{ins} from
  scratch, e.g. from a \texttt{pdb} file.\footnote{You should be using
    SHELXPRO for that.}) There is, however, a mechanism to insert
  SHELX instructions into the the new \texttt{ins} file: with the
  scripting command \texttt{(add-shelx-string-to-molecule \emph{imol
      string})}\\It is also accessible through the SHELX menu (after
  running the script\linebreak[4]\texttt{shelx-extension.scm}).

\end{itemize}

To make SHELXL coordinate formats filterable in the file selection
window,\linebreak[4]\texttt{(add-coordinates-glob-extension ".res")}
and \\\texttt{(add-coordinates-glob-extension ".ins")} should be
included in the\linebreak[4]\texttt{.coot} file.

\subsection{Reading \texttt{fcf} files}

Coot can open \texttt{fcf} files and 
calculates$\sigma_A$-weighted electron density and
difference maps.

\subsection{Reading \texttt{lst} files}
% needs chain unsplitting. -- Urgh? Don't get it.
Listing files corresponding to \texttt{ins}/\texttt{res} files can be
opened with Coot and it will filter them for interesting features in
the model and refinement flagged by SHELXL. Typically, these are
potentially split atoms and disagreeable restraints, which are then
displayed in an interesting-things-GUI. Such visualization of errors
in the model provides a great validation tool in the context of a
SHELXL project. The scripting function \texttt{(read-shelx-lst-file
  \emph{lst-file imol})} can be used to read \texttt{lst} files;
alternatively, \texttt{shelx-extensions.scm} provides a GUI option:
\textsf{SHELX$\to$Read LST File\ldots}

\subsection{Opening SHELXL projects} 

% Yet another undocumented thing.
As a consequence of SHELXL's file naming convention, it is
straight-forward to open all relevant files of a project
(\texttt{res}, \texttt{fcf} and \texttt{lst}) in one go. This can be
done using \texttt{(read-shelx-project \emph{file-name})} where
filename's root is the SHELX project's name. There is also a GUI
option in the SHELX menu (after running the script
\texttt{shelx-extension.scm}).

%% .pdb was not recognized as project file. It is now. (.insh
%% disappeared. That was just a trick to fool silly gmail.)

\subsection{Running SHELXL from Coot}

SHELXL can be run from Coot using either the function
\texttt{(shelxl-refine \emph{imol hkl-file})} or
\textsf{SHELX$\to$SHELXL refine\ldots} Specifying a \texttt{hkl}
file is optional, to be used though if the \texttt{hkl} file does not
correspond to the \texttt{res/ins} file.

Before running SHELXL, Coot creates a time-stamped \texttt{ins} file
and a sym-link to the \texttt{hkl} file. These, together with the
resulting files are stored in the coot-shelxl directory. \texttt{res}
and \texttt{fcf} files generated by SHELXL will be read in
automatically after refinement has finished.


\section{03srv}
%% How do we get the files i.e. who's going to host them?

In the small molecular world, it is generally not necessary to display
electron density maps, as they can be represented by a simple peaklist
quite precisely. However, visual examination of especially difference
density maps can reveal subtle errors in refinement and can prove
quite educational. So, suppose we have a small molecule project in
some middle stage of refinement and wish to have a glance of the
electron density map and maybe fix problematic parts of the structure.

\subsection{Reading the project}
\begin{itemize}
\item We'll start Coot: \\\texttt{\$ coot}


\item \textsf{Extensions $\rightarrow$ Modules $\rightarrow$ SHELX}

  [\textsl{An additional menu item, SHELX appears in the menu bar}.]

\item Open the project 03srv with\\\textsf{SHELX$\to$Read SHELX
    project\ldots}\\and select any of the \texttt{03srv} files, i.e.
  \texttt{res}, \texttt{lst} or \texttt{hkl}.

  [\textsl{Coot now displays the 03srv molecule and the corresponding
    electron density maps. Note that it has also created
    \texttt{03srv.fcf.cif}, a macromolecular equivalent of
    \texttt{03srv.fcf}.}]\footnote{Alternatively, we could have opened
    the coordinates and the \texttt{fcf} file
    using\\\textsf{File$\to$Open coordinates\ldots} and
    \textsf{File$\to$Open MTZ, mmCIF, fcf and phs\ldots}}

\item Note that the fcf file from SHELXL should be make using "LIST
  6".

\item Note that Coot has read \texttt{03srv.lst} as well but it has
  decided that there was "nothing interesting" listed there.

\item Now, just for speed and convenience, let's make the map radius a
  bit smaller. Select \textsf{Edit$\to$Map Parameters\ldots} and set
  the Map Radius to 5{\AA} and click \textsf{OK}.
\end{itemize}

\subsection{Hydrogens}
\begin{itemize}
\item Let's examine the structure by moving around a bit in the
  density. We see large positive and negative difference density peaks
  around some of the atoms. Let's centre on C2BA (the methyl group
  with negative density on the hydrogen atoms and positive between
  them) by middle-clicking on it (Figure \ref{fig:hydrogen-badness}).

"Hydrogens in wrong place!" --- you might say.

\item Indeed, these hydrogens are incorrectly set: they are exactly 60
  degrees off. Let's get rid of them. Select
  \textsf{Calculate$\to$Model/Fit/Refine},

[\textsl{Coot displays the Model/Fit/Refine window}]

click \textsf{Delete}, 

[\textsl{Coot displays the Delete item window}]

click on the \textsf{Atom} and \textsf{Keep Delete Active} buttons and
click on the hydrogen atoms to delete them. Finally, close the
\textsf{Delete} window.

\item Now we need to tell SHELXL to set the hydrogens properly. As
  seasoned SHELXers, we know that methyl groups require the "HFIX 137"
  instruction. Select \textsf{SHELX$\to$Add SHELXL instruction\ldots}
  from the menu bar.

[\textsl{A window appears where we can select our molecule and type the new command.}]

Select 03srv.res for model and type \texttt{HFIX 137 C2BA} in the
instruction entry box. Click \textsf{OK}.

\item Let's undisplay the difference density --- just to see the
  improvement after refinement. Click \textsf{Display Manager} in the
  menu bar

[\textsl{The Display Control window appears}]

and click the \textsf{Display} button next to \textsf{03srv.fcf.cif Difference SigmaA}.

\item To rerun SHELXL on the model with the new instruction for
  hydrogens, click \textsf{SHELX$\to$SHELXL Refine\ldots} and select
  03srv.res as molecule to be refined. We do not need to set the
  \texttt{hkl} file as we already have a \texttt{03srv.hkl} to refine
  against. Click \textsf{Refine}.

  [\textsl{Coot now runs SHELXL in the background and displays the new
    structure and electron density maps when the refinement has
    finished. }(Figure \ref{fig:hydrogen-fixed})]

\begin{figure}[htbp]
    \begin{minipage}[t]{3in}
        \epsfxsize 2.5in
        \epsffile{hydrogens-incorrect.eps}
        \parbox{2.5in}{\caption{\label{fig:hydrogen-badness}
            Difference density peaks around C2BA indicating that
            hydrogen atoms were incorrectly set}}
    \end{minipage}
    \hfill
    \begin{minipage}[t]{3in}
        \epsfxsize 2.5in
        \epsffile{hydrogens-fixed.eps}
        \parbox{2.5in}{\caption{\label{fig:hydrogen-fixed} No
            difference density around the same region after SHELXL
            refinement with corrected hydrogens}}
    \end{minipage}
\end{figure}

SHELXL has set the hydrogens correctly and we see no difference
density around \texttt{C2BA}. Well done.

\item Can you fix any other hydrogens? (Sure you can. C9AA has similar
  issues, for instance.) This will be your homework.

\end{itemize}

\subsection{Symmetry}
It might be quite interesting to look at crystal packing and
intermolecular interactions by displaying symmetry equivalents. This
is how it is done in Coot:

\begin{itemize}
\item First, for clarity, we will switch off maps: in the
  \textsf{Display Control} window click on the \textsf{All} toggle
  button above the maps.
\item Select \textsf{Draw$\to$Cell and Symmetry\ldots} and click
  \textsf{Yes} in the master symmetry switch. Finally, click
  \textsf{Apply}.

\item You can switch on the unit cell too, if you like.

\end{itemize}
We are done with 03srv, you can close Coot now. 

%% Is not saying anything good enough in terms of political correctness?


\section{Viscotoxin-B2}
Let's take Viscotoxin-B2 (an atomic resolution structure of a small
peptide toxin, PDB-ID: 2V9B) as an example to show how Coot can be
used for validation and model building of a protein refined with
SHELXL.

\subsection{Analysing \texttt{lst} files}

\begin{itemize}
\item Start Coot with\\\texttt{\$ coot}


\item \textsf{Extensions $\rightarrow$ Modules $\rightarrow$ SHELX}

  [\textsl{An additional menu item, SHELX appears in the menu bar}.]


\item
  Open the project viscotoxin-b2 with \textsf{SHELX$\to$Read SHELX
    project\ldots}\\and select any of the \texttt{viscotoxin-b2}
  files, i.e. \texttt{res}, \texttt{lst} or \texttt{hkl}.

[\textsl{Coot has read the \texttt{res} and \texttt{fcf} files. It has
  also found quite a few interesting features in the \texttt{lst} file
  and displays them in the \textsf{Interesting things from SHELXL}
  window.}]

As we browse the Interesting things from SHELXL list, we see many
disagreeable bond distance, angle distance and even chiral volume
restrains (DFIX, DANG and CHIV) around residue 1024. Navigate there by
clicking on, say, the first disagreeable DANG button.

What a mess! --- the structure around 1024 is strongly distorted and
out of density.

\item Other validation tools within Coot, such as the Ramachandran
  plot, B-factor Variance Graph and Density Fit Graph show 1024 A as
  an outlier\linebreak[4](Figure \ref{fig:viscotoxin}). Click
  \textsf{Validate$\to$Temp. fact. variance analysis} and
  select \textsf{viscotoxin-b2.res}. Examine the plot by
  clicking on outliers. Similarly, you can take a look at the
  Ramachandran plot and Density fit analysis as well (you will find a
  similar degree of badness there).

  Note that Coot has split the molecule into four chains: A and B are
  protein chains, whereas C is for sulfate ions and D for waters.


\item Let's fix 1024 A now. The easiest way to do that is real space
  refinement. We select a map first: select
  \textsf{Calculate$\to$Model/Fit/Refine\ldots}

  [\textsl{The \textsf{Model/Fit/Refine} window pops up\ldots}]

  where we click on \textsf{Select map\ldots} and select
  viscotoxin-b2.fcf.cif SigmaA. Now click \textsf{Real Space Refine
    Zone} in the \textsf{Model/Fit/Refine} window and select the
  region 1023-1025 by clicking on these residues in the graphics
  window.

  [\textsl{Coot displays white intermediate atoms that migrate towards
    the density. The \textsf{Accept Refinement} window pops up.}]

  Try to pull the atoms into the density. It might be a good idea to
  start with the carbonyl oxygen of 1024.\footnote{Eagle-eyed cooters
    might have noticed that some of the hydrogen atoms have flown away
    upon real-space-refine. This is due to some library
    incompatibilities. Nothing to worry about, SHELXL will duly knock
    them into shape.} Click \textsf{Accept} when done.

  [\textsl{The Ramachandran plot and the Density fit graph look much
    better.}]
\end{itemize}

\begin{figure}[htbp]
    \begin{center}
    \epsfxsize 5in
    \epsffile{viscotoxin.eps}
    \parbox{5in}{\caption{\label{fig:viscotoxin} Mistraced region in
        \texttt{viscotoxin-b2.res.} When reading the project, Coot has
        extracted potentially split atoms and disagreeable restraints
        from the listing file and displayed them in an
        interesting-things-GUI (upper-right corner). You can navigate
        to the problematic regions by clicking on the items in the
        list. The B-factor Variance Graph (bottom-right corner) shows
        a couple of outliers, one of which is the residue shown in the
        graphics window.}}
    \end{center}
\end{figure}

\subsection{Adding Atoms}
\begin{itemize}

\item Let's locate missing bits of the structure now. Select
  \textsf{Validate$\to$Difference Map Peaks\ldots} and hit
  \textsf{Find blobs}.

  [\textsl{The new pop-up contains a clickable list of difference
    density blobs. We click on the first one and Coot navigates to a
    large green cloud.}]

  Let's examine the first blob: it is clearly tetrahedral. We know
  that the crystals grew from ammonium-sulfate, so we will try to fit
  a sulfate ion here.

\item Select \textsf{Place Atom At Pointer} in the \textsf{Model/Fit/Refine} window.

[\textsl{We are given a new window to pick what kind of atom/molecule we wish to place here.}]

Select \textsf{SO4}. Select viscotoxin-b2.res to put the sulfate into
and hit \textsf{OK}.

[\textsl{A pretty sulfate ion appears, but it is a bit out of density. Let's fix it.}]

\item Simply select \textsf{Real Space Refine} in the
  \textsf{Model/Fit/Refine} window and click twice on the new sulfate
  ion. And finally hit \textsf{Accept} in the \textsf{Accept
    Refinement?} window. Done.

%% Oh dear. I thought of showing properly set occupancies for SO4 (i.e. 11.00). And what do I get? 1.00. Baah.
%% In no time, the PDB will be full of residues with different occupancies for all atoms.

\end{itemize}

\subsection{Adding Alternate Conformations}

SHELXL gives us the flexibility to refine occupancies of discretely
disordered atoms, where free variables are used to control occupancies
of alternate atom pairs. Coot, on the other hand, can set free
variables and occupancies according to the SHELX convention.

\begin{itemize}
\item You might remember another outlier in the B-factor Variance
  Graph: Ile 2035 in chain B. Let's click on the corresponding bar in
  the graph (it is the salmon coloured bar in the second row).

  [\textsl{Coot centers on 2035. We see a side chain with some
    difference density blobs around it. It is a hint for an alternate
    conformation.}]

\item To add an alternate conformation here, we click on the ``Add
  Alternative Conformation to a Residue'' button in the Coot toolbar
  (down the right-hand side).

  In the pop-up
  you can choose to split the whole residue or only atoms beyond
  C$\alpha$. Click on any atom of residue 2035.

  [\textsl{Coot has now split the residue, displays a set of white
    atoms. We can pick a rotamer in the rotamer selection window.}]

\item In this case rotamer 3 fits the density best, so let's select
  that. We can set the occupancy of the new conformer to, say, 0.3.
  (We do not need to be very precise, it will be refined by SHELXL
  anyway.) Click \textsf{Accept}.

  [\textsl{Well, C$\delta$ is still not perfect.}]

\item We pull C$\delta$ to the density with real space refinement.

  But how about the occupancies? Select \textsf{Measures/Residue
    Info\ldots} and click on any atom of this residue to check what
  Coot did about the occupancies.

  [\textsl{In the \textsf{Residue Info} pop-up we see that occupancies
    are set to 161.00 and -161.00 (and a new free variable, nr. 16,
    has been assigned to them).}]

\end{itemize}

We can now save coordinates or perhaps re-run SHELXL to see how well
we managed to fix things (this should take a couple of minutes).

\end{document}




